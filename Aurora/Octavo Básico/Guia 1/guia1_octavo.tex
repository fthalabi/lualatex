\def\name{Prof. Fernando Halabi}
\def\mail{fernando.halabi@colegionuevaaurora.cl}
\def\title{Guía 1}
\def\subtitle{Operatoria con números negativos}
\def\colegio{Liceo Bicentenario Nueva Aurora de Chile}
\def\asignatura{Matemática}
\def\curso{Octavo Básico}

\documentclass{prueba}
\begin{document}
\titulo{Objetivo e instrucciones generales}

El siguiente material corresponde a una evaluación con {\bfseries nota acumulativa}, que 
tiene como intención preparar la próxima prueba de cierre de unidad. Específicamente,
aquí se abarcará: Suma, resta, multiplicación y valor absoluto de números enteros 
para la resolución de expresiones aritméticas y problemas contextualizados.    

La evaluación es para completarla en la clase y es necesario {\bfseries contestar con lápiz pasta}
para tener derecho a reclamo.

\titulo{Resumen de contenidos claves}

Para {\bfseries sumar números} enteros:
\begin{itemize}
    \item Si los números tienen el {\bfseries mismo signo}: Se suman los números, %
    ignorando los signos, y el resultado final mantiene el signo original de los números. %
    \begin{equation*}
        \boxed{2+3=5} \quad \text{ó}  \quad \boxed{-2-3=-5}
    \end{equation*}
    \item Si los números tienen {\bfseries distinto signo}: Se restan los números, %
    el más grande menos el más pequeño, ignorando los signos, y el resultado final tiene %
    el mismo signo que el número más grande. 
    \begin{equation*}
        \boxed{-2+3=1} \quad\text{ó}\quad \boxed{2-3=-1}
    \end{equation*}
\end{itemize}

Para {\bfseries multiplicar números} enteros:
\begin{itemize}
    \item Si los factores tienen el {\bfseries mismo signo}: El producto tiene signo %
    positivo.
    \begin{equation*}
        \boxed{2 \cdot 3=6} \quad\text{ó}\quad \boxed{(-2)\cdot(-3)=6}
    \end{equation*}
    \item Si los factores tienen {\bfseries distinto signo}: El producto tiene signo %
    negativo.
    \begin{equation*}
        \boxed{2 \cdot (-3)=(-6)} \quad\text{ó}\quad \boxed{(-2) \cdot 3 = (-6)}
    \end{equation*}
\end{itemize}

Para resolver expresiones aritméticas:
\begin{itemize}
    \item Primero resuelve los paréntesis, empezando desde el de más adentro hacia afuera, 
    luego los valores absolutos y las multiplicaciones, para terminar con las sumas 
    y restas.
    \item Un menos afuera de un paréntesis se interpreta como multiplicar por $-1$.  
    \begin{equation*}
        \boxed{-(-3) = (-1)\cdot(-3) = 3} \quad \text{ó} \quad
        \boxed{-[-(-3)] = (-1)\cdot(-1)\cdot(-3) = (-3)}
    \end{equation*}
\end{itemize}

El {\bfseries valor absoluto} de un número es la distancia, o cantidad de pasos, que
hay entre dicho número y el cero en la recta numérica.
\begin{equation*}
    \boxed{\left|2\right|=2} \quad \text{ó} \quad \boxed{\left|-2\right|=2}
\end{equation*}

\titulo{Pauta de cotejo}

Para la corrección de la guía, se le asignará puntaje a cada respuesta según los criterios
que se encuentran detallados en la tabla a continuación.

\pauta

\parte{} Calcula considerando la prioridad de operación. Recuerda incluir un desarrollo 
completo y ordenado en el espacio señalizado.

\pregunta{} $-[-(-3-20)+5\cdot(4-8)+(-6+4)\cdot 2]$
\desarrollo[3cm]

\pregunta{} $(-50)\cdot(5+6)+(-2)\cdot[(8+2)\cdot(-5+3)]$
\desarrollo[3cm]

\pregunta{} $-2\cdot\{-5+(5-1)\cdot(-3)+2-(2-5)\cdot(-2-1)\}$
\desarrollo[3cm]

\pregunta{} $-\left|-4\right|\cdot\left(-3\cdot\left|5-8\right|+2\cdot\left|17-9\right|\right)$
\desarrollo[3cm]

\parte{} Resuelve los siguientes problemas. Coloque su desarrollo, que puede incluir dibujos o
diagramas según sea necesario, y la respuesta en los espacios asignados.

\pregunta{} Un pozo de petróleo tiene una profundidad de 856 [m]. Si una bomba extrae el 
petróleo y lo deja en un depósito a 32 [m] de altura, ¿Qué distancia vertical recorre el 
líquido luego de ser extraído?
\desarrollo[3cm]
\respuesta

\pregunta{} Encuentre el o los números enteros que cumplen las siguientes condiciones: Su valor 
absoluto es menor que 9, y además, es o son divisibles por 2 y 3 al mismo tiempo. 
¿Cuáles son los números?
\desarrollo[3cm]
\respuesta

\end{document}