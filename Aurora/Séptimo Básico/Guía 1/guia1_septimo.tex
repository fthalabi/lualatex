\def\name{Prof. Fernando Halabi}
\def\mail{fernando.halabi@colegionuevaaurora.cl}
\def\title{Guía 1}
\def\subtitle{Operatoria de números enteros ($\mathbb{Z}$)}
\def\colegio{Liceo Bicentenario Nueva Aurora de Chile}
\def\asignatura{Matemática}
\def\curso{Séptimo Básico}

\documentclass{prueba}

\begin{document}
\titulo{Objetivo e instrucciones generales}

El siguiente material corresponde a una evaluación con {\bfseries nota acumulativa}, que 
tiene como intención preparar la próxima prueba de cierre de unidad. Específicamente,
el alumno deberá: Distinguir entre el signo del número y el símbolo de la adición o la
sustracción; Aplicar las distintas propiedades de adición para números enteros en 
la resolución de ejercicios;
Aplicar el concepto de valor absoluto para números enteros; y por último, 
resolver ejercicios combinados respetando la prioridad de operación.

La evaluación es para completarla en la clase y es necesario {\bfseries contestar con lápiz pasta}
para tener derecho a reclamo.

\titulo{Resumen de contenidos claves}

\lipsum[1]

\titulo{Pauta de cotejo}

Para la corrección de la guía, se le asignará puntaje a cada respuesta según los criterios
que se encuentran detallados en la tabla a continuación.

\pauta
\newpage

\parte{} Complete 

\end{document}