\def\name{Prof. Fernando Halabi}
\def\mail{fernando.halabi@colegionuevaaurora.cl}
\def\title{Taller 1}
\def\subtitle{Operatoria con números negativos}
\def\colegio{Liceo Bicentenario Nueva Aurora de Chile}
\def\asignatura{Matemática}
\def\curso{Octavo Básico}

\documentclass{prueba}
\begin{document}

\parte{} Calcula considerando la prioridad de operación. Recuerda incluir un desarrollo 
completo y ordenado en el espacio señalizado.

\pregunta{} $-2\cdot\left[\left(3-10\right)\cdot(5-12) + \left|-5\right|\right]$
\desarrollo[4.5cm]

\pregunta{} $\left|-3\cdot(-5-3) - 7\right|$
\desarrollo[4.5cm]

\pregunta{} $-\left[\left(-2-7\right)\cdot\left(5-8\right)\cdot\left(-2\right) + 5\right]$
\desarrollo[4.5cm]

\parte{} Resuelve los siguientes problemas. Coloque su desarrollo, que puede incluir dibujos o
diagramas según sea necesario, y la respuesta en los espacios asignados.

\pregunta{} La era de los romanos empieza en el año 754 antes de Cristo y la de los 
musulmanes en el año 622 después de Cristo. ¿Cuántos años transcurrieron desde el 
comienzo de la era romana hasta el comienzo de la era musulmana?
\desarrollo[4cm]
\respuesta

\pregunta{} Encuentre el o los números enteros que cumplen las siguientes condiciones
al mismo tiempo:
\begin{itemize}
    \item Valor absoluto mayor que 4.
    \item Valor absoluto menor que 12.
    \item Divisible por 5.
\end{itemize}
\desarrollo[4cm]
\respuesta

\end{document}