\def\name{Prof. Fernando Halabi}
\def\mail{fernando.halabi@colegionuevaaurora.cl}
\def\title{Guía 1}
\def\subtitle{Potencias}
\def\colegio{Liceo Bicentenario Nueva Aurora de Chile}
\def\asignatura{Matemática}
\def\curso{Primero Medio}

\documentclass{prueba}

\begin{document}
\titulo{Objetivo e instrucciones generales}

El siguiente material corresponde a una evaluación con {\bfseries nota acumulativa}, que 
tiene como intención preparar la próxima prueba de cierre de unidad. Específicamente,
el alumno deberá:

\begin{itemize}[noitemsep]
    \item Reconocer la potencia de una potencia como una multiplicación iterativa.
    \item Reconocer el significado del exponente 0 y de los exponentes enteros 
    negativos.
    \item Aplicar las propiedades de multiplicación, división y potenciación de 
    potencias.
    %\item Resolver problemas de la vida diaria relacionados con potencias de base
    %racional y exponente entero
\end{itemize}

La evaluación es para completarla en la clase y es necesario {\bfseries contestar con lápiz pasta}
para tener derecho a reclamo.

\titulo{Resumen de contenidos claves}

Sean $a$ y $b$ números racionales\footnote{Los números racionales ($\mathbb{Q}$) incluyen: $2$, $-8$, $\frac{3}{4}$, $-1.\overline{3}$, $\pi$, etc\dots},
con $b$ distinto de cero\footnote{Es necesario que $b$ sea distinto de cero, ya que no se puede dividir por cero.}
y $n$, $m$ números enteros\footnote{Los números enteros ($\mathbb{Z}$) incluyen: \dots,$-2$, $-1$, $0$, $1$, $2$,\dots}. 
Para estos números se cumplen las siguientes propiedades de potencias:

\begin{multicols}{3}
    \begin{itemize}[label={$\color{primarycolor}\diamond$}]
        \item $a^0 = 1$
        \item $a^1 = a$
        \item $\left(\dfrac{a}{b}\right)^{-1} = \dfrac{b}{a}$
        \item $a^n \cdot a^m = a^{n+m}$
        \item $\dfrac{a^n}{a^m} = a^{n-m}$
        \item $\left(a^n\right)^m = a^{n\cdot m}$
        \item $a^n \cdot b^n = \left(a\cdot b\right)^n$
        \item $\dfrac{a^n}{b^n} = \left(\dfrac{a}{b}\right)^n$
        \item[\vspace{\fill}]
    \end{itemize}
\end{multicols}

\titulo{Pauta de cotejo}

Para la corrección de la guía, se le asignará puntaje a cada respuesta según los criterios
que se encuentran detallados en la tabla a continuación.

\pauta
\newpage

\parte{} Simplifica las siguientes expresiones utilizando propiedades de potencias. 
No olvides incluir desarrollo completo y ordenado en el espacio señalizado. 

\pregunta{} $\left(\dfrac{2^{-1}\cdot 3^{\frac{1}{4}}}{2^{-3}\cdot 3^{\frac{1}{2}}}\right)^{-2}$
\desarrollo[5cm]

\pregunta{} $\left(\dfrac{3^{-4}\cdot 5^{-1}}{3^2\cdot 5^{-3}}\right)^{-\frac{1}{2}}\cdot\left(\dfrac{3^4\cdot 5^3}{3^2\cdot 5^4}\right)^{-1}$
\desarrollo[6cm]

\pregunta{} $\left(\dfrac{2^{-4}}{2^{-2}-2^{-3}}\right)^{-2}$
\desarrollo[6cm]

\pregunta{} $\left(\dfrac{1}{2^{-3}} - \dfrac{1}{2^{-1}}\right)^{-3}$
\desarrollo[6cm]

\pregunta{} $\left(\dfrac{7^{-1}}{2^{-1}+3^{-1}+6^{-1}}\right)^{-2}$
\desarrollo[7cm]

\end{document}