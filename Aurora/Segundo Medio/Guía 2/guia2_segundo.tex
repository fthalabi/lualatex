\def\name{Prof. Fernando Halabi}
\def\mail{fernando.halabi@colegionuevaaurora.cl}
\def\title{Guía 2}
\def\subtitle{Operatoria en $\mathbb{Q}$}
\def\colegio{Liceo Bicentenario Nueva Aurora de Chile}
\def\asignatura{Matemática}
\def\curso{Segundo Medio}

\documentclass{prueba}

\begin{document}
\titulo{Objetivo e instrucciones generales}

El siguiente material corresponde a una evaluación con {\bfseries nota acumulativa}, que 
tiene como intención preparar la próxima prueba de cierre de unidad. Específicamente,
el alumno deberá: Convertir números racionales a fracción, y aplicar propiedades 
de estas últimas para determinar el valor de una expresión aritmética.

La evaluación es para completarla en la clase y es necesario {\bfseries contestar con lápiz pasta}
para tener derecho a reclamo.

\titulo{Resumen de contenidos claves}

Convertir de decimal a fracción:
\begin{tcenter}
    \begin{tikzpicture}
        \node (A) {$13,421 =\dfrac{13421}{1000}$};
        \node (B) [above right=5mm of A,draw] {El número original sin coma};
        \node (C) [below right=5mm of A,draw] {Un 1 y tantos 0s como decimales hayan};
        \draw [->] (B.south west) -- (A.north east);
        \draw [->] (C.north west) -- (A.south east);
    \end{tikzpicture}    
\end{tcenter}

Convertir de periódico a fracción:
\begin{tcenter}
    \begin{tikzpicture}
        \node (A) {$132,\overline{421} =\dfrac{132421-132}{999}$};
        \node (B) [above right=5mm of A,draw,text width=6cm] {Al numero original sin %
        coma se le resta lo que esta antes del periodo};
        \node (C) [below right=5mm of A,draw] {Tantos 9s como números %
        en el periodo};
        \draw [->] (B.south west) -- (A.north east);
        \draw [->] (C.north west) -- (A.south east);
    \end{tikzpicture}
\end{tcenter}

Convertir de semi-periodo a fracción:
\begin{tcenter}
    \begin{tikzpicture}
        \node (A) {$132,67\overline{421} =\dfrac{13267421-13267}{99900}$};
        \node (B) [above right=5mm of A,draw,text width=6cm] {Al numero original sin %
        coma se le resta lo que esta antes del periodo};
        \node (C) [below right=5mm of A,draw,text width=7.5cm] {Tantos 9s como números %
        en el periodo y tantos 0s como decimales antes del periodo};
        \draw [->] (B.south west) -- (A.north east);
        \draw [->] (C.north west) -- (A.south east);
    \end{tikzpicture}
\end{tcenter}

Convertir de fracción mixta a fracción impropia\footnote{La fracción  es impropia cuando
el numerador es más grande que el denominador.}:
\begin{equation*}
    4\dfrac{3}{5} = \dfrac{4\cdot 5 + 3}{5} = \dfrac{23}{5}
\end{equation*}

Para sumar y restar fracciones, es importante que primero todos los denominadores sean
iguales, para lo cual se usa el mínimo común múltiplo (MCM). Una vez calculado el MCM, 
las fracciones 
se multiplican (arriba y abajo) de tal manera que el denominador toma el valor del MCM.
Finalmente, los numeradores se operan normalmente para calcular el resultado final.
\begin{equation*}
    \dfrac{3}{4} + \dfrac{1}{2} - \dfrac{5}{8} = \dfrac{6}{8} + \dfrac{4}{8} - \dfrac{5}{8} = \dfrac{6+4-5}{8} = \dfrac{5}{8}
\end{equation*}

Notar que en el ejemplo anterior, las fracciones se multiplicaron por 2, 4 y 1 
respectivamente para que los denominadores tomen el valor del MCM, que es 8 en este caso.  

Para multiplicar fracciones, se multiplica hacia al lado:
\begin{equation*}
\dfrac{2}{3}\cdot\dfrac{5}{7} = \dfrac{2\cdot 5}{3 \cdot 7} = \dfrac{10}{35}    
\end{equation*}

Para dividir fracciones, se invierte la que está dividiendo para transformar la operación
en una multiplicación, es decir:
\begin{equation*}
\dfrac{2}{3}\div \dfrac{5}{7} = \dfrac{2}{3}\cdot\dfrac{7}{5} = \dfrac{2\cdot 7}{3\cdot 5} = \dfrac{14}{15}
\end{equation*} 
\titulo{Pauta de cotejo}

Para la corrección de la guía, se le asignará puntaje a cada respuesta según los criterios
que se encuentran detallados en la tabla a continuación.

\pauta
\newpage

\parte{} Reduzca o simplifique las siguientes expresiones lo más posible usando 
las distintas propiedades de los números racionales.

\pregunta{} $\dfrac{3}{4} + \dfrac{1}{6} - \dfrac{11}{12}$
\desarrollo[6cm]

\pregunta{} $1\dfrac{3}{5} \times 4\dfrac{5}{8}$
\desarrollo[6cm]

\pregunta{} $3,2\overline{46}\div 5,\overline{46}$
\desarrollo[6cm]

\pregunta{} $\left(1,2\overline{6}-3\dfrac{4}{5}\right)\div\left(\dfrac{1}{3}-\dfrac{8}{12}\right)$
\desarrollo[8cm]

\pregunta{} $1 - \dfrac{1}{1+\dfrac{2}{3-\dfrac{1}{2}}} + \dfrac{1}{1+\dfrac{1}{1+\dfrac{1}{4}}}$
\desarrollo[10cm]



\end{document}