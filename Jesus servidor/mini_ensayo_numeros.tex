\def\name{Prof. Fernando Halabi}
\def\mail{correo@caes.cl}
\def\title{Mini ensayo 1}
\def\subtitle{Operatoria de números}
\def\colegio{Colegio Jesús Servidor}
\def\asignatura{Matemática}
\def\curso{Tercero y Cuarto medio}

\documentclass{caes}

\begin{document}
\datos
\pregunta $5\cdot \left(\dfrac{0.05}{0.5}\right)$
\desarrollo[3cm]
\begin{alternativas*}
    \item $0.5$
    \item $0.05$
    \item $0.005$
    \item $50$
    \item $500$
\end{alternativas*}

\pregunta El orden de los números $a = \frac{2}{3}$, $b = \frac{5}{6}$ y 
$c = \frac{3}{8}$ de menor a mayor es
\desarrollo[3cm]
\begin{alternativas*}
    \item $a < b < c$
    \item $b < c < a$
    \item $b < a < c$
    \item $c < a < b$
    \item $c < b < a$
\end{alternativas*}

\pregunta $\dfrac{1}{\dfrac{3}{8} -0.75} + \dfrac{1}{\dfrac{3}{8}-0.25}$
\desarrollo[3cm]
\begin{alternativas*}
    \item $\dfrac{15}{3}$
    \item $\dfrac{16}{3}$
    \item $-\dfrac{16}{3}$
    \item $4$
    \item $\dfrac{8}{3}$
\end{alternativas*}

\newpage
\pregunta $\dfrac{1}{x} + \dfrac{1}{x} + \dfrac{1}{x}$
\desarrollo[3cm]
\begin{alternativas*}
    \item $3$
    \item $\dfrac{1}{x^3}$
    \item $\dfrac{3}{x}$
    \item $\dfrac{1}{3x}$
    \item $\dfrac{3}{x^3}$
\end{alternativas*}

\pregunta $\dfrac{1}{3} + \dfrac{2}{1-\dfrac{1}{4}}$
\desarrollo[3cm]
\begin{alternativas*}
    \item $\dfrac{3}{2}$
    \item $\dfrac{1}{3}$
    \item $\dfrac{11}{6}$
    \item $1$
    \item $3$
\end{alternativas*} 

\pregunta Si $a$ es un número natural mayor que 1, ¿Cuál es la relación 
correcta entre las fracciones: $p = \frac{3}{a}$, $t = \frac{3}{a-1}$,
$r = \frac{3}{a+1}$?
\desarrollo[3cm]
\begin{alternativas*}
    \item $p < t < r$
    \item $r < p < t$
    \item $t < r < p$
    \item $r < t < p$
    \item $p < r < t$
\end{alternativas*} 

\pregunta Se mezclan 2 litros de un licor $P$ con 3 litros de un licor $Q$. Si
6 litros del licor $P$ valen \$$a$ y 9 litros del licor $Q$ valen \$$b$, 
¿Cuál es el precio de los 5 litros de la mezcla?
\desarrollo[3cm]
\begin{alternativas*}
    \item $\$\;\dfrac{a+b}{3}$
    \item $\$\;\dfrac{a+b}{5}$
    \item $\$\;(2a + 3b)$
    \item $\$\;\dfrac{3a+2b}{18}$
    \item $\$\;\dfrac{5\cdot(3a+2b)}{18}$
\end{alternativas*} 

\pregunta Juan tiene un bidón de 5 litros de capacidad, llenado hasta
los $2\dfrac{1}{3}$, ¿Cuántos litros le faltan para \\llenarlo?
\desarrollo[3cm]
\begin{alternativas*}
    \item $2\dfrac{1}{3}$
    \item $2\dfrac{2}{3}$
    \item $2\dfrac{3}{2}$
    \item $3\dfrac{1}{3}$
    \item $1\dfrac{2}{3}$
\end{alternativas*} 

\pregunta $\dfrac{3^{-1} + 4^{-1}}{5^{-1}}$
\desarrollo[3cm]
\begin{alternativas*}
    \item $\dfrac{12}{35}$
    \item $\dfrac{35}{12}$
    \item $\dfrac{7}{5}$
    \item $\dfrac{5}{7}$
    \item $\dfrac{5}{12}$
\end{alternativas*} 

\pregunta $\dfrac{0.0009 \cdot 0.0000002}{6\cdot 0.0003}$
\desarrollo[3cm]
\begin{alternativas*}
    \item $10^{-15}$
    \item $10^{-12}$
    \item $10^{-7}$
    \item $10^{-6}$
    \item Ninguna de las anteriores
\end{alternativas*} 

\pregunta El orden de los números: $M = 4.51\cdot 10^{-6}$, 
$N = 45.1\cdot 10^{-5}$ y $P = 451\cdot 10^{-7}$, de menor a 
mayor, es
\desarrollo[3cm]
\begin{alternativas*}
    \item $M$, $N$, $P$
    \item $P$, $M$, $N$
    \item $N$, $M$, $P$
    \item $P$, $N$, $M$
    \item $M$, $P$, $N$
\end{alternativas*} 

\newpage
\pregunta $4^{-2} + 2^{-3} - 2^{-4}$
\desarrollo[3cm]
\begin{alternativas*}
    \item $\dfrac{1}{8}$
    \item $\dfrac{1}{4}$
    \item $\dfrac{1}{6}$
    \item $-8$
    \item $-6$
\end{alternativas*} 

\pregunta En la igualdad $4^n + 4^n + 4^n + 4^n = 2^{44}$, el valor
de $n$ es
\desarrollo[3cm]
\begin{alternativas*}
    \item $\dfrac{11}{2}$
    \item $11$
    \item $21$
    \item $22$
    \item Ninguna de las anteriores
\end{alternativas*} 

\pregunta $5\sqrt{12}-2\sqrt{27}$
\desarrollo[3cm]
\begin{alternativas*}
    \item $16\sqrt{3}$
    \item $4\sqrt{3}$
    \item $2\sqrt{3}$
    \item $3\sqrt{3}$
    \item No se puede determinar
\end{alternativas*} 

\pregunta $\sqrt{\dfrac{2}{\sqrt[3]{2}}} =$
\desarrollo[3cm]
\begin{alternativas*}
    \item $\sqrt[3]{4}$
    \item $\sqrt[3]{2}$
    \item $\sqrt[6]{8}$
    \item $\sqrt[6]{2}$
    \item $1$
\end{alternativas*} 

\newpage
\pregunta Al simplificar la expresión 
$\dfrac{2\sqrt[2]{7}+ \sqrt{14}}{\sqrt{7}}$ resulta
\desarrollo[3cm]
\begin{alternativas*}
    \item $2\sqrt{3}$
    \item $2+\sqrt{14}$
    \item $2+\sqrt{2}$
    \item $2\sqrt{7}+\sqrt{2}$
    \item $4$
\end{alternativas*} 

\pregunta $\dfrac{\sqrt{5^5+5^5+5^5+5^5+5^5}}{\sqrt[3]{5^5+5^5+5^5+5^5+5^5}} =$
\desarrollo[3cm]
\begin{alternativas*}
    \item $5$
    \item $5^{\frac{5}{6}}$
    \item $1$
    \item $5^{\frac{2}{3}}$
    \item $5^{\frac{3}{2}}$
\end{alternativas*} 

\pregunta El valor de la expresión $18-(-45):(-3)^2 + (-2)\cdot(-1)^5$ es
\desarrollo[3cm]
\begin{alternativas*}
    \item $-9$
    \item $9$
    \item $-5$
    \item $25$
    \item $5$
\end{alternativas*} 

\pregunta El valor de la suma $0.\overline{5} + 0.\overline{7}$ 
es igual a
\desarrollo[3cm]
\begin{alternativas*}
    \item $0.\overline{12}$
    \item $1.\overline{2}$
    \item $1.\overline{3}$
    \item $1.12$
    \item $1.2$
\end{alternativas*} 

\newpage
\pregunta Si $a$ y $b$ son números enteros positivos, la expresión
$\dfrac{a^2 + b}{a}$ representa un número entero si
\begin{enumerate}[label=(\arabic*), leftmargin=3cm]
    \item $a^2 + b$ es un número entero
    \item $\dfrac{b}{a}$ representa un número entero
\end{enumerate}
\begin{alternativas}
    \item (1) por sí sola
    \item (2) por sí sola
    \item Ambas juntas, (1) y (2)
    \item Cada una por sí sola, (1) ó (2)
    \item Se requiere información adicional
\end{alternativas}

\end{document}
