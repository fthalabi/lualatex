\def\name{Prof. Fernando Halabi}
\def\mail{mail@caes.cl}
\def\title{Guía 1}
\def\subtitle{Introducción al análisis de datos}
\def\colegio{Colegio Jesús Servidor}
\def\asignatura{Matemática}
\def\curso{Probabilidades y estadística descriptiva e inferencial}

\documentclass{caes}

\begin{python}
    import pandas as pd
    import numpy as np
    
    np.random.seed(123)
    sample = np.round(np.random.normal(1.65, 0.1, 30),2)
    pd.DataFrame(sample.reshape((3, 10))).to_csv(
        "sample.csv",
        index=False,
        header=False
    )
    value, count = np.unique(sample, return_counts=True)
    dt = pd.DataFrame({"Altura": value, "Frecuencia": count})
    dt["Probabilidad"] = np.round(count/np.sum(count),3)
    dt["Frecuencia Acumulada"] = np.cumsum(count)
    dt["Probabilidad Acumulada"] = np.round(dt["Frecuencia Acumulada"]/np.sum(count),3)
    dt.to_csv("tabla_resumen.csv", index=False)
\end{python}

\begin{document}

\pregunta{} Considere las siguientes alturas para rellenar la tabla a continuación.

\csvreader[
    centered tabularray = {
        vlines={wd=1pt,fg=primarycolor},
        hlines={wd=1pt,fg=primarycolor},
        hline{2} = {2}{-}{wd=1pt,fg=primarycolor},
    }, 
    table head = Altura & Frecuencia & Probabilidad & Frecuencia Acumulada & %
        Probabilidad Acumulada\\,
]{tabla_resumen.csv}{1=\alt, 2=\frec, 3=\prob, 4=\frecCum, 5=\probCum}%
{\csvexpval\alt & \frec & {\prob} &\frecCum & \probCum}


% \begin{center}
%     \begin{tblr}{vlines={wd=1pt,fg=primarycolor},hlines={wd=1pt,fg=primarycolor}}
%         1.68 & 1.65 & 1.55 & 1.68 & 1.56 & 1.72 & 1.60 & 1.52 & 1.68 & 1.45 \\
%         1.56 & 1.64 & 1.70 & 1.60 & 1.53 & 1.90 & 1.62 & 1.43 & 1.43 & 1.82 \\
%         1.56 & 1.50 & 1.57 & 1.75 & 1.58 & 1.61 & 1.46 & 1.66 & 1.56 & 1.58    
%     \end{tblr}        
% \end{center}

% \begin{center}
% \begin{tblr}{width = 0.4\linewidth, vlines={wd=1pt,fg=primarycolor},hlines={wd=1pt,fg=primarycolor},
%     hline{2} = {1}{-}{wd=1pt,fg=primarycolor},hline{2} = {2}{-}{wd=1pt,fg=primarycolor}, colspec={cc}}
%     Altura & Frecuencia & Probabilidad & Frecuencia Acumulada & Probabilidad Acumulada \\
%     1.43 & 2 & & & \\
%     1.45 & 1 & & & \\
%     1.46 & 1 & & & \\
%     1.50  & 1 & & & \\
%     1.52 & 1 & & & \\
%     1.53 & 1 & & & \\
%     1.55 & 1 & & & \\
%     1.56 & 4 & & & \\
%     1.57 & 1 & & & \\
%     1.58 & 2 & & & \\
%     1.60  & 2 & & & \\
%     1.61 & 1 & & & \\
%     1.62 & 1 & & & \\
%     1.64 & 1 & & & \\
%     1.65 & 1 & & & \\
%     1.66 & 1 & & & \\
%     1.68 & 3 & & & \\
%     1.70  & 1 & & & \\
%     1.72 & 1 & & & \\
%     1.75 & 1 & & & \\
%     1.82 & 1 & & & \\
%     1.90  & 1 & & &   
% \end{tblr}
% \end{center}

\newpage

\pregunta{} Use los datos de la tabla anterior, para dibujar un {\bfseries diagrama 
de caja} que represente la distribución de las alturas entregadas al inicio.
\desarrollo[8cm]

\pregunta{} Por último, use los datos de la tabla para crear un 
{\bfseries histograma} con las alturas. Denote en la gráfica donde 
se encuentra la {\bfseries media} de los datos.
\desarrollo[11cm]

\begin{equation}
    f(t \in {\mathbb R}) = \lim_{N\to\infty} f(0) \cdot \left(1 + \dfrac{r}{N} \right)^{N \cdot t} = f(0)\cdot {\mathrm e}^{r \cdot t}
\end{equation}

\begin{equation}
    \int_0^\infty \mathrm{e}^{-x}\,\mathrm{d}x
\end{equation}

\end{document}
