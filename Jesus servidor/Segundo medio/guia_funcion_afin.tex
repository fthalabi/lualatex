\documentclass{ceas2}
%\usepackage{lua-visual-debug}

\begin{document}
\datos
\section{Introducción}

En esta guía veremos como la función afín (ecuación de la recta) se puede usar para 
modelar (representar) situaciones de la vida real, y como este proceso nos permite
obtener nueva información sobre nuestro objeto de estudio.\par

Más adelante veremos herramientas matemáticas más sofisticadas, que nos permitirán crear
modelos más apegados a la realidad, pero de momento, supongamos que la función afín 
es suficiente para solucionar los problemas que se encuentran a continuación.

\section{Problemas resueltos}

\e Cinco metros de tela tienen un costo de \$300, encuentra un modelo
lineal para el costo y determina ¿Cuánto cuestan 25 metros? y ¿Cuántos metros
de tela se pueden comprar con \$18000? 

{\bfseries Solución:} Nuestro primer paso será reconocer los 
datos que entrega el enunciado como una tabla de puntos. En la siguiente 
tabla, $x$ representa la cantidad de tela en metros e $\boldsymbol y$ 
el costo de la tela.

\def\tabla{%
\begin{mtabla}{}
    x & y \\
    5 & 300 \\
    0 & 0 
\end{mtabla}
}

\begin{center}    
\begin{tikzpicture}[line width=1pt]
    \tikzset{msg/.style={draw, dashed,line width = 1pt, text width=11cm,inner sep=10pt, rounded corners=5pt}}
    \node (t) {\tabla};
    \node[right=2cm of t,yshift=1cm,msg] (p1) {Este punto representa que 5 metros de tela cuestan 300 pesos};
    \node[right=2cm of t,yshift=-1cm,msg] (p2) {Si no se compra tela (la cantidad es igual a 0 metros) el \mbox{precio} a pagar es 0 pesos. Esto no lo dice explicitamente el enunciado, pero, se entiende como algo logico y \mbox{esperable}};
    \draw[->,dashed] (p1.west) to[bend right=15] (t.east);
    \draw[->,dashed] (p2.west) to[bend left=15] ($(t.south east) + (0,15pt)$);
\end{tikzpicture}
\end{center}
Con los datos de la tabla, se puede calcular la pendiente de la forma
\begin{equation*}
    m = \dfrac{\text{largo lado vertical}}{\text{largo lado horizontal}} = \dfrac{300 - 0}{5 - 0} = \dfrac{300}{5} = 60.
\end{equation*}
Por otro lado, 
el coeficiente de posición en este caso no es necesario calcularlo, ya que, se
encuentra en la tabla, y es 0. Así, la ecuación de la recta que
representa el precio a pagar por la tela es 
\begin{equation}\label{eq:1}
    y = 60\cdot x.
\end{equation}

Con el modelo ya echo, nos preguntamos: 
¿Cuánto cuestan 25 metros de tela? Para responder esto, basta reemplazar $x=25$ en la
ecuación \mref{eq:1} y despejar el valor de la incógnita $y$
\begin{equation*}
    y = 60\cdot x = 60\cdot 25 = 1625.
\end{equation*} 
Así, comprar 25 metros de tela cuesta 1625 pesos. Para la segunda pregunta, ¿Cuántos
metros de tela se pueden comprar con 18000 pesos? Basta reemplazar $y=18000$
en la ecuación \mref{eq:1} y despejar el valor de $x$
\begin{equation*}
    60\cdot x = y \quad\Rightarrow\quad 60\cdot x = 18000 \quad\Rightarrow\quad x =\dfrac{18000}{60} = 300.
\end{equation*}
Lo cual nos dice que con 1800 pesos se puede comprar 300 metros de tela.

\e Un delfín mular mide 1.5 metros al nacer y pesa alrededor de 30 kilogramos. Los 
defines jóvenes son amamantados durante 15 meses, al final de dicho periodo 
estos cetáceos miden 2.7 metros y pesan 375 kilogramos. Con estos datos, 
¿Cuál es el aumento 
diario de la longitud para un delfín joven? ¿Cuál es el peso de un delfín de 
cinco meses de edad?

{\bfseries Solución:} Son necesarias dos ecuaciones para solucionar este problema,
una para representar el largo de un delfín mular en el tiempo y otra para su peso en 
el tiempo. Así, moviendo los datos de enunciado a una tabla se obtiene:

\def\tabla{%
\begin{mtabla}{}
    t & L \\
    0 & 1.5 \\
    15 & 2.7
\end{mtabla}
}

\begin{center}    
\begin{tikzpicture}[line width=1pt]
    \tikzset{msg/.style={draw, dashed,line width = 1pt, text width=8cm,inner sep=10pt, rounded corners=5pt}}
    \node (t) {\tabla};
    \node[right=-1cm of t,yshift=2.5cm,msg, text width=14cm] (p3) {Como son dos las ecuaciones que necesitamos y para no confundirlas, en lugar de usar $y$, se uso $L$ para denotar el largo en metros y $t$ el tiempo en meses};
    \node[right=2cm of t,yshift=0.5cm,msg] (p1) {A los 0 meses el delfín mide 1.5 metros};
    \node[right=2cm of t,yshift=-1cm,msg] (p2) {Para los 15 meses el delfín mide 2.7 metros};
    \draw[->,dashed] (p1.west) to[bend right=15] (t.east);
    \draw[->,dashed] (p2.west) to[bend left=15] ($(t.south east) + (0,15pt)$);
    \draw[->,dashed] (p3.west) to[bend right=40] (t.north);
\end{tikzpicture}
\end{center}

\def\tabla{%
\begin{mtabla}{}
    t & P \\
    0 & 30 \\
    15 & 375
\end{mtabla}
}

\begin{center}    
\begin{tikzpicture}[line width=1pt]
    \tikzset{msg/.style={draw, dashed,line width = 1pt, text width=9cm,inner sep=10pt, rounded corners=5pt}}
    \node (t) {\tabla};
    \node[right=0cm of t,yshift=2cm,msg, text width=11cm] (p3) {$P$ denota el peso en kilogramos del delfín y $t$ el tiempo en meses};
    \node[right=2cm of t,yshift=0.5cm,msg] (p1) {Al nacer (0 meses) el delfín pesa 30 kilogramos};
    \node[right=2cm of t,yshift=-1cm,msg] (p2) {Con 15 meses el delfín pesa 375 kilogramos};
    \draw[->,dashed] (p1.west) to[bend right=15] (t.east);
    \draw[->,dashed] (p2.west) to[bend left=15] ($(t.south east) + (0,15pt)$);
    \draw[->,dashed] (p3.west) to[bend right=25] (t.north);
\end{tikzpicture}
\end{center}


Usando los datos que se encuentran en estas tablas, las ecuaciones para el largo y peso 
del delfín son:

\begin{multicols}{2}\setlength{\columnseprule}{0.4pt}
    \noindent
        \begin{equation*}
            m = \dfrac{2.7 - 1.5}{15 - 0} = \dfrac{1.2}{15} = \dfrac{2}{25}
        \end{equation*}
        \begin{equation}\label{eq:l}
            L = \dfrac{2}{25}\cdot t + \dfrac{3}{2}
        \end{equation}
        \begin{equation*}
            m = \dfrac{375 - 30}{15 - 0} = \dfrac{345}{15} = 23
        \end{equation*}
        \begin{equation}\label{eq:p}
            P = 23\cdot t + 30
        \end{equation}
\end{multicols}

Entonces, con las ecuaciones para el largo ($L$) y el peso ($P$) ya calculadas, ¿Cuál 
es el aumento diario de longitud para un delfín? Este dato se puede calcular usando la
pendiente para la longitud, ya que $m=\frac{2}{25}=0.08$ representa lo que un delfín
crece en 1 mes. Así, lo que crece un delfín en un día es $0.08/30=0.00267$ metros.

Por último, para encontrar el peso de un delfín a los 5 meses de haber nacido,
basta reemplazar $t=5$ en la ecuación \mref{eq:p}, para obtener que 
$P=25\cdot 5 + 30 = 145$, y así la cantidad buscada es 145 kilogramos.

\section{Problemas propuestos}

\begin{enumerate}[label=(\arabic*)]
    \item Un hombre recibe \$120 por 3 horas de trabajo. Expresa el sueldo $S$ en (pesos)
    en términos del tiempo $t$ (horas).
    \desarrollo
    \item Un bebé pesa 3.5 kg al nacer y 3 años después alcanza 10.5 kg. Suponiendo 
    que el peso $P$ (en kg) en la infancia puede representarse con una recta,
    ¿Cuánto pesará el niño cuando cumpla 9 años? ¿A qué edad pesará 28 kg?
    \desarrollo[5cm]
    \item La cantidad de calor $C$ (en calorías), requerida para convertir un gramo de 
    agua en vapor, se relaciona linealmente con la temperatura $T$ (en °$F$) de la 
    atmósfera. A 50°$F$ esta conversión requiere 592 calorías y cada aumento de 
    $15°F$ aumenta 9.5 calorías la cantidad de calor. Expresa $C$ en términos de $T$.
    \desarrollo[5cm]
    \item \lipsum[1]
\end{enumerate}

\e 

\e Un bebe 
\e asdasd

\end{document}