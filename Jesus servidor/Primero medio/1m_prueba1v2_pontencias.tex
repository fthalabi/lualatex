\def\title{Prueba 1 (recuperativa)}
\def\subtitle{Propiedades de potencias}
\def\curso{Primero medio}

\documentclass{caes}

\begin{document}
\datos
\titulo{Objetivo e instrucciones generales}

Esta evaluación tiene como objetivo medir la aplicación de propiedades de 
potencias en el desarrollo expresiones aritméticas.  

\underline{La evaluación es individual y con nota al libro}, no está permitido 
el uso del celular o la calculadora. Además, para tener derecho a 
reclamo es necesario \underline{contestar con lápiz pasta}.

\titulo{Pauta de cotejo}

Para la corrección de la evaluación, se le asignará puntaje a cada respuesta 
según los criterios que se encuentran detallados en la tabla a continuación.

\pauta

\parte Simplifica las siguientes expresiones utilizando propiedades de potencias. 
No olvides incluir desarrollo completo y ordenado en el espacio señalizado.

\pregunta $-5^2\cdot (-5)^2$
\desarrollo

\newpage
\pregunta $2^3 \cdot 3^5 \cdot 2^4 \cdot 3^2$
\desarrollo[4cm]

\pregunta $5^{-3}\cdot 7^{-2} \cdot 5^{5} \cdot 7^{-4}$
\desarrollo[4.5cm]

\pregunta $\dfrac{4^{-2}}{4^{-6}}$
\desarrollo[4.5cm]

\pregunta $\dfrac{3^2 \cdot 3^5}{3^{-4} \cdot 3^{1}}$
\desarrollo[4.5cm]

\pregunta $\left(\dfrac{2^5 \cdot 4^3}{8\cdot 2^2 \cdot 32^2}\right)^{-2}$
\desarrollo[4.5cm]

\pregunta $(2^4 \cdot 4^{-6} \cdot 8^2)^{-\frac{1}{2}}$
\desarrollo[4.5cm]

\pregunta $\left(\dfrac{2^{\frac{3}{2}}}{2^{\frac{1}{3}}}\right)^6$
\desarrollo[4.5cm]

\end{document}