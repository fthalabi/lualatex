\providecommand{\tituloDocumento}{Evaluación}
\providecommand{\subtituloDocumento}{Ecuaciones de primer grado}
\documentclass{sn-guia}

\begin{document} 
\raggedright
\begin{tcbraster}[enhanced,raster columns=4,raster width=\linewidth,raster column skip=3pt,raster force size=false]
    \begin{caja}[title={\sffamily\scshape\bfseries Nombre},height=35pt,add to width=4.5cm]
    \end{caja}
    \begin{caja}[title={\sffamily\scshape\bfseries Curso},height=35pt,add to width=-1.5cm]
    \end{caja}    
    \begin{caja}[title={\sffamily\scshape\bfseries Puntaje},height=35pt,add to width=-1.5cm]
    \end{caja}
    \begin{caja}[title={\sffamily\scshape\bfseries Nota},height=35pt,add to width=-1.5cm]
    \end{caja}      
\end{tcbraster}
\vspace*{5mm}

\begin{problemas}
    \problema ¿Cuál es el valor de $x$ en la ecuación $0,3 + 10x = 0,5$?
    \begin{alternativas}[]
        \item 8
        \item 2
        \item 0,08
        \item 0,02
    \end{alternativas}

    \problema Un vendedor recibe un sueldo base de \$ 215.000, al mes, más un 8\% de las 
    ventas por comisión. ¿Cuánto debe vender para ganar \$ 317.000 en el mes?
    \begin{alternativas}
        \item \$ 254.625
        \item \$ 532.000
        \item \$ 1.275.000
        \item \$ 1.812.500
        \item \$ 3.962.500
    \end{alternativas}
    \problema La señora Marta compró 3 kilogramos de azúcar y 2 kilogramos de harina y 
    pagó \$ $s$. Si el kilogramo de azúcar vale \$ $p$, ¿cuánto cuesta el kilogramo de 
    harina?
    \begin{alternativas}
        \item \$ $\left( s-3p \right)$
        \item \$ $\left( s-3p \right)/2$
        \item \$ $\left( s+3p \right)/2$
        \item \$ $\left( s-p \right)/2$
        \item \$ $\left( s+3p \right)$
    \end{alternativas}

    \problema La expresión: ``para que el doble de $(a+c)$ sea igual a 18, le faltan
    4 unidades'', se expresa como
    \begin{alternativas}[]
        \item $2a + c + 4 = 18$
        \item $2(a+c) -4 = 18$
        \item $2(a+c) + 4 = 18$
        \item $4 - 2(a+c) = 18$
        \item $2a +c -4 = 18$
    \end{alternativas}

    \problema Un grupo de amigos salen a almorzar a un restaurante y desean repartir 
    la cuenta en partes iguales. Si cada uno pone \$ 5.500 faltan \$ 3.500 para 
    pagar la cuenta y si cada uno pone \$ 6.500 sobran \$ 500. ¿Cuál es el valor 
    de la cuenta?
    \begin{alternativas}
        \item \$ 20.000
        \item \$ 22.000
        \item \$ 25.500
        \item \$ 26.000
        \item \$ 29.500
    \end{alternativas}
    \problema Si tuviera \$ 80 más de los que tengo podría comprar exactamente 4 
    pasteles de \$ 240 cada uno. ¿Cuánto dinero me falta si quiero comprar 
    6 chocolates de \$ 180 cada uno?
    \begin{alternativas}
        \item \$ 280
        \item \$ 200
        \item \$ 120
        \item \$ 100
        \item \$ 40
    \end{alternativas}
    \problema Al sumar $\dfrac{x}{t}$ con $m$ se obtiene $\dfrac{x}{t+2}$, entonces ¿cuál 
    es el valor de $m$?
    \begin{alternativas}[]
        \item $0$
        \item $\dfrac{2x}{t(t+2)}$
        \item $\dfrac{-x}{t+2}$
        \item $\dfrac{-2x}{t(t+2)}$
        \item $\dfrac{-2}{t(t+2)}$
    \end{alternativas}

    \problema Compré $x$ kg de café en \$ 36.000 y compré 40 kg más de té que de 
    café en \$ 48.000. ¿Cómo se expresa el valor de 1 kg de café más 1 kg de té, 
    en función de $x$?
    \begin{alternativas}[]
        \item $\dfrac{36.000}{x}+\dfrac{48.000}{x+40}$
        \item $\dfrac{36.000}{x}+\dfrac{48.000}{x-40}$
        \item $\dfrac{x}{36.000}+\dfrac{x+40}{48.000}$
        \item $\dfrac{x}{36.000}+\dfrac{x-40}{48.000}$
        \item $\dfrac{36.000}{x}+\dfrac{48.000}{40}$
    \end{alternativas}
    \problema Hace 3 años Luisa tenía 5 años y Teresa $a$ años. ¿Cuál será la 
    suma de sus edades en $a$ años más?
    \begin{alternativas}[]
        \item $(11+3a)$ años
        \item $(11+2a)$ años
        \item $(11+a)$ años
        \item $(8+3a)$ años
        \item $(5+3a)$ años
    \end{alternativas}
    \problema Jorge compró tres artículos distintos en \$ $(4a+b)$. El primero 
    le costó \$ $a$ y el segundo \$ $(2a-b)$. ¿Cuánto le costó el tercero?
    \begin{alternativas}[]
        \item \$ $a$
        \item \$ $7a$
        \item \$ $(3a-b)$
        \item \$ $(3a+2b)$
        \item \$ $(a+2b)$
    \end{alternativas}
    \problema Se mezclan 2 litros de un licor $P$ con 3 litros de un licor $Q$. Si 6 
    litros del licor $P$ valen \$ $a$ y 9 litros del licor $Q$ valen \$ $b$, entonces 
    ¿cuál es el precio de los 5 litros de mezcla?
    \begin{alternativas}[]
        \item \$ $\dfrac{a+b}{3}$
        \item \$ $\dfrac{a+b}{5}$
        \item \$ $(2a+3b)$
        \item \$ $\dfrac{3a+2b}{18}$
        \item \$ $\dfrac{5\cdot(3a+2b)}{18}$
    \end{alternativas}
    \problema En un colegio se necesita colocar en la cocina 70 $\textrm{m}^2$ de 
    cerámica y 100  $\textrm{m}^2$ de piso flotante para la sala de computación. Si 
    el metro cuadrado de cerámica cuesta \$ $P$ y el metro cuadrado de piso flotante 
    es un 75\% más caro que la cerámica, entonces el costo total es de
    \begin{alternativas}[]
        \item \$ $145\cdot P$
        \item \$ $170\cdot P$
        \item \$ $175\cdot P$
        \item \$ $245\cdot P$
        \item \$ $195\cdot P$
    \end{alternativas}
    \problema Si $4(3x+3) = 5(6+2x)$, entonces $2x$ es
    \begin{alternativas}[]
        \item 9
        \item 16
        \item 18
        \item 27/10
        \item Ninguno de los valores anteriores
    \end{alternativas}
    \problema Juan en 10 años más tendrá el doble de la edad que tenía hace 5 años.
    ¿Qué edad tendrá Juan en un año más?
    \begin{alternativas}[]
        \item 21 años
        \item 20 años
        \item 16 años
        \item 15 años
        \item 11 años
    \end{alternativas}
    \problema En un supermercado trabajan reponedores, cajeros y supervisores. El 60\% 
    corresponde a reponedores, los supervisores son 18 y estos son un tercio de los 
    cajeros. ¿Cuántos trabajadores tiene el supermercado?
    \begin{alternativas}[]
        \item 54
        \item 72
        \item 108
        \item 120
        \item 180
    \end{alternativas}
    \problema Se corta una tabla de 3 metros de largo en dos partes, de modo que una 
    de ellas es 50 cm más larga que la otra. ¿Cuáles son las longitudes de cada parte?
    \begin{alternativas}[]
        \item 250 cm y 50 cm
        \item 150 cm y 150 cm
        \item 175 cm y 125 cm
        \item 200 cm y 100 cm
        \item Ninguna de las medidas anteriores.
    \end{alternativas}
    \problema La suma de los cuadrados de tres números enteros consecutivos es igual a
    291. ¿Cuál de las siguientes expresiones representa al planteamiento algebraico de 
    este problema?
    \begin{alternativas}[]
        \item $[x + (x+1) + (x+2)]^2 = 291$
        \item $x^2 +(x^2+1) + (x^2 +2) = 291$
        \item $(x-1)^2 + x^2 + (x+1)^2 = 291$
        \item $(x-1)^2x^2(x+1)^2=291$
        \item $x^2(x^2+1)(x^2+2)=291$
    \end{alternativas}
    \problema ¿Cuál debe ser el valor de $x$ para que la expresión $\dfrac{9}{2}-\dfrac{3}{x}$
    sea igual al inverso aditivo de $-3$?
    \begin{alternativas}[]
        \item $2$
        \item $\dfrac{6}{15}$
        \item $-\dfrac{6}{15}$
        \item $1$
        \item $\dfrac{18}{25}$
    \end{alternativas}
    \problema Una fábrica de zapatos debe entregar un pedido de $T$ pares de zapatos en 
    tres días. Si el primer día entrega 2/5 de él, el segundo día 1/3 de lo que resta
    y el tercer día 1/4 del resto, entonces lo que quedó sin entregar es
    \begin{alternativas}[]
        \item $\dfrac{1}{10}T$
        \item $\dfrac{9}{10}T$
        \item $\dfrac{3}{10}T$
        \item $\dfrac{1}{5}T$
        \item $\dfrac{1}{60}T$
    \end{alternativas}
    \problema Si al doble de 108 se le resta $m$ se obtiene $n$ y el triple de $n$ es 123,
    ¿cuál es el valor de $m$?
    \begin{alternativas}[]
        \item 93
        \item 67
        \item 175/2
        \item -175
        \item 175
    \end{alternativas}
    \problema ¿Cuál de las siguientes opciones es verdadera con respecto al conjunto solución
    de la ecuación $|3x-2|=1$?
    \begin{alternativas}[]
        \item Tiene dos soluciones reales positivas y distintas.
        \item Tiene una solución real positiva y otra real negativa.
        \item Tiene solo una solución real positiva.
        \item Tiene solo una solución real negativa.
        \item No tiene solución en los números reales.
    \end{alternativas}
    \problema Se repartirá un premio de \$ 624.000 entre Ingrid, Gerardo y Jaime. Ingrid
    recibe 3/8 del total, Gerardo recibe 2/3 de lo que quedará y Jaime el resto.
    ¿Cuánto reciben Gerardo y Jaime, respectivamente?
    \begin{alternativas}[]
        \item \$ 243.000 y \$ 260.000
        \item \$ 156.000 y \$ 134.000
        \item \$ 260.000 y \$ 364.000
        \item \$ 260.000 y \$ 130.000
        \item \$ 416.000 y \$ 208.000
    \end{alternativas}
    \problema Se tienen \$ 16.000 en monedas de \$ 500 y de \$ 50. Si el total de monedas 
    es 50, entonces la cantidad de monedas de \$ 500 es
    \begin{alternativas}[]
        \item 32
        \item 30
        \item 27
        \item 20
        \item 18
    \end{alternativas}
    \problema Todo el líquido contenido en un barril se reparte en 96 vasos iguales 
    hasta su capacidad máxima. Se quiere verter la misma cantidad de líquido de otro 
    barril idéntico al anterior en vasos iguales a los usados, pero solo hasta las 
    3/4 partes de su capacidad. ¿Cuántos vasos más se necesitarán para ello? 
    \begin{alternativas}[]
        \item 288
        \item 120
        \item 48
        \item 32
    \end{alternativas}

    \problema Un bidón tiene ocupada con gasolina la mitad de su capacidad máxima. Al 
    agregar 8 litros de gasolina, se llega a las 5/6 partes de su capacidad. ¿Cuál 
    es la capacidad máxima del bidón?
    \begin{alternativas}[]
        \item 10 litros
        \item 12 litros
        \item 20 litros
        \item 24 litros
        \item 48 litros
    \end{alternativas}
    \problema En una frutería cada durazno cuesta \$ 480 y cada mango cuesta \$ 400.
    Una persona gastó \$ 6800 en total comprando solo 16 frutas entre duraznos y mangos.
    ¿Cuál de las siguientes ecuaciones permite determinar la cantidad de $x$ duraznos 
    que compró la persona?
    \begin{alternativas}[]
        \item $480x + 400(16-x) = 6800$
        \item $480x + 400(x-16) = 6800$
        \item $480x + 400x = 16$
        \item $(480 + 400)x = 6800 + 16$
    \end{alternativas}
    \problema La expresión $P-\frac{Q}{R}t^2$ representa el volumen de agua, en 
    $\textrm{m}^3$, que queda en un pozo en el instante t, en segundos, desde que el pozo
    está en su máxima capacidad. Si, $P$, $Q$ y $R$ son constantes positivas, ¿cuál de 
    las siguientes expresiones representa la cantidad de segundos que el pozo tarde 
    en quedarse sin agua?
    \begin{alternativas}[]
        \item $\dfrac{PR}{Q}$
        \item $-\sqrt{\dfrac{PR}{Q}}$
        \item $\sqrt{\dfrac{PR}{Q}}$
        \item $\sqrt{\dfrac{-PR}{Q}}$
        \item $\dfrac{PQ}{R}$
    \end{alternativas}
\end{problemas}

\vspace*{\fill}
\begin{center}
    \begin{tikzpicture}[ampersand replacement=\&,]
        %\node (A) [opacity=0.4] {\includegraphics[width=2cm]{../flork3.jpg}};
        \node (B) [font=\slshape,text width=12cm]
        {``Cree en ti mismo y en lo que eres. Sé consciente de que hay algo en tu interior %
        que es más grande que cualquier obstáculo''};
        \node [left=0mm of B,opacity=0.4] {\pgfornament[width=2cm]{37}};
        \node [right=0mm of B,opacity=0.4] {\pgfornament[width=2cm]{38}};
    \end{tikzpicture}
\end{center}
\vspace*{\fill}

\end{document}