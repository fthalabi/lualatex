\documentclass[sin nombre]{plantilla-evaluacion-v1}

\begin{document}

\begin{luacode}
  require('estadistica.lua')
  function generar_lista()
    repeat
        m = math.random(13,23)
        d = math.random(3,5)
        lista = Generar_muestras(30,m,d)
        v_unicos = Obtener_unicos(lista)
        divisiones = Divisor_ideal(v_unicos)
    until divisiones >2 and divisiones <7 and Rango(lista) < 15
    return lista
  end
  lista = generar_lista()
  divisiones = Divisor_ideal(Obtener_unicos(lista))
  function lista_a_clist()
    lista_str = table.concat(lista, ",")
    tex.sprint(lista_str)
  end
\end{luacode}

\ExplSyntaxOn

%% DECLARAR LISTA
\clist_new:N \l_lista_lua
\clist_set:Nx \l_lista_lua {\directlua{lista_a_clist()}}
\seq_set_from_clist:NN \l_seq_lua \l_lista_lua

%% ORDENAR LISTA
\seq_clear_new:N \l_lista_ordenada
\seq_set_eq:NN \l_lista_ordenada \l_seq_lua
\seq_sort:Nn \l_lista_ordenada {
  \int_compare:nNnTF { #1 } > { #2 }
  { \sort_return_swapped: }
  { \sort_return_same: }
}

%% VALORES UNICOS
\seq_clear_new:N \l_valores_unicos
\seq_set_eq:NN \l_valores_unicos \l_lista_ordenada
\seq_remove_duplicates:N \l_valores_unicos
Valores~unicos:~\seq_use:Nn \l_valores_unicos { ,~} \par

\cs_new_protected:Npn \__cuantil:n #1 {
  \fp_zero_new:N \l_cuantil_prob_acum_actual
  \seq_clear_new:N \l_cuantil_prob_acum
  \seq_map_inline:Nn \l_valores_unicos {
    \int_zero_new:N \l_cuantil_n
    \seq_map_inline:Nn \l_seq_lua {
      \int_compare:nT {##1=####1} {\int_gincr:N \l_cuantil_n}
    }
    \fp_gadd:Nn \l_cuantil_prob_acum_actual
      {\fp_eval:n {\int_use:N \l_cuantil_n / \seq_count:N \l_seq_lua}}
    \seq_put_right:Nx \l_cuantil_prob_acum {\fp_use:N \l_cuantil_prob_acum_actual}
  }
  \seq_log:N \l_cuantil_prob_acum
  \seq_clear_new:N \l_cuantil_menores
  \seq_set_filter:NNn \l_cuantil_menores \l_cuantil_prob_acum {
    \fp_compare_p:nNn {##1} < {#1}
  }
  \seq_log:N \l_cuantil_menores
  \tl_clear_new:N \l_cuantil_resultado
  \int_zero_new:N \l_cuantil_pos
  \int_set:Nn \l_cuantil_pos {\int_eval:n {\seq_count:N \l_cuantil_menores + 1}}
  \tl_set:Nx \l_cuantil_resultado {\seq_item:Nn \l_valores_unicos \l_cuantil_pos}
  \tl_use:N \l_cuantil_resultado
}

\NewDocumentCommand{\cuantil}{m}{\__cuantil:n {#1}}

\cs_new_protected:Npn \__cuantil_directo:n #1 {
  \fp_zero_new:N \l_cd_fp_indice
  \fp_set:Nn \l_cd_fp_indice
    {\fp_eval:n {ceil(#1 * \seq_count:N \l_seq_lua ,0)}}
  \int_zero_new:N \l_cd_int_indice
  \int_set:Nn \l_cd_int_indice {\fp_to_int:N \l_cd_fp_indice}
  \tl_set:Nx \l_cd_valor {\seq_item:Nn \l_lista_ordenada \l_cd_int_indice}
  \tl_use:N \l_cd_valor
}

\NewDocumentCommand{\cuantilAlt}{m}{\__cuantil_directo:n {#1}}

\int_gzero_new:N \l_frecuencia
\int_gzero_new:N \l_frecuencia_acumulada
\fp_gzero:N \l_probabilidad
\fp_gzero:N \l_probabilidad_acumulada

\iow_new:N \g_archivo_tabla
\iow_open:Nn \g_archivo_tabla { datos_tabla.tmp }
\seq_map_inline:Nn \l_valores_unicos
  {
    \int_gzero:N \l_frecuencia
    \seq_map_inline:Nn \l_seq_lua {
      \int_compare:nT {#1=##1} {\int_gincr:N \l_frecuencia}
    }
    \int_gadd:Nn \l_frecuencia_acumulada {\int_use:N \l_frecuencia}
    \fp_gset:Nn \l_probabilidad {\int_use:N \l_frecuencia / \seq_count:N \l_seq_lua }
    \fp_gadd:Nn \l_probabilidad_acumulada {\fp_use:N \l_probabilidad}

    \iow_now:Nx \g_archivo_tabla
      { #1 ~&~ \int_use:N \l_frecuencia ~&~ \int_use:N \l_frecuencia_acumulada ~&~
      \fp_use:N \l_probabilidad ~&~ \fp_use:N \l_probabilidad_acumulada ~\\~ }
  }
\iow_close:N \g_archivo_tabla

\begingroup
\pgfkeys{/pgf/number~format/.cd,fixed,fixed~zerofill,precision=3}
\begin{tblr}[evaluate=\fileInput]{colspec={ccccc},
  cell{2-Z}{4,5}={cmd=\pgfmathprintnumber},hlines,hline{1,2,Z}={black,1pt},
  row{1}={valign=m},rows={rowsep+=2pt}}
  Datos & Frecuencia & {Frecuencia\\Acumulada} & Probabilidad & {Probabilidad\\Acumulada} \\
  \fileInput{datos_tabla.tmp}
\end{tblr}
\endgroup \par

Q1:~\directlua{tex.sprint(Cuantil(lista,0.25))} \quad \cuantil{0.25} \quad \cuantilAlt{0.25} \newline
Q2:~\directlua{tex.sprint(Cuantil(lista,0.5))}  \quad \cuantil{0.5} \quad \cuantilAlt{0.5} \newline
Q3:~\directlua{tex.sprint(Cuantil(lista,0.75))} \quad \cuantil{0.75} \quad \cuantilAlt{0.75} \newline \par

\seq_new:N \l_puntos
\int_gzero_new:N \g_n
\seq_map_inline:Nn \l_valores_unicos {
  \int_gzero:N \g_n
  \seq_map_inline:Nn \l_seq_lua {
    \int_compare:nT {#1=##1} {\int_gincr:N \g_n}
    }
    \seq_put_right:Nx \l_puntos {(#1,~\int_use:N \g_n)}
}
Puntos:~\seq_map_inline:Nn \l_puntos {
  \mbox{#1}~
}

\tl_new:N \l_datos_para_grafica
\tl_set:Nx \l_datos_para_grafica { \seq_use:Nn \l_puntos {~} }

\begin{tikzpicture}[baseline=(current~axis.north)]
  \begin{axis}[
      ybar,
      title={Gráfico~de~barras},
      ylabel={Frecuencia},
      xlabel={Datos},
      ymin=0,
      %xtick=data,
      %nodes~near~coords,
      %nodes~near~coords~align={vertical},
      ]
      \addplot~[ybar,bar~width=1,pattern={Lines[angle=-45,distance=5pt]}]
      ~coordinates~{\l_datos_para_grafica};
  \end{axis}
\end{tikzpicture} \hspace*{1cm}

\iow_new:N \g_archivo
\iow_open:Nn \g_archivo { datos_histograma.csv }
\seq_map_inline:Nn \l_seq_lua
  {
    \iow_now:Nx \g_archivo { #1 }
  }
\iow_close:N \g_archivo

\tl_new:N \l_n_barras
\tl_set:Nx \l_n_barras {\directlua{tex.sprint(divisiones)}}

\begin{tikzpicture}[baseline=(current~axis.north)]
  \begin{axis}[
      ybar~interval,
      title={Histograma},
      ylabel={Frecuencia},
      xlabel={Datos},
      ymin=0,
      grid=none,
      xticklabel= $[\pgfmathprintnumber\tick-\pgfmathprintnumber\nexttick[$,
      x~tick~label~style={rotate=45,anchor=east}
      %xtick=data,
      %nodes~near~coords,
      %nodes~near~coords~align={vertical},
      ]
      \addplot~[hist={bins=\l_n_barras},pattern={Lines[angle=-45,distance=5pt]}]
      ~table[y~index=0]~{datos_histograma.csv};
  \end{axis}
\end{tikzpicture} \par

\int_new:N \l_min
\int_set:Nn \l_min {\seq_item:Nn \l_valores_unicos {1}}
\int_new:N \l_max
\int_set:Nn \l_max {\seq_item:Nn \l_valores_unicos {-1}}
\seq_new:N \l_rango
\int_step_inline:nnn {\l_min} {\int_eval:n {\l_max + 1}} {
  \seq_put_right:Nn \l_rango {\fp_eval:n {#1-0.5}}
}
\begin{tikzpicture}[baseline=(current~axis.north)]
  \begin{axis}[
      ybar,
      title={Gráfico~de~barras},
      ylabel={Frecuencia},
      xlabel={Datos},
      ymin=0,
      grid=major,
      xmajorgrids=false,
      x~grid~style={draw},
      %tick~style={draw=none},
      major~grid~style={draw=black},
      extra~x~ticks={\seq_use:Nn \l_rango {,}},
      extra~tick~style={grid=major,xticklabel={\empty},tick~style={draw=none}},
      %xtick=data,
      %nodes~near~coords,
      %nodes~near~coords~align={vertical},
      ]
      \addplot~[ybar,bar~width=1,draw=none]
      ~coordinates~{\l_datos_para_grafica};
  \end{axis}
\end{tikzpicture} \hspace*{1cm}

\begin{tikzpicture}[baseline=(current~axis.north)]
  \begin{axis}[
      ybar~interval,
      title={Histograma},
      ylabel={Frecuencia},
      xlabel={Datos},
      ymin=0,
      grid=major,
      major~grid~style={draw=black},
      %tick~style={draw=none},
      xticklabel= $[\pgfmathprintnumber\tick-\pgfmathprintnumber\nexttick[$,
      x~tick~label~style={rotate=45,anchor=east}

      ]
      \addplot~[draw=none,hist={bins=\l_n_barras}]
      ~table[y~index=0]~{datos_histograma.csv};
  \end{axis}
\end{tikzpicture} \par



\ExplSyntaxOff

El tercer cuartil de los datos es: \cuantil{0.75}

\ExplSyntaxOn
\seq_clear_new:N \l_frecuencias
\seq_map_inline:Nn \l_valores_unicos {
  \int_zero_new:N \l_frecuencia_actual
  \seq_map_inline:Nn \l_seq_lua {
    \int_compare:nT {#1=##1} {\int_gincr:N \l_frecuencia_actual}
  }
  \seq_put_right:Nx \l_frecuencias {\int_use:N \l_frecuencia_actual}
}
\seq_log:N \l_frecuencias
\seq_sort:Nn \l_frecuencias {
  \int_compare:nNnTF { #1 } > { #2 }
  { \sort_return_swapped: }
  { \sort_return_same: }
}
\seq_log:N \l_frecuencias
\tl_new:N \l_frecuencia_maxima
\tl_set:Nx \l_frecuencia_maxima {\seq_item:Nn \l_frecuencias {-1}}
\edef\frecuenciaMaxima{\tl_use:N \l_frecuencia_maxima}
\tl_new:N \l_dato_minimo
\tl_set:Nx \l_dato_minimo {\seq_item:Nn \l_lista_ordenada {1}}
\edef\datoMinimo{\tl_use:N \l_dato_minimo}
\ExplSyntaxOff

\edef\primerCuartil{\cuantil{0.25}}
Q1: \primerCuartil

\ExplSyntaxOn
\tl_log:N \l_datos_para_grafica
\edef\datosParaGraficoDeBarras{\tl_use:N \l_datos_para_grafica}
\fp_zero_new:N \l_media_de_los_datos
\fp_set:Nn \l_media_de_los_datos
  {\fp_eval:n {(\seq_use:Nn \l_seq_lua {+})/\seq_count:N \l_seq_lua}}
\fp_log:N \l_media_de_los_datos
\edef\mediaDeLosDatos{\fp_use:N \l_media_de_los_datos}
\edef\puntosDeLaMedia{(\mediaDeLosDatos,-3) (\mediaDeLosDatos,10)}
\ExplSyntaxOff

\begin{tikzpicture}[baseline=(current axis.north)]
  \begin{axis}[
    ytickmin=0,
    ylabel={Frecuencia},
    %ylabel style={at={(axis cs: 0,\frecuenciaMaxima/2 -| yticklabel cs:0)}},
    ylabel style={xshift=3mm},
    xlabel={Datos},
  ]
    %\addplot+ [boxplot={
    %  draw position=-1,
    %}] table[y index=0] {datos_histograma.csv};
    \addplot+ [boxplot={
      draw position=-1,
      whisker range=10,
    }] table[y index=0] {datos_histograma.csv};

    \addplot [ybar,bar width=1,pattern={Lines[angle=-45,distance=5pt]}]
     coordinates {\datosParaGraficoDeBarras};

    \addplot [sharp plot,update limits=false,dashed]
      coordinates {(\mediaDeLosDatos,-3) (\mediaDeLosDatos,10)};
    \coordinate (YT) at (yticklabel cs:1);
    \coordinate (M) at (axis cs: \mediaDeLosDatos,0);

  \end{axis}
  \node[pin={[pin distance=1mm,pin edge={>={Triangle[width=8pt]},<-}]above:Media}] at (M |- YT) {};
\end{tikzpicture}


\ExplSyntaxOn


\end{document}