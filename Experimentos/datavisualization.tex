\documentclass{experimento}

\begin{document}

  \begin{tikzpicture}[scale=1]
  \draw [->,name path=EjeX] (-2,0) -- (3,0) node [right] {$x$};
  \draw [->,name path=EjeY] (0,-2) -- (0,2) node [above] {$y$};
  %%% punto destacado
  %\coordinate (P) at ($(A)!.5!(B)$);
  %\draw[dashed] (P) -- (P -| 0,0) node [left] {$-3$};
  %\draw[dashed] (P) -- (P |- 0,0) node [above] {$2$};
  %\fill (P) circle[radius=2pt];
  %% curva
  \coordinate (A) at (-1,-1);
  \coordinate (B) at (2,-1);
  \draw[name path=Parabola] (A) parabola[bend pos=0.5] bend +(0,2) (B);
  \node[name intersections={of=EjeX and Parabola,name=I},above left] at (I-1) {$-2$};
  \node[name intersections={of=EjeX and Parabola,name=I},above right] at (I-2) {$5$};
  \end{tikzpicture}

  \begin{tikzpicture}[scale=1]
  \draw [->] (-0.5,0) -- (2,0) node [right] {$x$};
  \draw [->] (0,-2) -- (0,2) node [above] {$y$};
  \coordinate (A) at (0,-1.3);
  \coordinate (B) at (1.3,0);
  \draw[shorten >=-30pt, shorten <=-30pt] (A) node [left] {$-6$} -- (B) node [above] {4};
  %% cuadrado
  %%%\draw (0,0) --  ++(-7pt,0) -- ([turn]-90:7pt) -- ([turn]-90:7pt);
  %% punto destacado
  \coordinate (P) at ($(A)!.5!(B)$);
  \draw[dashed] (P) -- (P -| 0,0) node [left] {$-3$};
  \draw[dashed] (P) -- (P |- 0,0) node [above] {$2$};
  \fill (P) circle[radius=2pt];
  %% segunda recta
  \coordinate (C) at (0,1);
  \draw[shorten >=-30pt, shorten <=-20pt] (C) node [left] {$5$} -- (P);
  \end{tikzpicture}

  \begin{tikzpicture}
  \datavisualization [school book axes, visualize as smooth line/.list={f1,f2},
  all axes={length=4cm},
  x axis={label=$x$,ticks={major at={2,5}}},
  y axis={label=$y$,ticks={major at={4}}},
  f1={label in data={text=$L_1$, when=x is 5}},
  f2={label in data={text=$L_2$, when=x is 1}}]
  data [format=function,set=f2] {
  var x : interval [0:6];
  func y = -4*(\value x)/5 + 4;
  }
  data [format=function,set=f1] {
  var x : interval [0.5:6];
  func y = 5*(\value x - 2)/4;
  };
\end{tikzpicture}

\begin{tikzpicture}
  \datavisualization [school book axes, visualize as smooth line]
  data [format=function] {
  var x : interval [0:2];
  func y = \value x*\value x*\value x;
  };
\end{tikzpicture}

\begin{tikzpicture}
  \datavisualization [school book axes={standard labels},
    visualize as smooth line,
    clean ticks,
    x axis={label=$x$},
    y axis={label=$f(x)$}]
    data [format=function] {
    var x : interval [-1:1];
    func y = \value x*\value x + 1;
    };
\end{tikzpicture}

\tikz \datavisualization data group {function classes} = {
  data [set=log, format=function] {
    var x : interval [0.2:2.5];
    func y = ln(\value x);
  }
  data [set=lin, format=function] {
    var x : interval [-2:2.5];
    func y = 0.5*\value x;
  }
  data [set=squared, format=function] {
    var x : interval [-1.5:1.5];
    func y = \value x*\value x;
  }
  data [set=exp, format=function] {
    var x : interval [-2.5:1];
    func y = exp(\value x);
  }
};

\tikz \datavisualization [
school book axes, all axes={unit length=7.5mm},
visualize as smooth line/.list={log, lin, squared, exp},
style sheet=vary dashing]
data group {function classes};

\tikz \datavisualization [
  school book axes, all axes={unit length=7.5mm},
  x axis={label=$x$},
  y axis={label=$y$},
  visualize as smooth line/.list={log, lin, squared, exp},
  log= {label in legend={text=$\log x$}},
  lin= {label in legend={text=$x/2$}},
  squared={label in legend={text=$x^2$}},
  exp= {label in legend={text=$e^x$}},
  style sheet=vary dashing]
data group {function classes};

\tikz \datavisualization [
  school book axes,all axes={unit length=5mm, ticks={step=2}},
  visualize as smooth line]
data [format=function] {
  var t : interval [0:2*pi];
  func x = \value t * cos(\value t r);
  func y = \value t * sin(\value t r);
};

\tikz \datavisualization [
  scientific axes=clean,
  y axis={ticks={style={
  /pgf/number format/fixed,
  /pgf/number format/fixed zerofill,
  /pgf/number format/precision=2}}},
  x axis={ticks={tick suffix=${}^\circ$}},
  visualize as smooth line/.list={1,2,3,4,5,6},
  style sheet=vary hue]
data [format=function] {
var set : {1,...,6};
var x : interval [0:50];
func y = sin(\value x * (\value{set}+10))/(\value{set}+5);
};

\tikz \datavisualization
[school book axes, visualize as smooth line/.list={seno,coseno},
x axis={length=0.8\textwidth,ticks={tick suffix=${}^\circ$},label=$x$,
  ticks and grid={major at={0, 30, 45, 60, 90, 120, 135, 150, 180, 210, 225, 240, 270, 300, 315, 330, 360}}},
y axis={length=5cm,label=$y$,ticks and grid={major at={
  -1,-0.86 as $-\sqrt{3}/2$,-0.7 as $-\sqrt{2}/2$,-0.5 as $-1/2$,0,0.5 as $1/2$,0.7 as $\sqrt{2}/2$,0.86 as $\sqrt{3}/2$,1
  }}},
  seno={pin in data={text=sen($\theta^\circ$),when=x is 155,pin angle=70,pin length=1cm}},
  coseno={pin in data={text=cos($\theta^\circ$),when=x is 100,pin angle=225,pin length=2cm}},
  ]
data [format=function,set=seno] {
  var x : interval [0:360];
  func y = sin(\value x);
}
data [format=function,set=coseno] {
  var x : interval [0:360];
  func y = cos(\value x);
};

\tikz \datavisualization
[school book axes, visualize as smooth line/.list={seno,coseno},
  x axis={length=0.8\textwidth,ticks={tick suffix=${}^\circ$},label=$x$,
  ticks and grid={major at={0, 30, 45, 60, 90, 120, 135, 150, 180, 210, 225, 240, 270, 300, 315, 330, 360}}},
  y axis={length=5cm,label=$y$,ticks and grid={major at={
  -1,-0.86 as $-\sqrt{3}/2$,-0.7 as $-\sqrt{2}/2$,-0.5 as $-1/2$,0,0.5 as $1/2$,0.7 as $\sqrt{2}/2$,0.86 as $\sqrt{3}/2$,1
  }}},
  legend=above,
  clean ticks,
  seno={label in legend={text=sen($\theta^\circ$)}},
  coseno={label in legend={text=cos($\theta^\circ$)}},
  style sheet=vary dashing
  ]
data [format=function,set=seno] {
  var x : interval [0:360];
  func y = sin(\value x);
}
data [format=function,set=coseno] {
  var x : interval [0:360];
  func y = cos(\value x);
};


\tikz \datavisualization [scientific axes,
  x axis={attribute=people, length=2.5cm, ticks=few},
  y axis={attribute=year},
  visualize as scatter]
data {
year, people
1900, 100
1910, 200
1950, 200
1960, 250
2000, 150
};

%\tikz \datavisualization [
%  scientific axes=clean,
%  x axis={attribute=time, ticks={tick unit=ms},
%  label={elapsed time}},
%  y axis={attribute=v, ticks={tick unit=m/s},
%  label={speed of disc}},
%  all axes=grid,
%  visualize as line]
%data {
%  time, v
%  0, 0
%  1, 0.001
%  2, 0.002
%  3, 0.004
%  4, 0.0035
%  5, 0.0085
%  6, 0.0135
%};

%\tikz \datavisualization [
%  scientific axes={clean, end labels},
%  visualize as smooth line,
%  x axis={label=degree $d$,
%  ticks={tick unit={}^\circ}},
%  y axis={label=$\tan d$}]
%data [format=function] {
%  var x : interval [-80:80];
%  func y = tan(\value x);
%};

\tikz \datavisualization [scientific axes,
  all axes={length=2.5cm}, x axis={ticks=stack},
  visualize as smooth line]
data [format=function] {
  var y : interval[-100:100];
  func x = \value y*\value y;
};

\tikz \datavisualization data group {sin functions} = {
data [format=function] {
var set : {1,...,8};
var x : interval [0:50];
func y = sin(\value x * (\value{set}+10))/(\value{set}+5);
}
};

\tikzdatavisualizationset {
example visualization/.style={
scientific axes=clean,
y axis={ticks={style={
/pgf/number format/fixed,
/pgf/number format/fixed zerofill,
/pgf/number format/precision=2}}},
x axis={ticks={tick suffix=${}^\circ$}},
1={label in legend={text=$\frac{1}{6}\sin 11x$}},
2={label in legend={text=$\frac{1}{7}\sin 12x$}},
3={label in legend={text=$\frac{1}{8}\sin 13x$}},
4={label in legend={text=$\frac{1}{9}\sin 14x$}},
5={label in legend={text=$\frac{1}{10}\sin 15x$}},
6={label in legend={text=$\frac{1}{11}\sin 16x$}},
7={label in legend={text=$\frac{1}{12}\sin 17x$}},
8={label in legend={text=$\frac{1}{13}\sin 18x$}}
}
}

\tikz \datavisualization [
  visualize as smooth line/.list=
  {1,2,3,4,5,6,7,8},
  example visualization,
  style sheet=vary dashing]
data group {sin functions};

\begin{tikzpicture}[baseline]
  \datavisualization [ school book axes, visualize as smooth line ]
  data [format=function] {
  var x : interval [-0.1*pi:4];
  func y = sin(\value x r);
  }
  info' {
  \fill [red] (visualization cs: x={(.5*pi)}, y=1) circle [radius=2mm];
    %\draw (visualization cs: x=1,y=-.5) -- (visualization cs: x=2,y=-.5);
  \draw [decorate,decoration={brace,mirror}]
    (visualization cs: x=1,y=-.5) -- (visualization cs: x=2,y=-.5)
    node [midway,xshift=0pt,yshift=-7pt,font=\tiny] {50\%};
  };

\end{tikzpicture}

\tikz[
  declare function={
    f(\x) = sin(\x r)/\x;
  }
] \datavisualization [school book axes, visualize as smooth line,
  all axes={length=8cm,ticks=few}]
  data [format=function] {
  var x : interval [0.01:8*pi] samples 1000;
  func y = f(\value x);
};

\begin{equation*}
  \lim_{x \to \infty} \dfrac{\sin(x)}{x}
\end{equation*}


\tikz \datavisualization [
  school book axes,
  visualize as smooth line/.list={y11,y21,y31,y41,y51,y61},
  x axis={length=.8\textwidth},
  y axis={length=5cm},
  y61={pin in data={text=$x^6\cdot(1-x)^1$,when=x is 0.88,pin angle=30,pin length=1cm}},
  y51={pin in data={text=$x^5\cdot(1-x)^1$,when=x is 0.89,pin angle=30,pin length=1cm}},
  y41={pin in data={text=$x^4\cdot(1-x)^1$,when=x is 0.90,pin angle=30,pin length=1.1cm}},
  y31={pin in data={text=$x^3\cdot(1-x)^1$,when=x is 0.91,pin angle=30,pin length=1.3cm}},
  y21={pin in data={text=$x^2\cdot(1-x)^1$,when=x is 0.93,pin angle=30,pin length=1.4cm}},
  y11={pin in data={text=$x^1\cdot(1-x)^1$,when=x is 0.97,pin angle=30,pin length=1cm}},
  ]
data [format=function,set=y61] {
  var x : interval [0:1] samples 200;
  func y = 56*(\value x)^6*(1-\value x);
}
data [format=function,set=y51] {
  var x : interval [0:1] samples 200;
  func y = 42*(\value x)^5*(1-\value x);
}
data [format=function,set=y41] {
  var x : interval [0:1] samples 200;
  func y = 30*(\value x)^4*(1-\value x);
}
data [format=function,set=y31] {
  var x : interval [0:1] samples 200;
  func y = 20*(\value x)^3*(1-\value x);
}
data [format=function,set=y21] {
  var x : interval [0:1] samples 200;
  func y = 12*(\value x)^2*(1-\value x);
}
data [format=function,set=y11] {
  var x : interval [0:1] samples 200;
  func y = 6*(\value x)^1*(1-\value x);
};

\end{document}

