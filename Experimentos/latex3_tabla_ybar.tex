\documentclass[sin nombre]{plantilla-evaluacion-v1}
\usepackage{expl3}

\begin{document}
  \tikzset{
    mynode/.style={
      minimum height=1cm,
      minimum width=1cm,
      draw,
      anchor=north west
    }
  }
  
  \ExplSyntaxOn
  \begin{tikzpicture}
  \int_step_inline:nn {6} {
    \int_step_inline:nn {8} {
      \node[mynode] at (##1, -#1) {\tiny \int_eval:n {(#1 - 1) * 8 + ##1}};
    }
  }
  \end{tikzpicture}
  \ExplSyntaxOff

  \ExplSyntaxOn
  \NewExpandableDocumentCommand \randomint { m m }
    { \int_rand:nn { #1 } { #2 } }
  \ExplSyntaxOff

  \randomint{8-8/4}{8+8/4}

  \LaTeX{} can now compute: $ \frac{\sin (3.5)}{2} + 2\cdot 10^{-3}
  = \ExplSyntaxOn \fp_to_decimal:n {sin(3.5)/2 + 2e-3} $.

\ExplSyntaxOn

\int_rand:nn {5} {20} \par
\fp_eval:n { rand() } \par
\cs_set:Npn \muestra:nn #1#2 {
  \fp_to_int:n {#1 + sqrt(-2*ln(rand()))*cos(2*pi*rand())*#2}
}
\muestra:nn {20}{5} \par


%% GENERAR LISTA
\seq_clear_new:N \l_mi_lista
\int_step_inline:nn {40} {
    \seq_put_right:Nx \l_mi_lista {\muestra:nn {20}{5}}
}
\seq_use:Nn \l_mi_lista { ,~ } \par

%% ORDENAR LISTA
\seq_clear_new:N \l_lista_ordenada
\seq_set_eq:NN \l_lista_ordenada \l_mi_lista
\seq_sort:Nn \l_lista_ordenada {
  \int_compare:nNnTF { #1 } > { #2 }
  { \sort_return_swapped: }
  { \sort_return_same: }
}
\seq_use:Nn \l_lista_ordenada { ,~} \par

%% VALORES UNICOS
\seq_clear_new:N \l_valores_unicos
\seq_set_eq:NN \l_valores_unicos \l_lista_ordenada
\seq_remove_duplicates:N \l_valores_unicos
\seq_use:Nn \l_valores_unicos { ,~} \par


\begingroup
\pgfkeys{/pgf/number~format/.cd,fixed,fixed~zerofill,precision=3}
\begin{tabular}{ccccc}
  Datos & Cantidad & F.~Acumulada & Probabilidad & P.~Acumulada\\
  \int_gzero_new:N \l_frecuencia_acumulada
  \int_gzero_new:N \l_probabilidad_acumulada
  \int_gzero_new:N \l_frecuencia
  \fp_gzero:N \l_probabilidad
  \fp_gzero:N \l_probabilidad_acumulada
  \seq_map_inline:Nn \l_valores_unicos
  {
    #1 &
    \int_gzero:N \l_frecuencia
    \seq_map_inline:Nn \l_mi_lista {
      \int_compare:nT {#1=##1} {\int_gincr:N \l_frecuencia}
    }
    \int_use:N \l_frecuencia & 
    \int_gadd:Nn \l_frecuencia_acumulada {\int_use:N \l_frecuencia}
    \int_use:N \l_frecuencia_acumulada &
    \fp_gset:Nn \l_probabilidad {\int_use:N \l_frecuencia / \seq_count:N \l_mi_lista }
    \pgfmathprintnumber{\fp_use:N \l_probabilidad} &
    \fp_gadd:Nn \l_probabilidad_acumulada {\fp_use:N \l_probabilidad}
    \pgfmathprintnumber{\fp_use:N \l_probabilidad_acumulada} \\
  }
\end{tabular} \par
\endgroup
%%%

\seq_new:N \l_puntos
\int_gzero_new:N \g_n
\seq_map_inline:Nn \l_valores_unicos {
  \int_gzero:N \g_n
  \seq_map_inline:Nn \l_mi_lista {
    \int_compare:nT {#1=##1} {\int_gincr:N \g_n}
    }
    \seq_put_right:Nx \l_puntos {(#1,~\int_use:N \g_n)}
}

lista:~\seq_use:Nn \l_puntos {~} \par
\tl_new:N \l_datos_para_grafica
\tl_set:Nx \l_datos_para_grafica { \seq_use:Nn \l_puntos {~} }

\begin{tikzpicture}[baseline=(current~bounding~box.north)]
  \begin{axis}[
      ybar,
      ylabel={Cantidad},
      xlabel={Datos},
      ymin=0,
      %xtick=data,
      %nodes~near~coords,
      %nodes~near~coords~align={vertical},
      ]
      \addplot~[ybar,bar~width=1,pattern={Lines[angle=-45,distance=5pt]}]
      ~coordinates~{\l_datos_para_grafica};
  \end{axis}
\end{tikzpicture} 

\iow_new:N \g_archivo
\iow_open:Nn \g_archivo { datos_histograma.csv }
\seq_map_inline:Nn \l_mi_lista
  {
    \iow_now:Nx \g_archivo { #1 }
  }
\iow_close:N \g_archivo

\begin{tikzpicture}[baseline=(current~bounding~box.north)]
  \begin{axis}[
      ybar~interval,
      ylabel={Cantidad},
      xlabel={Datos},
      ymin=0,
      grid=none,
      xticklabel= $[\pgfmathprintnumber\tick-\pgfmathprintnumber\nexttick[$,
      x~tick~label~style={rotate=45,anchor=east}
      %xtick=data,
      %nodes~near~coords,
      %nodes~near~coords~align={vertical},
      ]
      \addplot~[hist,pattern={Lines[angle=-45,distance=5pt]}]
      ~table[y~index=0]~{datos_histograma.csv};
  \end{axis}
\end{tikzpicture} \par

% Create a sequence to hold our data points
\seq_new:N \l_data_seq
% Function to generate random data points
\cs_new_protected:Nn \generate_data:nn
  {
    \seq_clear:N \l_data_seq
    \int_step_inline:nn {#1}
      {
        \seq_put_right:Nx \l_data_seq 
          { 
            (##1,~ \fp_eval:n { round(rand() * #2, 2) })
          }
      }
  }
% Generate 10 random data points with max value 100
\generate_data:nn {10} {100}
% Convert sequence to comma-separated list for pgfplots
\tl_new:N \l_plot_data_tl
\tl_set:Nx \l_plot_data_tl { \seq_use:Nn \l_data_seq {~} }

% For debugging: print the generated data


Generated~data:~\l_plot_data_tl \par

\bigskip

\begin{tikzpicture}
\begin{axis}[
    ybar,
    ylabel={Value},
    xlabel={Index},
    ymin=0,
    xtick=data,
    nodes~near~coords,
    nodes~near~coords~align={vertical},
    ]
    \addplot~coordinates~{\l_plot_data_tl};
\end{axis}
\end{tikzpicture} \par

\begin{luacode}
  require('estadistica.lua')
  function obtener_lista()
    repeat
      lista = Generar_muestras(60,20,4)
      v_unicos = Obtener_unicos(lista)
    until Divisor_ideal(v_unicos) == 6
    lista_str = table.concat(lista, ",")
    tex.sprint(lista_str) 
  end
\end{luacode}


\clist_new:N \l_lista_lua
\clist_set:Nx \l_lista_lua {\directlua{obtener_lista()}}
\seq_set_from_clist:NN \l_seq_lua \l_lista_lua 
Lista~desde~lua:~\seq_use:Nn \l_seq_lua {|~}

\ExplSyntaxOff



\end{document}