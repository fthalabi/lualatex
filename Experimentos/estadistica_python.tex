\documentclass[pagina vacia]{srs2}

\begin{python}
import numpy as np
import polars as pl

np.random.seed()
aleatorio_n99 = np.random.default_rng().normal(10,3,size=99)
muestras_n99 = np.floor(aleatorio_n99).astype(int)

def frecuencia(datos, x):
  valores,_,_,conteo = np.unique_all(datos)
  indice = np.where(valores == x)
  return conteo[indice].item(0)

def frecuencia_acumulada(datos, x):
  valores,_,_,conteo = np.unique_all(datos)
  conteo_acumulado = np.cumsum(conteo)
  indice = np.where(valores == x)
  return conteo_acumulado[indice].item(0)

def probabilidad(datos, x):
  return frecuencia(datos, x)*100/(datos.size)

def probabilidad_acumulada(datos, x):
  return frecuencia_acumulada(datos, x)*100/(datos.size)

def cuantil(datos, q):
  valores,_,_,conteo = np.unique_all(datos)
  p_acumulada = np.cumsum(conteo)/(datos.size)
  indices = np.where(p_acumulada >= q)
  return valores[indices[0].item(0)]

def puntos_grafico_barras(datos):
  valores,_,_,conteo = np.unique_all(datos)
  lista_puntos = [f"({x}, {y})" for x,y in zip(valores,conteo)]
  return " ".join(lista_puntos)

def histograma(muestras,ancho=3):
  minimo = np.min(muestras)
  rango = np.max(muestras) - np.min(muestras)
  n_barras = np.ceil(rango/ancho).astype(int)
  maximo = (minimo + ancho*n_barras).astype(int)
  frecuencias, cortes = np.histogram(muestras, bins=n_barras, range=(minimo,maximo))
  intervalos = [f"$\\left[{int(a)}-{int(b)}\\right[$" for a,b in zip(cortes[0:-1],cortes[1:])]
  puntos = [f"({int(a)},{int(b)})" for a,b in enumerate(frecuencias)]
  return [", ".join(intervalos), " ".join(puntos), n_barras]

def generar_tabla(archivo,datos):
  tabla = pl.DataFrame({'valores': datos})
  tabla = tabla.group_by('valores').len().sort('valores')
  tabla = tabla.rename({'len':'frecuencia'})
  tabla = tabla.with_columns(
    pl.col('frecuencia').cum_sum().alias('f_cumulativa'),
    (pl.col('frecuencia')*100/pl.col('frecuencia').sum()).alias('probabilidad'),
  )
  tabla = tabla.with_columns(
    pl.col('probabilidad').cum_sum().alias('p_acumulada')
  )
  tabla.write_csv(archivo,include_header=False)
generar_tabla(archivo='tabla.csv', datos=muestras_n99)

\end{python}

\begin{document}

%\pgfkeys{/pgf/number format/.cd,fixed,fixed zerofill,precision=2,verbatim,use comma}
\csvreader[no head,
  centered tabularray={
  cells={valign=m},
  %cell{1}{2,4} = {r=1,c=2}{c},
  %cell{1}{1-Z}={font=\bfseries},
  %colspec={X[1,c]X[2,c]X[2,c]X[2,c]X[2,c]},
  colspec={cccQ[si={table-format=3.2},c]Q[si={table-format=3.2},c]},
  width=0.7\linewidth,
  hlines,
  hline{1,2,Z}={black,1pt},
  rows={rowsep+=2pt},
  %vline{1,2,4,6} = {1pt,solid},
  %hline{1,2,Z} = {1pt,solid},
  % hline{2}={1}{-}{solid},
  % hline{2}={2}{-}{solid},
},
  table head={Datos & {Frecuencia\\absoluta} & {F. absoluta\\acumulada} & {Frecuencia\\relativa} & {F. relativa\\acumulada} \\}
]{tabla.csv}{1=\a, 2=\b, 3=\c, 4=\d, 5=\e}%
{\a & \b & \c & \d \%  & \e \%}


El cuantil del 25\% es \py{cuantil(muestras_n99,0.25)}.\\
El cuantil del 50\% es \py{cuantil(muestras_n99,0.5)}.\\
El cuantil del 75\% es \py{cuantil(muestras_n99,0.75)}.\\
El cuantil del 100\% es \py{cuantil(muestras_n99,1)}.\\


\begin{columnas}[0.5][t]
\begin{tikzpicture}[baseline=(current axis.north)]
  \begin{axis}[
      ybar,
      title={Gráfico de barras},
      ylabel={Frecuencia},
      xlabel={Datos},
      ymin=0,
      %xtick=data,
      %nodes~near~coords,
      %nodes~near~coords~align={vertical},
      ]
    \addplot[ybar,bar width=1,pattern={Lines[angle=-45,distance=5pt]}]
      coordinates {\py{puntos_grafico_barras(muestras_n99)}};
  \end{axis}
\end{tikzpicture}
\siguiente
\begin{tikzpicture}[baseline=(current axis.north)]
  \begin{axis}[
      ybar,
      title={Histograma},
      ylabel={Frecuencia},
      xlabel={Datos},
      ymin=0,
      xtick=data,
      xticklabels/.expanded={\py{histograma(muestras_n99)[0]}},
      x tick label style={rotate=45,anchor=north east,yshift=2pt,xshift=2pt},
      nodes near coords,
      nodes near coords align={vertical},
      xmin=-1,
      xmax/.expanded={\py{histograma(muestras_n99)[2]}},
      ]
    \addplot[ybar,bar width=1,pattern={Lines[angle=-45,distance=5pt]}]
      coordinates {\py{histograma(muestras_n99)[1]}};
  \end{axis}
\end{tikzpicture}
\end{columnas}


\end{document}