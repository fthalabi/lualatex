\documentclass{experimento}

\newcounter{nproblema}
\setcounter{nproblema}{1}

\NewDocumentCommand{\superbox}{m}{%
\ifthenelse{\value{nproblema} < 100}{%
\tcbox[colback=black!60, colframe=black!60, coltext=white,%
  on line,boxsep=0pt, left=1pt, right=1pt, top=1pt, bottom=1pt,%
  width=1cm]{\makebox[\widthof{22}][c]{\sffamily\small\bfseries #1}}}{%
\tcbox[colback=black!60, colframe=black!60, coltext=white,%
  on line,boxsep=0pt, left=1pt, right=1pt, top=1pt, bottom=1pt,%
  width=1cm]{\sffamily\small\bfseries #1}}}

\NewDocumentEnvironment{preguntas}{O{}+b}{%
  \NewDocumentCommand{\pregunta}{O{}}{\tcbitem[##1,blankest]%
    \superbox{\thenproblema}\stepcounter{nproblema} \hspace{1em}}
  \begin{tcbitemize}[raster columns=1,lefttitle=0pt,blankest,
    %fonttitle=\itshape, halign=center,
    valign=center,leftupper=2em,top=0pt,boxsep=0pt,
    breakable=false,nobeforeafter,#1]
      #2
}{%
  \end{tcbitemize}
}

\newtcolorbox{malla}[1][]{enhanced,inherit height,colback=white,colframe=black,boxrule=1pt,underlay={\begin{tcbclipinterior}
  \draw[help lines,black!20,step=5mm,yshift=0pt] (interior.south west) grid (interior.north east);\end{tcbclipinterior}},
  height=#1cm}

\newtcolorbox{respuesta}[1][]{enhanced,inherit height,colback=white,colframe=black,boxrule=1pt,underlay boxed title={\begin{tcbclipinterior}
  \draw[help lines,step=5mm] (interior.south west) grid[xstep=0] (interior.north east);\end{tcbclipinterior}},title=Respuesta,attach boxed title to top left={yshift=-\tcboxedtitleheight/2,xshift=10pt},
  boxed title style={colback=white},coltitle=black,height=#1cm}

\newtcolorbox{centrado}[1][]{enhanced,blank,nobeforeafter,halign=center,valign=center,
top=7pt,bottom=7pt,breakable=false,before={\newline},
before upper={\tikzset{baseline=(current bounding box.center)}},
after upper={\tikz[overlay] \node[xshift=3em] {#1};}}

%% Setting values
\newlength{\myitemsep}
\setlength{\myitemsep}{5pt}
\newlength{\mytopsep}
\setlength{\mytopsep}{10pt}

\NewDocumentEnvironment{alternativas}{O{}+b}%
{%
    \newline
    \newcommand{\alternativa}{\item}
    \begin{minipage}{\linewidth}
      \vspace*{\mytopsep}
      \begin{enumerate}[label={{\itshape\alph*})},
        labelsep=1em,leftmargin=2em,labelwidth=2em,
        labelindent=0pt,itemsep=\myitemsep,topsep=\mytopsep,
        listparindent=0pt,before=\raggedright,#1]
          #2
}{%
    \end{enumerate}
    \vspace*{\mytopsep}
  \end{minipage}
}

\NewDocumentEnvironment{alternativasgraficas}{O{}+b}{%
  \vspace*{\mytopsep}
  \NewDocumentCommand{\alternativa}{O{}}{\tcbitem[##1]}
  \begin{tcbitemize}[raster columns=2,raster equal height=rows,blank,lefttitle=0pt,
    %fonttitle=\itshape, halign=center,
    valign=center,leftupper=2em,top=-10pt,boxsep=0pt,
    breakable=false,nobeforeafter,
    coltitle=black,title={{\itshape\alphalph{\thetcbrasternum}})},#1]
      #2
}{%
  \end{tcbitemize}\phantom{.}\vspace*{\mytopsep}
}

% We define a command that does nothing, just acts as a marker.
% This is what we will split the environment's content on.
\newcommand{\siguiente}{}

\ExplSyntaxOn % Activate LaTeX3 programming syntax

% --- Variable Declarations ---
% A sequence to hold the parts of the body after splitting.
\seq_new:N \l_myenv_body_parts_seq
% A token list to hold the first part of the content.
\tl_new:N \l_myenv_part_one_tl
% A token list to hold the second part of the content.
\tl_new:N \l_myenv_part_two_tl

% --- Internal Processing Function ---
% This function takes the body of the environment (#1) and splits it.
\cs_new_protected:Npn \myenv_process_body:n #1
  {
    % Split the input (#1) by the \siguiente command and store the parts in our sequence.
    \seq_set_split:Nnn \l_myenv_body_parts_seq { \siguiente } { #1 }

    % Take the first item from the sequence and store it in \l_myenv_part_one_tl
    \seq_pop_left:NN \l_myenv_body_parts_seq \l_myenv_part_one_tl

    % Take all remaining items in the sequence, join them together,
    % and store the result in \l_myenv_part_two_tl.
    % This correctly handles cases with no \siguiente or multiple \siguiente's.
    \tl_set:Nx \l_myenv_part_two_tl { \seq_use:Nn \l_myenv_body_parts_seq { } }
  }

% --- The Environment Definition ---
\NewDocumentEnvironment{columnas}{ O{0.5} +b }
  {
    % 1. Process the entire body (#1) of the environment using our helper function.
    \myenv_process_body:n { #2 }

    % 2. Now you can use the processed parts.
    % We use \tl_use:N to output the content stored in our token list variables.
    % For this example, we'll format them directly.
    \begin{tcolorbox}[sidebyside,sidebyside~adapt=right,blankest,lefthand~ratio=#1]
    \tl_use:N \l_myenv_part_one_tl
    \tcblower
    \tl_use:N \l_myenv_part_two_tl
    \end{tcolorbox}
  }
  {
    % The "end" part of the environment is empty in this case.
  }

\ExplSyntaxOff % Deactivate LaTeX3 syntax

\NewTasksEnvironment[label={{\itshape\alph*})},%
label-width=2em,item-indent=3em,item-format={\raggedright},%
label-offset=1em]{talternativas}[\alternativa](1)

\NewTasksEnvironment[label={{\Roman*.}},
label-width=2em,item-indent=6em,item-format={\raggedright},%
label-offset=1em]{topciones}[\opcion](1)

\NewTasksEnvironment[label={(\arabic*)},
label-width=2em,item-indent=6em,item-format={\raggedright},%
label-offset=1em]{topcionesn}[\opcion](1)

\DeclareTotalTColorBox{\doscolumnas}{ O{0.5} +m +m }{%
  sidebyside,sidebyside adapt=right,blankest,lefthand ratio=#1}%
  {#2\tcblower #3}



\begin{document}

\doscolumnas[0.7]{%
  \begin{topciones}%
  \opcion asdasd hola hola hola hola hola hola hola hola hola hola hola
  \opcion asdasd hola hola hola hola hola
  \opcion asdasd hola hola
  \end{topciones}%
  \begin{talternativas}
  \alternativa as asdasdasd
  \alternativa as asdasdasd
  \alternativa as asdasdasd
  \alternativa as asdasdasd
  \end{talternativas}
}{
  \begin{tikzpicture}[scale=0.75]
  \draw [->] (-2.5,0) -- (2.5,0) node [right] {$x$};
  \draw [->] (0,-2.5) -- (0,2.5) node [above] {$y$};
  \draw[shorten >=-20pt, shorten <=-20pt] (-1.5,0) node [below] {$-6$} -- (0,1) node [right] {1};
  \end{tikzpicture}
}

\begin{columnas}[0.7]
  \begin{topciones}
  \opcion asdasd hola hola hola hola hola hola hola hola hola hola hola
  \opcion asdasd hola hola hola hola hola
  \opcion asdasd hola hola
  \end{topciones}
  \begin{talternativas}
  \alternativa as asdasdasd
  \alternativa as asdasdasd
  \alternativa as asdasdasd
  \alternativa as asdasdasd
  \end{talternativas}
\siguiente
  \begin{tikzpicture}[scale=0.75]
  \draw [->] (-2.5,0) -- (2.5,0) node [right] {$x$};
  \draw [->] (0,-2.5) -- (0,2.5) node [above] {$y$};
  \draw[shorten >=-20pt, shorten <=-20pt] (-1.5,0) node [below] {$-6$} -- (0,1) node [right] {1};
  \end{tikzpicture}
\end{columnas}

\begin{talternativas}
\alternativa as asdasdasd
\alternativa as asdasdasd
\alternativa as asdasdasd
\alternativa as asdasdasd
\alternativa as asdasdasd
\end{talternativas}

\begin{topciones}
\opcion asdasd
\opcion asdasd
\opcion asdasd
\opcion asdasd
\end{topciones}

\begin{topcionesn}
\opcion asdasd
\opcion asdasd
\opcion asdasd
\opcion asdasd
\end{topcionesn}

%\begin{myenv}
%Here is the first part of the text, which will become argument \#1.
%\siguiente
%And this is all the text that comes after the mybreak command. It will become argument \#2.
%\end{myenv}
%
%\section*{Case with no mybreak}
%\begin{myenv}
%This entire block of text contains no mybreak command. The whole content goes to the first part.
%\end{myenv}
%
%\section*{Case with multiple breaks}
%\begin{myenv}
%This is the first part.
%\siguiente
%This is the second part.
%\siguiente
%This part is also included in the second part. Everything after the first \texttt{\string\siguiente} is grouped together.
%\end{myenv}
%

hola \superbox{1}

\begin{preguntas}
\pregunta \lipsum[1]
\begin{malla}[2]
\end{malla}
\pregunta \lipsum[1-2]
\pregunta \lipsum[3]
\begin{respuesta}[2]
\end{respuesta}
\pregunta hola como estas?
\begin{centrado}[Holas :)]
  \begin{tikzpicture}[scale=1]
  \draw [->,name path=EjeX] (-2,0) -- (3,0) node [right] {$x$};
  \draw [->,name path=EjeY] (0,-2) -- (0,2) node [above] {$y$};
  %%% punto destacado
  %\coordinate (P) at ($(A)!.5!(B)$);
  %\draw[dashed] (P) -- (P -| 0,0) node [left] {$-3$};
  %\draw[dashed] (P) -- (P |- 0,0) node [above] {$2$};
  %\fill (P) circle[radius=2pt];
  %% curva
  \coordinate (A) at (-1,-1);
  \coordinate (B) at (2,-1);
  \draw[name path=Parabola] (A) parabola[bend pos=0.5] bend +(0,2) (B);
  \node[name intersections={of=EjeX and Parabola,name=I},above left] at (I-1) {$-2$};
  \node[name intersections={of=EjeX and Parabola,name=I},above right] at (I-2) {$5$};
  \end{tikzpicture}
\end{centrado}
\begin{alternativas}
  \alternativa aasdk
  \alternativa aasdk
  \alternativa aasdk
  \alternativa aasdk
\end{alternativas}
\pregunta \lipsum[4-7]
\end{preguntas}

hoaskdasok s

\begin{preguntas}
\pregunta \lipsum[1]
\begin{alternativas}
  \alternativa aasdk
  \alternativa aasdk
  \alternativa aasdk
  \alternativa aasdk
\end{alternativas}
\pregunta \lipsum[1-2]
\pregunta \lipsum[3]
\begin{tasks}(2)
\task \\
\begin{tikzpicture}[scale=0.75]
  \draw [->] (-2.5,0) -- (2.5,0) node [right] {$x$};
  \draw [->] (0,-2.5) -- (0,2.5) node [above] {$y$};
  \draw[shorten >=-20pt, shorten <=-20pt] (-1.5,0) node [below] {$-6$} -- (0,1) node [right] {1};
\end{tikzpicture}
\task \\
\begin{tikzpicture}[scale=0.75]
  \draw [->] (-2.5,0) -- (2.5,0) node [right] {$x$};
  \draw [->] (0,-2.5) -- (0,2.5) node [above] {$y$};
  \draw[shorten >=-20pt, shorten <=-20pt] (-1,0) node [below] {$-1$} -- (0,1.5) node [right] {6};
\end{tikzpicture}
\task \\
\begin{tikzpicture}[scale=0.75]
  \draw [->] (-2.5,0) -- (2.5,0) node [right] {$x$};
  \draw [->] (0,-2.5) -- (0,2.5) node [above] {$y$};
  \draw[shorten >=-20pt, shorten <=-20pt] (0,-1) node [left] {$-1$} -- (1.5,0) node [below] {6};
\end{tikzpicture}
\task \\
\begin{tikzpicture}[scale=0.75]
  \draw [->] (-2.5,0) -- (2.5,0) node [right] {$x$};
  \draw [->] (0,-2.5) -- (0,2.5) node [above] {$y$};
  \draw[shorten >=-20pt, shorten <=-20pt] (0,1) node [left] {$1$} -- (1.5,0) node [below] {6};
\end{tikzpicture}
\task \\
\begin{tikzpicture}[scale=0.75]
  \draw [->] (-2.5,0) -- (2.5,0) node [right] {$x$};
  \draw [->] (0,-2.5) -- (0,2.5) node [above] {$y$};
  \draw[shorten >=-20pt, shorten <=-20pt] (0,1.5) node [left] {6} -- (0.7,0) node [below] {1};
\end{tikzpicture}
\end{tasks}
\pregunta \lipsum[4-5]
\begin{columnas}[0.7]
  \begin{topciones}
  \opcion asdasd hola hola hola hola hola hola hola hola hola hola hola
  \opcion asdasd hola hola hola hola hola
  \opcion asdasd hola hola
  \end{topciones}
  \begin{talternativas}
  \alternativa as asdasdasd
  \alternativa as asdasdasd
  \alternativa as asdasdasd
  \alternativa as asdasdasd
  \end{talternativas}
\siguiente
  \begin{tikzpicture}[scale=0.75]
  \draw [->] (-2.5,0) -- (2.5,0) node [right] {$x$};
  \draw [->] (0,-2.5) -- (0,2.5) node [above] {$y$};
  \draw[shorten >=-20pt, shorten <=-20pt] (-1.5,0) node [below] {$-6$} -- (0,1) node [right] {1};
  \end{tikzpicture}
\end{columnas}
\pregunta \lipsum[4-7]

\end{preguntas}


\end{document}