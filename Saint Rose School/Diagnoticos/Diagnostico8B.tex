\def\titulo{Evaluación diagnóstica}
\def\subtitulo{Números y álgebra}
\def\curso{Octavo básico B}

\documentclass[sin curso]{srs}

\begin{document}

\section{Objetivo}

La siguiente evaluación tiene como objetivo medir el dominio de contenidos y habilidades
de cada estudiante, para poder diseñar un plan de estudio acorde de las necesidades del
grupo curso.

Particularmente, se medirá:
\begin{itemize}[nosep]
  \item Operatoria de números enteros y fracciones.
  \item Interpretar y utilizar porcentajes.
  \item Utilizar lenguaje algebraico para construir ecuaciones.
  \item Resolución de problemas que involucran proporciones directas o inversas.
\end{itemize}

La nota es solo referencial y no afecta el promedio del estudiante de ninguna manera. Cada
pregunta tiene un punto.

\section{Operatoria de números y fracciones}

Determine el valor de las siguientes expresiones.
\begin{preguntas}
  \pregunta $100-6-5-4-3-42-51 =$
  \begin{malla}[height=3cm]
  \end{malla}
  \pregunta $-(-24)+(-13)-(9) =$
  \begin{malla}[height=3cm]
  \end{malla}
\end{preguntas}

Represente los siguientes números como fracciones y simplifique en caso de ser posible.
\begin{preguntas}
  \pregunta $3,25 =$
  \begin{malla}[height=3cm]
  \end{malla}
  \pregunta $3,12252525...=$
  \begin{malla}[height=3cm]
  \end{malla}
\end{preguntas}

Represente las siguientes fracciones como porcentajes.
\begin{preguntas}
  \pregunta $\dfrac{3}{5} =$
  \begin{malla}[height=3cm]
  \end{malla}
  \pregunta $\dfrac{1}{8} =$
  \begin{malla}[height=3cm]
  \end{malla}
\end{preguntas}

Represente la siguiente cantidad como un número.
\begin{preguntas}
  \pregunta $15.62\% =$
  \begin{malla}[height=3cm]
  \end{malla}
\end{preguntas}

Determine las cantidades solicitadas en cada enunciado.
\begin{preguntas}
  \pregunta En un circo hay 200 personas. El 17\% ha comprado un paquete de gomitas, $8/25$
  no ha comprado nada y el resto ha comprado palomitas. ¿Cuántos han comprado cada cosa?
  \begin{malla}[height=5cm]
  \end{malla}
  \pregunta En una colonia de Jalisco hay 87 personas con coche y 130 personas sin coche. Si
  de las que tienen coche, el $25\%$ son mujeres y de las que no tienen el $40\%$ son hombres,
  indica la cantidad de mujeres y hombres con coche.
  \begin{malla}[height=5cm]
  \end{malla}
\end{preguntas}



\section{Proporciones}

Determine el valor desconocido en las siguientes proporciones.
\begin{preguntas}
  \pregunta $\dfrac{15}{x}=\dfrac{5}{20}$
  \begin{malla}[height=4.5cm]
  \end{malla}
  \pregunta
    \begin{tblr}{colspec={ccc},vlines,hlines,cells={valign=m}}
      y & 5 & 3 \\
      49 & 35 & 21 \\
    \end{tblr}
    \begin{malla}[height=4.5cm]
    \end{malla}

  \pregunta
    \begin{tblr}{colspec={ccc},vlines,hlines,cells={valign=m}}
      2 & 5 & 6 \\
      21 & w & 7 \\
    \end{tblr}
    \begin{malla}[height=4.5cm]
    \end{malla}
\end{preguntas}

Resuelve los siguientes problemas.
\begin{preguntas}
  \pregunta Un auto recorre en una hora y cuarto una distancia de 100 km. ¿Cuánto recorrerá
  en cinco horas?
  \begin{malla}[height=5cm]
  \end{malla}
  \pregunta Si 12 albañiles construyen una obra en 5 días, ¿en cuántos días la realizarán
  20 albañiles?
  \begin{malla}[height=5cm]
  \end{malla}
\end{preguntas}

\section{Ecuaciones}

Resuelve las siguientes ecuaciones.
\begin{preguntas}
  \pregunta $12-(6x-7)=11$
  \begin{malla}[height=5cm]
  \end{malla}
  \pregunta $\dfrac{x}{10}-12=5$
  \begin{malla}[height=5cm]
  \end{malla}
\end{preguntas}

Encuentra la solución al siguiente problema.

\begin{preguntas}
  \pregunta El total de ventas de un almacén el día lunes es la mitad de las ventas
  del día martes. Si
  el miércoles se vendieron $\$50000$ más que el día lunes, y en estos tres días se vendieron
  $\$350000$, ¿cuál fue el total de ventas del martes?
  \begin{malla}[height=4cm]
  \end{malla}

\end{preguntas}

\end{document}