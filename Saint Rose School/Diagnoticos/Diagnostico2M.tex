\def\titulo{Evaluación diagnóstica}
\def\subtitulo{Números y álgebra}
\def\curso{Segundo medio}

\documentclass[sin curso]{srs}

\begin{document}

\section{Objetivo}

La siguiente evaluación tiene como objetivo medir el dominio de contenidos y habilidades
de cada estudiante, para poder diseñar un plan de estudio acorde de las necesidades del
grupo curso.

Particularmente, se medirá:
\begin{itemize}[nosep]
  \item Propiedades de potencias y raíces.
  \item Factorizar usando productos notables.
  \item Solucionar sistemas lineales de dos ecuaciones (2x2).
  \item Plantear y resolver problemáticas usando lenguaje algebraico.
\end{itemize}

La nota es solo referencial y no afecta el promedio del estudiante de ninguna manera. Cada
pregunta tiene un punto.

\section{Operatoria de números}

Determine el valor de las siguientes expresiones.
\begin{preguntas}
  \pregunta $4^2\cdot 2^3\cdot 8^2 =$
  \begin{malla}[height=4cm]
  \end{malla}
  \pregunta $\dfrac{3^5\cdot 4^{-6}}{3^7\cdot 4^{-8}} =$
  \begin{malla}[height=4cm]
  \end{malla}
  \pregunta $\left(\left(-5\right)^2\right)^3 =$
  \begin{malla}[height=4cm]
  \end{malla}
  \pregunta $\left(3^{-2}\cdot 5^2\right)^3\left(3^3\cdot 5^{-3}\cdot 7\right)^2 =$
  \begin{malla}[height=5cm]
  \end{malla}
  \pregunta $\left(\dfrac{2^2\cdot 3^5 \cdot 4^2}{2^4\cdot 3^2}\right)^2 =$
  \begin{malla}[height=6cm]
  \end{malla}
  \pregunta $\left(\dfrac{1}{2^{-3}}-\dfrac{1}{2^{-1}}\right)^{-3} =$
  \begin{malla}[height=6cm]
  \end{malla}
  \pregunta $\sqrt{729} =$
  \begin{malla}[height=5cm]
  \end{malla}
  \pregunta $\sqrt[4]{625} =$
  \begin{malla}[height=6cm]
  \end{malla}
  \pregunta $\left(\sqrt{\dfrac{27}{125}}\right)\left(\sqrt[3]{\dfrac{9}{25}}\right) =$
  \begin{malla}[height=6cm]
  \end{malla}
  \pregunta $\sqrt{\sqrt[4]{256}} =$
  \begin{malla}[height=5cm]
  \end{malla}

\end{preguntas}

\section{Álgebra}

Factorice las siguientes expresiones.
\begin{preguntas}
  \pregunta $x^2-49 =$
  \begin{malla}[height=3cm]
  \end{malla}
  \pregunta $4a^4-9b^2c^2 =$
  \begin{malla}[height=5cm]
  \end{malla}
  \pregunta $x^2 -\dfrac{16}{49} =$
  \begin{malla}[height=5cm]
  \end{malla}
\end{preguntas}

Encuentre las soluciones de cada ecuación.
\begin{preguntas}
  \pregunta $x^2-18-7x$
  \begin{malla}[height=5cm]
  \end{malla}
  \pregunta $3x^2-5x-2$
  \begin{malla}[height=6.5cm]
  \end{malla}
\end{preguntas}

Solucione el siguiente sistema de ecuaciones.
\begin{preguntas}
  \pregunta $\begin{cases}
    x = y-3 \\
    2y = 5+x
  \end{cases}$
  \begin{malla}[height=5cm]
  \end{malla}
\end{preguntas}

Plantee y encuentre la solución de los siguientes problemas.
\begin{preguntas}
  \pregunta La diferencia de dos números es 30 y $1/5$ de su suma es 26. Determina los
  números.
  \begin{malla}[height=7cm]
  \end{malla}
  \pregunta Carlos y Gabriel fueron al supermercado a comprar lo necesario para una
  reunión con amigos del colegio, llevaban un total de $\$500$ para gastar. Carlos gastó
  dos terceras partes de su dinero, mientras que Gabriel tres quintas partes, regresaron
  a casa con un total de $\$180$, ¿cuánto llevaba cada uno al ir al supermercado?
  \begin{malla}[height=7cm]
  \end{malla}
\end{preguntas}

\end{document}