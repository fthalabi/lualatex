\def\titulo{Funciones}
\def\subtitulo{Traslación y reflexion}
\def\colegio{Saint Rose School}

\documentclass[]{presentacion}

\begin{document}

\begin{frame}
\frametitle{Traslación}
\begin{tikzpicture}
\begin{axis}[
    eje escolar,
    % Declare the function f(x) as an option to the axis
    declare function={
        f(\x) = 6 + \x - 4*\x^2 + \x^3;
    },
    smooth
]
  % Plot f(x)

  \addplot[restrict y to domain=-5:10] {f(x)};
  %\addplot[restrict y to domain=-5:10] {f(x-2)};
  \addplot[restrict y to domain=-5:10,dotted] {f(x+3)};
  \legend{$f(x)$,$f(x+3)$};
\end{axis}
\end{tikzpicture}


\end{frame}

\begin{frame}
\frametitle{Traslación horizontal}
\centering
\begin{tikzpicture}
\begin{axis}[
    eje escolar,
    % Declare the function f(x) as an option to the axis
    declare function={
        f(\x) = 6 + \x - 4*\x^2 + \x^3;
    },
    smooth,
    no marks,
    xmin=-6, xmax=8,
    x post scale=1.4
    ]
  \addplot+[restrict y to domain=-5:10] {f(x)};
  \addplot+[update limits=false,domain=-2:10] {f(x-2)};
  \addplot+[update limits=false] {f(x+3)};

  \legend{$f(x)$,$f(x-2)$,$f(x+3)$};
\end{axis}
\end{tikzpicture}


\end{frame}


\begin{frame}
\frametitle{Traslación vertical}
\centering
\begin{tikzpicture}
\begin{axis}[
    eje escolar,
    % Declare the function f(x) as an option to the axis
    declare function={
        f(\x) = 6 + \x - 4*\x^2 + \x^3;
    },
    smooth,
    no marks,
    xmin=-2, xmax=4,
    ymin=-8, ymax=8,
    domain=-2:4,
    x post scale=1.4,
    ]
  \addplot+[] {f(x)};
  \addplot+[] {f(x)+2};
  \addplot+[] {f(x)-3};
  \legend{$f(x)$,$f(x)+2$,$f(x)-3$};
\end{axis}
\end{tikzpicture}
\end{frame}

\begin{frame}
\frametitle{Reflexión horizontal}
\centering
\begin{tikzpicture}
\begin{axis}[
    eje escolar,
    % Declare the function f(x) as an option to the axis
    declare function={
        f(\x) = 6 + \x - 4*\x^2 + \x^3;
    },
    smooth,
    no marks,
    xmin=-4, xmax=4,
    ymin=-8, ymax=8,
    domain=-4:4,
    x post scale=1.4,
    ]
  \addplot+[] {f(x)};
  \addplot+[update limits=false] {f(-x)};
  \legend{$f(x)$,$f(-x)$};
\end{axis}
\end{tikzpicture}
\end{frame}

\begin{frame}
\frametitle{Reflexión vertical}
\centering
\begin{tikzpicture}
\begin{axis}[
    eje escolar,
    % Declare the function f(x) as an option to the axis
    declare function={
        f(\x) = 6 + \x - 4*\x^2 + \x^3;
    },
    smooth,
    no marks,
    xmin=-2, xmax=4,
    ymin=-8, ymax=8,
    domain=-4:4,
    x post scale=1.4,
    ]
  \addplot+[] {f(x)};
  \addplot+[update limits=false] {-f(x)};
  \legend{$f(x)$,$-f(x)$};
\end{axis}
\end{tikzpicture}
\end{frame}

\begin{frame}

\frametitle{Reflexión asdasd}
% --- Declare the function f(x) for TikZ ---
% \tikzdeclarefunction{<name>}(<number of arguments>){<definition using #1, #2, ...>}
\tikzdeclarefunction{f}(1){exp(-(#1)^2);} % f(#1) = e^(-(#1)^2)


\begin{tikzpicture}
  \datavisualization [
    % --- General axis setup ---
    scientific axes, % or cartesian axes
    x axis={label={$x$}, min value=-6, max value=5, ticks={step=1}},
    y axis={label={$y$}, min value=0, max value=1.2, ticks={step=0.2}},
    all axes={grid}, % Add grid to all axes

    % --- Legend setup ---
    legend={
        style={draw=gray, fill=white!90!gray, font=\small}, % Style the legend box
        pos={north east}, % Position of the legend
        gap=2pt % Space between legend entries
    },

    % --- Styling for all lines (can be overridden per plot) ---
    visualize as line/.append style={thick} % Make all lines thick by default
  ]
  % --- Plot 1: f(x) ---
  data [format=function, samples=100] {
    var x : domain=-6:5; % Domain for this specific plot
    func y = f(x);       % Use the declared function f
  }
  visualize as line={
    style={blue}, % Color for this plot (and its legend marker)
    label in legend={text={\(f(x)\)}}
  }

  % --- Plot 2: f(x-2) ---
  data [format=function, samples=100] {
    var x : domain=-4:7; % Adjusted domain: original_domain + 2
    func y = f(x-2);     % Shifted function
  }
  visualize as line={
    style={red},  % Color for this plot (and its legend marker)
    label in legend={text={\(f(x-2)\)}}
  }

  % --- Plot 3: f(x+3) ---
  data [format=function, samples=100] {
    var x : domain=-9:2; % Adjusted domain: original_domain - 3
    func y = f(x+3);     % Shifted function
  }
  visualize as line={
    style={green!60!black}, % Color for this plot (and its legend marker)
    label in legend={text={\(f(x+3)\)}}
  }; % Semicolon after the last visualization in the datavisualization environment

\end{tikzpicture}

\end{frame}



\end{document}