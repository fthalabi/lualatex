\def\titulo{Probabilidad y Estadística}
\def\subtitulo{Técnicas de conteo}
\def\curso{Optativo de profundización}
\def\colegio{Saint Rose School}

\documentclass[]{presentacion}

\begin{document}

\begin{frame}

\frametitle{Permutaciones}

\begin{teorema}
  El \textbf{número de permutaciones} de \textbf{n} objetos es
  \begin{equation*}
    P_n^n = n!
  \end{equation*}
\end{teorema}

\begin{problema}
  Un inspector visita 6 máquinas diferentes durante el día. A fin
  de impedir a los operadores que sepan cuando inspeccionará, varía el orden
  de las visitas. ¿De cuántas maneras puede hacerlo?
\end{problema}



\end{frame}

\begin{frame}

\begin{problema}
  ¿De cuántas maneras se pueden colocar 10 chicas en una fila, de
  manera que dos chicas, en particular, no queden juntas?
\end{problema}

\begin{problema}
  ¿De cuántas maneras se pueden colocar 12 niños en una fila, de
  manera que cuatro niños, en particular, queden juntos?
\end{problema}

\end{frame}

\begin{frame}
  \begin{teorema}
    El \textbf{número de permutaciones} de \textbf{n} objetos diferentes en \textbf{r}
    posiciones distintas es
    \begin{equation*}
      P_n^r = \dfrac{n!}{(n-r)!}
    \end{equation*}
  \end{teorema}

  \begin{problema}
    Un grupo está formado por 5 personas y desean formar una comisión
    integrada por presidente y secretario. ¿De cuántas maneras puede nombrarse
    esta comisión?
  \end{problema}
\end{frame}

\begin{frame}
  \begin{problema}
    Encontrar el número total de enteros positivos que pueden formarse
    utilizando los dígitos \{1,2,3,4\}, si ningún dígito ha de repetirse cuando
    se forma un número.
  \end{problema}
\end{frame}

\begin{frame}[allowframebreaks]

\begin{problema}[1]
  En un ómnibus que posee 37 asientos en ocho filas de cuatro
  asientos cada una con un pasillo en el medio, y al final 5 asientos juntos,
  se desea ubicar 25 pasajeros.
  \begin{ejercicios*}
    * ¿De cuántas formas se pueden ubicar?
    * ¿De cuántas formas se pueden ubicar si deciden no ocupar los últimos 5 asientos?
    * ¿De cuántas formas se pueden ubicar si viajan cinco amigos que deciden
    viajar juntos en los últimos asientos?
  \end{ejercicios*}
\end{problema}

\end{frame}

\begin{frame}
\begin{continuacion}
    \begin{ejercicios*}[start=4]
      * ¿De cuántas formas se pueden ubicar si ocupan los 18 asientos que poseen
      ventanilla?
      * ¿De cuántas formas se pueden ubicar si 10 de los pasajeros están enfermos
      y deben viajar en asientos que poseen ventanilla?
    \end{ejercicios*}
\end{continuacion}
\end{frame}

\end{document}