\def\titulo{Guía}
\def\subtitulo{Función cuadrática}
\def\curso{Segundo medio}
\documentclass[sin nombre]{srs}
\begin{document}

\separador
\begin{preguntas}[after-item-skip=2cm]
\pregunta Las soluciones de la ecuación $2\left(x - 1\right)^{2} = 5$, están representadas en:
\begin{vertical}
\alternativa $1 \pm \dfrac{\sqrt{5}}{2}$
\alternativa $-1 \pm \dfrac{\sqrt{5}}{2}$
\alternativa $1 \pm \sqrt{\dfrac{5}{2}}$
\alternativa $-1 \pm \sqrt{\dfrac{5}{2}}$
\alternativa $\dfrac{1 \pm \sqrt{5}}{2}$
\end{vertical}

\pregunta ¿En cuál de las siguientes ecuaciones cuadráticas, las soluciones son reales e iguales?
\begin{vertical}
\alternativa $x^{2} - 4x = -1$
\alternativa $x^{2} - 2x = -4$
\alternativa $2x^{2}-9=0$
\alternativa $2x^{2} + x = 1$
\alternativa $4x^{2} + 4x =-1$
\end{vertical}

\pregunta ¿En cuál de las siguientes ecuaciones cuadráticas, las soluciones no son reales?
\begin{vertical}
\alternativa $x^{2} + x = 1$
\alternativa $x^{2} - 2x = 4$
\alternativa $2x^{2} - 5x = -2$
\alternativa $x^{2} + x = 2$
\alternativa $x^{2} + 4x = -8$
\end{vertical}

\pregunta Con respecto a las soluciones (o raíces) de la ecuación $x^{2} + 4x = 32$, ¿cuál(es) de las siguientes afirmaciones es (son) verdadera(s)?
\begin{verticali}
\alternativa Son racionales.
\alternativa Son positivas.
\alternativa Son números enteros.
\end{verticali}
\begin{vertical}
\alternativa Solo I
\alternativa Solo I y II
\alternativa Solo I y III
\alternativa Solo II y III
\alternativa I, II y III
\end{vertical}

\pregunta Si una de las soluciones de la ecuación en x, $3x^{2} + 5kx + 2 = 0$ es $-2$, entonces k =
\begin{vertical}
\alternativa $-1$
\alternativa $1$
\alternativa $\dfrac{7}{5}$
\alternativa $-\dfrac{7}{5}$
\alternativa $-\dfrac{1}{3}$
\end{vertical}

%\pregunta Si x es la solución de la ecuación $-x = \dfrac{3}{x-4}$, ¿cuál es el menor valor posible para la expresión: $\dfrac{3}{x-4}$?
%\begin{vertical}
%\alternativa $-4$
%\alternativa $-3$
%\alternativa $-1$
%\alternativa $1$
%\alternativa $3$
%\end{vertical}

\pregunta ¿Cuál(es) de las siguientes ecuaciones no tienen soluciones en los números reales?
\begin{verticali}
\alternativa $2\left(x - 2\right)^{2} + 3 = 0$
\alternativa $-\dfrac{3}{2}\left(x - 1\right)^{2} + 1 = 0$
\alternativa $2\left(x + \dfrac{1}{2}\right)^{2} + 5 = 0$
\end{verticali}
\begin{vertical}
\alternativa Solo I
\alternativa Solo II
\alternativa Solo I y II
\alternativa Solo I y III
\alternativa I, II y III
\end{vertical}

\pregunta ¿Cuál de las siguientes ecuaciones tiene como raíces (o soluciones) a $\left(2 + \sqrt{5}\right)$ y $\left(2 - \sqrt{5}\right)$?
\begin{vertical}
\alternativa $x^{2} - 4x + 9=0$
\alternativa $x^{2} + 4x + 9 =0$
\alternativa $x^{2} - 4x + 1 =0$
\alternativa $x^{2} - 4x - 1 = 0$
\alternativa $x^{2}-2x-1=0$
\end{vertical}

\pregunta ¿Cuál de las siguientes ecuaciones tiene raíces (o soluciones) $\left(a + b\right)$ y $\left(a - b\right)$?
\begin{vertical}
\alternativa $x^{2} + ax + a^{2} - b^{2} = 0$
\alternativa $x^{2} - ax + a^{2} - b^{2} = 0$
\alternativa $x^{2} + 2ax + a^{2} - b^{2} = 0$
\alternativa $x^{2}-2ax + a^{2} - b^{2} = 0$
\alternativa $x^{2}-2ax + a^{2} + b^{2} = 0$
\end{vertical}

\pregunta Con respecto a las soluciones de la ecuación $x + \dfrac{2}{x-1} = 4$, ¿cuál de las siguientes afirmaciones es verdadera?
\begin{vertical}
\alternativa Son reales de distinto signo.
\alternativa Son racionales positivas.
\alternativa No son reales.
\alternativa Son racionales negativas.
\alternativa Ninguna de ellas.
\end{vertical}

\pregunta Las soluciones de la ecuación en x, $2x^{2} - 4x + k = 0$ son reales y distintas, entonces:
\begin{vertical}
\alternativa $k > 2$
\alternativa $k < 2$
\alternativa $k \leq 2$
\alternativa $k < \dfrac{1}{2}$
\alternativa $k > 1$
\end{vertical}

\pregunta Las soluciones de la ecuación en x, $bx^{2} - bx + b + 1 = 0$, con $b \neq 0$, son reales e iguales, entonces $b =$
\begin{vertical}
\alternativa $-\dfrac{3}{4}$
\alternativa $\dfrac{3}{4}$
\alternativa $\dfrac{4}{3}$
\alternativa $-\dfrac{4}{3}$
\alternativa No existe tal valor de b.
\end{vertical}

\pregunta Dada la ecuación en x, $\left(k - 1\right)x^{2} + 2\left(k - 2\right)x + \left(k - 1\right) = 0$, ¿qué valor debe tomar k para que las raíces o soluciones sean reales e iguales?
\begin{vertical}
\alternativa $\dfrac{3}{2}$
\alternativa $-\dfrac{2}{3}$
\alternativa $\dfrac{3}{2}$
\alternativa $\dfrac{1}{2}$
\alternativa No existe tal valor de k.
\end{vertical}

\pregunta La ecuación en x, $\left(k - 2\right)x^{2} + 2\left(k - 4\right)x + k - 4 = 0$, con k un número real distinto de 2, tiene dos soluciones que no son números reales, entonces:
\begin{vertical}
\alternativa $k > 4$
\alternativa $k=4$
\alternativa $k < 4$
\alternativa $k > 2$
\alternativa $k < 2$
\end{vertical}

\pregunta Sea la ecuación cuadrática en x, $a\left(x - b\right)^{2} + b = c$, se puede determinar que las soluciones de esta ecuación son reales y distintas, sabiendo que:
\begin{verticaln}
\alternativa $c > b$
\alternativa $a\left(b - c\right) < 0$
\end{verticaln}
\begin{vertical}
\alternativa (1) por sí sola
\alternativa (2) por sí sola
\alternativa Ambas juntas, (1) y (2)
\alternativa Cada una por sí sola, (1) ó (2)
\alternativa Se requiere información adicional
\end{vertical}

\pregunta Dada la ecuación $x^{2} + 10x - 15 = 0$, ¿qué número real $p$ se debe sumar a ambos lados de la ecuación para completar el cuadrado de un binomio en el lado izquierdo de ella y cuáles son las soluciones de esta ecuación?
\begin{vertical}
\alternativa $p$ = 40 y las soluciones son $\left(-5 - \sqrt{115}\right)$ y $\left(- 5 + \sqrt{115}\right)$.
\alternativa $p$ = -10 y las soluciones son $\left(10 - \sqrt{5}\right)$ y $\left(10 + \sqrt{5}\right)$.
\alternativa $p$ = 40 y las soluciones son $\left(-5 - \sqrt{40}\right)$ y $\left(- 5 + \sqrt{40}\right)$.
\alternativa $p$ = -25 y las soluciones no son reales.
\alternativa $p$ = 25 y las soluciones no son reales.
\end{vertical}

\pregunta $a$ y $b$ son números reales, ¿cuál (es) de las siguientes ecuaciones en x, tiene(n) siempre solución(es) en el conjunto de los números reales?
\begin{verticali}
\alternativa $\left(x-b\right)^{2} - \dfrac{a}{b} = 0$, con $ab>0$.
\alternativa $ax^{2} + b = a$, con $a > b$.
\alternativa $ax^{2} + b = 0$, con $ab < 0$.
\end{verticali}
\begin{vertical}
\alternativa Solo I
\alternativa Solo II
\alternativa Solo I y II
\alternativa Solo I y III
\alternativa I, II y III
\end{vertical}

\pregunta El área de un rectángulo es 50 cm² y su perímetro es 30 cm. ¿Cuál de las siguientes ecuaciones permite determinar su largo “$x$”?
\begin{vertical}
\alternativa $x^{2} - 15x - 50 = 0$
\alternativa $x^{2} + 15x + 50 = 0$
\alternativa $x^{2} - 15x + 50 = 0$
\alternativa $x^{2} - 30x + 50 = 0$
\alternativa $x^{2} + 30x + 50 = 0$
\end{vertical}

\pregunta Se tienen tres números consecutivos donde el menor es "$x$". Si el doble del producto de los dos menores tiene 20 unidades más que el cuadrado del mayor, ¿cuál de las siguientes ecuaciones permite determinar el menor de los términos?
\begin{vertical}
\alternativa $2x\left(x + 1\right) + 20 = \left(x + 2\right)^{2}$
\alternativa $2x\left(x + 1\right) - 20 = \left(x + 2\right)^{2}$
\alternativa $2x\left(x + 1\right) = 20 + \left(x + 2\right)^{2}$
\alternativa $2x\left(x + 1\right) = 20 - \left(x + 2\right)^{2}$
\alternativa $2x\left(x + 1\right) = \left(20 - \left(x + 2\right)\right)^{2}$
\end{vertical}

\pregunta La edad de un hermano es el doble de la edad del otro más cuatro años. Si el producto de sus edades es 160, ¿cuál es la edad del mayor?
\begin{vertical}
\alternativa 8 años
\alternativa 10 años
\alternativa 16 años
\alternativa 20 años
\alternativa 24 años
\end{vertical}

\pregunta En un rectángulo, el largo mide 2 cm más que el ancho. Si los lados se aumentan en 2 cm, se forma un segundo rectángulo cuya área sumada con la del primero resulta 288 cm². ¿Cuánto mide el ancho del rectángulo original?
\begin{vertical}
\alternativa 8 cm
\alternativa 10 cm
\alternativa 12 cm
\alternativa 14 cm
\alternativa 18 cm
\end{vertical}

\pregunta Las aristas de un cubo disminuyen en 2 cm, disminuyendo el volumen del cubo en 296 cm³. ¿Cuánto medían inicialmente las aristas?
\begin{vertical}
\alternativa 4 cm
\alternativa 6 cm
\alternativa 8 cm
\alternativa 36 cm
\alternativa 48 cm
\end{vertical}

\pregunta Un número tiene dos cifras, tales que la de las decenas tiene una unidad más que el doble de la otra. Si al número se le suma el producto de las cifras resulta 94, entonces ¿cuál es la diferencia de las cifras?
\begin{vertical}
\alternativa 2
\alternativa 3
\alternativa 4
\alternativa 7
\alternativa 8
\end{vertical}

\pregunta Por el arriendo de una casa en la playa, a un grupo de amigos le cobran \$60 000 por el fin de semana. Para cancelar este valor lo dividieron en partes iguales, pero posteriormente dos de ellos no pudieron asistir por lo que la cuota tuvo que subir en \$1 500 para reunir el total del arriendo, entonces ¿cuántos amigos iban a ir al comienzo?
\begin{vertical}
\alternativa 7
\alternativa 8
\alternativa 10
\alternativa 12
\alternativa 15
\end{vertical}

\pregunta Un campesino ha plantado lechugas en filas, poniendo en cada una de ellas la misma cantidad, de modo que la cantidad de lechugas por fila supera en dos a la cantidad de filas. Al otro año decide aumentar en cuatro la cantidad de filas y disminuir en dos la cantidad de lechugas por fila. Si la cantidad de lechugas plantadas durante los dos años es 756, ¿cuántas fueron plantadas en cada fila en el primer año?
\begin{vertical}
\alternativa 20
\alternativa 22
\alternativa 24
\alternativa 25
\alternativa 26
\end{vertical}

\pregunta La gráfica de la función f definida en los reales mediante $f\left(x\right) = x^{2} + a$, pasa por el punto $\left(a, 2\right)$, entonces el (los) valor(es) de a es (son):
\begin{vertical}
\alternativa Solo 1
\alternativa Solo -1
\alternativa -2 o 1
\alternativa Solo -2
\alternativa No existen tales valores.
\end{vertical}

\pregunta Con respecto a la parábola de ecuación: $y = -x^{2} + 4x - 3$, se afirma que:
\begin{verticali}
\alternativa Intercepta al eje y en $\left(0,-3\right)$.
\alternativa Intercepta al eje x en dos puntos.
\alternativa Su vértice es el punto $\left(-2,-7\right)$.
\end{verticali}
¿Cuál(es) de las afirmaciones anteriores es (son) verdadera(s)?
\begin{vertical}
\alternativa Solo I
\alternativa Solo II
\alternativa Solo I y II
\alternativa Solo II y III
\alternativa I, II y III
\end{vertical}

\pregunta ¿Cuál de los siguientes gráficos representa mejor a la función cuadrática: $y = x^{2} - 6x + 9$?
\begin{alternativasgraficas}
\alternativa%
 \begin{tikzpicture}[declare function={f(\x)=(\x)^2;}]
  \datavisualization [school book axes, visualize as smooth line,
  all axes={length=3cm,ticks={major at={0}}},
  x axis={label=$x$,min value=-3,max value=3},
  y axis={label=$y$,min value=-0.5,max value=3},
  %f1={label in data={text=$L_1$, when=x is 5}},
  %f2={label in data={text=$L_2$, when=x is 1}}
  ]
  data [format=function] {
  var x : interval [-2.5:0.5];
  func y = f(\value x + 1) + 0.5;
  };
\end{tikzpicture}
\alternativa%
 \begin{tikzpicture}[declare function={f(\x)=(\x)^2;}]
  \datavisualization [school book axes, visualize as smooth line,
  all axes={length=3cm,ticks={major at={0}}},
  x axis={label=$x$,min value=-3,max value=3},
  y axis={label=$y$,min value=-0.5,max value=3},
  %f1={label in data={text=$L_1$, when=x is 5}},
  %f2={label in data={text=$L_2$, when=x is 1}}
  ]
  data [format=function] {
  var x : interval [-0.5:2.5];
  func y = f(\value x - 1) + 0.5;
  };
\end{tikzpicture}
\alternativa%
 \begin{tikzpicture}[declare function={f(\x)=(\x)^2;}]
  \datavisualization [school book axes, visualize as smooth line,
  all axes={length=3cm,ticks={major at={0}}},
  x axis={label=$x$,min value=-3,max value=3},
  y axis={label=$y$,min value=-0.5,max value=3},
  %f1={label in data={text=$L_1$, when=x is 5}},
  %f2={label in data={text=$L_2$, when=x is 1}}
  ]
  data [format=function] {
  var x : interval [-2.5:0.5];
  func y = f(\value x + 1);
  };
\end{tikzpicture}
\alternativa%
 \begin{tikzpicture}[declare function={f(\x)=(\x)^2;}]
  \datavisualization [school book axes, visualize as smooth line,
  all axes={length=3cm,ticks={major at={0}}},
  x axis={label=$x$,min value=-3,max value=3},
  y axis={label=$y$,min value=-0.5,max value=3},
  %f1={label in data={text=$L_1$, when=x is 5}},
  %f2={label in data={text=$L_2$, when=x is 1}}
  ]
  data [format=function] {
  var x : interval [-0.5:2.5];
  func y = f(\value x - 1);
  };
\end{tikzpicture}
\alternativa%
 \begin{tikzpicture}[declare function={f(\x)=(\x)^2;}]
  \datavisualization [school book axes, visualize as smooth line,
  all axes={length=3cm,ticks={major at={0}}},
  x axis={label=$x$,min value=-3,max value=3},
  y axis={label=$y$,min value=-0.5,max value=3},
  %f1={label in data={text=$L_1$, when=x is 5}},
  %f2={label in data={text=$L_2$, when=x is 1}}
  ]
  data [format=function] {
  var x : interval [-2:1];
  func y = -f(\value x + 0.5)+2;
  };
\end{tikzpicture}
\end{alternativasgraficas}

\pregunta Sea la función f definida en los reales, mediante $f\left(x\right) = -2\left(x - 3\right)\left(x - 5\right)$, entonces las coordenadas del vértice de la parábola asociada a su gráfica son:
\begin{vertical}
\alternativa $\left(4,-2\right)$
\alternativa $\left(4, 2\right)$
\alternativa $\left(4,-1\right)$
\alternativa $\left(4, 1\right)$
\alternativa $\left(2,-6\right)$
\end{vertical}

\pregunta ¿Cuál de los siguientes gráficos representa mejor a la función: $f\left(x\right) = \left(x + 2\right)^{2} + 1$?
\begin{alternativasgraficas}
\alternativa%
 \begin{tikzpicture}[declare function={f(\x)=(\x)^2;}]
  \datavisualization [school book axes, visualize as smooth line,
  all axes={length=3cm,ticks={major at={0}}},
  x axis={label=$x$,min value=-3,max value=3},
  y axis={label=$y$,min value=-0.5,max value=3},
  %f1={label in data={text=$L_1$, when=x is 5}},
  %f2={label in data={text=$L_2$, when=x is 1}}
  ]
  data [format=function] {
  var x : interval [-0.5:2.5];
  func y = f(\value x - 1) + 0.5;
  };
\end{tikzpicture}
\alternativa%
 \begin{tikzpicture}[declare function={f(\x)=(\x)^2;}]
  \datavisualization [school book axes, visualize as smooth line,
  all axes={length=3cm,ticks={major at={0}}},
  x axis={label=$x$,min value=-3,max value=3},
  y axis={label=$y$,min value=-0.5,max value=3},
  %f1={label in data={text=$L_1$, when=x is 5}},
  %f2={label in data={text=$L_2$, when=x is 1}}
  ]
  data [format=function] {
  var x : interval [-2.5:0.5];
  func y = f(\value x + 1) + 0.5;
  };
\end{tikzpicture}
\alternativa%
 \begin{tikzpicture}[declare function={f(\x)=(\x)^2;}]
  \datavisualization [school book axes, visualize as smooth line,
  all axes={length=3cm,ticks={major at={0}}},
  x axis={label=$x$,min value=-3,max value=3},
  y axis={label=$y$,min value=-3,max value=0.5},
  %f1={label in data={text=$L_1$, when=x is 5}},
  %f2={label in data={text=$L_2$, when=x is 1}}
  ]
  data [format=function] {
  var x : interval [-1.5:1.5];
  func y = -f(\value x) -0.5;
  };
\end{tikzpicture}
\alternativa%
 \begin{tikzpicture}[declare function={f(\x)=(\x)^2;}]
  \datavisualization [school book axes, visualize as smooth line,
  all axes={length=3cm,ticks={major at={0}}},
  x axis={label=$x$,min value=-3,max value=3},
  y axis={label=$y$,min value=-1,max value=2.5},
  %f1={label in data={text=$L_1$, when=x is 5}},
  %f2={label in data={text=$L_2$, when=x is 1}}
  ]
  data [format=function] {
  var x : interval [-1.5:1.5];
  func y = f(\value x) -0.8;
  };
\end{tikzpicture}
\alternativa%
 \begin{tikzpicture}[declare function={f(\x)=(\x)^2;}]
  \datavisualization [school book axes, visualize as smooth line,
  all axes={length=3cm,ticks={major at={0}}},
  x axis={label=$x$,min value=-3,max value=3},
  y axis={label=$y$,min value=-3,max value=0.5},
  %f1={label in data={text=$L_1$, when=x is 5}},
  %f2={label in data={text=$L_2$, when=x is 1}}
  ]
  data [format=function] {
  var x : interval [-0.5:2.5];
  func y = -f(\value x - 1);
  };
\end{tikzpicture}
\end{alternativasgraficas}

\pregunta ¿Cuál de las siguientes afirmaciones es FALSA con respecto a la función $f\left(x\right) = -\left(x^{2} + 4\right)$ si el dominio son todos los números reales?
\begin{vertical}
\alternativa La gráfica no intersecta al eje x.
\alternativa El vértice de la parábola asociada a esta función está en el eje y.
\alternativa El vértice de la parábola asociada a esta función está en el eje x.
\alternativa Su gráfica tiene al eje y como eje de simetría.
\alternativa El valor de x donde alcanza su máximo es $x = 0$.
\end{vertical}

\pregunta ¿Cuál de las siguientes funciones definidas en los reales, tiene como gráfico la parábola de la figura?
\begin{centrado}
 \begin{tikzpicture}[declare function={f(\x)=(\x^2)/2 -3*\x +4;}]
  \datavisualization [school book axes, visualize as smooth line,
  all axes={length=4cm},
  x axis={label=$x$,min value=-1,max value=7,ticks={major at={2,4}}},
  y axis={label=$y$,min value=-1,max value=6,ticks={major at={4}}},
  %f1={label in data={text=$L_1$, when=x is 5}},
  %f2={label in data={text=$L_2$, when=x is 1}}
  ]
  data [format=function] {
  var x : interval [-0.5:6.4];
  func y = f(\value x);
  };
\end{tikzpicture}
\end{centrado}
\begin{vertical}
\alternativa $g\left(x\right) = \left(x - 3\right)^{2} + 1$
\alternativa $h\left(x\right) = -\left(x - 3\right)^{2} - 1$
\alternativa $j\left(x\right) = \left(x - 3\right)^{2} + 2$
\alternativa $k\left(x\right) = 2\left(x - 2\right)\left(x - 4\right)$
\alternativa $m\left(x\right)=\dfrac{1}{2}\left(x-2\right)\left(x - 4\right)$
\end{vertical}

\pregunta ¿Cuál de las siguientes funciones definidas en los reales, tiene como recorrido los reales menores o iguales que $-1$?
\begin{vertical}
\alternativa $g\left(x\right) = \left(x - 3\right)^{2} - 1$
\alternativa $h\left(x\right) = -\left(x - 3\right)^{2} + 1$
\alternativa $j\left(x\right) = -\left(x - 1\right)^{2} + 2$
\alternativa $k\left(x\right) = -\left(x - 1\right)^{2} - 2$
\alternativa $t\left(x\right) = -\left(x - 4\right)^{2} - 1$
\end{vertical}

\pregunta Sea f una función cuyo dominio es el conjunto de los números reales, definida por $f\left(x\right) = a\left(x - 2\right)^{2} + 1$, con a un número real distinto de cero. ¿Cuál (es) de las siguientes afirmaciones es (son) verdadera(s)?
\begin{verticali}
\alternativa Si $a > 0$, el valor mínimo de f se alcanza para $x = 2$.
\alternativa Si $a < 0$, el recorrido de f es $]-\infty, 1]$.
\alternativa Si la gráfica pasa por el origen, entonces $a = -\dfrac{1}{4}$.
\end{verticali}
\begin{vertical}
\alternativa Solo I
\alternativa Solo II
\alternativa Solo I y II
\alternativa Solo II y III
\alternativa I, II y III
\end{vertical}

\pregunta Se puede determinar la función cuadrática, definida en los reales mediante $f\left(x\right) = ax^{2} + c$, sabiendo que:
\begin{verticaln}
\alternativa La gráfica asociada a esta función pasa por el punto $\left(1,\,4\right)$.
\alternativa Su mínimo es $y = 1$.
\end{verticaln}
\begin{vertical}
\alternativa (1) por sí sola
\alternativa (2) por sí sola
\alternativa Ambas juntas, (1) y (2)
\alternativa Cada una por sí sola, (1) ó (2)
\alternativa Se requiere información adicional
\end{vertical}

\pregunta ¿Cuál(es) de las siguientes afirmaciones es (son) verdadera(s), con respecto a las funciones de la forma $f\left(x\right) = \left(a - 1\right)x^{2} - a$ con dominio los números reales?
\begin{verticali}
\alternativa Si $a > 1$, entonces la gráfica de la función es una parábola que se abre hacia arriba.
\alternativa La gráfica de f intersecta al eje de las ordenadas en el punto $\left(0, -a\right)$.
\alternativa Si $a < 1$, entonces el mínimo de la función es $-a$.
\end{verticali}
\begin{vertical}
\alternativa Solo I
\alternativa Solo II
\alternativa Solo I y II
\alternativa Solo II y III
\alternativa I, II y III
\end{vertical}

\pregunta Sea f una función cuyo dominio es el conjunto de los números reales, definida por $f\left(x\right) = ax^{2} + \left(a + 2\right)x + 2$, con $a \neq 0$. ¿Cuál de las siguientes relaciones se debe cumplir, para que la gráfica de la función intersecte al eje x en un solo punto?
\begin{vertical}
\alternativa $a = -2$
\alternativa $a = 2$
\alternativa $a^{2}-4a + 4 > 0$
\alternativa $a^{2} - 4a + 4 < 0$
\alternativa $\dfrac{-\left(a + 2\right) + \sqrt{\left(a + 2\right)^{2} - 8}}{2a}$
\end{vertical}

\pregunta Sea $f$ una función definida en los reales mediante $f\left(x\right) = x^{2} - 4bx - 2$, con $b \neq 0$, entonces el valor de $x$ donde la función alcanza su valor mínimo es:
\begin{vertical}
\alternativa $2b$
\alternativa $-2b$
\alternativa $b$
\alternativa $4b^{2} + 2$
\alternativa $-4b^{2} - 2$
\end{vertical}

\pregunta ¿Cuál es el conjunto de todos los valores de a, para que la función definida por $f\left(x\right) = \left(x - a\right)^{2} + 4a$, intersecte al eje x en dos puntos?
\begin{vertical}
\alternativa $]0, \infty[$
\alternativa $]-\infty, 0[$
\alternativa $]-\infty, 0]$
\alternativa $[0, \infty[$
\alternativa $\emptyset$
\end{vertical}

\pregunta La gráfica de la función $f\left(x\right) = \left(a - 2\right)x^{2} + 2\left(a - 1\right)x + a - 1$, con $a \neq 2$ y dominio los números reales, intersecta en dos puntos al eje x, si:
\begin{vertical}
\alternativa $a < 1$
\alternativa $a = 1$
\alternativa $a > 1$
\alternativa $a > 2$
\alternativa $a < 2$
\end{vertical}

\pregunta Sea la función definida en los reales, mediante $f\left(x\right) = a\left(x - h\right)^{2} + k$, con $a \neq 0$. Se puede determinar el eje de simetría de la parábola que representa a la gráfica de esta función sabiendo que:
\begin{verticaln}
\alternativa $h = 3$.
\alternativa El vértice de la parábola es el punto $\left(3,2\right)$.
\end{verticaln}
\begin{vertical}
\alternativa (1) por sí sola
\alternativa (2) por sí sola
\alternativa Ambas juntas, (1) y (2)
\alternativa Cada una por sí sola, (1) ó (2)
\alternativa Se requiere información adicional
\end{vertical}

\pregunta Sea la función cuadrática $f\left(x\right) = x^{2} - ax - 2a^{2}$ con $a \neq 0$ y dominio el conjunto de los números reales. ¿Cuál (es) de las siguientes afirmaciones es (son) verdadera(s)?
\begin{verticali}
\alternativa La gráfica intercepta al eje x en dos puntos, para todo valor de $a$.
\alternativa El valor mínimo de la función es $-\dfrac{9a^{2}}{4}$.
\alternativa La gráfica asociada a esta función pasa por el punto $\left(-2a, -4a^{2}\right)$.
\end{verticali}
\begin{vertical}
\alternativa Solo I
\alternativa Solo II
\alternativa Solo I y II
\alternativa Solo II y III
\alternativa I, II y III
\end{vertical}

\pregunta ¿Cuál(es) de las siguientes afirmaciones es (son) siempre verdadera(s) con respecto a la función definida en los reales mediante $f\left(x\right) = ax^{2} + bx + c$, con $a \neq 0$?
\begin{verticali}
\alternativa Si $b = 0$, el mínimo es $y = c$.
\alternativa Si $c = 0$, uno de los ceros de la función es $x = - \dfrac{b}{a}$.
\alternativa Si $b=0$ y $c = 0$, entonces su gráfico intersecta a los ejes en el origen.
\end{verticali}
\begin{vertical}
\alternativa Solo I
\alternativa Solo II
\alternativa Solo I y II
\alternativa Solo II y III
\alternativa I, II y III
\end{vertical}

\pregunta ¿Cuál(es) de las siguientes afirmaciones es (son) siempre verdadera(s) con respecto a la función definida en los reales mediante $f\left(x\right) = \left(x - p\right)^{2}$?
\begin{verticali}
\alternativa El vértice de la parábola asociada a su gráfica está en el eje x.
\alternativa La ordenada del punto donde la gráfica intercepta al eje y es positiva.
\alternativa El eje de simetría de la gráfica es la recta de ecuación $x = p$.
\end{verticali}
\begin{vertical}
\alternativa Solo I
\alternativa Solo II
\alternativa Solo I y II
\alternativa Solo I y III
\alternativa I, II y III
\end{vertical}

\pregunta Sea f una función definida en los reales mediante $f\left(x\right) = x^{2} - ax + 6$, con $a \neq 0$. Si el valor de $x$ donde la función alcanza su valor mínimo es $-2$, entonces $a =$
\begin{vertical}
\alternativa $4$
\alternativa $-8$
\alternativa $-4$
\alternativa $4~ \text{ ó }~ -4$
\alternativa $-\sqrt{32} ~\text{ ó }~ \sqrt{32}$.
\end{vertical}

\pregunta La función $h\left(t\right) = pt - 5t^{2}$, modela la altura (en metros) que alcanza un proyectil al ser lanzado verticalmente hacia arriba a los $t$ segundos. Se puede determinar esta función si se sabe que:
\begin{verticaln}
\alternativa A los 2 segundos alcanza una altura de 30 metros.
\alternativa La altura máxima la alcanza a los 2,5 segundos.
\end{verticaln}
\begin{vertical}
\alternativa (1) por sí sola
\alternativa (2) por sí sola
\alternativa Ambas juntas, (1) y (2)
\alternativa Cada una por sí sola, (1) ó (2)
\alternativa Se requiere información adicional
\end{vertical}

\pregunta Las ganancias de una empresa, medidas en millones de dólares, se modelan según la función cuadrática $G\left(t\right) = - \frac{6}{32}\left(t-9\right)^{2} + 12$, donde $t$ es la cantidad de años desde que fue inaugurada. ¿Cuál de las siguientes afirmaciones es FALSA?
\begin{vertical}
\alternativa A los 9 años se obtuvo la máxima ganancia.
\alternativa Al primer año no obtuvo ganancia.
\alternativa A los 8 y a los 10 años obtuvo la misma ganancia.
\alternativa Después de los 9 años sus ganancias empezaron a disminuir.
\alternativa La ganancia anual siempre fue inferior a 12 millones de dólares.
\end{vertical}

\pregunta La altura $h\left(t\right)$ alcanzada, medida en metros, de un proyectil se modela mediante la función $h\left(t\right) = 20t - 5t^{2}$, donde t es la cantidad de segundos que transcurren hasta que alcanza dicha altura. ¿Cuál(es) de las siguientes afirmaciones es (son) verdadera(s)?
\begin{verticali}
\alternativa A los 4 segundos llega al suelo.
\alternativa A los 2 segundos alcanza su altura máxima.
\alternativa Al primer y tercer segundo después de ser lanzado alcanza la misma altura.
\end{verticali}
\begin{vertical}
\alternativa Solo I
\alternativa Solo II
\alternativa Solo I y II
\alternativa Solo II y III
\alternativa I, II y III
\end{vertical}

\pregunta Se puede determinar el valor numérico del máximo de la función cuadrática $f\left(x\right) = -x^{2} + 2ax - a$, si se conoce:
\begin{verticaln}
\alternativa El valor numérico de la abscisa del vértice de la parábola asociada a la gráfica de esta función.
\alternativa El valor numérico de uno de los ceros de esta función.
\end{verticaln}
\begin{vertical}
\alternativa (1) por sí sola
\alternativa (2) por sí sola
\alternativa Ambas juntas, (1) y (2)
\alternativa Cada una por sí sola, (1) ó (2)
\alternativa Se requiere información adicional
\end{vertical}

\end{preguntas}
\end{document}