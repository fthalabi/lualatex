\def\curso{Segundo medio A}
\def\puntaje{20}
\def\titulo{Prueba}
\def\subtitulo{Racionalización y estimación de raíces}
\def\fecha{14 de Abril, 2025}
\documentclass[]{srs}

\begin{document}

\section*{Objetivo}
  Realizar cálculos y estimar raíces usando procesos como la racionalización y/o el método
  de Héron.

\section*{Instrucciones generales}
  Tiene 1 hora y 30 minutos para responder la evaluación. Esta es individual y debe
  usar solo sus materiales personales para trabajar durante este periodo, no los solicite
  a un compañero durante la evaluación.

\section{Opciones múltiples}

\section*{Instrucciones}
Lee con atención y escoge la alternativa que responde la pregunta en cada enunciado.

\section*{Criterios de evaluación}
En la corrección de esta sección, se asignará 2 puntos al marcar la alternativa correcta.
Las alternativas corregidas serán consideradas incorrectas, es decir, marque solo una
alternativa por enunciado.

\separador[2mm]

\begin{preguntas}[after-item-skip=2cm]
  \pregunta Una expresión equivalente a $\sqrt{20}$ es: \\
  \begin{vertical}
    \alternativa $\sqrt{5}$
    \alternativa $2\sqrt{5}$
    \alternativa $3\sqrt{5}$
    \alternativa $4\sqrt{5}$
    \alternativa $2\sqrt{10}$

  \end{vertical}

  \pregunta Si $\sqrt{3}$ es aproximadamente 1,73205, entonces $\sqrt{0,12}$ aproximado
  por redondeo a la centésima es: \\
  \begin{vertical}
    \alternativa $0,03$
    \alternativa $0,04$
    \alternativa $0,34$
    \alternativa $0,35$
    \alternativa $0,36$
  \end{vertical}

  \pregunta Si $a=3\sqrt{2}$, $b=2\sqrt[3]{8}$, $c=2\sqrt{3}$, ¿en cuál de las siguientes
  alternativas se presentan en orden creciente?: \\
  \begin{vertical}
    \alternativa a, b, c
    \alternativa c, b, a
    \alternativa b, a, c
    \alternativa c, a, b
    \alternativa a, c, b
  \end{vertical}

  \pregunta Al reducir la expresión: $\sqrt{108} + \sqrt{27} - \sqrt{48} - 2\sqrt{12}$,
  ¿cuál es el resultado? \\
  \begin{vertical}
    \alternativa $\sqrt{30}$
    \alternativa $5\sqrt{3}$
    \alternativa $4\sqrt{3}$
    \alternativa $3\sqrt{3}$
    \alternativa $\sqrt{3}$
  \end{vertical}

  \pregunta Si $a=9$ y $b=18$, ¿cuál es el valor de la expresión
  $\left(\sqrt{a}+2\sqrt{b}\right)\left(\sqrt{a}-2\sqrt{b}\right)$?: \\
  \begin{vertical}
    \alternativa $-9$
    \alternativa $9$
    \alternativa $27$
    \alternativa $-27$
    \alternativa $-63$
  \end{vertical}

  \pregunta Al desarrollar la expresión:
  $\left(2\div\dfrac{\sqrt{5}}{\sqrt{35}}-\sqrt{7}-2\right)\cdot\sqrt{7} -1$,
  ¿cuál es su resultado? \\
  \begin{vertical}
    \alternativa $4\sqrt{7}$
    \alternativa $6-9\sqrt{35}$
    \alternativa $-6+9\sqrt{35}$
    \alternativa $6-2\sqrt{7}$
    \alternativa $2\sqrt{42}-2\sqrt{7} -8$
  \end{vertical}

  \pregunta ¿Cuál(es) de las siguientes afirmaciones es (son) verdadera(s)?: \\
  \begin{vertical*}
    \alternativa $\left(\sqrt{2}-3\right)^2 = 11$
    \alternativa $\dfrac{\sqrt{45}+\sqrt{125}}{\sqrt{20}} = 4$
    \alternativa $\sqrt{\sqrt{17}-1}\cdot\sqrt{\sqrt{17}+1} = 4$
  \end{vertical*}
  \begin{vertical}
    \alternativa Solo I.
    \alternativa Solo II.
    \alternativa Solo III.
    \alternativa Solo II y III.
    \alternativa I, II y III.
  \end{vertical}

\end{preguntas}

\section{Preguntas abiertas}

\section*{Instrucciones}
Al responder, incluya desarrollo en todas sus respuestas, y recuerde marcar o señalizar su
resultado final en cada pregunta.

\section*{Criterios de evaluación}
  En la corrección de esta sección, cada pregunta tiene 3 puntos y se asignará
  el puntaje de cada una según los siguientes criterios:
\begin{center}
  \begin{tblr}{width=\linewidth,colspec={X[1,c]|X[6]}, hline{1,Z} = {1}{-}{}, hline{1,Z} = {2}{-}{},
      hlines, cells={valign=m}, row{1} = {bg=black!15}}
      Puntaje asignado & \SetCell{c} Criterios o indicadores \\
      +50\% & Señala clara y correctamente cuál es la solución o el resultado de la pregunta hecha
      en el enunciado.\\
      +50\% & Incluye un desarrollo que relata de manera clara y ordenada los procedimientos
      \mbox{necesarios} para solucionar la problemática. En caso de estar incompleto o con
      errores el desarrollo, se asignará puntaje parcial si se muestra dominio de los
       contenidos y conceptos involucrados.\\
      0\% &  La respuesta es incorrecta. De haber desarrollo, este tiene errores conceptuales.\\
  \end{tblr}
\end{center}
\separador[2mm]

\begin{preguntas}(1)
  \pregunta Aproxime a la centésima la siguiente expresión: $-\sqrt{12}+2\sqrt{6}\approx$
  \begin{malla}[height=6cm]
  \end{malla}
  \pregunta Desarrolle y racionalice:
  $\left(3\sqrt{\dfrac{25}{2}}-5\sqrt{\dfrac{9}{16}}\right)\div\left(3\sqrt{2}\right)=$
  \begin{malla}[height=15cm]
  \end{malla}
\end{preguntas}





\end{document}