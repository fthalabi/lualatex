\def\titulo{Guía}
\def\subtitulo{Probabilidad}
\def\curso{Segundo medio}
\documentclass[sin nombre]{srs2}

\tcbset{raster before skip=10pt,raster after skip=30pt, raster row skip=30pt}

\begin{document}
\vspace*{20pt}
\begin{preguntas}
\pregunta ¿Cuál de los siguientes eventos es más probable al lanzar un dado?
\begin{alternativas}
\alternativa Que salga un número impar.
\alternativa Que salga un múltiplo de seis.
\alternativa Que salga un número divisor de seis.
\alternativa Que salga un número primo.
\alternativa Que salga un número menor que cuatro.
\end{alternativas}

\pregunta Al lanzar un dado, la probabilidad que el resultado sea par o primo es:
\begin{alternativas}
\alternativa $1$
\alternativa $\dfrac{2}{3}$
\alternativa $\dfrac{2}{5}$
\alternativa $\dfrac{3}{5}$
\alternativa $\dfrac{5}{6}$
\end{alternativas}

\pregunta Si se lanzan dos dados, ¿cuál es la probabilidad de que el primero sea un número par y el segundo sea un múltiplo de tres?
\begin{alternativas}
\alternativa $\dfrac{1}{3}$
\alternativa $\dfrac{1}{4}$
\alternativa $\dfrac{1}{6}$
\alternativa $\dfrac{5}{6}$
\alternativa $\dfrac{5}{12}$
\end{alternativas}

\pregunta Dos eventos son complementarios si su intersección es vacía y la unión es igual al espacio muestral. ¿Cuál de las siguientes pares de eventos son complementarios?
\begin{alternativas}
\alternativa Al lanzar un dado, “que salga un primo” y “que salga un número compuesto”.
\alternativa Elegir “un ausente” o “un presente” al elegir un estudiante al azar de la lista del curso.
\alternativa Al lanzar dos monedas, “que salga más de una cara” o “más de un sello”.
\alternativa Elegir una prueba corregida al azar y esta tenga una nota “mayor que 4” o “menor que 4”.
\alternativa Todos los anteriores.
\end{alternativas}

\pregunta La probabilidad de que un evento ocurra es p, ¿cuál es la probabilidad de que no ocurra en dos intentos seguidos?
\begin{alternativas}
\alternativa $1 - p^2$
\alternativa $\left(1 - p\right)^2$
\alternativa $1 + p^2$
\alternativa $p^2$
\alternativa $p^2 - 1$
\end{alternativas}

\pregunta Si se lanzan tres dados, ¿cuál es la probabilidad que la suma sea menor que 18?
\begin{alternativas}
\alternativa $\dfrac{1}{216}$
\alternativa $\dfrac{1}{18}$
\alternativa $\dfrac{215}{216}$
\alternativa $\dfrac{17}{18}$
\alternativa $1$
\end{alternativas}

\pregunta En una caja se tienen diez bolitas numeradas del 0 al 9. Si se extraen dos con reposición, ¿cuál es la probabilidad de que las bolitas extraídas sean impares e iguales?
\begin{alternativas}
\alternativa $\dfrac{1}{5}$
\alternativa $\dfrac{1}{10}$
\alternativa $\dfrac{5}{81}$
\alternativa $\dfrac{2}{9}$
\alternativa $\dfrac{1}{20}$
\end{alternativas}

\pregunta En una caja hay 7 bolitas numeradas del 1 al 7, si se saca una bolita al azar, ¿cuál es la probabilidad de que sea impar o mayor que 4?
\begin{alternativas}
\alternativa $\dfrac{3}{7}$
\alternativa $\dfrac{4}{7}$
\alternativa $\dfrac{5}{7}$
\alternativa $\dfrac{6}{7}$
\alternativa $1$
\end{alternativas}

\pregunta En un juego se tira un dado y una moneda y gana aquel que obtiene una cara y un seis, ¿cuál es la probabilidad de ganar en este juego?
\begin{alternativas}
\alternativa $\dfrac{1}{12}$
\alternativa $\dfrac{2}{12}$
\alternativa $\dfrac{8}{12}$
\alternativa $\dfrac{1}{8}$
\alternativa $\dfrac{2}{8}$
\end{alternativas}

\pregunta En una tómbola hay 15 bolitas negras y 15 rojas, cada color numeradas del 1 al 15, ¿cuál es la probabilidad de que al elegir una bolita salga el número 13 o salga de color negro?
\begin{alternativas}
\alternativa $\dfrac{14}{30}$
\alternativa $\dfrac{15}{30}$
\alternativa $\dfrac{16}{30}$
\alternativa $\dfrac{17}{30}$
\alternativa $\dfrac{18}{30}$
\end{alternativas}

\pregunta La distribución por género en los dos cursos de primero medio de un colegio, se muestra en la siguiente tabla:
Si de estos alumnos(as) se elige un(a) alumno(a) al azar, ¿cuál(es) de las siguientes afirmaciones es (son) verdadera(s)?
\begin{opciones}
\opcion La probabilidad de que sea mujer es $\dfrac{22}{60}$
\opcion La probabilidad de que sea del 1°A es $\dfrac{1}{2}$
\opcion La probabilidad de que sea un estudiante varón del 1°B es $\dfrac{18}{30}$
\end{opciones}
\begin{columnas}[0.6]
\begin{alternativas}
\alternativa Solo I
\alternativa Solo II
\alternativa Solo I y II
\alternativa Solo II y III
\alternativa I, II y III
\end{alternativas}
\siguiente
\begin{tblr}{|c|c|c|}
\hline
& hombres & mujeres \\
\hline
1°A & 20 & 10 \\
1°B & 18 & 12 \\
\hline
\end{tblr}
\end{columnas}

\pregunta Se lanzan 4 dados, ¿cuál es la probabilidad de que tres marquen un número par y uno marque un número impar?
\begin{alternativas}
\alternativa $\dfrac{1}{16}$
\alternativa $\dfrac{1}{8}$
\alternativa $\dfrac{1}{4}$
\alternativa $\dfrac{3}{16}$
\alternativa $\dfrac{3}{8}$
\end{alternativas}

\pregunta Se tienen dos cajas con bolitas negras y blancas. La primera tiene 3 blancas y 4 negras y la segunda dos blancas y 6 negras. Si se elige una bolita de cada caja (cada caja tiene la misma probabilidad de ser elegida), ¿cuál es la probabilidad de que por lo menos una de las bolitas sea negra?
\begin{alternativas}
\alternativa $\dfrac{3}{7}$
\alternativa $\dfrac{25}{28}$
\alternativa $\dfrac{19}{28}$
\alternativa $\dfrac{9}{28}$
\alternativa $\dfrac{3}{28}$
\end{alternativas}

\pregunta En una caja hay solo bolitas de colores verdes, rojas y blancas todas del mismo tipo. Se puede determinar la probabilidad de extraer una bolita blanca o una roja sabiendo que:
\begin{opciones*}
\opcion La probabilidad de sacar una bolita blanca es un $40\%$.
\opcion La probabilidad de sacar una bolita verde es el doble de la probabilidad de sacar una roja.
\end{opciones*}
\begin{alternativas}
\alternativa (1) por sí sola
\alternativa (2) por sí sola
\alternativa Ambas juntas, (1) y (2)
\alternativa Cada una por sí sola, (1) ó (2)
\alternativa Se requiere información adicional
\end{alternativas}

\pregunta Se tienen tres cajas cada una con tres bolitas de colores: verde, rojo y amarillo. Si se extrae una bolita de cada caja, ¿cuál es la probabilidad de que sean de distinto color?
\begin{alternativas}
\alternativa $\dfrac{1}{27}$
\alternativa $\dfrac{1}{9}$
\alternativa $\dfrac{2}{9}$
\alternativa $\dfrac{1}{3}$
\alternativa $\dfrac{2}{3}$
\end{alternativas}

\pregunta Un portal de ventas de automóviles de internet ofrece $200$ vehículos entre camionetas y autos, de los cuales hay nuevos y usados. Se sabe que, los autos nuevos son $40$, el total de camionetas $98$ y los vehículos usados corresponden al $55\%$. Si se elige un vehículo al azar, ¿cuál(es) de las siguientes afirmaciones es (son) verdadera(s)?
\begin{opciones}
\opcion La probabilidad de que sea una camioneta nueva es $0,25$.
\opcion La probabilidad de que sea una camioneta usada es $0,24$.
\opcion La probabilidad de que sea un auto o un vehículo nuevo es $0,96$.
\end{opciones}
\begin{alternativas}
\alternativa Solo I
\alternativa Solo II
\alternativa Solo I y II
\alternativa Solo I y III
\alternativa I, II y III
\end{alternativas}

\pregunta Se ha lanzado un dado dos veces, ¿cuál(es) de las siguientes afirmaciones es (son) verdadera(s)?
\begin{opciones}
\opcion La probabilidad de que la suma sea $8$ es $\dfrac{5}{36}$
\opcion La probabilidad de que las dos veces salga el mismo número es $\dfrac{1}{6}$
\opcion La probabilidad de que el número que salga la primera vez sea a lo sumo igual al número que salga la segunda vez es $\dfrac{15}{36}$
\end{opciones}
\begin{alternativas}
\alternativa Solo I
\alternativa Solo II
\alternativa Solo I y II
\alternativa Solo I y III
\alternativa I, II y III
\end{alternativas}

\pregunta Se lanzan dos dados y se suman los números obtenidos, ¿cuál(es) de las siguientes afirmaciones es (son) verdadera(s)?
\begin{opciones}
\opcion La probabilidad de que la suma sea $5$ es igual a la probabilidad de que la suma sea $9$.
\opcion $7$ es la suma con mayor probabilidad.
\opcion La probabilidad de que la suma sea a lo más $6$ es igual a la probabilidad de que la suma sea mayor que $7$.
\end{opciones}
\begin{alternativas}
\alternativa Solo I
\alternativa Solo II
\alternativa Solo I y II
\alternativa Solo II y III
\alternativa I, II y III
\end{alternativas}

\pregunta En una caja hay 4 bolitas blancas y 2 negras y en otra hay 4 bolitas blancas y 6 negras. Si se saca una bolita de cada caja, ¿cuál es la probabilidad de sacar por lo menos una bolita blanca?
\begin{alternativas}
\alternativa $\dfrac{1}{5}$
\alternativa $\dfrac{2}{3}$
\alternativa $\dfrac{4}{5}$
\alternativa $\dfrac{4}{15}$
\alternativa $\dfrac{11}{15}$
\end{alternativas}

\pregunta En una caja hay bolitas marcadas con los números del 1 al 4. En el siguiente gráfico se muestra la frecuencia relativa de algunos de estos números. ¿Cuál(es) de las siguientes afirmaciones es (son) verdadera(s)?
\begin{opciones}
\opcion La probabilidad de sacar una bolita marcada con el 1 o el 4 es igual a la probabilidad de sacar una bolita marcada con el 2 o el 3.
\opcion La probabilidad de sacar un Impar tiene la misma probabilidad que sacar un 4.
\opcion La probabilidad de sacar a lo más un 2 es igual a la probabilidad de sacar un 3.
\end{opciones}
\begin{columnas}[0.5]
\begin{alternativas}
\alternativa Solo I
\alternativa Solo II
\alternativa Solo I y II
\alternativa Solo I y III
\alternativa I, II y III
\end{alternativas}
\siguiente
\begin{tikzpicture}
\begin{axis}[footnotesize,eje escolar, xtick={1,2,3},ytick={0.1, 0.2, 0.3},
  enlarge x limits=0.4, enlarge y limits=0.2,xlabel=Número,ylabel=Frecuencia relativa]
\addplot+ [ybar,pattern] coordinates {(1, 0.1) (2, 0.2) (3, 0.3)};
\end{axis}
\end{tikzpicture}
\end{columnas}

\pregunta En una cartera, la señora Eugenia tiene 3 monedas, una de $\$50$, una de $\$100$ y una de $\$500$. Si saca dos al azar, ¿cuál es la probabilidad de que saque más de $\$150$?
\begin{alternativas}
\alternativa $\dfrac{1}{3}$
\alternativa $\dfrac{2}{3}$
\alternativa $\dfrac{1}{2}$
\alternativa $\dfrac{1}{4}$
\alternativa $\dfrac{3}{4}$
\end{alternativas}

\pregunta En una caja hay 3 bolitas blancas y 2 rojas, en otra caja hay 2 blancas y 4 rojas. Si se saca una bolita de cada caja, ¿cuál es probabilidad de que sean del mismo color?
\begin{alternativas}
\alternativa $\dfrac{1}{5}$
\alternativa $\dfrac{4}{15}$
\alternativa $\dfrac{7}{15}$
\alternativa $\dfrac{8}{15}$
\alternativa $\dfrac{14}{15}$
\end{alternativas}

\pregunta Un dado cargado es tal que la probabilidad de que salga un número par y múltiplo de tres es $0,1$; que salga un impar y divisor de seis es $0,2$; y que salga un número par y divisor de $12$ es $0,6$, ¿cuál es la probabilidad de que salga el número $5$?
\begin{alternativas}
\alternativa $0,1$
\alternativa $0,2$
\alternativa $0,25$
\alternativa $0,3$
\alternativa $0,4$
\end{alternativas}

\pregunta Sean A y B dos eventos, se puede determinar la probabilidad de que ocurra B, sabiendo:
\begin{opciones*}
\opcion La probabilidad de que ocurra A o B.
\opcion La probabilidad de que ocurra A y no B.
\end{opciones*}
\begin{alternativas}
\alternativa (1) por sí sola
\alternativa (2) por sí sola
\alternativa Ambas juntas, (1) y (2)
\alternativa Cada una por sí sola, (1) ó (2)
\alternativa Se requiere información adicional
\end{alternativas}


\pregunta Un jardín infantil tiene 43 bebés en sala cuna, de los cuales 23 son de género femenino. Si 9 de los bebés se enfermaron en el invierno pasado, de los cuales 4 eran niñitos, ¿cuál es la probabilidad de que si elige un bebé al azar este sea de género femenino o se enfermó en invierno?
\begin{alternativas}
\alternativa $\dfrac{32}{43}$
\alternativa $\dfrac{27}{43}$
\alternativa $\dfrac{9}{43} \cdot \dfrac{27}{43}$
\alternativa $\dfrac{1}{43}$
\alternativa $\dfrac{5}{43}$
\end{alternativas}

\pregunta En una caja hay bolitas rojas, verdes y amarillas. Si se saca una bolita al azar, se sabe que la probabilidad de sacar una que no sea amarilla es $\dfrac{2}{3}$, la probabilidad de sacar una roja o amarilla es $\dfrac{4}{5}$, ¿cuál es la probabilidad de sacar una bolita roja?
\begin{alternativas}
\alternativa $\dfrac{1}{5}$
\alternativa $\dfrac{1}{3}$
\alternativa $\dfrac{7}{15}$
\alternativa $\dfrac{8}{15}$
\alternativa $\dfrac{3}{4}$
\end{alternativas}

\pregunta En una tómbola hay bolitas marcadas con números enteros positivos, se sabe que hay 8 bolitas amarillas y 12 rojas, de las amarillas hay 5 marcadas con números impares y de las rojas hay 4 marcadas con números pares. Si se elige una bolita al azar, ¿cuál es la probabilidad de que sea amarilla o par?
\begin{alternativas}
\alternativa $\dfrac{3}{4}$
\alternativa $\dfrac{1}{5}$
\alternativa $\dfrac{1}{2}$
\alternativa $\dfrac{3}{20}$
\alternativa $\dfrac{3}{5}$
\end{alternativas}

\pregunta Un plantel de fútbol está formado por 3 arqueros, 6 defensas, 8 mediocampistas y 4 delanteros. Si se eligen dos jugadores al azar, ¿cuál(es) de las siguientes afirmaciones es (son) verdadera(s)?
\begin{opciones}
\opcion La probabilidad de que los dos sean delanteros es $\dfrac{1}{35}$
\opcion La probabilidad de que uno sea delantero y el otro arquero es $\dfrac{1}{35}$
\opcion La probabilidad de que ninguno sea defensa es $\dfrac{1}{2}$
\end{opciones}
\begin{alternativas}
\alternativa Solo I
\alternativa Solo III
\alternativa Solo I y II
\alternativa Solo I y III
\alternativa I, II y III
\end{alternativas}

\pregunta Se lanza una moneda 4 veces, ¿cuál de las siguientes afirmaciones NO es verdadera?
\begin{alternativas}
\alternativa La probabilidad de que salgan a lo menos 2 caras es $\dfrac{11}{16}$
\alternativa La probabilidad de que salgan a lo más 3 sellos es $\dfrac{15}{16}$
\alternativa La probabilidad de que salgan tantas caras como sellos es $\dfrac{3}{8}$
\alternativa La probabilidad de que salga solo 1 sello es menor a la probabilidad de que salgan por lo menos 3 sellos.
\alternativa La probabilidad de que salgan menos de 2 sellos es igual a la probabilidad de que salgan a lo más 2 caras.
\end{alternativas}

\pregunta Una moneda está cargada de tal forma que es cuatro veces más probable que se obtenga una cara que un sello. Si la moneda se lanza dos veces, ¿cuál es la probabilidad de NO obtener dos caras?
\begin{alternativas}
\alternativa $\dfrac{1}{16}$
\alternativa $\dfrac{7}{16}$
\alternativa $\dfrac{1}{25}$
\alternativa $\dfrac{8}{25}$
\alternativa $\dfrac{9}{25}$
\end{alternativas}

\pregunta En un comedor hay 4 ampolletas, las cuales pueden estar encendidas o apagadas independientemente unas de otras y para cada ampolleta la probabilidad de estar prendida es igual a la probabilidad de estar apagada. Si un niño juega con las 4 luces prendiéndolas y apagándolas, ¿cuál es la probabilidad de que queden dos encendidas y dos apagadas?
\begin{alternativas}
\alternativa $\dfrac{1}{16}$
\alternativa $\dfrac{1}{4}$
\alternativa $\dfrac{3}{8}$
\alternativa $\dfrac{3}{4}$
\alternativa $\dfrac{1}{2}$
\end{alternativas}

\pregunta Se lanzan dos dados y una moneda, ¿cuál es probabilidad de que en la moneda salga sello y que los dados sumen 6?
\begin{alternativas}
\alternativa $\dfrac{1}{18}$
\alternativa $\dfrac{1}{24}$
\alternativa $\dfrac{7}{12}$
\alternativa $\dfrac{5}{72}$
\alternativa $\dfrac{32}{36}$ % This seems like a typo, possibly 32/36 or 5/36
\end{alternativas}

\pregunta Los alumnos de un cierto colegio deben optar por la asignatura de Artes o Música, pudiendo elegir una de ellas, ambas o ninguna. Si se elige un alumno al azar, se puede determinar la probabilidad de que elija Música y no Arte, sabiendo:
\begin{opciones*}
\opcion La probabilidad de que elija Arte o Música es $0,8$
\opcion La probabilidad de que elija Arte y no Música $0,5$
\end{opciones*}
\begin{alternativas}
\alternativa (1) por sí sola
\alternativa (2) por sí sola
\alternativa Ambas juntas, (1) y (2)
\alternativa Cada una por sí sola, (1) ó (2)
\alternativa Se requiere información adicional
\end{alternativas}

\pregunta Una moneda está cargada de modo que la probabilidad de que salga cara es $0,6$, si se lanza dos veces, ¿cuál es la probabilidad de que salga una cara y un sello?
\begin{alternativas}
\alternativa $0,24$
\alternativa $0,25$
\alternativa $0,48$
\alternativa $0,5$
\alternativa $0,52$
\end{alternativas}

\pregunta Se toma una muestra a las máquinas A y B para estudiar su efectividad en la producción de ciertos artículos. Los resultados se muestran en la siguiente tabla:
¿Cuál(es) de las siguientes afirmaciones es (son) verdadera(s)?
\begin{opciones}
\opcion Es más probable hallar un artículo fallado en la muestra de B que en la muestra de A.
\opcion Si se elige un artículo al azar la probabilidad de que esté fallado y que haya sido producido por A es $\dfrac{1}{20}$
\opcion Si se elige un artículo al azar la probabilidad de que esté fallado o sea producido por B es $\dfrac{156}{270}$
\end{opciones}
\begin{columnas}[0.5]
\begin{alternativas}
\alternativa Solo I
\alternativa Solo II
\alternativa Solo I y III
\alternativa Solo II y III
\alternativa I, II y III
\end{alternativas}
\siguiente
\begin{tblr}{|c|c|c|}
\hline
Máquina & No fallados & fallados \\
\hline
A & 114 & 6 \\
B & 142 & 8 \\
\hline
\end{tblr}
\end{columnas}
\pregunta La probabilidad de que salga cara en una moneda cargada es el triple de que salga sello. Si se lanza la moneda $800$ veces, por la Ley de los Grandes Números, la cantidad de caras que deberían salir es cercano a:
\begin{alternativas}
\alternativa $200$
\alternativa $250$
\alternativa $300$
\alternativa $400$
\alternativa $600$
\end{alternativas}

\pregunta Se toma un test de 4 preguntas de verdadero o falso a $1\,600$ personas y cada una de ellas las contesta al azar. La Ley de los Grandes Números permite afirmar que:
\begin{alternativas}
\alternativa en un grupo de $400$ personas, hay $100$ que tienen $3$ buenas.
\alternativa en un grupo de $500$ personas hay aproximadamente $100$ obtienen $2$ preguntas buenas.
\alternativa aproximadamente el $20\%$ de las personas obtuvo $4$ preguntas buenas.
\alternativa aproximadamente el $25\%$ obtuvo una pregunta buena.
\alternativa aproximadamente un $40\%$ obtuvo más de $2$ preguntas buenas.
\end{alternativas}

\pregunta Se lanzan $12\,000$ veces dos dados no cargados, entonces según la Ley de los Grandes Números, ¿cuál de las siguientes afirmaciones es verdadera?
\begin{alternativas}
\alternativa $1\,000$ veces la suma saldrá un $4$.
\alternativa aproximadamente, $6\,000$ veces la suma será a lo sumo igual a $6$.
\alternativa aproximadamente, $2\,000$ veces la suma será a lo menos $10$.
\alternativa por cada $1\,200$ veces, $200$ veces la suma será un $7$.
\alternativa por cada $1\,100$ veces aproximadamente, en $1\,000$ la suma será $8$.
\end{alternativas}

\pregunta Según la Ley de los Grandes Números, ¿cuál(es) de las siguientes afirmaciones es (son) verdadera(s)?
\begin{opciones}
\opcion Si se lanzan tres monedas $8\,000$ veces en $1\,000$ de ellas saldrá solo una cara.
\opcion Si se lanzan dos monedas $18\,000$ veces, en la mitad de las veces saldrá solo una cara.
\opcion Si se lanzan $4$ monedas $16\,000$ veces, en un $25\%$ de las veces saldrán solo $3$ caras.
\end{opciones}
\begin{alternativas}
\alternativa Solo I
\alternativa Solo II
\alternativa Solo III
\alternativa Solo II y III
\alternativa Ninguna de ellas.
\end{alternativas}

\pregunta En la tabla adjunta se muestra la distribución por color de las bolitas que se encuentran en una caja. Si los colores que aparecen en la tabla son los únicos que tienen las bolitas, las bolitas son de igual tamaño y tienen un solo color, ¿cuál es la probabilidad de elegir una bolita que sea roja o blanca?
\begin{centrado}
\begin{tblr}{|l|c|c|c|}
\hline
color & Frecuencia & Frecuencia acumulada & Frecuencia relativa \\
\hline
rojo & & & 0,2 \\
verde & 3 & & 0,15 \\
azul & & 12 & \\
blanco & & & \\
\hline
\end{tblr}
\end{centrado}
\begin{alternativas}
\alternativa $0,2$
\alternativa $0,24$
\alternativa $0,25$
\alternativa $0,3$
\alternativa $0,6$
\end{alternativas}

\pregunta Se sabe que el $40\%$ de la población fuma y el $10\%$ fuma y es hipertensa. ¿Cuál es la probabilidad de que un fumador sea hipertenso?
\begin{alternativas}
\alternativa $4\%$
\alternativa $10\%$
\alternativa $25\%$
\alternativa $40\%$
\alternativa $50\%$
\end{alternativas}

\pregunta En una pequeña empresa hay $20$ funcionarios que se dividen en administrativos y vendedores. Se sabe que hay $6$ mujeres de las cuales hay $2$ administrativos y el total de vendedores es $12$. Si se elige una persona al azar de esta empresa, ¿cuál(es) de las siguientes afirmaciones es (son) verdadera?
\begin{opciones}
\opcion La probabilidad de que sea una mujer vendedora es $0,2$.
\opcion La probabilidad de que sea un administrativo de cualquier género o de género femenino es $0,7$.
\opcion La probabilidad de elegir un administrativo de cualquier género es $0,4$.
\end{opciones}
\begin{alternativas}
\alternativa Solo I
\alternativa Solo III
\alternativa Solo I y II
\alternativa Solo I y III
\alternativa I, II y III
\end{alternativas}

\pregunta En una caja hay solo bolitas de colores verdes, rojas y blancas todas del mismo tipo. Se puede determinar la probabilidad de extraer una bolita blanca, sabiendo:
\begin{opciones*}
\opcion La probabilidad de sacar una bolita verde o blanca es $0,65$.
\opcion La probabilidad de sacar una bolita roja o blanca es $0,55$.
\end{opciones*}
\begin{alternativas}
\alternativa (1) por sí sola
\alternativa (2) por sí sola
\alternativa Ambas juntas, (1) y (2)
\alternativa Cada una por sí sola, (1) ó (2)
\alternativa Se requiere información adicional
\end{alternativas}

\pregunta Se lanza una moneda no cargada 5 veces, ¿cuál(es) de las siguientes afirmaciones es (son) verdadera(s)?
\begin{opciones}
\opcion La probabilidad de obtener más de tres sellos es igual a la probabilidad de obtener menos de 2 sellos.
\opcion La probabilidad de obtener más de 4 caras es igual a la probabilidad de obtener más de 4 sellos.
\opcion La probabilidad de obtener a lo sumo 3 caras es igual a la probabilidad de obtener menos de 3 sellos.
\end{opciones}
\begin{alternativas}
\alternativa Solo I
\alternativa Solo II
\alternativa Solo I y II
\alternativa Solo I y III
\alternativa I, II y III
\end{alternativas}

\pregunta En una oficina trabajan 8 hombres y 6 mujeres. Si se va a elegir una comisión formada por tres integrantes, ¿cuál es la probabilidad de que esté integrada por 2 hombres y una mujer?
\begin{alternativas}
\alternativa $\dfrac{2}{13}$
\alternativa $\dfrac{6}{13}$
\alternativa $\dfrac{8}{13}$
\alternativa $\dfrac{6}{49}$
\alternativa $\dfrac{144}{243}$
\end{alternativas}

\pregunta En una caja hay solo bolitas negras y blancas todas del mismo tipo, numeradas con números enteros. Se puede determinar la probabilidad de extraer una bolita par y negra, sabiendo:
\begin{opciones*}
\opcion La probabilidad de sacar una bolita blanca es $\dfrac{5}{9}$
\opcion La probabilidad de sacar una negra e impar es $\dfrac{1}{3}$
\end{opciones*}
\begin{alternativas}
\alternativa (1) por sí sola
\alternativa (2) por sí sola
\alternativa Ambas juntas, (1) y (2)
\alternativa Cada una por sí sola, (1) ó (2)
\alternativa Se requiere información adicional
\end{alternativas}

\end{preguntas}

\end{document}