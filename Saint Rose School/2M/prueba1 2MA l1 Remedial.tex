\def\curso{Segundo medio A}
\def\puntaje{16}
\def\titulo{Prueba recuperativa}
\def\subtitulo{Racionalización y estimación de raíces}
\def\fecha{22 de abril, 2025}
\documentclass[]{srs}

\begin{document}

\subsection*{Objetivo}
  Realizar cálculos utilizando propiedades de raíces y/o racionalización.

\subsection*{Instrucciones generales}
  Tiene 40 minutos para responder la evaluación. Esta es individual y debe
  usar solo sus materiales personales para trabajar durante este periodo, no los solicite
  a un compañero durante la evaluación.

  Lee con atención y escoge la alternativa que responde la pregunta en cada enunciado.

\subsection*{Criterios de evaluación}
En la corrección de esta sección, se asignará 2 puntos al marcar la alternativa correcta.
Las alternativas corregidas serán consideradas incorrectas, es decir, marque solo una
alternativa por enunciado.

\separador[2mm]

\begin{preguntas}[after-item-skip=3cm]

\pregunta \(\dfrac{\sqrt{20} + \sqrt{45}}{\sqrt{5}} =\)
\begin{vertical}
\alternativa \(5\)
\alternativa \(7\)
\alternativa \(\sqrt{5}\)
\alternativa \(\sqrt{13}\)
\alternativa \(2 + 3\sqrt{5}\)
\end{vertical}

\pregunta Si \(x = \dfrac{1}{2\sqrt{3}}\), \(y = \dfrac{\sqrt{7}}{3}\), \(z = \dfrac{\sqrt{10}}{4}\), \(w = \dfrac{\sqrt{18}}{5}\), entonces:
\begin{vertical}
\alternativa \(z<x<w<y\)
\alternativa \(z<w<y<x\)
\alternativa \(z<w<x<y\)
\alternativa \(w<z<x<y\)
\alternativa \(y<x<w<z\)
\end{vertical}

\pregunta Si \(a > 0\), entonces \(\dfrac{\sqrt{a^2}}{\sqrt{a}} =\)
\begin{vertical}
\alternativa \(\sqrt[3]{a^2}\)
\alternativa \(\sqrt{a^3}\)
\alternativa \(\sqrt{a}\)
\alternativa \(\sqrt[3]{a}\)
\alternativa \(\sqrt[6]{a}\)
\end{vertical}

\pregunta Si \(x > 0\), entonces \(\sqrt{\sqrt{x+9}+\sqrt{x}} \cdot \sqrt{\sqrt{x+9}-\sqrt{x}} =\)
\begin{vertical}
\alternativa 3
\alternativa 9
\alternativa \(\sqrt{3}\)
\alternativa \(2\sqrt{3}\)
\alternativa \(3\sqrt{3}\)
\end{vertical}

\pregunta Sean los números: \(x = \sqrt{3} - \sqrt{2}\) ; \(y = \sqrt{3} + \sqrt{2}\) ; \(z = \dfrac{\sqrt{3}}{\sqrt{2}}\), entonces \(xyz =\)
\begin{vertical}
\alternativa \(1 + \sqrt{6}\)
\alternativa \(\sqrt{3} + \sqrt{2}\)
\alternativa \(\sqrt{3}\)
\alternativa \(\dfrac{\sqrt{6}}{2}\)
\alternativa \(\sqrt{6}\)
\end{vertical}


\pregunta \(\sqrt{\dfrac{\sqrt{75} + \sqrt{48}}{\sqrt{3}}} =\)
\begin{vertical}
\alternativa 3
\alternativa 9
\alternativa \(\sqrt{3}\)
\alternativa \(2\sqrt{3}\)
\alternativa \(4\sqrt{3}\)
\end{vertical}

\pregunta Si \(P = \sqrt{4+\sqrt{7}} + \sqrt{4-\sqrt{7}}\), entonces \(P^2 =\)
\begin{vertical}
\alternativa 4
\alternativa 8
\alternativa 14
\alternativa 16
\alternativa \(2\sqrt{2}\)
\end{vertical}


\pregunta \(\dfrac{1}{\sqrt{2}-1} - \dfrac{1}{\sqrt{2}} =\)
\begin{vertical}
\alternativa \(1 + \sqrt{2}\)
\alternativa \(\dfrac{1}{2}\)
\alternativa \(\dfrac{1}{3}\)
\alternativa \(\dfrac{2 + \sqrt{2}}{2}\)
\alternativa \(-\dfrac{2 + \sqrt{2}}{2}\)
\end{vertical}

\end{preguntas}

\newpage
~

\end{document}