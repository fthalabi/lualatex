\def\curso{Segundo medio A}
\def\puntaje{8}
\def\titulo{Control}
\def\subtitulo{Propiedades de raíces y ecuaciones}
\documentclass[]{srs}

\begin{document}

\section*{Objetivos}
  \begin{itemize}[nosep]
    \item Desarrollar expresiones aritméticas usando propiedades de raíces y racionalización.
    \item Aproximar el valor de una raíz usando el método de Héron.
    \item Determinar el valor de una incógnita cuando se encuentra al interior de raíz.
  \end{itemize}

\section*{Instrucciones}
  Cada pregunta tiene 2 puntos y cuenta con 30 minutos para completar
  la evaluación. Incluya desarrollo en todas sus respuestas, y recuerde marcar o señalizar
  el resultado final en cada pregunta.

\section*{Criterios de evaluación}
  En la corrección, se asignará el puntaje a cada pregunta según los siguientes criterios.
\begin{center}
  \begin{tblr}{width=\linewidth,colspec={X[1,c]|X[6]}, hline{1,Z} = {1}{-}{}, hline{1,Z} = {2}{-}{},
      hlines, cells={valign=m}, row{1} = {bg=black!15}}
      Puntaje asignado & \SetCell{c} Criterios o indicadores \\
      +50\% & Señala clara y correctamente cuál es la solución o el resultado de la pregunta hecha
      en el enunciado.\\
      +50\% & Incluye un desarrollo que relata de manera clara y ordenada los procedimientos
      \mbox{necesarios} para solucionar la problemática. En caso de estar incompleto o con
      errores el desarrollo, se asignará puntaje parcial si se muestra dominio de los
       contenidos y conceptos involucrados.\\
      0\% &  La respuesta es incorrecta. De haber desarrollo, este tiene errores conceptuales.\\
  \end{tblr}
\end{center}
\separador[2mm]

\begin{preguntas}(1)
  \pregunta Reduzca y racionalice la siguiente expresión:
  \begin{mcaja}
    \dfrac{5}{\sqrt{7}} -\,\dfrac{3}{\sqrt{28}}
  \end{mcaja}
  \begin{malla}[height=8cm]
  \end{malla}
  \pregunta Utilice el método de Héron para aproximar la siguiente expresión:
  \begin{mcaja}[]
   \sqrt{7}-2\sqrt{14}
  \end{mcaja}
  \begin{malla}[height=9cm]
  \end{malla}
  \pregunta Determine el valor de la incógnita en:
  \begin{mcaja}
    \sqrt{5x+2} + 1 = 9
  \end{mcaja}
  \begin{malla}[height=13cm]
  \end{malla}
  \pregunta Reduzca y racionalice la siguiente expresión:
  \begin{mcaja}
    \left(\sqrt{27}+\sqrt{96}-\sqrt{75}\right)\div\sqrt{15}
  \end{mcaja}
  \begin{malla}[height=20cm]
  \end{malla}
\end{preguntas}




\end{document}