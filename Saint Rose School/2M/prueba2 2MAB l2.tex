\def\curso{Segundo medio}
\def\puntaje{40}
\def\titulo{Prueba}
\def\subtitulo{Raíz enésima y logaritmos}
\def\fecha{29 de mayo, 2025}
\documentclass[]{srs}

\begin{document}

\subsection*{Objetivo}
Realizar cálculos y solucionar problemas utilizando propiedades de raíces, potencias y
logaritmos.

\subsection*{Instrucciones generales}
Tiene 1 hora y 30 minutos para responder la evaluación. Esta es individual y debe
usar solo sus materiales personales para trabajar durante este periodo, no los solicite
a un compañero durante la evaluación.

Para cada pregunta, lea con atención el enunciado y escoja la alternativa que lo
responde correctamente. Solo hay una alternativa correcta por cada pregunta.

\subsection*{Criterios de evaluación}
En la corrección de esta sección, se asignará 2 puntos al marcar la alternativa correcta.
Las alternativas corregidas serán consideradas incorrectas, es decir, marque solo una
alternativa por enunciado.

\separador[2mm]


\begin{preguntas}

\pregunta $3^3 + 3^3 + 3^3 =$
\begin{vertical}
\alternativa $3^4$
\alternativa $3^5$
\alternativa $3^9$
\alternativa $9^3$
\alternativa $9^9$
\end{vertical}

\pregunta $2^{10} + 2^{11} =$
\begin{vertical}
\alternativa $2^{21}$
\alternativa $2^{22}$
\alternativa $4^{21}$
\alternativa $6^{10}$
\alternativa $3 \cdot 2^{10}$
\end{vertical}

\pregunta $\left(\dfrac{1}{2}a^{-2}\right)^{-3} =$
\begin{vertical}
\alternativa $8a^6$
\alternativa $8a^{-5}$
\alternativa $\dfrac{1}{2}a^{-5}$
\alternativa $\dfrac{1}{8}a^{-6}$
\alternativa $\dfrac{1}{2}a^6$
\end{vertical}

\pregunta $\dfrac{1}{\sqrt{2} - 1} - \dfrac{1}{\sqrt{2}} =$
\begin{vertical}
\alternativa $1 + \sqrt{2}$
\alternativa $\dfrac{1}{2}$
\alternativa $\dfrac{1}{3}$
\alternativa $\dfrac{2 + \sqrt{2}}{2}$
\alternativa $-\dfrac{2 + \sqrt{2}}{2}$
\end{vertical}

\pregunta ¿Cuál(es) de los siguientes números corresponden a números racionales?
\begin{verticali}
\alternativa $\dfrac{\sqrt{50}}{\sqrt{8}}$
\alternativa $\left(1 + \sqrt{2}\right)^2$
\alternativa $\dfrac{1}{\sqrt{\sqrt{16}}}$
\end{verticali}
\begin{vertical}
\alternativa Solo I
\alternativa Solo II
\alternativa Solo I y III
\alternativa Solo II y III
\alternativa I, II y III
\end{vertical}

\pregunta La expresión $a^4 - b^4$ se puede escribir como
\begin{vertical}
\alternativa $\left(a-b\right)^4$
\alternativa $\left(a+b\right)^2\left(a-b\right)^2$
\alternativa $\left(a^3-b^3\right)\left(a+b\right)$
\alternativa $\left(a^2+b^2\right)\left(a^2-b^2\right)$
\alternativa $\left(a-b\right)\left(a^3+b^3\right)$
\end{vertical}

\pregunta Se tienen los números reales: $x = \dfrac{1}{\sqrt{2}}$; $y = \dfrac{2}{\sqrt{2} - 1}$; $z = \dfrac{4}{\sqrt{2} + 1}$; $w = \dfrac{\sqrt{2}}{\sqrt{2} - 1}$ ¿Cuál de las siguientes afirmaciones es (son) verdadera(s)?
\begin{verticali}
\alternativa El mayor es y.
\alternativa $y > z > x$.
\alternativa $w > z > x$.
\end{verticali}
\begin{vertical}
\alternativa Solo I
\alternativa Solo II
\alternativa Solo I y II
\alternativa Solo II y III
\alternativa I, II y III
\end{vertical}


\pregunta Si $\dfrac{2^{x+1} + 2^x}{3^x - 3^{x-2}} = \dfrac{4}{9}$, entonces el valor de $2x + 1$ es:
\begin{vertical}
\alternativa $5$
\alternativa $15$
\alternativa $14$
\alternativa $13$
\alternativa $11$
\end{vertical}

\pregunta Si $ab = \sqrt{3}$ y $b = \sqrt{3} - \sqrt{2}$, entonces $a:$
\begin{vertical}
\alternativa $3 + \sqrt{6}$
\alternativa $3 + \sqrt{3}$
\alternativa $\sqrt{3} + \sqrt{2}$
\alternativa $-\left(1 + \sqrt{2}\right)$
\alternativa $-\sqrt{2}$
\end{vertical}

\pregunta $\left(\sqrt{2}\right)^{20} \cdot \left(1 + \dfrac{1}{\sqrt{2}}\right)^{10} \cdot \left(1 - \dfrac{1}{\sqrt{2}}\right)^{10} =$
\begin{vertical}
\alternativa $1$
\alternativa $\dfrac{1}{4}$
\alternativa $\dfrac{9}{4}$
\alternativa $\dfrac{3}{4}$
\alternativa $\dfrac{9}{16}$
\end{vertical}


\pregunta ¿Cuál(es) de las siguientes igualdades es (son) verdadera(s)?
\begin{verticali}
\alternativa $\sqrt{3} \cdot \sqrt[3]{3^2} = 3$
\alternativa $\dfrac{\sqrt[3]{3}}{\sqrt[4]{3}} = \sqrt[12]{3}$
\alternativa $\sqrt[3]{3} \cdot \sqrt[4]{3} = \sqrt[7]{3}$
\end{verticali}
\begin{vertical}
\alternativa Solo I
\alternativa Solo II
\alternativa Solo I y II
\alternativa Solo II y III
\alternativa I, II y III
\end{vertical}

\pregunta Si $\log_2 8 = x$, entonces $x =$
\begin{vertical}
\alternativa $-3$
\alternativa $2\sqrt{2}$
\alternativa $3$
\alternativa $4$
\alternativa $5$
\end{vertical}

\pregunta $\log 2 + \log 8 - \log 4 =$
\begin{vertical}
\alternativa $\log 4$
\alternativa $\log 6$
\alternativa $\log 8$
\alternativa $\log 12$
\alternativa $\log \left(\dfrac{5}{2}\right)$
\end{vertical}

\pregunta Si $\log_3 x = -2$, entonces $x =$
\begin{vertical}
\alternativa $-9$
\alternativa $-6$
\alternativa $0,\overline{1}$
\alternativa $0,\overline{3}$
\alternativa $9$
\end{vertical}

\pregunta Si $\log \left(x + 1\right) = 2$, entonces $x =$
\begin{vertical}
\alternativa $19$
\alternativa $21$
\alternativa $99$
\alternativa $101$
\alternativa $1\,023$
\end{vertical}

\pregunta Sean $P = \log_2 \sqrt[3]{4}$, $Q = \log_4 \sqrt[3]{4}$ y $R = \log_8 \sqrt[3]{4}$, ¿cuál(es) de las siguientes afirmaciones es (son) verdadera(s)?
\begin{verticali}
\alternativa $Q = \dfrac{P}{2}$
\alternativa $R = \dfrac{P}{3}$
\alternativa $PQ = R$
\end{verticali}
\begin{vertical}
\alternativa Solo I
\alternativa Solo II
\alternativa Solo I y II
\alternativa Solo II y III
\alternativa I, II y III
\end{vertical}




\pregunta $\log_2 \left(\log_9 \left(\log_5 125\right)\right) =$
\begin{vertical}
\alternativa $2$
\alternativa $-2$
\alternativa $1$
\alternativa $-1$
\alternativa $0$
\end{vertical}

\pregunta Si a y b son números positivos, se puede determinar que $a = b^2$, si:
\begin{verticaln}
\alternativa $\log a = 2 \log b$
\alternativa $\log \left(\dfrac{a}{b^2}\right) = 0$
\end{verticaln}
\begin{vertical}
\alternativa (1) por sí sola
\alternativa (2) por sí sola
\alternativa Ambas juntas, (1) y (2)
\alternativa Cada una por sí sola, (1) ó (2)
\alternativa Se requiere información adicional
\end{vertical}

\pregunta $\log \left(\dfrac{\sqrt{6} + 3}{\sqrt{2} + \sqrt{3}}\right) =$
\begin{vertical}
\alternativa $\dfrac{1}{2} \log 3$
\alternativa $\log 3$
\alternativa $2 \log 3$
\alternativa $\log 6$
\alternativa $\log 2$
\end{vertical}

\pregunta La masa de un material radioactivo medida en kilogramos, está dada por la expresión $m\left(t\right) = 4 \cdot \left(0,2\right)^t$, donde t es el tiempo medido en años. ¿Cuántos años deben transcurrir para que la masa del material quede reducida a dos kilogramos?
\begin{vertical}
\alternativa $\log 2,5$
\alternativa $\dfrac{\log 5}{\log 2}$
\alternativa $\log 5 - \log 2$
\alternativa $\dfrac{\log 2}{1 - \log 2}$
\alternativa Todas las anteriores.
\end{vertical}


\end{preguntas}



\end{document}

