\def\curso{Segundo medio B}
\def\puntaje{6}
\def\titulo{Control}
\def\subtitulo{Ecuaciones con raíces}
\documentclass[]{srs}

\begin{document}

\section*{Objetivo}
  Solucionar ecuaciones donde la incógnita se encuentra al interior de una raíz.

\section*{Instrucciones}
  Cada pregunta tiene 2 puntos y cuenta con 30 minutos para completar
  la evaluación. Incluya desarrollo en todas sus respuestas, y recuerde marcar o señalizar
  el resultado final en cada pregunta.

\section*{Criterios de evaluación}
  En la corrección, se asignará el puntaje a cada pregunta según los siguientes criterios.
\begin{center}
  \begin{tblr}{width=\linewidth,colspec={X[1,c]|X[6]}, hline{1,Z} = {1}{-}{}, hline{1,Z} = {2}{-}{},
      hlines, cells={valign=m}, row{1} = {bg=black!15}}
      Puntaje asignado & \SetCell{c} Criterios o indicadores \\
      +50\% & Señala clara y correctamente cuál es la solución o el resultado de la pregunta hecha
      en el enunciado.\\
      +50\% & Incluye un desarrollo que relata de manera clara y ordenada los procedimientos
      \mbox{necesarios} para solucionar la problemática. En caso de estar incompleto o con
      errores el desarrollo, se asignará puntaje parcial si se muestra dominio de los
       contenidos y conceptos involucrados.\\
      0\% &  La respuesta es incorrecta. De haber desarrollo, este tiene errores conceptuales.\\
  \end{tblr}
\end{center}
\separador[2mm]

Determine el valor de la incógnita ($x$) en cada caso.
\begin{preguntas}(1)
  \pregunta $\sqrt{\dfrac{3x}{2}}=6$
  \begin{malla}[height=9cm]
  \end{malla}
  \pregunta $\sqrt{5x+2} + 1 = 9$
  \begin{malla}[height=9cm]
  \end{malla}
  \pregunta $\dfrac{7}{11}\sqrt{\dfrac{2x-3}{4}}-5=\dfrac{3}{2}$
  \begin{malla}[height=15cm]
  \end{malla}
\end{preguntas}




\end{document}