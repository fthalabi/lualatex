\def\titulo{Guía}
\def\subtitulo{Función cuadrática}
\def\curso{Segundo medio A}
\documentclass[sin nombre]{srs}
\begin{document}

\separador
\begin{preguntas}[after-item-skip=2cm]
\pregunta Las soluciones de la ecuación $2\left(x - 1\right)^{2} = 5$, están representadas en:
\begin{vertical}
\alternativa $1 \pm \dfrac{\sqrt{5}}{2}$
\alternativa $-1 \pm \dfrac{\sqrt{5}}{2}$
\alternativa $1 \pm \sqrt{\dfrac{5}{2}}$
\alternativa $-1 \pm \sqrt{\dfrac{5}{2}}$
\alternativa $\dfrac{1 \pm \sqrt{5}}{2}$
\end{vertical}

\pregunta ¿En cuál de las siguientes ecuaciones cuadráticas, las soluciones son reales e iguales?
\begin{vertical}
\alternativa $x^{2} - 4x = -1$
\alternativa $x^{2} - 2x = -4$
\alternativa $2x^{2}-9=0$
\alternativa $2x^{2} + x = 1$
\alternativa $4x^{2} + 4x =-1$
\end{vertical}

\pregunta ¿En cuál de las siguientes ecuaciones cuadráticas, las soluciones no son reales?
\begin{vertical}
\alternativa $x^{2} + x = 1$
\alternativa $x^{2} - 2x = 4$
\alternativa $2x^{2} - 5x = -2$
\alternativa $x^{2} + x = 2$
\alternativa $x^{2} + 4x = -8$
\end{vertical}

\pregunta ¿Cuál(es) de las siguientes ecuaciones no tienen soluciones en los números reales?
\begin{verticali}
\alternativa $2\left(x - 2\right)^{2} + 3 = 0$
\alternativa $-\dfrac{3}{2}\left(x - 1\right)^{2} + 1 = 0$
\alternativa $2\left(x + \dfrac{1}{2}\right)^{2} + 5 = 0$
\end{verticali}
\begin{vertical}
\alternativa Solo I
\alternativa Solo II
\alternativa Solo I y II
\alternativa Solo I y III
\alternativa I, II y III
\end{vertical}

\pregunta ¿Cuál de las siguientes ecuaciones tiene como raíces (o soluciones) a $\left(2 + \sqrt{5}\right)$ y $\left(2 - \sqrt{5}\right)$?
\begin{vertical}
\alternativa $x^{2} - 4x + 9=0$
\alternativa $x^{2} + 4x + 9 =0$
\alternativa $x^{2} - 4x + 1 =0$
\alternativa $x^{2} - 4x - 1 = 0$
\alternativa $x^{2}-2x-1=0$
\end{vertical}

\pregunta Con respecto a la parábola de ecuación: $y = -x^{2} + 4x - 3$, se afirma que:
\begin{verticali}
\alternativa Intercepta al eje y en $\left(0,-3\right)$.
\alternativa Intercepta al eje x en dos puntos.
\alternativa Su vértice es el punto $\left(-2,-7\right)$.
\end{verticali}
¿Cuál(es) de las afirmaciones anteriores es (son) verdadera(s)?
\begin{vertical}
\alternativa Solo I
\alternativa Solo II
\alternativa Solo I y II
\alternativa Solo II y III
\alternativa I, II y III
\end{vertical}

\pregunta ¿Cuál de los siguientes gráficos representa mejor a la función cuadrática: $y = x^{2} - 6x + 9$?
\begin{alternativasgraficas}
\alternativa%
 \begin{tikzpicture}[declare function={f(\x)=(\x)^2;}]
  \datavisualization [school book axes, visualize as smooth line,
  all axes={length=3cm,ticks={major at={0}}},
  x axis={label=$x$,min value=-3,max value=3},
  y axis={label=$y$,min value=-0.5,max value=3},
  %f1={label in data={text=$L_1$, when=x is 5}},
  %f2={label in data={text=$L_2$, when=x is 1}}
  ]
  data [format=function] {
  var x : interval [-2.5:0.5];
  func y = f(\value x + 1) + 0.5;
  };
\end{tikzpicture}
\alternativa%
 \begin{tikzpicture}[declare function={f(\x)=(\x)^2;}]
  \datavisualization [school book axes, visualize as smooth line,
  all axes={length=3cm,ticks={major at={0}}},
  x axis={label=$x$,min value=-3,max value=3},
  y axis={label=$y$,min value=-0.5,max value=3},
  %f1={label in data={text=$L_1$, when=x is 5}},
  %f2={label in data={text=$L_2$, when=x is 1}}
  ]
  data [format=function] {
  var x : interval [-0.5:2.5];
  func y = f(\value x - 1) + 0.5;
  };
\end{tikzpicture}
\alternativa%
 \begin{tikzpicture}[declare function={f(\x)=(\x)^2;}]
  \datavisualization [school book axes, visualize as smooth line,
  all axes={length=3cm,ticks={major at={0}}},
  x axis={label=$x$,min value=-3,max value=3},
  y axis={label=$y$,min value=-0.5,max value=3},
  %f1={label in data={text=$L_1$, when=x is 5}},
  %f2={label in data={text=$L_2$, when=x is 1}}
  ]
  data [format=function] {
  var x : interval [-2.5:0.5];
  func y = f(\value x + 1);
  };
\end{tikzpicture}
\alternativa%
 \begin{tikzpicture}[declare function={f(\x)=(\x)^2;}]
  \datavisualization [school book axes, visualize as smooth line,
  all axes={length=3cm,ticks={major at={0}}},
  x axis={label=$x$,min value=-3,max value=3},
  y axis={label=$y$,min value=-0.5,max value=3},
  %f1={label in data={text=$L_1$, when=x is 5}},
  %f2={label in data={text=$L_2$, when=x is 1}}
  ]
  data [format=function] {
  var x : interval [-0.5:2.5];
  func y = f(\value x - 1);
  };
\end{tikzpicture}
\alternativa%
 \begin{tikzpicture}[declare function={f(\x)=(\x)^2;}]
  \datavisualization [school book axes, visualize as smooth line,
  all axes={length=3cm,ticks={major at={0}}},
  x axis={label=$x$,min value=-3,max value=3},
  y axis={label=$y$,min value=-0.5,max value=3},
  %f1={label in data={text=$L_1$, when=x is 5}},
  %f2={label in data={text=$L_2$, when=x is 1}}
  ]
  data [format=function] {
  var x : interval [-2:1];
  func y = -f(\value x + 0.5)+2;
  };
\end{tikzpicture}
\end{alternativasgraficas}

\pregunta Sea la función f definida en los reales, mediante $f\left(x\right) = -2\left(x - 3\right)\left(x - 5\right)$, entonces las coordenadas del vértice de la parábola asociada a su gráfica son:
\begin{vertical}
\alternativa $\left(4,-2\right)$
\alternativa $\left(4, 2\right)$
\alternativa $\left(4,-1\right)$
\alternativa $\left(4, 1\right)$
\alternativa $\left(2,-6\right)$
\end{vertical}

\pregunta ¿Cuál de los siguientes gráficos representa mejor a la función: $f\left(x\right) = \left(x + 2\right)^{2} + 1$?
\begin{alternativasgraficas}
\alternativa%
 \begin{tikzpicture}[declare function={f(\x)=(\x)^2;}]
  \datavisualization [school book axes, visualize as smooth line,
  all axes={length=3cm,ticks={major at={0}}},
  x axis={label=$x$,min value=-3,max value=3},
  y axis={label=$y$,min value=-0.5,max value=3},
  %f1={label in data={text=$L_1$, when=x is 5}},
  %f2={label in data={text=$L_2$, when=x is 1}}
  ]
  data [format=function] {
  var x : interval [-0.5:2.5];
  func y = f(\value x - 1) + 0.5;
  };
\end{tikzpicture}
\alternativa%
 \begin{tikzpicture}[declare function={f(\x)=(\x)^2;}]
  \datavisualization [school book axes, visualize as smooth line,
  all axes={length=3cm,ticks={major at={0}}},
  x axis={label=$x$,min value=-3,max value=3},
  y axis={label=$y$,min value=-0.5,max value=3},
  %f1={label in data={text=$L_1$, when=x is 5}},
  %f2={label in data={text=$L_2$, when=x is 1}}
  ]
  data [format=function] {
  var x : interval [-2.5:0.5];
  func y = f(\value x + 1) + 0.5;
  };
\end{tikzpicture}
\alternativa%
 \begin{tikzpicture}[declare function={f(\x)=(\x)^2;}]
  \datavisualization [school book axes, visualize as smooth line,
  all axes={length=3cm,ticks={major at={0}}},
  x axis={label=$x$,min value=-3,max value=3},
  y axis={label=$y$,min value=-3,max value=0.5},
  %f1={label in data={text=$L_1$, when=x is 5}},
  %f2={label in data={text=$L_2$, when=x is 1}}
  ]
  data [format=function] {
  var x : interval [-1.5:1.5];
  func y = -f(\value x) -0.5;
  };
\end{tikzpicture}
\alternativa%
 \begin{tikzpicture}[declare function={f(\x)=(\x)^2;}]
  \datavisualization [school book axes, visualize as smooth line,
  all axes={length=3cm,ticks={major at={0}}},
  x axis={label=$x$,min value=-3,max value=3},
  y axis={label=$y$,min value=-1,max value=2.5},
  %f1={label in data={text=$L_1$, when=x is 5}},
  %f2={label in data={text=$L_2$, when=x is 1}}
  ]
  data [format=function] {
  var x : interval [-1.5:1.5];
  func y = f(\value x) -0.8;
  };
\end{tikzpicture}
\alternativa%
 \begin{tikzpicture}[declare function={f(\x)=(\x)^2;}]
  \datavisualization [school book axes, visualize as smooth line,
  all axes={length=3cm,ticks={major at={0}}},
  x axis={label=$x$,min value=-3,max value=3},
  y axis={label=$y$,min value=-3,max value=0.5},
  %f1={label in data={text=$L_1$, when=x is 5}},
  %f2={label in data={text=$L_2$, when=x is 1}}
  ]
  data [format=function] {
  var x : interval [-0.5:2.5];
  func y = -f(\value x - 1);
  };
\end{tikzpicture}
\end{alternativasgraficas}

\pregunta ¿Cuál de las siguientes afirmaciones es FALSA con respecto a la función $f\left(x\right) = -\left(x^{2} + 4\right)$ si el dominio son todos los números reales?
\begin{vertical}
\alternativa La gráfica no intersecta al eje x.
\alternativa El vértice de la parábola asociada a esta función está en el eje y.
\alternativa El vértice de la parábola asociada a esta función está en el eje x.
\alternativa Su gráfica tiene al eje y como eje de simetría.
\alternativa El valor de x donde alcanza su máximo es $x = 0$.
\end{vertical}

\pregunta ¿Cuál de las siguientes funciones definidas en los reales, tiene como gráfico la parábola de la figura?
\begin{centrado}
 \begin{tikzpicture}[declare function={f(\x)=(\x^2)/2 -3*\x +4;}]
  \datavisualization [school book axes, visualize as smooth line,
  all axes={length=4cm},
  x axis={label=$x$,min value=-1,max value=7,ticks={major at={2,4}}},
  y axis={label=$y$,min value=-1,max value=6,ticks={major at={4}}},
  %f1={label in data={text=$L_1$, when=x is 5}},
  %f2={label in data={text=$L_2$, when=x is 1}}
  ]
  data [format=function] {
  var x : interval [-0.5:6.4];
  func y = f(\value x);
  };
\end{tikzpicture}
\end{centrado}
\begin{vertical}
\alternativa $g\left(x\right) = \left(x - 3\right)^{2} + 1$
\alternativa $h\left(x\right) = -\left(x - 3\right)^{2} - 1$
\alternativa $j\left(x\right) = \left(x - 3\right)^{2} + 2$
\alternativa $k\left(x\right) = 2\left(x - 2\right)\left(x - 4\right)$
\alternativa $m\left(x\right)=\dfrac{1}{2}\left(x-2\right)\left(x - 4\right)$
\end{vertical}

\pregunta Sea f una función definida en los reales mediante $f\left(x\right) = x^{2} - ax + 6$, con $a \neq 0$. Si el valor de $x$ donde la función alcanza su valor mínimo es $-2$, entonces $a =$
\begin{vertical}
\alternativa $4$
\alternativa $-8$
\alternativa $-4$
\alternativa $4~ \text{ ó }~ -4$
\alternativa $-\sqrt{32} ~\text{ ó }~ \sqrt{32}$.
\end{vertical}

\pregunta La función $h\left(t\right) = pt - 5t^{2}$, modela la altura (en metros) que alcanza un proyectil al ser lanzado verticalmente hacia arriba a los $t$ segundos. Se puede determinar esta función si se sabe que:
\begin{verticaln}
\alternativa A los 2 segundos alcanza una altura de 30 metros.
\alternativa La altura máxima la alcanza a los 2,5 segundos.
\end{verticaln}
\begin{vertical}
\alternativa (1) por sí sola
\alternativa (2) por sí sola
\alternativa Ambas juntas, (1) y (2)
\alternativa Cada una por sí sola, (1) ó (2)
\alternativa Se requiere información adicional
\end{vertical}

\pregunta La altura $h\left(t\right)$ alcanzada, medida en metros, de un proyectil se modela mediante la función $h\left(t\right) = 20t - 5t^{2}$, donde t es la cantidad de segundos que transcurren hasta que alcanza dicha altura. ¿Cuál(es) de las siguientes afirmaciones es (son) verdadera(s)?
\begin{verticali}
\alternativa A los 4 segundos llega al suelo.
\alternativa A los 2 segundos alcanza su altura máxima.
\alternativa Al primer y tercer segundo después de ser lanzado alcanza la misma altura.
\end{verticali}
\begin{vertical}
\alternativa Solo I
\alternativa Solo II
\alternativa Solo I y II
\alternativa Solo II y III
\alternativa I, II y III
\end{vertical}

\end{preguntas}
\end{document}