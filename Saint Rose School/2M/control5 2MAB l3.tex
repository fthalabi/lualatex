\def\curso{Segundo medio}
\def\puntaje{8}
\def\titulo{Control}
\def\subtitulo{Factorización de trinomios}
\def\fecha{23 de mayo, 2025}
\documentclass[]{srs}

\begin{document}

\subsection*{Objetivo}
  Factorizar ecuaciones cuadráticas.

\subsection*{Instrucciones}
  Cada pregunta tiene 2 puntos y cuenta con 40 minutos para completar
  la evaluación. Incluya desarrollo en todas sus respuestas, y recuerde marcar o señalizar
  el resultado final en cada pregunta.

\subsection*{Criterios de evaluación}
  En la corrección, se asignará el puntaje a cada pregunta según los siguientes criterios.
\begin{center}
  \begin{tblr}{width=\linewidth,colspec={X[1,c]|X[6]}, hline{1,Z} = {1}{-}{}, hline{1,Z} = {2}{-}{},
      hlines, cells={valign=m}, row{1} = {bg=black!15}}
      Puntaje asignado & \SetCell{c} Criterios o indicadores \\
      +50\% & Señala clara y correctamente cuál es la solución o el resultado de la pregunta hecha
      en el enunciado.\\
      +50\% & Incluye un desarrollo que relata de manera clara y ordenada los procedimientos
      \mbox{necesarios} para solucionar la problemática. En caso de estar incompleto o con
      errores el desarrollo, se asignará puntaje parcial si se muestra dominio de los
       contenidos y conceptos involucrados.\\
      0\% &  La respuesta es incorrecta. De haber desarrollo, este tiene errores conceptuales.\\
  \end{tblr}
\end{center}
\separador[2mm]

Factorice cada una de las siguientes expresiones.

\begin{preguntas}
  \pregunta $x^2+3x+2$
  \begin{malla}[height=7cm]
  \end{malla}
  \pregunta $x^2+x-30$
  \begin{malla}[height=7cm]
  \end{malla}
  \pregunta $x^2+3x-550$
  \begin{malla}[height=7cm]
  \end{malla}
  \pregunta $6x^2-7x-3$
  \begin{malla}[height=9cm]
  \end{malla}
\end{preguntas}




\end{document}