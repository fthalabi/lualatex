\def\curso{Segundo medio A}
\def\puntaje{23}
\def\titulo{Prueba}
\def\subtitulo{Función cuadrática (parte 1)}
\def\fecha{25 de julio, 2025}
\documentclass[]{srs}

\newsavebox{\rubricaUno}
\begin{lrbox}{\rubricaUno}
\def\stop{{\FA\char"E204}\space}
\begin{centrado}
\begin{tblr}{width=176mm,colspec={X[6]X[1,c]X[1,c]}, hline{2,Z} = {1}{-}{}, hline{2,Z} = {2}{-}{},
    hlines, cells={valign=m}, row{2} = {bg=black!15},vlines,cell{3-Z}{1} = {font=\raggedright}}
    \SetCell[c=3]{c} \stop No rayar la tabla, es solo para uso del docente \stop &  &\\
    \SetCell{c} Criterios para la corrección & Puntaje & Asignación \\
    Factoriza correctamente la expresión y encuentra las soluciones & $2$ & \\
    Señala claramente los resultados, e incluye ordenadamente los procedimientos
    necesarios para solucionar la problemática. & $1$ & \\
\end{tblr}
\end{centrado}
\end{lrbox}

\newsavebox{\rubricaDos}
\begin{lrbox}{\rubricaDos}
\def\stop{{\FA\char"E204}\space\normalfont}
\begin{centrado}
\begin{tblr}{width=176mm,colspec={X[6]X[1,c]X[1,c]}, hline{2,Z} = {1}{-}{}, hline{2,Z} = {2}{-}{},
    hlines, cells={valign=m}, row{2} = {bg=black!15},vlines,cell{3-Z}{1} = {font={\raggedright}}}
    \SetCell[c=3]{c} \stop No rayar la tabla, es solo para uso del docente \stop &  &\\
    \SetCell{c} Criterios para la corrección & Puntaje & Asignación \\
    Encuentra las raíces o soluciones de la ecuación cuadrática. & $2$ & \\
    Determina la intersección con el eje $y$. & $1$ & \\
    Determina si la función tiene máximo o mínimo y donde se encuentra. & $1$ & \\
    Utiliza los datos anteriores para representar correctamente la
    gráfica de la función. & $1$ & \\
    Señala claramente los resultados, e incluye ordenadamente los procedimientos
    necesarios para solucionar la problemática. & $2$ & \\
\end{tblr}
\end{centrado}
\end{lrbox}

\begin{document}

\subsection*{Objetivo}
  Describir todas las parte de la función cuadrática y encontrar sus soluciones, ya sea
  factorizando o utilizando la formula general.

\subsection*{Instrucciones generales}
  Cuenta con 40 minutos para completar la evaluación. Esta es individual y debe usar solo
  sus materiales personales para trabajar durante este periodo, no los solicite a un compañero
  durante la evaluación.

  Lea atentamente cada enunciado, siga las instrucciones, y responda cada
  pregunta siguiendo los criterios descritos en su correspondiente rubrica. Cumplir
  con todos estos criterios, es necesario para obtener el puntaje completo de cada pregunta.
  En algunos casos, se puede asignar puntaje parcial por un criterio medianamente logrado.

\separador[2mm]

Factorice cada una de las siguientes ecuaciones, y encuentre las soluciones o raíces
en cada uno de los casos.

\begin{preguntas}
  \pregunta $x^2 -13x+30$
  \begin{malla}[7]

  \end{malla}
  \usebox{\rubricaUno}
  \pregunta $x^2-7x-18$
  \begin{malla}[7]

  \end{malla}
  \usebox{\rubricaUno}
  \pregunta $x^2+x-132$
  \begin{malla}[7]

  \end{malla}
  \usebox{\rubricaUno}

\end{preguntas}

\newpage
Describa completamente cada una de las siguientes funciones cuadráticas y grafique
en los espacios señalizados.

\begin{preguntas}
  \pregunta $y=-x^2+3x+4$
  \begin{centrado}
  \begin{tikzpicture}
  \datavisualization [school book axes={standard labels},
    visualize as smooth line,
    clean ticks,
    all axes={length=6cm,ticks=none},
    x axis={label=$x$},
    y axis={label=$f(x)$}]
     data {};
  \end{tikzpicture}
  \end{centrado}
  \begin{malla}[10]

  \end{malla}
  \usebox{\rubricaDos}

  \pregunta $y=x^2-\dfrac{1}{12}x -\dfrac{1}{12}$
  \begin{centrado}
  \begin{tikzpicture}
  \datavisualization [school book axes={standard labels},
    visualize as smooth line,
    clean ticks,
    all axes={length=6cm,ticks=none},
    x axis={label=$x$},
    y axis={label=$f(x)$}]
     data {};
  \end{tikzpicture}
  \end{centrado}
  \begin{malla}[11]

  \end{malla}
  \usebox{\rubricaDos}

\end{preguntas}

\end{document}