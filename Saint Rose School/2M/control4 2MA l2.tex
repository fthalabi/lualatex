\def\curso{Segundo medio A}
\def\puntaje{12}
\def\titulo{Control}
\def\subtitulo{Raíz enésima}
\def\fecha{25 de Abril, 2025}
\documentclass[]{srs}

\begin{document}

\section*{Objetivo}
  Calcular y reducir expresiones con raíces de radical mayor a dos.

\section*{Instrucciones}
  Cada pregunta tiene 2 puntos y cuenta con 45 minutos para completar
  la evaluación. Incluya desarrollo en todas sus respuestas, y recuerde marcar o señalizar
  el resultado final en cada pregunta.

\section*{Criterios de evaluación}
  En la corrección, se asignará el puntaje a cada pregunta según los siguientes criterios.
\begin{center}
  \begin{tblr}{width=\linewidth,colspec={X[1,c]|X[6]}, hline{1,Z} = {1}{-}{}, hline{1,Z} = {2}{-}{},
      hlines, cells={valign=m}, row{1} = {bg=black!15}}
      Puntaje asignado & \SetCell{c} Criterios o indicadores \\
      +50\% & Señala clara y correctamente cuál es la solución o el resultado de la pregunta hecha
      en el enunciado.\\
      +50\% & Incluye un desarrollo que relata de manera clara y ordenada los procedimientos
      \mbox{necesarios} para solucionar la problemática. En caso de estar incompleto o con
      errores el desarrollo, se asignará puntaje parcial si se muestra dominio de los
       contenidos y conceptos involucrados.\\
      0\% &  La respuesta es incorrecta. De haber desarrollo, este tiene errores conceptuales.\\
  \end{tblr}
\end{center}
\separador[2mm]

\begin{preguntas}(1)
  \pregunta Reduce la expresión a una sola raíz.
  \begin{mcaja}
    \sqrt[5]{\sqrt[4]{8}}
  \end{mcaja}
  \begin{malla}[height=9cm]
  \end{malla}

  \pregunta Extrae el mayor factor posible fuera de la raíz.
  \begin{mcaja}
    \sqrt[3]{750}
  \end{mcaja}
  \begin{malla}[height=10cm]
  \end{malla}
  \pregunta Introduce el factor en la raíz.
  \begin{mcaja}
    \dfrac{3}{5}\sqrt[4]{1,\overline{6}}
  \end{mcaja}
  \begin{malla}[height=10cm]
  \end{malla}
  \pregunta Calcule y reduzca lo más posible la expresión.
  \begin{mcaja}
    \dfrac{\sqrt[4]{15^{24}}\sqrt[5]{15^{70}}}{\sqrt[6]{3^{54}}}
  \end{mcaja}
  \begin{malla}[height=10cm]
  \end{malla}
  \pregunta Calcule y reduzca lo más posible la expresión.
  \begin{mcaja}
    \sqrt[3]{2\sqrt{3}+2}\sqrt[3]{2\sqrt{3}-2}
  \end{mcaja}
  \begin{malla}[height=10cm]
  \end{malla}
  \pregunta Calcule y reduzca lo más posible la expresión.
  \begin{mcaja}
    \dfrac{\sqrt[11]{5^2}+\sqrt[11]{5^2}+\sqrt[11]{5^2}+\sqrt[11]{5^2}+\sqrt[11]{5^2}}{\sqrt[13]{5^5}}
  \end{mcaja}
  \begin{malla}[height=10cm]
  \end{malla}
\end{preguntas}




\end{document}