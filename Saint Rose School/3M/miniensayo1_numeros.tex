\def\titulo{Miniensayo}
\def\subtitulo{Eje de números}
\def\curso{Tercero medio}
\documentclass[sin fecha]{srs}

\begin{document}

\separador
\begin{preguntas}[after-item-skip=1cm]

\pregunta El $30\%$ de un número es $45$, ¿cuál es su $12\%$?
\begin{vertical}
\alternativa $25$
\alternativa $18$
\alternativa $16$
\alternativa $12$
\alternativa $8$
\end{vertical}

\pregunta El $15\%$ de $1\dfrac{2}{3}$ es:
\begin{vertical}
\alternativa $0,125$
\alternativa $0,75$
\alternativa $0,5$
\alternativa $0,45$
\alternativa $0,25$
\end{vertical}


\pregunta El $50\%$ de la mitad de un número es $20$, entonces el número es:
\begin{vertical}
\alternativa $5$
\alternativa $10$
\alternativa $20$
\alternativa $40$
\alternativa $80$
\end{vertical}

\pregunta $103$ es el $10\%$ de:
\begin{vertical}
\alternativa $1\,000$
\alternativa $1\,020$
\alternativa $1\,030$
\alternativa $1\,040$
\alternativa $1\,050$
\end{vertical}

\pregunta ¿Qué porcentaje es $0,4\overline{2}$ de $0,\overline{76}$ ?
\begin{vertical}
\alternativa $32,41\%$
\alternativa $50\%$
\alternativa $55\%$
\alternativa $60,8\%$
\alternativa $181,81\%$
\end{vertical}

\pregunta Una camisa con un $20\%$ de descuento cuesta $\$4\,000$. ¿Cuánto costaría sin la rebaja?
\begin{vertical}
\alternativa $\$4\,800$
\alternativa $\$5\,000$
\alternativa $\$5\,200$
\alternativa $\$5\,400$
\alternativa $\$5\,500$
\end{vertical}

\pregunta La cuarta parte de $0,\overline{2}$ es:
\begin{vertical}
\alternativa $0,0\overline{4}$
\alternativa $0,05$
\alternativa $0,0\overline{5}$
\alternativa $0,\overline{5}$
\alternativa $0,\overline{8}$
\end{vertical}

\pregunta En un curso hay una mujer cada $4$ hombres. ¿Qué $\%$ del curso son mujeres?
\begin{vertical}
\alternativa $20\%$
\alternativa $25\%$
\alternativa $30\%$
\alternativa $40\%$
\alternativa $80\%$
\end{vertical}

\pregunta Se ha cancelado $\$42\,000$, que corresponde al $60\%$ de una deuda. ¿Cuánto falta por pagar?
\begin{vertical}
\alternativa $\$14\,000$
\alternativa $\$28\,000$
\alternativa $\$30\,000$
\alternativa $\$70\,000$
\alternativa $\$112\,000$
\end{vertical}

\pregunta Solo $12$ alumnas, de un curso de $30$, han pagado una cuota para un paseo. ¿Qué $\%$ del curso falta por pagar?
\begin{vertical}
\alternativa $40\%$
\alternativa $45\%$
\alternativa $55\%$
\alternativa $60\%$
\alternativa $65\%$
\end{vertical}

\pregunta El estadio A de una ciudad tiene capacidad para 40.000 personas sentadas y otro B para 18.000. Se hacen eventos simultáneos; el A se ocupa hasta el 25\% de su capacidad y el B llena sólo el 50\%. ¿Cuál(es) de las siguientes afirmaciones es(son) verdadera(s) ?
\begin{verticali}
\alternativa El estadio A registró mayor asistencia de público que el B.
\alternativa Si se hubiese llevado a los asistentes de ambos estadios al A, habría quedado en éste, menos del 50\% de sus asientos vacíos.
\alternativa Los espectadores que asistieron en conjunto a los dos estadios superan en 1.000 a la capacidad de B.
\end{verticali}
\begin{vertical}
\alternativa Sólo I
\alternativa Sólo II
\alternativa Sólo III
\alternativa Sólo I y II
\alternativa Sólo I y III
\end{vertical}

\pregunta La fila (linea horizontal) de la tabla adjunta significa \q{3 gallinas
comen 6 kilos de ración en 12 días}. Siendo esta afirmación verdadera, ¿cuál
alternativa contiene tambien información verdadera?\\
\vspace*{5pt}\hspace*{2em}\begin{tblr}{colspec={XXX},width=20em,cells={halign=c},vlines,hlines}
  Gallinas & Kilos & Días \\
  3 & 6 & 12 \\
\end{tblr}
\begin{vertical}
  \alternativa \begin{tblr}{colspec={XXX},width=20em,cells={halign=c},vlines,hlines} 1 & 6 & 24 \\ \end{tblr}
  \alternativa \begin{tblr}{colspec={XXX},width=20em,cells={halign=c},vlines,hlines} 1 & 1 & 12 \\ \end{tblr}
  \alternativa \begin{tblr}{colspec={XXX},width=20em,cells={halign=c},vlines,hlines} 3 & 3 & 3 \\ \end{tblr}
  \alternativa \begin{tblr}{colspec={XXX},width=20em,cells={halign=c},vlines,hlines} 6 & 6 & 6 \\ \end{tblr}
\end{vertical}


%\alternativa Sólo I
%\alternativa Sólo I y II
%\alternativa Sólo I y III
%\alternativa Sólo II y III
%\alternativa I, II y III
%\end{vertical}

\pregunta En una quinta hay naranjos, manzanos y duraznos que suman en total 300 árboles. Si hay 120 naranjos y la razón entre los duraznos y manzanos es $7:3$, entonces ¿cuántos duraznos hay en la quinta?
\begin{vertical}
\alternativa 54
\alternativa 77
\alternativa 84
\alternativa 126
\alternativa 210
\end{vertical}

\pregunta $y$ es inversamente proporcional al cuadrado de $x$, cuando $y = 16$, $x = 1$. Si $x = 8$, entonces $y =$
\begin{vertical}
\alternativa $\dfrac{1}{2}$
\alternativa $\dfrac{1}{4}$
\alternativa $2$
\alternativa $4$
\alternativa $9$
\end{vertical}

\pregunta Se desea cortar un alambre de 720 mm en tres trozos de modo que la razón de sus longitudes sea $8:6:4$. ¿Cuánto mide cada trozo de alambre, de acuerdo al orden de las razones dadas?
\begin{vertical}
\alternativa 180 mm \quad 120 mm \quad 90 mm
\alternativa 420 mm \quad 180 mm \quad 120 mm
\alternativa 320 mm \quad 240 mm \quad 160 mm
\alternativa 510 mm \quad 120 mm \quad 90 mm
\alternativa Ninguna de las medidas anteriores
\end{vertical}

\pregunta Se sabe que $a$ es directamente proporcional al número $\dfrac{1}{b}$ y cuando $a$ toma el valor $15$, el valor de $b$ es $4$. Si $a$ toma el valor $6$, entonces el valor de $b$ es:
\begin{vertical}
\alternativa $10$
\alternativa $\dfrac{8}{5}$
\alternativa $\dfrac{5}{8}$
\alternativa $\dfrac{1}{10}$
\alternativa $\dfrac{15}{4}$
\end{vertical}

\pregunta En un mapa (a escala) se tiene que 2 cm en él corresponden a 25 km en la realidad. Si la distancia en el mapa entre dos ciudades es $5{,}4$ cm, entonces la distancia real es
\begin{vertical}
\alternativa 50 km
\alternativa 65 km
\alternativa $67{,}5$ km
\alternativa $62{,}5$ km
\alternativa ninguno de los valores anteriores.
\end{vertical}

\pregunta En la tabla adjunta $z$ es directamente proporcional a $\dfrac{1}{y}$, según los datos registrados, el valor de $\dfrac{a}{b}$ es:
\begin{centrado}
\begin{tblr}{|c|c|}
\hline
$z$ & $y$ \\
\hline
$8$ & $2$ \\
\hline
$a$ & $4$ \\
\hline
$1$ & $16$ \\
\hline
$\dfrac{1}{4}$ & $b$ \\
\hline
\end{tblr}
\end{centrado}
\begin{vertical}
\alternativa $256$
\alternativa $16$
\alternativa $\dfrac{1}{16}$
\alternativa $64$
\alternativa $\dfrac{1}{64}$
\end{vertical}

\pregunta La escala de un mapa es $1:500\,000$. Si en el mapa la distancia entre dos ciudades es $3{,}5$ cm, ¿cuál es la distancia real entre ellas?
\begin{vertical}
\alternativa $1{,}75$ km
\alternativa $17{,}5$ km
\alternativa $175$ km
\alternativa $1\,750$ km
\alternativa $17\,500$ km
\end{vertical}

\pregunta Los cajones $M$ y $S$ pesan juntos $K$ kilogramos. Si la razón entre los pesos de $M$ y $S$ es $3:4$, entonces $S:K =$
\begin{vertical}
\alternativa $4:7$
\alternativa $4:3$
\alternativa $7:4$
\alternativa $3:7$
\alternativa $3:4$
\end{vertical}

\pregunta La ley combinada que rige el comportamiento ideal de un gas es $\dfrac{P \cdot V}{T} = \text{constante}$, donde $P$ es la presión del gas, $V$ su volumen y $T$ su temperatura absoluta. ¿Cuál(es) de las siguientes afirmaciones es(son) verdadera(s)?
\begin{verticali}
\alternativa A volumen constante la presión es directamente proporcional a la temperatura
\alternativa A temperatura constante la presión es inversamente proporcional al volumen
\alternativa A presión constante el volumen es inversamente proporcional a la temperatura
\end{verticali}
\begin{vertical}
\alternativa Solo I
\alternativa Solo II
\alternativa Solo I y II
\alternativa Solo I y III
\alternativa I, II y III
\end{vertical}

\pregunta Una nutricionista mezcla tres tipos de jugos de fruta de modo que sus volúmenes están en la razón $1:2:3$. Si el volumen del segundo tipo es de 4 litros, ¿cuántos litros tiene la mezcla total?
\begin{vertical}
\alternativa 6 litros
\alternativa 10 litros
\alternativa 12 litros
\alternativa 14 litros
\alternativa 16 litros
\end{vertical}

\pregunta Entre tres hermanos compran un número de rifa que cuesta \$ $1\,000$. Juan aporta con \$ $240$, Luis con \$ $360$ y Rosa aporta el resto. El premio es de \$ $60\,000$ Deciden, en caso de ganarlo repartirlo en forma directamente proporcional al aporte de cada uno, ¿Qué cantidad de dinero le correspondería a Rosa?
\begin{vertical}
\alternativa \$ $30\,000$
\alternativa \$ $18\,000$
\alternativa \$ $24\,000$
\alternativa \$ $20\,000$
\alternativa \$ $40\,000$
\end{vertical}

\pregunta $3^3 + 3^3 + 3^3 =$
\begin{vertical}
\alternativa $3^4$
\alternativa $3^5$
\alternativa $3^9$
\alternativa $9^3$
\alternativa $9^9$
\end{vertical}

\pregunta $2^{10} + 2^{11} =$
\begin{vertical}
\alternativa $2^{21}$
\alternativa $2^{22}$
\alternativa $4^{21}$
\alternativa $6^{10}$
\alternativa $3 \cdot 2^{10}$
\end{vertical}

\pregunta $\left(\dfrac{1}{2}a^{-2}\right)^{-3} =$
\begin{vertical}
\alternativa $8a^6$
\alternativa $8a^{-5}$
\alternativa $\dfrac{1}{2}a^{-5}$
\alternativa $\dfrac{1}{8}a^{-6}$
\alternativa $\dfrac{1}{2}a^6$
\end{vertical}

\pregunta $\dfrac{1}{\sqrt{2} - 1} - \dfrac{1}{\sqrt{2}} =$
\begin{vertical}
\alternativa $1 + \sqrt{2}$
\alternativa $\dfrac{1}{2}$
\alternativa $\dfrac{1}{3}$
\alternativa $\dfrac{2 + \sqrt{2}}{2}$
\alternativa $-\dfrac{2 + \sqrt{2}}{2}$
\end{vertical}

\pregunta ¿Cuál(es) de los siguientes números corresponden a números racionales?
\begin{verticali}
\alternativa $\dfrac{\sqrt{50}}{\sqrt{8}}$
\alternativa $\left(1 + \sqrt{2}\right)^2$
\alternativa $\dfrac{1}{\sqrt{\sqrt{16}}}$
\end{verticali}
\begin{vertical}
\alternativa Solo I
\alternativa Solo II
\alternativa Solo I y III
\alternativa Solo II y III
\alternativa I, II y III
\end{vertical}

\pregunta La expresión $a^4 - b^4$ se puede escribir como
\begin{vertical}
\alternativa $\left(a-b\right)^4$
\alternativa $\left(a+b\right)^2\left(a-b\right)^2$
\alternativa $\left(a^3-b^3\right)\left(a+b\right)$
\alternativa $\left(a^2+b^2\right)\left(a^2-b^2\right)$
\alternativa $\left(a-b\right)\left(a^3+b^3\right)$
\end{vertical}

\pregunta Se tienen los números reales: $x = \dfrac{1}{\sqrt{2}}$; $y = \dfrac{2}{\sqrt{2} - 1}$; $z = \dfrac{4}{\sqrt{2} + 1}$; $w = \dfrac{\sqrt{2}}{\sqrt{2} - 1}$ ¿Cuál de las siguientes afirmaciones es (son) verdadera(s)?
\begin{verticali}
\alternativa El mayor es y.
\alternativa $y > z > x$.
\alternativa $w > z > x$.
\end{verticali}
\begin{vertical}
\alternativa Solo I
\alternativa Solo II
\alternativa Solo I y II
\alternativa Solo II y III
\alternativa I, II y III
\end{vertical}


\pregunta Si $\dfrac{2^{x+1} + 2^x}{3^x - 3^{x-2}} = \dfrac{4}{9}$, entonces el valor de $2x + 1$ es:
\begin{vertical}
\alternativa $5$
\alternativa $15$
\alternativa $14$
\alternativa $13$
\alternativa $11$
\end{vertical}

\pregunta Si $ab = \sqrt{3}$ y $b = \sqrt{3} - \sqrt{2}$, entonces $a:$
\begin{vertical}
\alternativa $3 + \sqrt{6}$
\alternativa $3 + \sqrt{3}$
\alternativa $\sqrt{3} + \sqrt{2}$
\alternativa $-\left(1 + \sqrt{2}\right)$
\alternativa $-\sqrt{2}$
\end{vertical}

\pregunta $\left(\sqrt{2}\right)^{20} \cdot \left(1 + \dfrac{1}{\sqrt{2}}\right)^{10} \cdot \left(1 - \dfrac{1}{\sqrt{2}}\right)^{10} =$
\begin{vertical}
\alternativa $1$
\alternativa $\dfrac{1}{4}$
\alternativa $\dfrac{9}{4}$
\alternativa $\dfrac{3}{4}$
\alternativa $\dfrac{9}{16}$
\end{vertical}


\pregunta ¿Cuál(es) de las siguientes igualdades es (son) verdadera(s)?
\begin{verticali}
\alternativa $\sqrt{3} \cdot \sqrt[3]{3^2} = 3$
\alternativa $\dfrac{\sqrt[3]{3}}{\sqrt[4]{3}} = \sqrt[12]{3}$
\alternativa $\sqrt[3]{3} \cdot \sqrt[4]{3} = \sqrt[7]{3}$
\end{verticali}
\begin{vertical}
\alternativa Solo I
\alternativa Solo II
\alternativa Solo I y II
\alternativa Solo II y III
\alternativa I, II y III
\end{vertical}

\pregunta Si $\log_2 8 = x$, entonces $x =$
\begin{vertical}
\alternativa $-3$
\alternativa $2\sqrt{2}$
\alternativa $3$
\alternativa $4$
\alternativa $5$
\end{vertical}

\pregunta $\log 2 + \log 8 - \log 4 =$
\begin{vertical}
\alternativa $\log 4$
\alternativa $\log 6$
\alternativa $\log 8$
\alternativa $\log 12$
\alternativa $\log \left(\dfrac{5}{2}\right)$
\end{vertical}

\pregunta Si $\log_3 x = -2$, entonces $x =$
\begin{vertical}
\alternativa $-9$
\alternativa $-6$
\alternativa $0,\overline{1}$
\alternativa $0,\overline{3}$
\alternativa $9$
\end{vertical}

\pregunta Si $\log \left(x + 1\right) = 2$, entonces $x =$
\begin{vertical}
\alternativa $19$
\alternativa $21$
\alternativa $99$
\alternativa $101$
\alternativa $1\,023$
\end{vertical}

\pregunta Sean $P = \log_2 \sqrt[3]{4}$, $Q = \log_4 \sqrt[3]{4}$ y $R = \log_8 \sqrt[3]{4}$, ¿cuál(es) de las siguientes afirmaciones es (son) verdadera(s)?
\begin{verticali}
\alternativa $Q = \dfrac{P}{2}$
\alternativa $R = \dfrac{P}{3}$
\alternativa $PQ = R$
\end{verticali}
\begin{vertical}
\alternativa Solo I
\alternativa Solo II
\alternativa Solo I y II
\alternativa Solo II y III
\alternativa I, II y III
\end{vertical}

\pregunta $\log_2 \left(\log_9 \left(\log_5 125\right)\right) =$
\begin{vertical}
\alternativa $2$
\alternativa $-2$
\alternativa $1$
\alternativa $-1$
\alternativa $0$
\end{vertical}

\pregunta Si a y b son números positivos, se puede determinar que $a = b^2$, si:
\begin{verticaln}
\alternativa $\log a = 2 \log b$
\alternativa $\log \left(\dfrac{a}{b^2}\right) = 0$
\end{verticaln}
\begin{vertical}
\alternativa (1) por sí sola
\alternativa (2) por sí sola
\alternativa Ambas juntas, (1) y (2)
\alternativa Cada una por sí sola, (1) ó (2)
\alternativa Se requiere información adicional
\end{vertical}

\pregunta $\log \left(\dfrac{\sqrt{6} + 3}{\sqrt{2} + \sqrt{3}}\right) =$
\begin{vertical}
\alternativa $\dfrac{1}{2} \log 3$
\alternativa $\log 3$
\alternativa $2 \log 3$
\alternativa $\log 6$
\alternativa $\log 2$
\end{vertical}

\pregunta La masa de un material radioactivo medida en kilogramos, está dada por la expresión $m\left(t\right) = 4 \cdot \left(0,2\right)^t$, donde t es el tiempo medido en años. ¿Cuántos años deben transcurrir para que la masa del material quede reducida a dos kilogramos?
\begin{vertical}
\alternativa $\log 2,5$
\alternativa $\dfrac{\log 5}{\log 2}$
\alternativa $\log 5 - \log 2$
\alternativa $\dfrac{\log 2}{1 - \log 2}$
\alternativa Todas las anteriores.
\end{vertical}

\end{preguntas}


\end{document}