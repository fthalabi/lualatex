\def\curso{Octavo básico B}
\def\puntaje{XXXXXXXXXXXXX}
\def\titulo{Tarea}
\def\subtitulo{Teorema de Moivre}
\def\fecha{YYYYYYYYYYYYYY, 2025}
\documentclass[sin nombre]{srs}

\begin{document}

\section*{Objetivo}
  Resolver operaciones con números complejos en su forma trigonométrica y comparar
  donde sea posible con la metodología de la forma binomial.

\section*{Instrucciones generales}
  Entregue junto con este enunciado un informe detallando la solución a las
  distintas problemáticas planteadas. La tarea es individual, con nota y tiene
  que entregarla en la clase el día ZZ de mayo. Se penalizará entregas atrasadas
  con un punto menos en la nota final por cada día de atraso.

\section*{Criterios de evaluación}
  Para la corrección de esta tarea, se asignará el puntaje según los criterios que
  se encuentran explicados en la siguiente rubrica.

\begin{center}
  \begin{tblr}{width=\linewidth,colspec={XXX}, hline{1,Z} = {1}{-}{}, hline{1,Z} = {2}{-}{},
      hlines,row{3,5,7}={font={\raggedright\small}},row{2,4,6} = {bg=black!15,halign=c},
      row{1}={halign=c},rows={rowsep=5pt}}
      %%% CABEZERA
      Logrado & Suficiente & Insuficiente \\
      %%% PRESENTACION
      \SetCell[c=3]{c} Presentación (puntaje por informe)& & \\
      La tarea se presenta de forma impecable: letra clara y fácilmente
      legible en todo el documento, sin errores de ortografía.
      El papel está limpio, sin arrugas, dobleces ni manchas. \mbox{(5 puntos)}&
      La tarea es legible en su mayor parte, aunque podría mejorar la claridad
      de la letra en algunas secciones. Presenta errores ortográficos mínimos (1-2)
      y/o pequeñas imperfecciones en el papel (manchas leves, arrugas menores) que no
      dificultan significativamente la lectura. \mbox{(3 puntos)}&
      La letra es difícil de entender en varias secciones, presenta múltiples
      errores de ortografía (3 o más) y/o el papel está descuidado
      (arrugado, manchado, doblado) dificultando la revisión y lectura. \mbox{(0 puntos)}\\
      %%% PROCESOS Y JUSTICACACION
      \SetCell[c=3]{c} Procesos y justificación (puntaje por pregunta) & & \\
      Detalla cada paso del procedimiento de forma clara, ordenada y sistemática.
      Justifica adecuadamente las operaciones, propiedades o estrategias matemáticas
      utilizadas, demostrando una comprensión profunda del problema y cómo se alcanza
      la solución. El razonamiento es fácil de seguir. \mbox{(3 puntos)}&
      Presenta la mayoría de los pasos del procedimiento, aunque algunos podrían ser
      más detallados o claros. La justificación es adecuada en general, pero puede
      haber omisiones menores o falta de precisión en alguna explicación.
      El proceso general es comprensible, aunque con algunas dificultades para
      seguir el razonamiento en puntos específicos. \mbox{(2 puntos)}&
      Los pasos del procedimiento son confusos, incompletos, desordenados o ausentes.
      La justificación es escasa, ausente o incorrecta. No es posible seguir el
      razonamiento para entender cómo se intentó llegar al resultado. \mbox{(0 puntos)}\\
      %%% RESULTADOS
      \SetCell[c=3]{c} Resultados (puntaje por pregunta)& & \\
      El resultado alcanzado es correcto y está desarrollado completamente, es decir,
      no se dejaron valores expresados y/o sin simplificar. \mbox{(3 puntos)} &
      El resultado está casi correcto. El error es aritmético y no conceptual,
      o simplemente falto desarrollar la respuesta. \mbox{(2 puntos)}&
      El resultado no responde de ninguna manera al enunciado y/o hay errores que
      son fundamentales. \mbox{(0 puntos)}\\
  \end{tblr}
\end{center}

  Para verificar que la tarea se realizó de manera honesta, es posible que tenga
  explicar sus procesos y/o resultados en una interrogación.

\newpage

%\begin{aviso}[after skip=3cm]
%  Debe responder solo los problemas destacados, ignore el resto.
%\end{aviso}
%hola

\newcounter{mytaskcounter}


\NewDocumentCommand{\superbox}{m}{%
\ifthenelse{\value{mytaskcounter} < 100}{%
\tcbox[colback=black!60, colframe=black!60, coltext=white,%
  on line,boxsep=0pt, left=1pt, right=1pt, top=1pt, bottom=1pt,%
  width=1cm]{\makebox[\widthof{22}][c]{#1}}}{%
\tcbox[colback=black!60, colframe=black!60, coltext=white,%
  on line,boxsep=0pt, left=1pt, right=1pt, top=1pt, bottom=1pt,%
  width=1cm]{#1}}}

\NewTasksEnvironment[label=\superbox{\sffamily\bfseries\arabic*},
label-width=\labelwidth,item-indent=30pt,resume=true,
label-offset=10pt,column-sep=10pt,counter=mytaskcounter]{preguntitas}[\preguntita](1)



%\superbox{1}hoas\\
%\superbox{8}asas\\
%\superbox{12}asdasd\\
%\superbox{22}adsads\\
%\superbox{105}ads\\
%\superbox{111}ads\\
%\superbox{999}asd\\
%\superbox{1294}asd\\
%\superbox{9999}asd\\

\begin{preguntitas}
  \preguntita asdasd
  \preguntita asdasd
  \preguntita asdasd
  \preguntita asdasd
  \preguntita asdasd
  \preguntita asdasd
  \preguntita asdasd
  \preguntita asdasd
  \preguntita asdasd
  \preguntita asdasd
  \preguntita asdasd
  \preguntita asdasd
  \preguntita asdasd
  \preguntita asdasd
  \preguntita asdasd
  \preguntita asdasd
  \preguntita asdasd
  \preguntita asdasd
  \preguntita asdasd
  \preguntita asdasd
  \preguntita asdasd
  \preguntita asdasd
  \preguntita asdasd
  \preguntita asdasd
  \preguntita asdasd
  \preguntita asdasd
  \preguntita asdasd
  \preguntita asdasd
  \preguntita asdasd
  \preguntita asdasd
  \preguntita asdasd
  \preguntita asdasd
  \preguntita asdasd
  \preguntita asdasd
  \preguntita asdasd
  \preguntita asdasd
  \preguntita asdasd
  \preguntita asdasd
  \preguntita asdasd
  \preguntita asdasd
  \preguntita asdasd
  \preguntita asdasd
  \preguntita asdasd
  \preguntita asdasd
  \preguntita asdasd
  \preguntita asdasd
  \preguntita asdasd
  \preguntita asdasd
  \preguntita asdasd
  \preguntita asdasd
  \preguntita asdasd
  \preguntita asdasd
  \preguntita asdasd
  \preguntita asdasd
  \preguntita asdasd
  \preguntita asdasd
  \preguntita asdasd
  \preguntita asdasd
  \preguntita asdasd
  \preguntita asdasd
  \preguntita asdasd
  \preguntita asdasd
  \preguntita asdasd
  \preguntita asdasd
  \preguntita asdasd
  \preguntita asdasd
  \preguntita asdasd
  \preguntita asdasd
  \preguntita asdasd
  \preguntita asdasd
  \preguntita asdasd
  \preguntita asdasd
  \preguntita asdasd
  \preguntita asdasd
  \preguntita asdasd
  \preguntita asdasd
  \preguntita asdasd
  \preguntita asdasd
  \preguntita asdasd
  \preguntita asdasd
  \preguntita asdasd
  \preguntita asdasd
  \preguntita asdasd
  \preguntita asdasd
  \preguntita asdasd
  \preguntita asdasd
  \preguntita asdasd
  \preguntita asdasd
  \preguntita asdasd
  \preguntita asdasd
  \preguntita asdasd
  \preguntita asdasd
  \preguntita asdasd
  \preguntita asdasd
  \preguntita asdasd
  \preguntita asdasd
  \preguntita asdasd
  \preguntita asdasd
  \preguntita asdasd
  \preguntita asdasd
  \preguntita asdasd
  \preguntita asdasd
  \preguntita asdasd
  \preguntita asdasd
  \preguntita asdasd
  \preguntita asdasd
  \preguntita asdasd
  \preguntita asdasd
  \preguntita asdasd
  \preguntita asdasd
  \preguntita asdasd
  \preguntita asdasd
  \preguntita asdasd
  \preguntita asdasd
  \preguntita asdasd
  \preguntita asdasd
  \preguntita asdasd
  \preguntita asdasd
  \preguntita asdasd
  \preguntita asdasd
  \preguntita asdasd
  \preguntita asdasd
  \preguntita asdasd
  \preguntita asdasd
  \preguntita asdasd
  \preguntita asdasd
  \preguntita asdasd
  \preguntita asdasd
  \preguntita asdasd
  \preguntita asdasd
  \preguntita asdasd
  \preguntita asdasd
  \preguntita asdasd
  \preguntita asdasd
  \preguntita asdasd
  \preguntita asdasd
  \preguntita asdasd
  \preguntita asdasd
  \preguntita asdasd
  \preguntita asdasd
  \preguntita asdasd
  \preguntita asdasd
  \preguntita asdasd


\end{preguntitas}

\newcounter{testcounter}
\setcounter{testcounter}{3}

\ifnum\value{testcounter}=0
the counter is 0
\else
the counter is not 0
\fi


\end{document}