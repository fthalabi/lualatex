\documentclass[12pt]{article}
\usepackage[utf8]{inputenc}
\usepackage{amsmath, amssymb}
\usepackage{geometry}
\geometry{margin=1in}
\usepackage{longtable}
\usepackage{graphicx}

\title{Cuadernillo de Repaso: Números}
\author{}
\date{}

\begin{document}

\maketitle

\section*{Conjuntos Numéricos}

Los conjuntos numéricos que has estudiado son los siguientes:

\begin{itemize}
  \item \textbf{Números naturales}: $\mathbb{N} = \{1, 2, 3, \dots\}$
  \item \textbf{Números enteros}: $\mathbb{Z} = \{\dots, -3, -2, -1, 0, 1, 2, 3, \dots\}$
  \item \textbf{Números racionales}: Se pueden expresar como fracción, incluyen decimales finitos, periódicos, semiperiódicos y enteros. $\mathbb{Q} = \frac{a}{b}$ con $a, b \in \mathbb{Z}$ y $b \neq 0$.
  \item \textbf{Números irracionales}: No se pueden expresar como fracción, tienen infinitas cifras decimales sin período (ej.: $\sqrt{3}, 2\sqrt{3}-1, \pi$). Se designan $\mathbb{Q}'$ o $\mathbb{I}$.
  \item \textbf{Números reales}: $\mathbb{R} = \mathbb{Q} \cup \mathbb{Q}'$
  \item \textbf{Números complejos}: De la forma $a + bi$ con $a, b \in \mathbb{R}$ e $i^2 = -1$, $\mathbb{C} = \{z = a + bi \mid a, b \in \mathbb{R}\}$
\end{itemize}

\textbf{Observación}: No son números reales las raíces de índice par de negativos (ej.: $\sqrt{-9}, \sqrt{-164}$) ni divisiones por cero.

\section*{Conversión de Decimal a Fracción}

\begin{itemize}
  \item Decimal finito: Escribe el número sin coma en el numerador; en el denominador, un 1 seguido de tantos ceros como cifras decimales.
  \item Decimal infinito periódico: Numerador: número sin coma; denominador: tantos nueves como cifras periódicas.
  \item Decimal infinito semiperiódico: Numerador: resta del número sin coma menos el anteperíodo; denominador: nueves por cifras periódicas seguidos de ceros por cifras de anteperíodo.
\end{itemize}

\section*{Propiedad de Clausura}

Las operaciones básicas en $\mathbb{R}$ cumplen clausura (salvo división por cero). Los irracionales no cumplen clausura bajo suma, resta, multiplicación ni división.

\section*{Aproximaciones de Números Reales}

\begin{itemize}
  \item Redondeo: Observa la cifra siguiente; si $\geq 5$, aumenta en 1.
  \item Truncamiento: Solo se consideran las cifras pedidas.
  \item Por defecto: Aproximación menor más cercana.
  \item Por exceso: Aproximación mayor más cercana.
\end{itemize}

\section*{Orden en los Números Reales}

Se puede comparar usando denominadores comunes, numeradores comunes o convirtiendo a decimal.

\section*{Ejercicios Resueltos y de Práctica}

\textit{\textbf{(Aquí se colocan todos los ejemplos y ejercicios resueltos como en el PDF original, numerados y sin alteración alguna)}}.

\section*{Respuestas}

\begin{tabular}{|c|p{12cm}|}
\hline
\textbf{Capítulo 1} & 1.B 2.D 3.C 4.D 5.B 6.E 7.D 8.A 9.C 10.A 11.D 12.C 13.C 14.C 15.E 16.D 17.C 18.D 19.B 20.D 21.C 22.D 23.E 24.C 25.B 26.A 27.E 28.C 29.E 30.D 31.E 32.C 33.E 34.B 35.E 36.D 37.D 38.D 39.E 40.C 41.D 42.A 43.C 44.B 45.A 46.C 47.D 48.C 49.D
\end{tabular}

\end{document}
