\def\curso{Tercero medio}
\def\puntaje{20}
\def\titulo{Miniensayo}
\def\subtitulo{Eje de números (1)}
\def\fecha{6 de junio, 2025}
\documentclass[revolver]{srs}

\begin{document}

\subsection*{Objetivo}
Realizar cálculos y solucionar problemas utilizando propiedades de números racionales,
raíces, potencias y logaritmos.

\subsection*{Instrucciones generales}
Tiene 1 hora y 30 minutos para responder la evaluación. Esta es individual y debe
usar solo sus materiales personales para trabajar durante este periodo, no los solicite
a un compañero durante la evaluación.

Para cada pregunta, lea con atención el enunciado y escoja la alternativa que lo
responde correctamente. Solo hay una alternativa correcta por cada pregunta.

\subsection*{Criterios de evaluación}
Se asignará 1 puntos por cada pregunta contestada correctamente. En caso de marcar múltiples
alternativas en una misma pregunta, esto invalidará la respuesta y se considerará incorrecta.

\separador[2mm]

\begin{preguntas}[after-item-skip=2cm]
\pregunta Si el producto $0,22 \cdot 0,1\overline{6}$ se trunca a dos decimales resulta:
\begin{vertical}
\alternativa $0,02$
\alternativa $0,03$
\alternativa $0,04$
\alternativa $0,05$
\alternativa $0,35$
\end{vertical}

\pregunta El $20\%$ del área de un cuadrado es $20 \text{ cm}^2$, ¿Cuál es su perímetro?
\begin{vertical}
\alternativa $100 \text{ cm}$
\alternativa $40 \text{ cm}$
\alternativa $25 \text{ cm}$
\alternativa $20 \text{ cm}$
\alternativa $10 \text{ cm}$
\end{vertical}

\pregunta ¿Para cuál(es) de los siguientes números reales, su raíz cuadrada es un número racional?
\begin{verticali}
\alternativa $16,9 \cdot 10^{-5}$
\alternativa $1\,960\,000$
\alternativa $\dfrac{196 \cdot 10^{-3}}{169 \cdot 10^{-7}}$
\end{verticali}
\begin{vertical}
\alternativa Solo I
\alternativa Solo II
\alternativa Solo I y II
\alternativa Solo II y III
\alternativa I, II y III
\end{vertical}

\pregunta Si $P = 0,\overline{24}$, $Q = \dfrac{121}{500}$ y $R = \dfrac{11}{45}$, entonces al ordenarlos en forma creciente, resulta:
\begin{vertical}
\alternativa $P < Q < R$
\alternativa $R < Q < P$
\alternativa $Q < R < P$
\alternativa $R < P < Q$
\alternativa $Q < P < R$
\end{vertical}

\pregunta Si $h$ hombres pueden fabricar 50 artículos en un día, ¿cuántos hombres se necesitan para fabricar $x$ artículos en un día?
\begin{vertical}
\alternativa $\dfrac{hx}{50}$
\alternativa $\dfrac{50x}{h}$
\alternativa $\dfrac{x}{50h}$
\alternativa $\dfrac{h}{50x}$
\alternativa Ninguna de las alternativas
\end{vertical}

\pregunta ¿Qué $\%$ es $\dfrac{6}{25}$ de $\dfrac{3}{5}$?
\begin{vertical}
\alternativa $20\%$
\alternativa $25\%$
\alternativa $40\%$
\alternativa $45\%$
\alternativa $60\%$
\end{vertical}

\pregunta A un evento asistieron 56 personas. Si había 4 mujeres por cada 3 hombres, ¿cuántas mujeres asistieron al evento?
\begin{vertical}
\alternativa 8
\alternativa 21
\alternativa 24
\alternativa 28
\alternativa 32
\end{vertical}

\pregunta El $20\%$ de $\left(x + y\right)$ equivale a los $\dfrac{4}{5}$ de $\left(x - y\right)$, entonces $\dfrac{x}{y} =$
\begin{vertical}
\alternativa $\dfrac{3}{4}$
\alternativa $\dfrac{3}{5}$
\alternativa $\dfrac{4}{3}$
\alternativa $1$
\alternativa $\dfrac{5}{3}$
\end{vertical}

\pregunta El $30\%$ de $a$ equivale al $20\%$ de $b$. Si $b = 150$, ¿qué parte es $a$ de $b$?
\begin{vertical}
\alternativa $\dfrac{2}{3}$
\alternativa $\dfrac{3}{2}$
\alternativa $\dfrac{1}{2}$
\alternativa $\dfrac{2}{5}$
\alternativa $\dfrac{1}{4}$
\end{vertical}

\pregunta Un artículo tiene un costo de \$A y se vende en \$B ($B > A$), ¿cuál es el porcentaje de ganancia?
\begin{vertical}
\alternativa $\left(\dfrac{A - B}{A}\right) \cdot 100\%$
\alternativa $\left(\dfrac{B - A}{A}\right) \cdot 100\%$
\alternativa $\left(\dfrac{B - A}{B}\right) \cdot 100\%$
\alternativa $\left(\dfrac{B - A}{A + B}\right) \cdot 100\%$
\alternativa $\left(AB\right) \cdot 100\%$
\end{vertical}

\pregunta En la siguiente tabla, se muestra la distribución de ausentes/presentes por género en un día de clases, siendo $n$ el total de alumnos:
\begin{centrado}
\begin{tblr}{|l|c|c|}
\hline
 & Presentes & Ausentes \\
\hline
Hombres & a & c \\
Mujeres & b & d \\
\hline
\end{tblr}
\end{centrado}
¿Cuál de las siguientes afirmaciones es FALSA?
\begin{vertical}
\alternativa El porcentaje de presentes ese día fue $\left(\dfrac{a + b}{n}\right) \cdot 100\%$
\alternativa El porcentaje de mujeres del curso es $\left(\dfrac{b + d}{n}\right) \cdot 100\%$
\alternativa De las mujeres, el porcentaje que asistió ese día fue $\left(\dfrac{b}{n}\right) \cdot 100\%$
\alternativa Del curso, el porcentaje de los hombres ausentes ese día fue $\left(\dfrac{c}{n}\right) \cdot 100\%$
\end{vertical}

\pregunta En un rectángulo, el largo aumenta un $30\%$ y el ancho disminuye un $30\%$, entonces su área
\begin{vertical}
\alternativa queda igual.
\alternativa aumenta un $3\%$.
\alternativa disminuye en un $9\%$.
\alternativa sube en un $10\%$.
\alternativa disminuye en un $10\%$.
\end{vertical}

\pregunta En una reserva forestal, la cantidad de hectáreas de árboles disminuye a una tasa de un $20\%$ anual. ¿Cuál de las siguientes ecuaciones nos permite determinar la cantidad de años t que deben transcurrir para que la cantidad de hectáreas iniciales C se haya reducido a un $1\%$?
\begin{vertical}
\alternativa $C \cdot \left(0,8\right)^t = 0,01 C$
\alternativa $C \cdot \left(0,8\right)^t = 0,99 C$
\alternativa $C \cdot \left(0,2\right)^t = 0,99 C$
\alternativa $C \cdot \left(1,2\right)^t = 0,99 C$
\alternativa $C \cdot \left(1,2\right)^t = 1 + 0,99 C$
\end{vertical}

\pregunta $\dfrac{2^4 + 2^5}{2^6 + 2^7} =$
\begin{vertical}
\alternativa $2^{-4}$
\alternativa $2^{-2}$
\alternativa $2^{-1}$
\alternativa $2^2$
\alternativa $2^3$
\end{vertical}


\pregunta Se puede determinar la potencia $a^n$, con a y n racionales y $a \neq 0$, si se sabe que:
\begin{verticaln}
\alternativa $a^{-2n} = 9$
\alternativa $a^{3n} = -\dfrac{1}{27}$
\end{verticaln}
\begin{vertical}
\alternativa (1) por sí sola
\alternativa (2) por sí sola
\alternativa Ambas juntas, (1) y (2)
\alternativa Cada una por sí sola, (1) ó (2)
\alternativa Se requiere información adicional
\end{vertical}

\pregunta El resultado de $\dfrac{1 + \frac{1}{\sqrt{2}}}{\sqrt{2} - 1}$ es un número real que está entre:
\begin{vertical}
\alternativa $1$ y $2$
\alternativa $2$ y $3$
\alternativa $3$ y $4$
\alternativa $4$ y $5$
\alternativa $5$ y $6$
\end{vertical}

\pregunta $\left(\sqrt{3} - \sqrt{2}\right) \cdot \sqrt{5 + 2\sqrt{6}} =$
\begin{vertical}
\alternativa $1$
\alternativa $2$
\alternativa $\sqrt{6}$
\alternativa $2\sqrt{6}$
\alternativa $7$
\end{vertical}

\pregunta Si $0 < a < 2$, entonces $\sqrt{a^2 - 4a + 4} + \sqrt{a^2 + 4a + 4} =$
\begin{vertical}
\alternativa $2a$
\alternativa $4a$
\alternativa $2$
\alternativa $4$
\alternativa $-2$
\end{vertical}

\pregunta Si $x = \dfrac{1}{2\sqrt{3}}$, $y = \dfrac{\sqrt{7}}{3}$, $z = \dfrac{\sqrt{10}}{4}$ y $w = \dfrac{\sqrt{18}}{5}$, entonces:
\begin{vertical}
\alternativa $z < x < w < y$
\alternativa $z < w < y < x$
\alternativa $z < w < x < y$
\alternativa $w < z < x < y$
\alternativa $y < x < w < z$
\end{vertical}

\pregunta $\log_{3} \sqrt{0,\overline{1}} =$
\begin{vertical}
\alternativa $-1$
\alternativa $1$
\alternativa $2$
\alternativa $-2$
\alternativa $\dfrac{2}{3}$
\end{vertical}

\pregunta Si $p = \log_4 \sqrt{2}$, $4 = \log_q 16$ y $2 = \log_4 r$, entonces ¿cuál(es) de las siguientes afirmaciones es (son) verdadera(s)?
\begin{verticali}
\alternativa $pr = 2q$
\alternativa $pqr = 8$
\alternativa $r^p = q$
\end{verticali}
\begin{vertical}
\alternativa Solo I
\alternativa Solo II
\alternativa Solo I y II
\alternativa Solo II y III
\alternativa I, II y III
\end{vertical}

\pregunta $\log_2\left(\log_4 \left(\log_2 \sqrt[3]{4^6}\right)\right) =$
\begin{vertical}
\alternativa $-1$
\alternativa $1$
\alternativa $0$
\alternativa $2$
\alternativa $\log 2$
\end{vertical}


\end{preguntas}

\end{document}