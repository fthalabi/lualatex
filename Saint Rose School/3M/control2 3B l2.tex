\def\curso{Tercero medio B}
\def\puntaje{7}
\def\titulo{Control}
\def\subtitulo{Trigonometría}
\def\fecha{28 de abril, 2025}
\documentclass[]{srs}

\begin{document}

\section*{Objetivo}
Usar las propiedades geométricas del triángulo rectángulo para determinar el seno y coseno de un ángulo. Además de, usar la
simetría de las funciones trigonométricas para determinar valores entre $90^{\circ}$ y
$360^{\circ}$, y la periodicidad para calcular valores afuera del rango $0^{\circ}$ a
$360^{\circ}$.

\section*{Instrucciones}
  Cuenta con 40 minutos para completar  la evaluación. Incluya desarrollo en todas
  sus respuestas, y recuerde marcar o señalizar el resultado final en cada pregunta.

\section*{Criterios de evaluación}
  En la corrección, se asignará el puntaje a cada pregunta según los siguientes criterios.
\begin{center}
  \begin{tblr}{width=\linewidth,colspec={X[1,c]|X[6]}, hline{1,Z} = {1}{-}{}, hline{1,Z} = {2}{-}{},
      hlines, cells={valign=m}, row{1} = {bg=black!15}}
      Puntaje asignado & \SetCell{c} Criterios o indicadores \\
      +50\% & Señala clara y correctamente cuál es la solución o el resultado de la pregunta hecha
      en el enunciado.\\
      +50\% & Incluye un desarrollo que relata de manera clara y ordenada los procedimientos
      \mbox{necesarios} para solucionar la problemática. En caso de estar incompleto o con
      errores el desarrollo, se asignará puntaje parcial si se muestra dominio de los
       contenidos y conceptos involucrados.\\
      0\% &  La respuesta es incorrecta. De haber desarrollo, este tiene errores conceptuales.\\
  \end{tblr}
\end{center}
\separador[2mm]

\begin{preguntas}(1)
  \pregunta Utilice la geometría del triángulo rectángulo para determinar y demostrar
  el valor de la función coseno para los ángulos de 30 y 60 grados. [2 puntos]
  \begin{malla}[height=9cm]
  \end{malla}
  \pregunta Llene la tabla con las coordenadas de cada punto marcado en la gráfica
  de la función coseno. \\\mbox{[3 puntos]}
\tcbsidebyside[sidebyside adapt=left,blank]{%
\begin{tikzpicture}
  \begin{axis}[
    title=Gráfica de la función coseno,
    eje escolar,
    x post scale=1.5,
    y=3cm,
    cycle list name=linestyles,
    extra tick style={grid=major,yticklabel=\empty,xticklabel=\empty},
    ytick=\empty,
    xtick=\empty,
    extra y ticks={-1,-0.86,-0.7,-0.5,0.5,0.7,0.86,1},
    extra x ticks={pi/6,pi/4,pi/3,pi/2,4*pi/6,3*pi/4,5*pi/6,pi,7*pi/6,5*pi/4,
    4*pi/3,3*pi/2,5*pi/3,7*pi/4,11*pi/6,2*pi},
    %xticklabels={$\dfrac{\pi}{2}$,$\pi$,$\dfrac{3\pi}{2}$,$2\pi$},
    smooth,
    enlarge y limits=0.2,
    trig format plots=rad,
    legend style={
      at={(0.95,0.95)},
      anchor=north east,
      cells={anchor=west},
      nodes={
        inner xsep=0.1em,  % Reduce horizontal spacing between symbol and text
        inner ysep=0.1em,  % Reduce vertical spacing between legend entries
      }
    }
    ]
    \addplot[domain=0:2*pi,no marks] {cos(x)};
    \addplot[nodes near coords,
    nodes near coords style={yshift=10pt,fill=white,rounded corners,inner sep=2pt},
    only marks,point meta=explicit symbolic,
    only marks,mark=*,mark options={scale=1, fill=black}]
      coordinates {
        (0,1) [$A$]
        (pi/6,0.86) [$B$]
        (pi/4,0.7) [$C$]
        (pi/3,0.5) [$D$]
        (pi/2,0) [$E$]
        (4*pi/6,-0.5) [$F$]
        (3*pi/4,-0.7) [$G$]
        (5*pi/6,-0.86) [$H$]
        (pi,-1) [$I$]
        (7*pi/6,-0.86) [$J$]
        (5*pi/4,-0.7) [$K$]
        (4*pi/3,-0.5) [$L$]
        (3*pi/2,0) [$M$]
        (5*pi/3,0.5) [$N$]
        (7*pi/4,0.7) [$O$]
        (11*pi/6,0.86) [$P$]
        (2*pi,1) [$Q$]
      };
  \end{axis}
\end{tikzpicture}%
}{%
\begin{tblr}{colspec={X[1,c]X[2,c]},
  hlines,vlines, cells={valign=m}, row{1} = {bg=black!15}}
  Punto & Coordenadas \\
  $A$ & $\left(\;0^{\circ}\;,\;\; 1\;\right)$ \\
  $B$ & \\
  $C$ & \\
  $D$ & \\
  $E$ & $\left(\;90^{\circ}\;,\;\; 0\;\right)$ \\
  $F$ & \\
  $G$ & \\
  $H$ & \\
  $I$ & \\
  $J$ & \\
  $K$ & \\
  $L$ & \\
  $M$ & \\
  $N$ & \\
  $O$ & \\
  $P$ & \\
  $Q$ & \\
\end{tblr}
}
\end{preguntas}

Utilice la periodicidad de las funciones trigonométricas para determinar los siguientes
valores. \\\mbox{[0.5 puntos C/U]}

\begin{preguntas}(2)
  \pregunta $\text{sen}\,(780^{\circ})$
  \begin{malla}[height=3cm]
  \end{malla}
  \pregunta $\text{cos}\,(-225^{\circ})$
  \begin{malla}[height=3cm]
  \end{malla}
  \pregunta $\text{sen}\,(1080^{\circ})$
  \begin{malla}[height=3cm]
  \end{malla}
  \pregunta $\text{cos}\,(1140^{\circ})$
  \begin{malla}[height=3cm]
  \end{malla}
\end{preguntas}

\end{document}