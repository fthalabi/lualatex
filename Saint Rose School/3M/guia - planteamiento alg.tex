\def\titulo{Guía}
\def\subtitulo{Planteamiento algebraico}
\def\curso{Tercero medio A}
\documentclass[sin nombre]{srs}
\begin{document}

\separador

\begin{preguntas}[after-item-skip=2cm]
\pregunta ¿Qué número entero es tal que al sumarle el triple de su antecesor da 77?
\begin{vertical}
\alternativa $19$
\alternativa $20$
\alternativa $21$
\alternativa $22$
\alternativa $24$
\end{vertical}

\pregunta ¿Qué número par es tal que al sumarlo con su par sucesor da 42?
\begin{vertical}
\alternativa $10$
\alternativa $12$
\alternativa $20$
\alternativa $22$
\alternativa $24$
\end{vertical}

\pregunta En un rectángulo el largo mide $3 \text{ cm}$ más que el ancho y su perímetro es $54 \text{ cm}$. ¿Cuánto mide su largo?
\begin{vertical}
\alternativa $12 \text{ cm}$
\alternativa $15 \text{ cm}$
\alternativa $18 \text{ cm}$
\alternativa $25,5 \text{ cm}$
\alternativa $28,5 \text{ cm}$
\end{vertical}

\pregunta Una piscina está llena hasta los $\dfrac{3}{5}$ de su capacidad. Si le faltan $1\,200$ litros para llenarla, ¿cuál es su capacidad?
\begin{vertical}
\alternativa $480 \text{ litros}$
\alternativa $800 \text{ litros}$
\alternativa $2\,000 \text{ litros}$
\alternativa $3\,000 \text{ litros}$
\alternativa $3\,200 \text{ litros}$
\end{vertical}

\pregunta Si se divide el sucesor del doble de un número con el antecesor del número resulta $3$, entonces ¿cuál es el sucesor del cuadrado del número?
\begin{vertical}
\alternativa $4$
\alternativa $5$
\alternativa $10$
\alternativa $17$
\alternativa $26$
\end{vertical}

\pregunta Un número entero sumado con el doble de su antecesor resulta 28. ¿Cuál es el antecesor el número?
\begin{vertical}
\alternativa $9$
\alternativa $10$
\alternativa $11$
\alternativa $21$
\alternativa $22$
\end{vertical}

\pregunta Se tienen tres números consecutivos, tales que la diferencia entre el cuadrado del intermedio, con el cuadrado del menor equivale al triple del mayor. Entonces uno de estos números puede ser:
\begin{vertical}
\alternativa $-6$
\alternativa $-5$
\alternativa $4$
\alternativa $5$
\alternativa $7$
\end{vertical}

\pregunta Si un sector rectangular tiene un perímetro de $40 \text{ m}$ y es de tal manera que su ancho tiene $4 \text{ m}$ menos que su largo, la ecuación que permite conocer el ancho “x” es:
\begin{vertical}
\alternativa $2x + 4 = 40$
\alternativa $4x + 4 = 40$
\alternativa $2x - 4 = 20$
\alternativa $2x + 4 = 20$
\alternativa $2x + 8 = 20$
\end{vertical}

\pregunta Raúl compró $1,2$ kilogramos de pan más una bolsa de papel de $\$50$, pagó con un billete de $\$1\,000$ y una moneda de $\$100$, y recibió un vuelto de $\$30$. Según la información dada, ¿cuánto cuesta un kilogramo de pan?
\begin{vertical}
\alternativa $\$833,3$
\alternativa $\$841,6$
\alternativa $\$850$
\alternativa $\$930$
\alternativa $\$933,3$
\end{vertical}

\pregunta Se pueden determinar tres números consecutivos, sabiendo que:
\begin{verticaln}
\alternativa La suma del menor con el mayor da el doble del central.
\alternativa La suma de los tres números es $33$.
\end{verticaln}
\begin{vertical}
\alternativa (1) por sí sola
\alternativa (2) por sí sola
\alternativa Ambas juntas, (1) y (2)
\alternativa Cada una por sí sola, (1) ó (2)
\alternativa Se requiere información adicional
\end{vertical}

\pregunta Las edades de dos hermanos suman $40$ años y uno tiene los $\dfrac{3}{5}$ de los que tiene el otro. ¿Cuál es la diferencia entre sus edades?
\begin{vertical}
\alternativa $10 \text{ años}$
\alternativa $15 \text{ años}$
\alternativa $20 \text{ años}$
\alternativa $25 \text{ años}$
\alternativa $30 \text{ años}$
\end{vertical}

\pregunta En un triángulo el ángulo menor mide $26^\circ$ menos que el del medio y $28^\circ$ menos que el mayor. ¿Cuánto mide el mayor de los ángulos interiores?
\begin{vertical}
\alternativa $42^\circ$
\alternativa $60^\circ$
\alternativa $68^\circ$
\alternativa $70^\circ$
\alternativa $78^\circ$
\end{vertical}

\pregunta En un jardín hay $31$ flores entre calas, orquídeas y pensamientos. Siendo las orquídeas un tercio de los pensamientos y éstos cuatro más que las calas. ¿Cuántos son los pensamientos?
\begin{vertical}
\alternativa $6$
\alternativa $7$
\alternativa $14$
\alternativa $15$
\alternativa $21$
\end{vertical}

\pregunta Una madre reparte $\$12\,000$ entre sus dos hijos de modo que el mayor recibió $\$3\,000$ más que el doble de lo que recibió el otro, ¿cuánto recibió el menor?
\begin{vertical}
\alternativa $\$2\,000$
\alternativa $\$3\,000$
\alternativa $\$6\,000$
\alternativa $\$7\,500$
\alternativa $\$9\,000$
\end{vertical}

\pregunta Cuando uno de dos hermanos nació, el mayor tenía nueve años. Si uno de ellos tiene un año más que el doble del otro, ¿cuánto suman sus edades?
\begin{vertical}
\alternativa $8$
\alternativa $12$
\alternativa $17$
\alternativa $25$
\alternativa $34$
\end{vertical}

\pregunta Un cuadrado tiene $14 \text{ cm}$ más de perímetro que un triángulo equilátero. Si la suma de los perímetros de ambas figuras es $26 \text{ cm}$, ¿cuál es el área del cuadrado?
\begin{vertical}
\alternativa $4 \text{ cm}^2$
\alternativa $6 \text{ cm}^2$
\alternativa $8 \text{ cm}^2$
\alternativa $20 \text{ cm}^2$
\alternativa $25 \text{ cm}^2$
\end{vertical}

\pregunta Dos cajas pesan $102$ kilogramos y si se sacan $7$ kilogramos de una y se depositan en la otra, quedan iguales. ¿Cuántos kilogramos tiene la más pesada?
\begin{vertical}
\alternativa $48$
\alternativa $51$
\alternativa $56$
\alternativa $58$
\alternativa $65$
\end{vertical}

\pregunta Dos libros han costado $\$13\,000$. El doble del precio del libro más barato es $\$200$ más que el precio del otro. ¿Cuál es la diferencia entre los precios de ambos libros?
\begin{vertical}
\alternativa $\$2\,200$
\alternativa $\$3\,200$
\alternativa $\$4\,200$
\alternativa $\$4\,400$
\alternativa $\$8\,600$
\end{vertical}

\pregunta Las edades de Pedro y Luis están en la razón de $5:4$ y hace tres años estaban en la razón $4:3$. ¿Cuánto suman sus edades actuales?
\begin{vertical}
\alternativa $19 \text{ años.}$
\alternativa $21 \text{ años}$
\alternativa $24 \text{ años}$
\alternativa $27 \text{ años.}$
\alternativa $36 \text{ años.}$
\end{vertical}

\pregunta Doña Pepa lleva $3$ kilogramos de tomates y $2$ de limones en $\$1\,300$. Si hubiese llevado $2$ kilogramos de tomates y $3$ de limones le habría costado $\$100$ menos. ¿Cuánto vale cada kilogramo de limones?
\begin{vertical}
\alternativa $\$126$
\alternativa $\$150$
\alternativa $\$200$
\alternativa $\$250$
\alternativa $\$300$
\end{vertical}

\pregunta Un taller mecánico vende aceite para autos en dos formatos, bidones de dos y cinco litros cada uno. En total en el taller hay $26$ bidones y $100$ litros de aceite. ¿Cuántos bidones de dos litros hay?
\begin{vertical}
\alternativa $8$
\alternativa $10$
\alternativa $12$
\alternativa $15$
\alternativa $16$
\end{vertical}

\pregunta Hace ”a” años las edades de dos hermanos sumaban “$10a$”. ¿Cuál será el promedio de sus edades en “a” años más?
\begin{vertical}
\alternativa $5,5a$
\alternativa $6a$
\alternativa $6,5a$
\alternativa $7a$
\alternativa $12a$
\end{vertical}

\pregunta Paula ahorró $\$11\,000$ en monedas de $\$100$ y $\$500$, se puede saber cuántas monedas de cada tipo ahorró sabiendo que:
\begin{verticaln}
\alternativa Son $30$ monedas.
\alternativa Lo ahorrado en monedas de $\$500$ es $10$ veces lo ahorrado en monedas de $\$100$.
\end{verticaln}
\begin{vertical}
\alternativa (1) por sí sola
\alternativa (2) por sí sola
\alternativa Ambas juntas, (1) y (2)
\alternativa Cada una por sí sola, (1) ó (2)
\alternativa Se requiere información adicional
\end{vertical}

\pregunta En un campeonato de fútbol, si un equipo gana un partido recibe $3$ puntos y si empata gana $1$ punto. Si en $6$ partidos un equipo permanece invicto con $14$ puntos, ¿cuántos partidos ha ganado?
\begin{vertical}
\alternativa $1$
\alternativa $2$
\alternativa $3$
\alternativa $4$
\alternativa $5$
\end{vertical}

\pregunta En una compra de útiles escolares, Pedro compra dos lápices de mina y cuatro de pasta en $\$1\,800$. Si el lápiz de pasta cuesta $\$150$ más que el lápiz de mina, ¿qué valor tiene este último?
\begin{vertical}
\alternativa $\$150$
\alternativa $\$200$
\alternativa $\$250$
\alternativa $\$350$
\alternativa $\$400$
\end{vertical}

\pregunta Un matrimonio tiene tres hijos: el mayor y dos gemelos. El mayor tenía dos años cuando nacieron los gemelos y actualmente sus edades suman $14$ años, ¿qué edad tienen los gemelos?
\begin{vertical}
\alternativa $2$
\alternativa $3$
\alternativa $4$
\alternativa $5$
\alternativa $6$
\end{vertical}

\pregunta Se puede determinar un número de dos cifras, sabiendo que:
\begin{verticaln}
\alternativa La suma de las cifras es $9$.
\alternativa Si se suma el número con el que resulta de invertir sus cifras resulta $99$.
\end{verticaln}
\begin{vertical}
\alternativa (1) por sí sola
\alternativa (2) por sí sola
\alternativa Ambas juntas, (1) y (2)
\alternativa Cada una por sí sola, (1) ó (2)
\alternativa Se requiere información adicional
\end{vertical}

\pregunta Un peluquero en tres días de trabajo recaudó $\$180\,000$. En el segundo día atendió a dos clientes más que en el primer día y en el tercer día atendió $4$ más que en el primer día. Si a cada uno de los clientes le cobró $\$6\,000$ por el corte, ¿cuántos atendió el segundo día?
\begin{vertical}
\alternativa $8$
\alternativa $10$
\alternativa $12$
\alternativa $18$
\alternativa $20$
\end{vertical}

\pregunta Con un hilo de $64 \text{ cm}$ se construye un rectángulo cuyo largo mide $4 \text{ cm}$ más que el ancho. ¿Cuál es el área de este rectángulo?
\begin{vertical}
\alternativa $252 \text{ cm}^2$
\alternativa $396 \text{ cm}^2$
\alternativa $572 \text{ cm}^2$
\alternativa $780 \text{ cm}^2$
\alternativa $1\,020 \text{ cm}^2$
\end{vertical}

\pregunta Felipe compra un ramo de flores que contenía $18$ claveles y $6$ rosas en $\$6\,600$. Si las rosas valen $\$100$ más que los claveles, ¿cuánto vale cada una de las rosas?
\begin{vertical}
\alternativa $\$200$
\alternativa $\$250$
\alternativa $\$300$
\alternativa $\$350$
\alternativa $\$450$
\end{vertical}

\pregunta En una fiesta hay $12$ mujeres más que hombres. Si se retiran $4$ mujeres y $2$ hombres, el número de hombres equivaldría a la mitad del número de mujeres. ¿Cuántos hombres había en un principio?
\begin{vertical}
\alternativa $10$
\alternativa $12$
\alternativa $14$
\alternativa $20$
\alternativa $24$
\end{vertical}

\pregunta En un juego de tiro al blanco se asignan $100$ puntos por cada acierto y se descuentan $50$ por cada error. Si un jugador lanzó $30$ veces obteniendo $1\,500$ puntos ¿cuál fue el número de aciertos?
\begin{vertical}
\alternativa $10$
\alternativa $15$
\alternativa $20$
\alternativa $25$
\alternativa $29$
\end{vertical}

\pregunta Pedro tiene \$A y su hermano Diego tiene \$B. Si Pedro le da \$200 a Diego quedan ambos con igual cantidad de dinero y si el padre de ellos le hubiese dado \$500 a Pedro y le hubiese quitado \$100 a Diego, entonces Pedro quedaría con el doble de lo que tendría Diego. ¿Cuál de los siguientes sistemas permite determinar el dinero que tenían inicialmente?
\begin{vertical}
\alternativa $\begin{cases} A + 200 = B - 200 \\ 2\left(A - 500\right) = B + 100 \end{cases}$
\alternativa $\begin{cases} A + 200 = B - 200 \\ 2\left(A + 500\right) = B - 100 \end{cases}$
\alternativa $\begin{cases} A - 200 = B + 200 \\ 2\left(A + 500\right) = B - 100 \end{cases}$
\alternativa $\begin{cases} A + 200 = B - 200 \\ A + 500 = 2\left(B - 100\right) \end{cases}$
\alternativa $\begin{cases} A - 200 = B + 200 \\ A + 500 = 2\left(B - 100\right) \end{cases}$
\end{vertical}

\pregunta Francisco tiene \$p en a monedas de \$50 y b monedas de \$100. Si el total de monedas son 10, ¿cuál de los siguientes sistemas permite determinar cuántas monedas tiene de cada denominación?
\begin{vertical}
\alternativa $\begin{cases} a + b = 10 \\ \dfrac{50}{a} + \dfrac{100}{b} = p \end{cases}$
\alternativa $\begin{cases} a + b = 10 \\ 100a + 50b = p \end{cases}$
\alternativa $\begin{cases} a + b = 10 \\ \dfrac{a}{50} + \dfrac{b}{100} = p \end{cases}$
\alternativa $\begin{cases} a + b = 10 \\ \dfrac{a}{50} + \dfrac{b}{100} = 10 \end{cases}$
\alternativa $\begin{cases} a + b = 10 \\ 50a + 100b = p \end{cases}$
\end{vertical}

\pregunta Si se suma la edad de Pablo con la de su hermano resultan $20$ años. Si se suma la edad de Pablo con el doble de la edad de su hermano resulta $28$ años. Si se suma la mitad de la edad de Pablo con un número resulta $40$. ¿cuál es el número?
\begin{vertical}
\alternativa $8$
\alternativa $12$
\alternativa $20$
\alternativa $34$
\alternativa $68$
\end{vertical}

\pregunta En un supermercado, $3$ kilogramos de paltas y $2$ kilogramos de tomate valen $\$8\,900$. Se puede determinar cuánto vale el kilogramo de paltas si se sabe que:
\begin{verticaln}
\alternativa $2$ kilogramos de paltas y $3$ kilogramos de tomates valen $\$7\,350$.
\alternativa El kilogramo de paltas vale $\$1\,550$ más que el kilogramo de tomates.
\end{verticaln}
\begin{vertical}
\alternativa (1) por sí sola
\alternativa (2) por sí sola
\alternativa Ambas juntas, (1) y (2)
\alternativa Cada una por sí sola, (1) ó (2)
\alternativa Se requiere información adicional
\end{vertical}

\pregunta El promedio de dos números es $29$ y si se dividen, el cuociente resulta $3$ y el resto $2$. ¿Cuál es el número mayor?
\begin{vertical}
\alternativa $14$
\alternativa $15$
\alternativa $29$
\alternativa $43$
\alternativa $44$
\end{vertical}

\pregunta Juan va a comprar bebidas y papas fritas, para ello lleva $\$3\,600$. Si comprara $3$ latas de bebida y $2$ bolsas de papas le faltarían $\$100$ y si comprara $2$ latas de bebida y $3$ bolsas de papas, le sobrarían $\$50$. ¿Cuánto vuelto recibiría, si compra una bolsa de papas y una lata de bebida?
\begin{vertical}
\alternativa $\$1\,450$
\alternativa $\$2\,150$
\alternativa $\$2\,250$
\alternativa $\$2\,350$
\alternativa $\$2\,170$
\end{vertical}

\pregunta En un cine la entrada normal vale $\$600$ más que la de estudiantes. A una función asisten $50$ personas de las cuales $10$ cancelaron entrada de estudiantes, recaudándose $\$114\,000$, ¿cuánto valía la entrada para estudiantes?
\begin{vertical}
\alternativa $\$1\,400$
\alternativa $\$1\,800$
\alternativa $\$2\,000$
\alternativa $\$2\,200$
\alternativa $\$2\,400$
\end{vertical}

\pregunta Un vaso está lleno de agua, si se bota el $20\%$ de su contenido, el vaso con el agua tendrían una masa de $320$ gramos y si se hubiese botado un tercio de su contenido, habrían tenido una masa de $300$ gramos. ¿Cuál es la masa del vaso?
\begin{vertical}
\alternativa $20 \text{ gramos}$
\alternativa $150 \text{ gramos}$
\alternativa $160 \text{ gramos}$
\alternativa $180 \text{ gramos}$
\alternativa $200 \text{ gramos}$
\end{vertical}

\pregunta Los alumnos de un curso deben reunir fondos para comprar un TV para su sala, para ello, se dividirá el dinero a recaudar en partes iguales. Se puede determinar cuánto es el dinero a reunir sabiendo que:
\begin{verticaln}
\alternativa Si cada uno aporta $\$4\,000$ sobrarían $\$12\,000$.
\alternativa Si cada uno aporta $\$3\,000$ faltarían $\$16\,000$.
\end{verticaln}
\begin{vertical}
\alternativa (1) por sí sola
\alternativa (2) por sí sola
\alternativa Ambas juntas, (1) y (2)
\alternativa Cada una por sí sola, (1) ó (2)
\alternativa Se requiere información adicional
\end{vertical}

\pregunta Un número tiene dos cifras donde la cifra de las unidades es p y la de las decenas es b. Si la suma de las cifras es $5$ y si al número se le suma $9$ resulta el número con las cifras invertidas, ¿cuál de los siguientes sistemas permite determinar las cifras del número?
\begin{vertical}
\alternativa $\begin{cases} b + p + p + b = 5 \\ 10b + p + 9 = 10p + b \end{cases}$
\alternativa $\begin{cases} 10b + p + 9 = 10b + p \\ 10b + p = 5 \end{cases}$
\alternativa $\begin{cases} b + p + 9 = p + b \\ b + p = 5 \end{cases}$
\alternativa $\begin{cases} 10p + b + 9 = 10b + p \\ b + p = 5 \end{cases}$
\alternativa $\begin{cases} 10b + p + 9 = 10p + b \\ b + p = 5 \end{cases}$
\end{vertical}

\pregunta Si a un número que tiene dos cifras se le resta la suma de sus cifras resulta $54$ y si al número se le resta el que resulta al invertir sus cifras resulta $27$. ¿Cuál el doble del número?
\begin{vertical}
\alternativa $12$
\alternativa $18$
\alternativa $63$
\alternativa $72$
\alternativa $126$
\end{vertical}

\pregunta Juan quiere instalar una enciclopedia en una biblioteca cuyos compartimientos son de igual tamaño. Al ponerla en los compartimientos se da cuenta que si coloca cuatro tomos en cada compartimiento le sobraría un tomo y si los pone de a cinco el último compartimiento quedaría vacío. ¿Cuántos compartimientos tiene la biblioteca?
\begin{vertical}
\alternativa $4$
\alternativa $5$
\alternativa $6$
\alternativa $7$
\alternativa $8$
\end{vertical}

\pregunta Un vehículo recorre una cierta distancia a una rapidez constante de $90 \text{ km/h}$, si hubiese ido a $100 \text{ km/h}$ se hubiese demorado $5$ minutos menos. ¿Qué longitud tenía el trayecto?
\begin{vertical}
\alternativa $50 \text{ km}$
\alternativa $60 \text{ km}$
\alternativa $70 \text{ km}$
\alternativa $75 \text{ km}$
\alternativa $120 \text{ km}$
\end{vertical}

\pregunta En un curso, la razón entre el número de hombres y el número de mujeres es $5 : 3$ y si se retiran $4$ hombres y se agregan tres mujeres, la razón es $7 : 6$. ¿Qué diferencia había inicialmente entre hombres y mujeres?
\begin{vertical}
\alternativa $3$
\alternativa $5$
\alternativa $10$
\alternativa $15$
\alternativa $20$
\end{vertical}

\pregunta Si x cuadernos de un mismo tipo valen \$p. Si comprara dos más le harían un descuento de un $5\%$, en este caso ¿cuánto hubiese pagado?
\begin{vertical}
\alternativa $\dfrac{\left(x + 2\right)p}{x} - \dfrac{5}{100}$
\alternativa $\left(0,95\right) \cdot \dfrac{\left(x + 2\right)p}{x}$
\alternativa $\left(0,05\right) \cdot \dfrac{\left(x + 2\right)p}{x}$
\alternativa $\left(0,95\right) \cdot \dfrac{xp}{x + 2}$
\alternativa $\left(0,9\right) \cdot \dfrac{\left(x + 2\right)p}{x}$
\end{vertical}

\pregunta Se obtienen $\$150\,000$ como capital final al invertir un monto “x” durante $48$ meses. Si al transcurrir un año el capital aumenta un $10\%$ respecto a lo acumulado el año anterior, ¿cuál de las siguientes ecuaciones permite determinar el capital x, suponiendo que no hubo depósitos ni retiros durante todo el período?
\begin{vertical}
\alternativa $150\,000 = x\left(1 + 0,1\right)^{48}$
\alternativa $150\,000 = x\left(1 + 0,1\right)^4$
\alternativa $150\,000 = x\left(1 + 0,01\right)^4$
\alternativa $x = 150\,000\left(1 + 0,1\right)^4$
\alternativa $x\left(1 + 0,1 \cdot 4\right) = 150\,000$
\end{vertical}

\pregunta El largo de un rectángulo se disminuye en $10 \text{ cm}$ y el ancho aumenta en $10 \text{ cm}$, obteniéndose un rectángulo que tiene $50 \text{ cm}^2$ más que el original. ¿Cuál es la diferencia en cm, entre los lados distintos del rectángulo original?
\begin{vertical}
\alternativa $5$
\alternativa $10$
\alternativa $15$
\alternativa $30$
\alternativa Falta información para determinarlo.
\end{vertical}

\pregunta En una compañía de electricidad, el cobro mensual consiste en un modelo lineal, donde se aplica un cargo fijo de \$F más un monto que depende de la cantidad de kWh consumidos. Si por el consumo de x kWh se ha cobrado un monto de \$T, ¿cuál de las siguientes expresiones corresponde(n) al valor de la cuenta por un consumo de $\left(x + 2\right)$ kWh?
\begin{verticali}
\alternativa $\dfrac{T - F}{x} \cdot \left(x + 2\right)$
\alternativa $F + \dfrac{\left(T - F\right) \cdot \left(x + 2\right)}{x}$
\alternativa $T + 2 \left(\dfrac{T - F}{x}\right)$
\end{verticali}
\begin{vertical}
\alternativa Solo I
\alternativa Solo II
\alternativa Solo I y II
\alternativa Solo II y III
\alternativa I, II y III
\end{vertical}

\pregunta El $50\%$ de la mitad de un número es $20$, entonces el número es:
\begin{vertical}
\alternativa $5$
\alternativa $10$
\alternativa $20$
\alternativa $40$
\alternativa $80$
\end{vertical}

\end{preguntas}
\end{document}