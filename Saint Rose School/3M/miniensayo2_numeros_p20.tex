\def\curso{Tercero medio}
\def\puntaje{30}
\def\titulo{Miniensayo}
\def\subtitulo{Eje de números (2)}
\def\fecha{13 de junio, 2025}
\documentclass[revolver]{srs}

\begin{document}

\subsection*{Objetivo}
Realizar cálculos y solucionar problemas utilizando propiedades de números racionales,
raíces y potencias.

\subsection*{Instrucciones generales}
Tiene 1 hora y 30 minutos para responder la evaluación. Esta es individual y debe
usar solo sus materiales personales para trabajar durante este periodo, no los solicite
a un compañero durante la evaluación.

Para cada pregunta, lea con atención el enunciado y escoja la alternativa que lo
responde correctamente. Solo hay una alternativa correcta por cada pregunta.

\subsection*{Criterios de evaluación}
Se asignará 1 puntos por cada pregunta contestada correctamente. En caso de marcar múltiples
alternativas en una misma pregunta, esto invalidará la respuesta y se considerará incorrecta.

\separador[2mm]

\begin{preguntas}[after-item-skip=2cm]
%\pregunta ¿Cuál es el valor de $1-3^2\cdot 2-7\cdot\left(-15\div-5\right)\div-7$?
%\begin{vertical}
%\alternativa $-10$
%\alternativa $-13$
%\alternativa $-14$
%\alternativa $-19$
%\end{vertical}

\pregunta Si $n=-3$~ y~ $m = 4$, ¿cuál es el valor de~ $-nm-\left(m-n\right)$?
\begin{vertical}
\alternativa $19$
\alternativa $5$
\alternativa $-11$
\alternativa $-13$
\end{vertical}

\pregunta $\left(\dfrac{5}{9}-\dfrac{2}{5}\right)\div\dfrac{14}{15}=$
\begin{vertical}
\alternativa $\dfrac{1}{14}$
\alternativa $\dfrac{45}{56}$
\alternativa $\dfrac{98}{675}$
\alternativa $\dfrac{1}{6}$
\alternativa $\dfrac{7}{10}$
\end{vertical}

\pregunta Un paquete de 24 rollos de papel higiénico de 50 metros cada uno, cuesta \$ 7.440 . ¿Cuál de las siguientes expresiones representa el valor de 1 metro de dicho papel, en pesos?
\begin{vertical}
\alternativa $\dfrac{7.440}{24}$
\alternativa $\dfrac{7.440}{50}$
\alternativa $\dfrac{7.440}{24\cdot 50}$
\alternativa $\dfrac{7.440}{24}\cdot 50$
\alternativa $\dfrac{7.440}{50}\cdot 24$
\end{vertical}

\pregunta Se repartió una herencia entre cinco hermanos, dos tíos y una prima. Si cada hermano recibió la séptima parte de la herencia y cada tío la mitad de lo que recibió cada uno de los hermanos, ¿qué parte de la herencia recibió la prima?
\begin{vertical}
\alternativa $\dfrac{2}{7}$
\alternativa $\dfrac{5}{7}$
\alternativa $\dfrac{11}{14}$
\alternativa $\dfrac{1}{7}$
\alternativa $\dfrac{3}{14}$
\end{vertical}

\pregunta Una caja vacía tiene una masa de 375 g. Luego se le agregan 6 paquetes de $\dfrac{3}{4}$ kg cada uno y 4 paquetes de $1\dfrac{1}{4}$ kg cada uno. ¿Cuál es la masa total de la caja con estos paquetes?
\begin{vertical}
\alternativa $9\dfrac{7}{8}$ kg
\alternativa $9\dfrac{1}{2}$ kg
\alternativa $6\dfrac{1}{8}$ kg
\alternativa $2\dfrac{5}{8}$ kg
\alternativa $2\dfrac{3}{8}$ kg
\end{vertical}

\pregunta El modelo RVA de colores, permite crear cualquier color mediante la mezcla de los distintos tonos de tres colores: rojo, verde y azul.
Los valores de la intensidad de cada uno de estos colores van desde el 0 al 225 y cada color creado tiene un código de tres números donde el
primero representa al rojo, el segundo al verde y el tercero al azul.
\newline
El código de la mezcla de dos colores se obtiene haciendo el promedio de cada uno de los valores de los colores originales tal como se presenta a continuación:
\begin{centrado}
\begin{tblr}{
  colspec={X[c,m] X[c,m]},
  hlines, vlines,
  width=.7\textwidth,
  rows={rowsep+=3pt}
}
Colores para mezclar & Color resultante \\
$\left(a, b, c\right)\;,\; \left(m, n, t\right)$ & $\left(\dfrac{a+m}{2}, \dfrac{b+n}{2}, \dfrac{c+t}{2}\right)$ \\
\end{tblr}
\end{centrado}
¿Con qué color hay que mezclar el color $\left(160, 60, 120\right)$ para obtener el color $\left(170, 80, 60\right)$?
\begin{vertical}
\alternativa $\left(10, 20, 60\right)$
\alternativa $\left(180, 100, 60\right)$
\alternativa $\left(180, 100, 0\right)$
\alternativa $\left(165, 70, 90\right)$
\end{vertical}

\pregunta ¿Cuál es el valor de $\left(1-\dfrac{3}{2}\right)\left(1-\dfrac{4}{3}\right)\left(1-\dfrac{1}{4}\right)\left(1-\dfrac{1}{5}\right)$?
\begin{vertical}
\alternativa $1$
\alternativa $3$
\alternativa $\dfrac{1}{5}$
\alternativa $\dfrac{1}{10}$
\end{vertical}

\pregunta Un estudiante recibe un préstamo del cual destina la sexta parte para pagar gastos comunes y el resto para compras.
Si gasta $\dfrac{3}{4}$ de la parte destinada a compras en alimento, ¿qué fracción del préstamo queda para alimento?
\begin{vertical}
\alternativa $\dfrac{7}{12}$
\alternativa $\dfrac{5}{24}$
\alternativa $\dfrac{5}{8}$
\alternativa $\dfrac{3}{8}$
\end{vertical}

\pregunta Considere los racionales $\dfrac{2}{5}, \dfrac{3}{4}, \dfrac{5}{7}, \dfrac{14}{15}$, ¿cuál de las siguientes
opciones representa el orden de mayor a menor?
\begin{vertical}
\alternativa $\dfrac{2}{5}, \dfrac{3}{4}, \dfrac{5}{7}, \dfrac{14}{15}$
\alternativa $\dfrac{14}{15}, \dfrac{5}{7}, \dfrac{3}{4}, \dfrac{2}{5}$
\alternativa $\dfrac{2}{5}, \dfrac{5}{7}, \dfrac{3}{4}, \dfrac{14}{15}$
\alternativa $\dfrac{14}{15}, \dfrac{3}{4}, \dfrac{5}{7}, \dfrac{2}{5}$
\end{vertical}

\pregunta ¿Cuál es el inverso multiplicativo de $-1 \div 2\dfrac{1}{5}$?
\begin{vertical}
\alternativa $-\dfrac{5}{2}$
\alternativa $-\dfrac{2}{5}$
\alternativa $-\dfrac{11}{5}$
\alternativa $-\dfrac{5}{11}$
\end{vertical}

%\pregunta ¿Cuántos vasos de $\dfrac{1}{5}$ litro se pueden llenar, exactamente, con diez botellas de agua mineral de dos y medio litros cada una?
%\begin{vertical}
%\alternativa 125
%\alternativa 150
%\alternativa 200
%\alternativa 250
%\end{vertical}

\pregunta ¿Cuál es el valor de $4-\dfrac{3}{4}\cdot\left(\dfrac{5}{2}-\dfrac{1}{3}\right)$?
\begin{vertical}
\alternativa $6\dfrac{1}{2}$
\alternativa $5\dfrac{9}{32}$
\alternativa $4\dfrac{7}{8}$
\alternativa $2\dfrac{3}{8}$
\end{vertical}

\pregunta Si $P = \dfrac{2\cdot3-2\cdot2\cdot3\cdot3}{-2\cdot3-2\cdot2\cdot3\cdot3}$, ¿cuál es el valor de $P^{-1}$?
\begin{vertical}
\alternativa $\dfrac{7}{5}$
\alternativa $1$
\alternativa $-\dfrac{7}{5}$
\alternativa $-1$
\end{vertical}

\pregunta Si $M=1,4+4,05$~;~ $P=5,\overline{6}-0,2\overline{1}$ ~y~ $Q= 3,\overline{21}+2,\overline{24}$, ¿cuál de las relaciones es verdadera?
\begin{vertical}
\alternativa $P>Q>M$
\alternativa $M=Q>P$
\alternativa $Q>P>M$
\alternativa $P>M>Q$
\alternativa $Q>M>P$
\end{vertical}

\pregunta En la tabla adjunta se muestran los tiempos que demoraron cuatro atletas en correr 100 metros. Según los datos de la
tabla, ¿cuál de los siguientes valores es la resta de los tiempos, en segundos, entre los dos atletas más rápidos?
\begin{centrado}
\begin{tblr}{
  colspec={X[c,m] X[c,m]},
  hlines, vlines,
  width=.7\textwidth,
  row{1}={bg=gray!20},
  rows={rowsep+=3pt},
}
Atleta & Tiempo en segundos \\
Andrés & $9,63$ \\
Bernardo & $\dfrac{39}{4}$ \\
Carlos & $\dfrac{979}{100}$ \\
Danilo & $9\dfrac{69}{100}$ \\
\end{tblr}
\end{centrado}
\begin{vertical}
\alternativa $3,42$
\alternativa $0,12$
\alternativa $0,06$
\alternativa $0,555$
\alternativa $0,04$
\end{vertical}

\pregunta ¿Cuál de las siguientes opciones NO representa un número entero?
\begin{vertical}
\alternativa $0,\overline{3}+0,\overline{6}$
\alternativa $1,5+0,5$
\alternativa $0,15\cdot 0,5$
\alternativa $\dfrac{0,18}{0,09}$
\end{vertical}

\pregunta Si $A = 2 \cdot \left(\dfrac{0,002}{0,02}\right)$, ¿a qué distancia de 2 se encuentra A en la recta numérica?
\begin{vertical}
\alternativa $1,8$
\alternativa $1,2$
\alternativa $0,8$
\alternativa $0,2$
\end{vertical}

\pregunta ¿Cuál es el 40 \% del 15 \% de 300?
\begin{vertical}
\alternativa 18
\alternativa 75
\alternativa 165
\alternativa 180
\end{vertical}

\pregunta Si el radio de un círculo aumenta en un 100 \%, entonces ¿en cuánto aumenta su área?
\begin{vertical}
\alternativa 100 \%
\alternativa 200 \%
\alternativa 300 \%
\alternativa 400 \%
\end{vertical}

\pregunta Si la diferencia entre el $p$ \% y el $q$ \% de 18 000 es 4500, entonces ¿cuál es el valor de $p-q$?
\begin{vertical}
\alternativa 2,5
\alternativa 25
\alternativa 250
\alternativa 2500
\end{vertical}

\pregunta Si $m=-2$ ~y~ $n=3$, ¿cuál de las siguientes igualdades es FALSA?
\begin{vertical}
\alternativa $m^3+n^3=19$
\alternativa $n^m = \dfrac{1}{9}$
\alternativa $n^{-m}-m^n=1$
\alternativa $\left(\dfrac{m}{n}\right)^m = \dfrac{9}{4}$
\end{vertical}

\pregunta Para $t$ distinto de cero, ¿cuál es el resultado de $t^{m-3}\cdot\left(t^{m-2}-t^{3-m}\right)$?
\begin{vertical}
\alternativa $1-t^{2m+6}$
\alternativa $t^{2m-5}-1$
\alternativa $t^{3m-3}-1$
\alternativa $t^{2m-5}-t^{-2m-6}$
\end{vertical}

\pregunta $\sqrt{\left(0,25\right)^{1-a}} =$
\begin{vertical}
\alternativa $\left(\dfrac{1}{2}\right)^{-a}$
\alternativa $\left(\dfrac{1}{2}\right)^{1-a}$
\alternativa $\left(\dfrac{1}{2}\right)^{-\frac{a}{2}}$
\alternativa $\left(\dfrac{1}{2}\right)^{\frac{a}{2}}$
\alternativa $\left(\dfrac{1}{2}\right)^{a}$
\end{vertical}

\pregunta La expresión $\sqrt[3]{a^2} \div \left(\sqrt[3]{a}\right)^{-1}$ es equivalente a
\begin{vertical}
\alternativa $\sqrt[3]{a}$
\alternativa $\dfrac{1}{a}$
\alternativa $-1$
\alternativa $-\sqrt[3]{a}$
\alternativa $a$
\end{vertical}

\pregunta $\sqrt{0,\overline{4}} \cdot \dfrac{x^{\frac{2}{3}}}{\sqrt[3]{x}}$
\begin{vertical}
\alternativa $0,2\cdot x$
\alternativa $\dfrac{2}{3}\cdot x^{\frac{1}{3}}$
\alternativa $\sqrt{\dfrac{4}{10}} \cdot x^{\frac{1}{3}}$
\alternativa $0,\overline{2} \cdot x^{\frac{1}{3}}$
\alternativa $\dfrac{2}{3} \cdot x$
\end{vertical}

\pregunta Si $a$,~ $b$,~ $n$~ y~ $p$ son números reales positivos, entonces $\sqrt[b]{a^n} \cdot \sqrt[n]{p^b}$ es igual a
\begin{vertical}
\alternativa $ap$
\alternativa $\left(ap\right)^{\frac{n^2+b^2}{nb}}$
\alternativa $\sqrt[bn]{a^{n^2} p^{b^2}}$
\alternativa $\sqrt[bn]{\left(ap\right)^{n+b}}$
\alternativa ninguna de las expresiones.
\end{vertical}

\pregunta ¿Cuál de las siguientes igualdades es FALSA?
\begin{vertical}
\alternativa $-25^{\frac{1}{2}} = -5$
\alternativa $4^{-\frac{1}{2}} = \dfrac{1}{2}$
\alternativa $32^{\frac{5}{3}} = 8$
\alternativa $8^{\frac{2}{3}} = 4$
\end{vertical}

%\pregunta ¿Cuál de las siguientes expresiones resulta un número racional?
%\begin{vertical}
%\alternativa $\dfrac{3\sqrt{2}}{2\sqrt{3}}$
%\alternativa $8\sqrt{3}-5\sqrt{3}$
%\alternativa $\dfrac{\sqrt{2}}{\sqrt{72}}$
%\alternativa Ninguna de ellas.
%\end{vertical}

\pregunta ¿Cuál es el valor de $\sqrt{289} - \sqrt[3]{-125} + \sqrt[4]{256}$?
\begin{vertical}
\alternativa 26
\alternativa 21
\alternativa 19
\alternativa 16
\end{vertical}

\pregunta ¿Cuál es el valor de $\left(\dfrac{\sqrt{72}}{6} + \dfrac{\sqrt{162}}{3} - \dfrac{\sqrt{32}}{2}\right)$?
\begin{vertical}
\alternativa $3\sqrt{2}$
\alternativa $2\sqrt{2}$
\alternativa $\sqrt{2}$
\alternativa $-\sqrt{2}$
\end{vertical}

%\pregunta Al reducir la expresión $\sqrt[4]{9}+\sqrt[6]{27}$ se obtiene
%\begin{vertical}
%\alternativa 3
%\alternativa $2\sqrt{3}$
%\alternativa $\sqrt[10]{36}$
%\alternativa $\sqrt[12]{9^3+27^2}$
%\end{vertical}

\pregunta ¿Cuál es el resultado de $\left(\sqrt[3]{t^4}\right)\left(\sqrt[6]{t}\right)\left(\sqrt[12]{t^5}\right)$?
\begin{vertical}
\alternativa $t \cdot \sqrt[12]{t^7}$
\alternativa $t \cdot \sqrt[12]{t^{11}}$
\alternativa $\sqrt[20]{t^{10}}$
\alternativa $\sqrt[30]{t^{117}}$
\end{vertical}

\pregunta La figura adjunta está formada por un cuadrado de lado $\dfrac{3}{\sqrt{5}}$ cm, dentro del cual se trazaron sus dos diagonales.
\begin{centrado}
\begin{tikzpicture}
  \node [shape=rectangle,draw,inner sep=15mm,name=S] at (0,0) {};
  \draw (S.north west) -- (S.south east);
  \draw (S.north east) -- (S.south west);
  \fill[pattern=north west lines] (S.north west) -- (S.south west) -- (S.center) -- cycle;
\end{tikzpicture}
\end{centrado}
¿Cuál es el área de la figura achurada, en cm$^2$?
\begin{vertical}
\alternativa $\dfrac{3\sqrt{5}+3\sqrt{10}}{5}$
\alternativa $\dfrac{9}{20}$
\alternativa $\dfrac{9}{5}$
\alternativa $\dfrac{12}{4\sqrt{5}}$
\end{vertical}

\end{preguntas}

\end{document}