\def\curso{Tercero medio B}
\def\puntaje{34}
\def\titulo{Prueba}
\def\subtitulo{Operatoria de números complejos}
\documentclass[]{srs}

\begin{document}

\section*{Objetivo}

\begin{itemize}[nosep]
  \item Describir números complejos en su binomial o como par ordenado.
  \item Resolver y reducir expresiones que involucran operatoria básica entre números complejos.
  \item Calcular el módulo de un número complejo.
  \item Ubicar un número complejo en el plano de Argand.
\end{itemize}

\section*{Instrucciones generales}
  Tiene 1 hora y 30 minutos para responder la evaluación. Esta es individual y debe
  usar solo sus materiales personales para trabajar durante este periodo, no los solicite
  a un compañero durante la evaluación.

\section{Opciones múltiples}

\section*{Instrucciones}
Lea atentamente cada enunciado y escoja la alternativa correcta en cada caso.

\section*{Criterios de evaluación}
En la corrección de esta sección, se asignará 2 puntos al marcar la alternativa correcta.
Las alternativas corregidas serán consideradas incorrectas, es decir, marque solo una
alternativa por enunciado.

\separador[2mm]

\begin{preguntas}[after-item-skip=2cm]
 \pregunta La suma de la parte real e imaginaria del número $3-2i$ es:\\
 \begin{vertical}
 \alternativa $3-2i$
 \alternativa $5$
 \alternativa $1$
 \alternativa $-1$
 \alternativa $i$
 \end{vertical}

 \pregunta Para $x > 2$, la representación del número $\sqrt{x-2}$ como par ordenado es:\\
\begin{vertical}
  \alternativa $\left(2\,,\;x\right)$
  \alternativa $\left(0\,,\;\sqrt{x-2}\right)$
  \alternativa $\left(\sqrt{x-2}\,,\;0\right)$
  \alternativa $\left(-\sqrt{x-2}\,,\;0\right)$
  \alternativa $\left(\sqrt{x-2}\,,\;\sqrt{x-2}\right)$
\end{vertical}

 \pregunta Respecto del número complejo que aparece en la imagen, es correcto
 afirmar que:\\
\begin{center}
 \begin{tikzpicture}[scale=0.5]
  \begin{axis}[nodes near coords, eje escolar,xlabel={$\mathbb{R}$},ylabel={$\mathbb{C}$},
    axis line style={line width=2pt},label style={font=\huge},
    ytick=\empty,xtick=\empty,xmin=-1,xmax=1,ymin=-1,ymax=1,
    nodes near coords style={yshift=5pt,font=\huge}]
    \addplot+[only marks,point meta=explicit symbolic,
    only marks,mark=*,mark options={scale=2, fill=black}]
      coordinates {(-0.5,0.5) [$z$]};
  \end{axis}
\end{tikzpicture}
\end{center}
\begin{vertical*}
  \alternativa Solo tiene parte real.
  \alternativa Su parte real es positiva.
  \alternativa Su parte imaginaria es positiva.
\end{vertical*}
\begin{vertical}
  \alternativa Solo I.
  \alternativa Solo II.
  \alternativa Solo III.
  \alternativa Solo I y II.
  \alternativa I, II y III.
\end{vertical}

\pregunta El valor de $i^{2019}$ es:\\
\begin{vertical}
\alternativa $1$
\alternativa $i$
\alternativa $-1$
\alternativa $-i$
\alternativa $3$
\end{vertical}

\pregunta Para $x < 3$, la representación del número $\sqrt{x-3}$ en el plano de Argand es:
\begin{alternativasgraficas}[raster columns=3]
  \alternativa
  \begin{tikzpicture}[scale=0.4]
    \begin{axis}[nodes near coords, eje escolar,xlabel={$\mathbb{R}$},ylabel={$\mathbb{C}$},
      axis line style={line width=2pt},label style={font=\huge},
      ytick=\empty,xtick=\empty,xmin=-1,xmax=1,ymin=-1,ymax=1,
      nodes near coords style={yshift=5pt,font=\huge}]
      \addplot+[only marks,point meta=explicit symbolic,
      only marks,mark=*,mark options={scale=2, fill=black}]
        coordinates {(0.5,0) [$z$]};
    \end{axis}
  \end{tikzpicture}
  \alternativa
  \begin{tikzpicture}[scale=0.4]
    \begin{axis}[nodes near coords, eje escolar,xlabel={$\mathbb{R}$},ylabel={$\mathbb{C}$},
      axis line style={line width=2pt},label style={font=\huge},
      ytick=\empty,xtick=\empty,xmin=-1,xmax=1,ymin=-1,ymax=1,
      nodes near coords style={yshift=5pt,xshift=-10pt,font=\huge}]
      \addplot+[only marks,point meta=explicit symbolic,
      only marks,mark=*,mark options={scale=2, fill=black}]
        coordinates {(0,0.5) [$z$]};
    \end{axis}
  \end{tikzpicture}
  \alternativa
  \begin{tikzpicture}[scale=0.4]
    \begin{axis}[nodes near coords, eje escolar,xlabel={$\mathbb{R}$},ylabel={$\mathbb{C}$},
      axis line style={line width=2pt},label style={font=\huge},
      ytick=\empty,xtick=\empty,xmin=-1,xmax=1,ymin=-1,ymax=1,
      nodes near coords style={yshift=5pt,font=\huge}]
      \addplot+[only marks,point meta=explicit symbolic,
      only marks,mark=*,mark options={scale=2, fill=black}]
        coordinates {(0.5,0.5) [$z$]};
    \end{axis}
  \end{tikzpicture}
  \alternativa
  \begin{tikzpicture}[scale=0.4]
    \begin{axis}[nodes near coords, eje escolar,xlabel={$\mathbb{R}$},ylabel={$\mathbb{C}$},
      axis line style={line width=2pt},label style={font=\huge},
      ytick=\empty,xtick=\empty,xmin=-1,xmax=1,ymin=-1,ymax=1,
      nodes near coords style={yshift=5pt,xshift=-10pt,font=\huge}]
      \addplot+[only marks,point meta=explicit symbolic,
      only marks,mark=*,mark options={scale=2, fill=black}]
        coordinates {(0,-0.5) [$z$]};
    \end{axis}
  \end{tikzpicture}
  \alternativa
  \begin{tikzpicture}[scale=0.4]
    \begin{axis}[nodes near coords, eje escolar,xlabel={$\mathbb{R}$},ylabel={$\mathbb{C}$},
      axis line style={line width=2pt},label style={font=\huge},
      ytick=\empty,xtick=\empty,xmin=-1,xmax=1,ymin=-1,ymax=1,
      nodes near coords style={yshift=5pt,font=\huge}]
      \addplot+[only marks,point meta=explicit symbolic,
      only marks,mark=*,mark options={scale=2, fill=black}]
        coordinates {(-0.5,0.5) [$z$]};
    \end{axis}
  \end{tikzpicture}
\end{alternativasgraficas}

\pregunta La representación en el plano de Argand de un número complejo se encuentra
en el tercer cuadrante. Entonces, es correcto afirmar que:\\
\begin{vertical*}
  \alternativa Su parte imaginaria es positiva.
  \alternativa Su parte real es negativa.
  \alternativa El resultado de la multiplicación entre su parte real y
  su parte imaginaria es positiva.
\end{vertical*}
\begin{vertical}
  \alternativa Solo I.
  \alternativa Solo II.
  \alternativa Solo III.
  \alternativa Solo II y III.
  \alternativa I, II y III.
\end{vertical}

\pregunta Si \hspace{3pt} $z=1+i$ \hspace{3pt} y \hspace{3pt} $w=3i-2$,
  ¿cuáles de las siguientes afirmaciones es correcta?\\
\begin{vertical*}
  \alternativa $|z|=\sqrt{2}$
  \alternativa $|z+w|\le|z|+|w|$
  \alternativa $|z\cdot w|=|z|\cdot|w|$
\end{vertical*}
\begin{vertical}
  \alternativa Solo I.
  \alternativa Solo II.
  \alternativa Solo III.
  \alternativa Solo II y III.
  \alternativa I, II y III.
\end{vertical}

\pregunta Si el módulo de un número complejo es tal que $|z|=5$ y su parte real es 4,
se puede decir sobre su parte imaginaria que:\\
\begin{vertical*}
  \alternativa $Im(z)=3$
  \alternativa $Im(z)=-3$
  \alternativa $Im(z)=\pm 3i$
\end{vertical*}
\begin{vertical}
  \alternativa Solo I.
  \alternativa Solo II.
  \alternativa Solo III.
  \alternativa Solo I y II.
  \alternativa Solo I y III.
\end{vertical}

\pregunta ¿Cuáles de los siguientes números es (son) solución(es) de la ecuación
cuadrática $x^2+x+1=0$?\\
\begin{vertical*}
  \alternativa $-\;\dfrac{1}{2}+\dfrac{\sqrt{3}}{2}i$
  \alternativa $-\;\dfrac{1}{2}-\,\dfrac{\sqrt{3}}{2}i$
  \alternativa $\dfrac{1}{2}+\dfrac{\sqrt{3}}{2}i$
\end{vertical*}
\begin{vertical}
  \alternativa Solo I.
  \alternativa Solo II.
  \alternativa Solo III.
  \alternativa Solo I y II.
  \alternativa Solo II y III.
\end{vertical}

\pregunta Si $z=3i-5$, la expresión $2z+3iz-z-4iz$ es:\\
\begin{vertical}
  \alternativa $z$
  \alternativa $-iz$
  \alternativa $3i+6$
  \alternativa $8i-2$
  \alternativa $4i$
\end{vertical}

\pregunta La parte imaginaria de la expresión
$(i-1)(i+1)(2i-1)(2i+1)$ es:\\
\begin{vertical}
\alternativa $6$
\alternativa $-6$
\alternativa $0$
\alternativa $1$
\alternativa $-1$
\end{vertical}

\pregunta Si $z=3i-1$, ¿cuál es el valor de la
expresión $z^3-z$?\\
\begin{vertical}
  \alternativa $27+21i$
  \alternativa $27-21i$
  \alternativa $-18-12i$
  \alternativa $9-22i$
  \alternativa $-27-21i$
\end{vertical}


\pregunta El número $(i^{36}-i^{54})^2$ es equivalente a:\\
\begin{vertical}
  \alternativa $0$
  \alternativa $2$
  \alternativa $4$
  \alternativa $-2i$
  \alternativa $2i$
\end{vertical}

\pregunta La expresión $(2i)^{28}$ es:\\
\begin{vertical}
  \alternativa $4^{28}$
  \alternativa $i^{28}$
  \alternativa $(-i)^{28}$
  \alternativa $-2^{28}$
  \alternativa $2^{28}$
\end{vertical}

\end{preguntas}

\section{Preguntas abiertas}

\section*{Instrucciones}
Lea atentamente el enunciado de cada pregunta, considere los datos entregados y
responda a la problemática planteada, explicando y detallando claramente
su proceso y resultados.

\section*{Criterios de evaluación}
  En la corrección de esta sección, cada pregunta tiene 3 puntos y se asignará
  el puntaje de cada una según los siguientes criterios:
\begin{center}
  \begin{tblr}{width=\linewidth,colspec={X[1,c]|X[6]}, hline{1,Z} = {1}{-}{}, hline{1,Z} = {2}{-}{},
      hlines, cells={valign=m}, row{1} = {bg=black!15}}
      Puntaje asignado & \SetCell{c} Criterios o indicadores \\
      +50\% & Señala clara y correctamente cuál es la solución o el resultado de la pregunta hecha
      en el enunciado.\\
      +50\% & Incluye un desarrollo que relata de manera clara y ordenada los procedimientos
      \mbox{necesarios} para solucionar la problemática. En caso de estar incompleto o con
      errores el desarrollo, se asignará puntaje parcial si se muestra dominio de los
       contenidos y conceptos involucrados.\\
      0\% &  La respuesta es incorrecta. De haber desarrollo, este tiene errores conceptuales.\\
  \end{tblr}
\end{center}
\separador[2mm]

\begin{preguntas}(1)
  \pregunta ¿Qué valor(s) de $k$ permite(n) que para el número complejo $z$
  se cumpla que $Re(z)+Im(z)=5$?
  \begin{mcaja}
    \text{con}\quad z = \left(\dfrac{4}{k}+1\right) + \left(-k+1\right)i
  \end{mcaja}
  \begin{malla}[height=11cm]
  \end{malla}
  \pregunta ¿Cuál o cuáles son los números complejos que tienen como parte real el doble
  de su parte imaginaria y su módulo tiene el valor de $\sqrt{45}$?
  \begin{malla}[height=11cm]
  \end{malla}
\end{preguntas}





\end{document}