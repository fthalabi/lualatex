\def\curso{Tercero medio}
\def\puntaje{11}
\def\titulo{Tarea}
\def\subtitulo{Regla de Laplace}
\def\fecha{16 de mayo, 2025}
\documentclass[]{srs}

\begin{document}

\section*{Objetivo}
  Resolver problemas y calcular probabilidades utilizando la regla de Laplace.

\section*{Instrucciones generales}
  Entregue junto con este enunciado un informe detallando la solución a las
  distintas problemáticas planteadas. La tarea se puede realizar en parejas
  (con una mayor escala de dificultad en la corrección), tiene nota acumulativa
  y debe entregarla en la clase el día 19 de mayo.

\section*{Criterios de evaluación}
  Para la corrección de esta tarea, se asignará el puntaje según los criterios que
  se encuentran explicados en la siguiente rubrica.

\begin{center}
  \begin{tblr}{width=\linewidth,colspec={XXX}, hline{1,Z} = {1}{-}{}, hline{1,Z} = {2}{-}{},
      hlines,row{3,5,7}={font={\small\raggedright}},row{2,4,6} = {bg=black!15,halign=c},
      row{1}={halign=c},rows={rowsep=5pt}}
      %%% CABEZERA
      Logrado & Suficiente & Insuficiente \\
      %%% PRESENTACION
      \SetCell[c=3]{c} Presentación (puntaje por informe)& & \\
      La tarea se presenta de forma impecable: letra clara y fácilmente
      legible en todo el documento, sin errores de ortografía.
      El papel está limpio, sin arrugas, dobleces ni manchas.\mbox{(5 puntos)}&
      La tarea es legible en su mayor parte, aunque podría mejorar la claridad
      de la letra en algunas secciones. Presenta errores ortográficos mínimos (1-2)
      y/o pequeñas imperfecciones en el papel (manchas leves, arrugas menores) que no
      dificultan significativamente la lectura. \mbox{(3 puntos)}&
      La letra es difícil de entender en varias secciones, presenta múltiples
      errores de ortografía (3 o más) y/o el papel está descuidado
      (arrugado, manchado, doblado) dificultando la revisión y lectura. \mbox{(0 puntos)}\\
      %%% PROCESOS Y JUSTICACACION
      \SetCell[c=3]{c} Procesos y justificación (puntaje por pregunta) & & \\
      Detalla cada paso del procedimiento de forma clara, ordenada y sistemática.
      Justifica adecuadamente las operaciones, propiedades o estrategias matemáticas
      utilizadas, demostrando una comprensión profunda del problema y cómo se alcanza
      la solución. El razonamiento es fácil de seguir. \mbox{(3 puntos)}&
      Presenta la mayoría de los pasos del procedimiento, aunque algunos podrían ser
      más detallados o claros. La justificación es adecuada en general, pero puede
      haber omisiones menores o falta de precisión en alguna explicación.
      El proceso general es comprensible, aunque con algunas dificultades para
      seguir el razonamiento en puntos específicos. \mbox{(2 puntos)}&
      Los pasos del procedimiento son confusos, incompletos, desordenados o ausentes.
      La justificación es escasa, ausente o incorrecta. No es posible seguir el
      razonamiento para entender cómo se intentó llegar al resultado. \mbox{(0 puntos)}\\
      %%% RESULTADOS
      \SetCell[c=3]{c} Resultados (puntaje por pregunta)& & \\
      El resultado alcanzado es correcto y está desarrollado completamente, es decir,
      no se dejaron valores expresados y/o sin simplificar. \mbox{(3 puntos)} &
      El resultado está casi correcto. El error es aritmético y no conceptual,
      o simplemente falto desarrollar la respuesta. \mbox{(2 puntos)}&
      El resultado no responde de ninguna manera al enunciado y/o hay errores que
      son fundamentales. \mbox{(0 puntos)}\\
  \end{tblr}
\end{center}

  Para verificar que la tarea se realizó de manera honesta, es posible que tenga
  explicar sus procesos y/o resultados en una interrogación.

\newpage

\begin{aviso}[after skip=4em]
  Debe responder solo dos de los tres problemas.
\end{aviso}

\begin{preguntas}[after-item-skip=20pt]
  \pregunta Considere una baraja francesa de 52 cartas ¿Cuál es la probabilidad de que
  al sacar 6 cartas, 3 sean rojas y 3 negras?
  \pregunta Se colocan 6 bolas, aleatoriamente, en tres cajas inicialmente vacías.
  ¿Cuál es la probabilidad que la primera caja contenga exactamente 2 bolas?
  \pregunta Se colocan, aleatoriamente, 8 libros en un estante. Entre ellos hay
  una obra en cuatro tomos y otra en tres tomos. ¿Cuál es la probabilidad de que los tomos
  de cada obra estén juntos?

\end{preguntas}

\end{document}