\def\curso{Tercero medio B}
\def\puntaje{8}
\def\titulo{Control}
\def\subtitulo{Lógica proposicional y conjuntos}
\def\fecha{28 de abril, 2025}
\documentclass[]{srs}

\begin{document}

\section*{Objetivo}
  Plantear una aseveración como una proposición lógica y describir
  la aseveración como una operación entre conjuntos.

\section*{Instrucciones}
  Cuenta con 40 minutos para completar  la evaluación. Incluya desarrollo en todas
  sus respuestas, y recuerde marcar o señalizar el resultado final en cada pregunta.

\section*{Criterios de evaluación}
  En la corrección, se asignará el puntaje a cada pregunta según los siguientes criterios.
\begin{center}
  \begin{tblr}{width=\linewidth,colspec={X[1,c]|X[6]}, hline{1,Z} = {1}{-}{}, hline{1,Z} = {2}{-}{},
      hlines, cells={valign=m}, row{1} = {bg=black!15}}
      Puntaje asignado & \SetCell{c} Criterios o indicadores \\
      +50\% & Señala clara y correctamente cuál es la solución o el resultado de la pregunta hecha
      en el enunciado.\\
      +50\% & Incluye un desarrollo que relata de manera clara y ordenada los procedimientos
      \mbox{necesarios} para solucionar la problemática. En caso de estar incompleto o con
      errores el desarrollo, se asignará puntaje parcial si se muestra dominio de los
       contenidos y conceptos involucrados.\\
      0\% &  La respuesta es incorrecta. De haber desarrollo, este tiene errores conceptuales.\\
  \end{tblr}
\end{center}
\separador[2mm]

\begin{preguntas}(1)
  \pregunta Rellene la siguiente tabla de verdad y determine el valor de verdad de la proposición. [2 puntos]
  \begin{mcaja}
    r \lor \neg\left(\neg p \lor q\right)
  \end{mcaja}
  \vspace*{5pt}
  \begin{tblr}{colspec={X[1,c]X[1,c]X[1,c]X[2,c]X[2,c]X[2,c]X[3,c]},
    hlines,vlines, cells={valign=m}, row{1} = {bg=black!15},cell{1}{4-6}={bg=white}}
    $p$ & $q$ & $r$ &  &  &  & $r \lor \neg\left(\neg p \lor q\right)$ \\
     &  &  &  &  &  & \\
     &  &  &  &  &  & \\
     &  &  &  &  &  & \\
     &  &  &  &  &  & \\
     &  &  &  &  &  & \\
     &  &  &  &  &  & \\
     &  &  &  &  &  & \\
     &  &  &  &  &  & \\
  \end{tblr}
  \pregunta Utilice la siguiente tabla de verdad para estudiar caso a caso la
  aseveración: \q{Ocurre alguno de los tres eventos pero no todos juntos}, como una proposición lógica.
  Finalmente, encuentre una fórmula lógica que represente los resultados obtenidos en la tabla, y
  como dicha fórmula se puede describir como una operación entre los conjuntos $A$, $B$ y $C$.
  [4 puntos] \\[10pt]
  \begin{tblr}{colspec={X[1,c]X[1,c]X[1,c]X[2,c]X[2,c]X[2,c]X[3,c]},
    hlines,vlines, cells={valign=m}, row{1} = {bg=black!15},cell{1}{4-7}={bg=white}}
    $p$ & $q$ & $r$ &  &  &  &  \\
     &  &  &  &  &  & \\
     &  &  &  &  &  & \\
     &  &  &  &  &  & \\
     &  &  &  &  &  & \\
     &  &  &  &  &  & \\
     &  &  &  &  &  & \\
     &  &  &  &  &  & \\
     &  &  &  &  &  & \\
  \end{tblr}
  \begin{respuesta}[height=4cm]

  \end{respuesta}

\end{preguntas}

Considere que cada una de las siguientes aseveraciones involucra tres eventos, y
describa cada aseveración utilizando álgebra de conjuntos. [0,5 puntos C/U]
\begin{preguntas}(1)
\pregunta \q{Ocurren todos los eventos}.
\begin{respuesta}[height=2cm]
\end{respuesta}
\pregunta \q{Ocurre por lo menos uno de los eventos}.
\begin{respuesta}[height=2cm]
\end{respuesta}
\pregunta \q{No ocurre más de uno de los eventos}.
\begin{respuesta}[height=2cm]
\end{respuesta}
\pregunta \q{Ocurren a lo más dos de los eventos}.
\begin{respuesta}[height=2cm]
\end{respuesta}

\end{preguntas}
\end{document}