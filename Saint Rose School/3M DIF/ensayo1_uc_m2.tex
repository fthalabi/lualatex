\def\titulo{Ensayo}
\def\subtitulo{(M2) versión 1}
\def\curso{Tercero medio}
\documentclass[sin nombre]{srs}

\begin{document}

\separador
\begin{preguntas}[after-item-skip=1cm]

\pregunta Si al producto entre $0,09$ y $0,9$ se divide por el producto entre
$9 \cdot 10^{-3}$ y el cuadrado del inverso aditivo de $\dfrac{3}{10}$,
¿qué número se obtiene?
\begin{vertical}
\alternativa $-100$
\alternativa $-15$
\alternativa $10$
\alternativa $100$
\alternativa $300$
\end{vertical}

\pregunta Considera los números reales $A = \dfrac{\sqrt{2}}{\sqrt{5}}$,
$G = \dfrac{3}{2\sqrt{7}}$, $O = \dfrac{\sqrt{5}}{\sqrt{7}}$ y
$T = \dfrac{1}{7\sqrt{2}}$. ¿Cuál es el orden decreciente de estos números?
\begin{vertical}
\alternativa $T$, $G$, $A$, $O$
\alternativa $A$, $O$, $G$, $T$
\alternativa $G$, $A$, $T$, $O$
\alternativa $T$, $G$, $O$, $A$
\alternativa $O$, $A$, $G$, $T$
\end{vertical}

\pregunta Si el producto de dos números reales $p$ y $q$ resulta un
número irracional, ¿cuál de las siguientes afirmaciones es
siempre verdadera?
\begin{vertical}
\alternativa $p$ y $q$ son números irracionales.
\alternativa El cociente entre $p$ y $q$ es un número irracional.
\alternativa La suma de $p$ y $q$ es un número racional.
\alternativa La potencia $p^q$ puede ser racional.
\end{vertical}

\pregunta ¿Cuál de las siguientes expresiones representa el $7,5\;\%$ del
$9,\overline{9}\;\%$ de $Q$?
\begin{vertical}
\alternativa $\dfrac{3}{40} \cdot \dfrac{1}{10} \cdot Q$
\alternativa $\dfrac{3}{40} \cdot \dfrac{99}{1000} \cdot Q$
\alternativa $\dfrac{75}{1000} \cdot \dfrac{9,9}{100} \cdot Q$
\alternativa $\dfrac{75}{100} \cdot \dfrac{1}{10} \cdot Q$
\end{vertical}

\pregunta ¿Cuál de las siguientes opciones presenta una resolución correcta
de $\dfrac{625^{\frac{1}{4}}}{81} \div \dfrac{243^{\frac{1}{5}}}{32}$ ?
\begin{vertical}
\alternativa $\dfrac{625^{\frac{1}{4}}}{81} \div \dfrac{243^{\frac{1}{5}}}{32} = \left(\dfrac{625}{81}\right)^{\frac{1}{4}} \div \left(\dfrac{243}{32}\right)^{\frac{1}{5}} = \dfrac{5}{3} \div \dfrac{3}{2} = \dfrac{10}{9}$
\alternativa $\dfrac{625^{\frac{1}{4}}}{81} \div \dfrac{243^{\frac{1}{5}}}{32} = \dfrac{\left(5^4\right)^{\frac{1}{4}} \cdot 32}{81 \cdot \left(3^5\right)^{\frac{1}{5}}} = \dfrac{5 \cdot 32}{81 \cdot 3} = \dfrac{160}{243}$
\alternativa $\dfrac{625^{\frac{1}{4}}}{81} \div \dfrac{243^{\frac{1}{5}}}{32} = \dfrac{5}{81} \div \dfrac{3}{32} = \dfrac{3 \cdot 81}{32 \cdot 5} = \dfrac{243}{160}$
\alternativa $\dfrac{625^{\frac{1}{4}}}{81} \div \dfrac{243^{\frac{1}{5}}}{32} = \dfrac{5 \cdot 3}{3^4 \cdot 2^5} = \dfrac{5}{3^3 \cdot 2^5} = \dfrac{5}{90}$
\end{vertical}

\pregunta Luisa, Martina, Lucas y Matías calculan el resultado de la
expresión $\left(\sqrt{5}-1\right)^2 \cdot \left(1+\sqrt{5}\right)^2$
utilizando diferentes procedimientos los cuales se muestran a continuación:
\begin{tcbraster}[raster columns=2,raster equal height=rows,colbacktitle=white,colback=white,colframe=black,coltitle=black]
\begin{tcolorbox}[title=Luisa]
\begin{+array}{ll}
&$\left(\sqrt{5}-1\right)^2 \cdot \left(1+\sqrt{5}\right)^2$ \\
$=$&$\left(\left(\sqrt{5}-1\right) \cdot \left(\sqrt{5}+1\right)\right)^2$ \\
$=$&$\left(\left(\sqrt{5}\right)^2 - 1^2\right)^2$ \\
$=$&$16$ \\
\end{+array}
\end{tcolorbox}
\begin{tcolorbox}[title=Martina]
\begin{+array}{ll}
&$\left(\sqrt{5}-1\right)^2 \cdot \left(1+\sqrt{5}\right)^2$ \\
$=$&$ \left(\left(\sqrt{5}\right)^2 - 1^2\right) \cdot \left(1^2 + \left(\sqrt{5}\right)^2\right)$ \\
$=$&$ \left(5-1\right) \cdot \left(1+5\right)$\\
$=$&$24$\\
\end{+array}
\end{tcolorbox}
\begin{tcolorbox}[title=Lucas]
\begin{+array}{ll}
&$\left(\sqrt{5}-1\right)^2 \cdot \left(1+\sqrt{5}\right)^2$ \\
$=$&$\left(\sqrt{5-1}\right)^2 \cdot \left(\sqrt{1+5}\right)^2$ \\
$=$&$4 \cdot 6$ \\
$=$&$24$
\end{+array}
\end{tcolorbox}
\begin{tcolorbox}[title=Matías]
\begin{+array}{ll}%
&$\left(\sqrt{5}-1\right)^2 \cdot \left(1+\sqrt{5}\right)^2$ \\
$=$&$\left(6-2\sqrt{5}\right) \cdot \left(6+2\sqrt{5}\right)$ \\
$=$&$36-20$ \\
$=$&$16$
\end{+array}
\end{tcolorbox}
\end{tcbraster}
¿Qué estudiante(s) realiza(n) un procedimiento correcto?
\begin{vertical}
\alternativa Martina y Lucas
\alternativa Luisa
\alternativa Matías y Luisa
\alternativa Martina
\end{vertical}

\pregunta Para racionalizar el denominador de la expresión $\dfrac{3}{\sqrt{\sqrt{3}+\sqrt{2}}}$ Juan realiza los siguientes pasos:\\[5pt]
\textbf{Paso 1}
\begin{equation*}
\dfrac{3}{\sqrt{\sqrt{3}+\sqrt{2}}} \cdot \dfrac{\sqrt{\sqrt{3}+\sqrt{2}}}{\sqrt{\sqrt{3}+\sqrt{2}}} = \dfrac{3\sqrt{\sqrt{3}+\sqrt{2}}}{\sqrt{3}+\sqrt{2}}
\end{equation*}
\textbf{Paso 2}
\begin{equation*}
\dfrac{3\sqrt{\sqrt{3}+\sqrt{2}}}{\sqrt{3}+\sqrt{2}}\cdot\dfrac{\sqrt{3}+\sqrt{2}}{\sqrt{3}+\sqrt{2}} = \dfrac{3\sqrt{\sqrt{3}+\sqrt{2}}\left(\sqrt{3}+\sqrt{2}\right)}{\sqrt{3}^2+\sqrt{2}^2}
\end{equation*}
\textbf{Paso 3}
\begin{equation*}
 \dfrac{3\sqrt{\sqrt{3}+\sqrt{2}}\left(\sqrt{3}+\sqrt{2}\right)}{\sqrt{3}^2+\sqrt{2}^2} =  \dfrac{3\sqrt{\sqrt{3}+\sqrt{2}}\left(\sqrt{3}+\sqrt{2}\right)}{5}
\end{equation*}
¿En cuál de los pasos efectuados por Juan se comete el primer error?
\begin{vertical}
\alternativa Paso 1
\alternativa Paso 2
\alternativa Paso 3
\alternativa Todos los pasos son correctos
\end{vertical}

\pregunta Una bonificación, en el comercio, se denomina a una rebaja aplicada
sobre una cantidad en un tanto por ciento. Esta tiene lugar cuando el
volumen de ventas es significativo y/o se tiene en consideración el
tipo de cliente. Si una empresa textil compra telas importadas desde
la India por un total de 10000 dólares americanos recibiendo una
bonificación del \mbox{16 \%}, ¿cuál es valor de la compra de textiles en
pesos chilenos? Considera que 1 dólar americano es equivalente a 792 pesos
chilenos.
\begin{vertical}
\alternativa $1\,267\,200$
\alternativa $6\,652\,800$
\alternativa $7\,793\,820$
\alternativa $8\,046\,720$
\alternativa $9\,187\,200$
\end{vertical}

\pregunta ¿Cuál es el valor de $\log_2 15 - \log_2 3 - \dfrac{\log_6 5}{\log_6 2}$ ?
\begin{vertical}
\alternativa $\dfrac{\left(\log_2 5\right)\left(\log_6 5\right)}{\log_6 2}$
\alternativa $\log_6 2$
\alternativa $\log_2 5 - \log_6 5 + \log_6 2$
\alternativa 1
\alternativa 0
\end{vertical}

\pregunta Si se considera que aproximadamente $\log \dfrac{1}{\sqrt{5}} = -0,349$, ¿cuál es el valor de $\log \sqrt{80}$?
\begin{vertical}
\alternativa $-2,55$
\alternativa $0,950$
\alternativa $0,953$
\alternativa $1,651$
\alternativa $3,047$
\end{vertical}

\pregunta El nivel de intensidad del sonido $B$ medido en decibeles se relaciona con la intensidad sonora $I$ medida en [Watts/metros$^2$] mediante la fórmula:
\begin{equation*}
B = 10 \cdot \log\left(\dfrac{I}{10^{-12}}\right).
\end{equation*}
Si el sonido de un avión durante el despegue tiene un nivel de intensidad de 150 decibeles, ¿cuál es su intensidad sonora en [Watts/metros$^2$]?
\begin{vertical}
\alternativa 10
\alternativa 100
\alternativa 1000
\alternativa 10000
\end{vertical}

\pregunta Un capital inicial $C_0$ es colocado a un interés compuesto anual de un $r\%$ y al transcurrir $n$ años se obtiene un capital final $C_f$. El interés compuesto anual está dado por la expresión $C_f = C_0 \cdot \left(1 + \dfrac{r}{100}\right)^n$. Si un capital es colocado a un interés compuesto anual del 2\%, ¿a los cuántos años este capital se duplicará?
\begin{vertical}
\alternativa $\log_{1,02} 2$
\alternativa $\log_2 1,02$
\alternativa $\log_{1,2} 2$
\alternativa $\log_2 1,2$
\end{vertical}

\pregunta Si $a$ es un número racional y $n$ un número entero, ¿cuál de las siguientes expresiones es igual a $3a^n - 6a^{n-3}$?
\begin{vertical}
\alternativa $3a^n(-2a^3)$
\alternativa $3a^{n-3}(a^{n-2}-2a)$
\alternativa $3a^n(a^2-2^{-3})$
\alternativa $3a^{n-3}(a^3-2)$
\end{vertical}

\pregunta En la siguiente tabla se muestran los valores para dos variables positivas $X$ e $Y$ que son inversamente proporcionales.
\begin{centrado}
\begin{tblr}{|c|c|}
\hline
$X$ & $Y$ \\
\hline
$2p + 2$ & 5 \\
\hline
$q$ & $2p$ \\
\hline
\end{tblr}
\end{centrado}
¿Qué expresión representa $p$ en términos de $q$, para $q > 5$?
\begin{vertical}
\alternativa $\dfrac{5}{q-5}$
\alternativa $\dfrac{2}{5(q-5)}$
\alternativa $\dfrac{5}{2(q-5)}$
\alternativa $\dfrac{5}{5-q}$
\end{vertical}

\pregunta Considera que las siguientes gráficas muestran los ingresos, en miles de pesos, de una empresa tecnológica por la venta de Tablets y celulares.
\begin{tcolorbox}[blank,sidebyside]
\begin{tikzpicture}
  \datavisualization [school book axes, visualize as line,
  all axes={min value=0,max value=5},
  x axis={length=3cm,label=Celulares,ticks and grid={major at={2 as $a$, 4 as $2a$}}},
  y axis={length=2.5cm,label={Miles de pesos},ticks and grid={major at={2 as $800$, 4 as $1600$}}},
  %f1={label in data={text=$L_1$, when=x is 5}},
  %f2={label in data={text=$L_2$, when=x is 1}}
  ]
  data point [x=0,y=0]
  data point [x=4,y=4];
\end{tikzpicture}
\tcblower
\begin{tikzpicture}
  \datavisualization [school book axes, visualize as line,
  all axes={min value=0,max value=5},
  x axis={length=3cm,label=Tablets,ticks and grid={major at={2 as $b$, 4 as $2b$}}},
  y axis={length=2.5cm,label={Miles de pesos},ticks and grid={major at={2 as $600$, 4 as $1200$}}},
  %f1={label in data={text=$L_1$, when=x is 5}},
  %f2={label in data={text=$L_2$, when=x is 1}}
  ]
  data point [x=0,y=0]
  data point [x=4,y=4];
\end{tikzpicture}
\end{tcolorbox}
¿Cuál de las siguientes expresiones representa la relación entre $a$ y $b$ si ambas gráficas tienen la misma constante de proporcionalidad?
\begin{vertical}
\alternativa $4a = 3b$
\alternativa $3a = 2b$
\alternativa $3a = 4b$
\alternativa $2a = 3b$
\end{vertical}

\pregunta ¿Cuál es el valor de $x$ en la ecuación $4 - \dfrac{x+1}{3} = 1$?
\begin{vertical}
\alternativa 0
\alternativa 8
\alternativa 10
\alternativa 12
\end{vertical}

\pregunta Considera un rectángulo de largo $a$ centímetros y ancho $b$
centímetros, tal que ~$7 < a < 15$~ y ~$3 < b < 6$.

Si el largo y el ancho
del rectángulo se triplican, ¿cuál de las siguientes desigualdades
contiene solamente a todos los posibles valores que puede tomar el
aumento $A$ del perímetro del rectángulo?
\begin{vertical}
\alternativa $30 < A < 63$
\alternativa $40 < A < 84$
\alternativa $18 < A < 106$
\alternativa $60 < A < 126$
\end{vertical}

\pregunta Sea el sistema de ecuaciones $~\begin{+cases}[columns={colsep=1pt}]2x+4y&=1\\-3x+y&=2\end{+cases}$,~ en $x$ e $y$, ¿cuál es el valor de $x+y$?
\begin{vertical}
\alternativa $-\dfrac{4}{5}$
\alternativa $-\dfrac{1}{2}$
\alternativa $-1$
\alternativa $0$
\end{vertical}

\pregunta ¿Cuáles son los valores de $a$ y $b$, respectivamente, para
los cuales el sistema
$\begin{+cases}[columns={colsep=1pt}] ax + by &= 8 \\ 3x - y &= 1 \end{+cases}$,~
 en $x$ e $y$, tiene infinitas soluciones?
\begin{vertical}
\alternativa $24~~\text{ y }~~-8$
\alternativa $-24~~\text{ y }~~8$
\alternativa $3~~\text{ y }~~-1$
\alternativa $-3~~\text{ y }~~1$
\end{vertical}

\pregunta Una tienda comercial tiene en promoción sets de potes plásticos en
tamaño pequeño o mediano.

La promoción consiste en llevar un set de 5 unidades de potes plásticos de
tamaño pequeño por \$1000 y un set de 3 potes plásticos de tamaño mediano
por \$1500.

Si una persona gasta \$18000 comprando un total de 54 potes plásticos,
¿cuántos sets de potes medianos compró?
\begin{vertical}
\alternativa 14
\alternativa 8
\alternativa 7
\alternativa 6
\end{vertical}

\pregunta Si ~$\dfrac{1}{3}$~ y ~$-\dfrac{1}{6}$~ son las soluciones de una ecuación de segundo grado, ¿cuál de las siguientes ecuaciones corresponde a ellas?
\begin{vertical}
\alternativa $18x^2+3x-1=0$
\alternativa $18x^2-3x-1=0$
\alternativa $18x^2-x-3=0$
\alternativa $18x^2+3x+1=0$
\end{vertical}

\pregunta ¿Cuál de las siguientes condiciones para $K$ permite asegurar que
las soluciones de la ecuación $x^2+Kx+4=0$, en $x$, no sean números reales?
\begin{vertical}
\alternativa $K \le 4$
\alternativa $K < 4$
\alternativa $0 < K < 4$
\alternativa $-4 < K < 4$
\alternativa $-4 < K < 0$
\end{vertical}

\pregunta Las dimensiones de un club de boxeo rectangular son $(3x-1)$ metros de ancho y $(3x+2)$ metros de largo. En él se construye un Ring de lucha, el cual ocupa una superficie cuadrada de lado $(2x-1)$ metros, tal como se muestra en la figura.
\begin{centrado}
\begin{tikzpicture}[scale=0.7]
\draw (-4,-2) rectangle (4,2);
\draw (-1,-1) coordinate (A) rectangle (2,1.5) coordinate (B);
\node at ($(A)!0.5!(B)$) {Ring};
\end{tikzpicture}
\end{centrado}
Si la superficie para entrenar fuera del ring es igual a $144 \text{ m}^2$, ¿cuál de las siguientes ecuaciones permite determinar todas las dimensiones involucradas?
\begin{vertical}
\alternativa $5x^2+7x=147$
\alternativa $x^2+3x=144$
\alternativa $5x^2-x=145$
\alternativa $13x^2-x=145$
\end{vertical}

\pregunta Si en un determinado momento un euro corresponde a 1,08 dólares
americanos y un dólar americano corresponde a 817 pesos chilenos,
¿cuál de las siguientes funciones permite calcular la cantidad de
 pesos chilenos si se cuenta con $x$ euros?
\begin{vertical}
\alternativa $f(x) = 817x + 1,08$
\alternativa $t(x) = 1,08x + 817$
\alternativa $p(x) = \dfrac{817}{1,08}x$
\alternativa $h(x) = 817 \cdot 1,08x$
\end{vertical}

\pregunta Considera los romboides de largo $b$ y lado $x$, cuyo
perímetro es $P$ y de área $A$. Si $f$ corresponde a la altura
del romboide, ¿cuál de las siguientes
funciones representa a $f$?
\begin{vertical}
\alternativa $f(x) = \dfrac{2A-P}{2x}$
\alternativa $f(x) = \dfrac{4A}{P-2x}$
\alternativa $f(x) = \dfrac{P-2A}{2x}$
\alternativa $f(x) = \dfrac{P}{2x}+2x$
\alternativa $f(x) = \dfrac{2A}{P-2x}$
\end{vertical}

\pregunta Dada la función cuadrática $f(x) = mx^2+nx+p$, ¿cuál de las siguientes opciones representa la ecuación de su eje de simetría?
\begin{vertical}
\alternativa $x = -\dfrac{n}{2m}$
\alternativa $x = \dfrac{p}{2m}$
\alternativa $x = n^2-4mp$
\alternativa $y = -\dfrac{n}{p}$
\alternativa $y = mx+n$
\end{vertical}

\pregunta Considera la función $g$ definida por $g(x)=(a-x)(x-b)$, con $a$ y $b$ constantes reales positivas, tal que $a > b$. ¿Cuál de los siguientes gráficos representa mejor la gráfica de $g$?
\begin{alternativasgraficas}
\alternativa%
 \begin{tikzpicture}[declare function={f(\x)=(\x)^2;}]
  \datavisualization [school book axes, visualize as smooth line,
  all axes={length=3cm,ticks={major at={0}}},
  x axis={label=$x$,min value=-1,max value=4,ticks={major at={1.29 as [node style={xshift=-5pt,anchor=south,yshift=-1em}] $a$, 2.7 as [node style={xshift=5pt,anchor=south,yshift=-1em}] $b$}}},
  y axis={label=$y$,min value=-1,max value=2.2},
  %f1={label in data={text=$L_1$, when=x is 5}},
  %f2={label in data={text=$L_2$, when=x is 1}}
  ]
  data [format=function] {
  var x : interval [0.5:3.5];
  func y = f(\value x - 2) - 0.5;
  };
\end{tikzpicture}
\alternativa%
 \begin{tikzpicture}[declare function={f(\x)=(\x)^2;}]
  \datavisualization [school book axes, visualize as smooth line,
  all axes={length=3cm,ticks={major at={0}}},
  x axis={label=$x$,min value=-1,max value=4,ticks={major at={1.29 as [node style={above left,yshift=3pt}] $b$, 2.7 as [node style={above right,yshift=3pt}] $a$}}},
  y axis={label=$y$,min value=-2,max value=1.2},
  %f1={label in data={text=$L_1$, when=x is 5}},
  %f2={label in data={text=$L_2$, when=x is 1}}
  ]
  data [format=function] {
  var x : interval [0.5:3.5];
  func y = -f(\value x - 2) + 0.5;
  };
\end{tikzpicture}
\alternativa%
 \begin{tikzpicture}[declare function={f(\x)=(\x)^2;}]
  \datavisualization [school book axes, visualize as smooth line,
  all axes={length=3cm,ticks={major at={0}}},
  x axis={label=$x$,min value=-1,max value=4,ticks={major at={1.29 as [node style={xshift=-5pt,anchor=south,yshift=-1em}] $b$, 2.7 as [node style={xshift=5pt,anchor=south,yshift=-1em}] $a$}}},
  y axis={label=$y$,min value=-1,max value=2.2},
  %f1={label in data={text=$L_1$, when=x is 5}},
  %f2={label in data={text=$L_2$, when=x is 1}}
  ]
  data [format=function] {
  var x : interval [0.5:3.5];
  func y = f(\value x - 2) - 0.5;
  };
\end{tikzpicture}
\alternativa%
 \begin{tikzpicture}[declare function={f(\x)=(\x)^2;}]
  \datavisualization [school book axes, visualize as smooth line,
  all axes={length=3cm,ticks={major at={0}}},
  x axis={label=$x$,min value=-4,max value=1,ticks={major at={-2.7 as [node style={anchor=south,yshift=5pt,xshift=-6pt}] $-b$, -1.3 as [node style={anchor=south,yshift=5pt,xshift=6pt}] $-a$}}},
  y axis={label=$y$,min value=-2,max value=1.2},
  %f1={label in data={text=$L_1$, when=x is 5}},
  %f2={label in data={text=$L_2$, when=x is 1}}
  ]
  data [format=function] {
  var x : interval [-3.5:-0.5];
  func y = -f(\value x + 2) + 0.5;
  };
\end{tikzpicture}
\end{alternativasgraficas}

\pregunta Considera un terreno rectangular en que su largo mide el doble de su ancho. Si la diagonal del terreno mide 25 metros, ¿cuál es la medida de su superficie en metros cuadrados?
\begin{vertical}
\alternativa 100
\alternativa 250
\alternativa $\dfrac{1250}{3}$
\alternativa $\dfrac{25\sqrt{3}}{3}$
\end{vertical}

\pregunta Considera una circunferencia de centro en $P$ y radio $r$, tal como se representa en la figura adjunta.
\begin{centrado}
 \begin{tikzpicture}[declare function={f(\x)=(\x)^2;}]
  \datavisualization [school book axes, visualize as smooth line,
  x axis={length=6cm,label=$x$,min value=-6,max value=1,ticks and grid={major at={,-4 as $n$}}},
  y axis={length=4cm,label=$y$,min value=-4,max value=1,ticks and grid={major at={0 as $O$,-2.5 as $-3m$}}},
  %f1={label in data={text=$L_1$, when=x is 5}},
  %f2={label in data={text=$L_2$, when=x is 1}}
  ]
  info' {
    \draw [name path=circulo] (visualization cs: x=-4,y=-2.5) circle [radius=1cm];
    \draw [name path=linea,help lines] (visualization cs: x=-4,y=-2.5) -- (visualization cs: x=0,y=0);
    \node at (visualization cs: x=-4,y=-2.5) [below left] {$P$};
    \node [name intersections={of=circulo and linea,by=x}] at (x) [above] {$Q$};
  };
\end{tikzpicture}
\end{centrado}
Si $m>0, n<0$ y la distancia de $P$ a $Q$ es de $\dfrac{1}{8}$ de la distancia del punto $P$ al origen $O$, con $Q$ un punto sobre la circunferencia, ¿cuál de las siguientes expresiones representa el área de la semicírculo de centro $P$?
\begin{vertical}
\alternativa $\pi \cdot \dfrac{9m^2+n^2}{128}$
\alternativa $\pi \cdot \dfrac{9m^2+n^2}{64}$
\alternativa $\pi \cdot \dfrac{9m^2+n^2}{16}$
\alternativa $\pi \cdot \dfrac{9m^2+n^2}{8}$
\end{vertical}

\pregunta Considera un cubo cuya área total es $\dfrac{8}{3}$ centímetros cuadrados. ¿Cuál es el volumen de dicho cubo, en centímetros cúbicos?
\begin{vertical}
\alternativa $\dfrac{16}{81}$
\alternativa $\dfrac{4}{9}$
\alternativa $\dfrac{8}{27}$
\alternativa $\dfrac{8}{3}$
\end{vertical}

\pregunta En el plano cartesiano se representan los vectores $-\vec{w}$ y $\vec{z}$.
\begin{centrado}
 \begin{tikzpicture}[declare function={f(\x)=(\x)^2;}]
  \datavisualization [school book axes, visualize as smooth line,
  x axis={length=6cm,label=$x$,min value=-3.5,max value=2.5,ticks and grid={major at={-3,-2}},ticks={minor at={-1,1,2}}},
  y axis={length=4cm,label=$y$,min value=-5.5,max value=2.5,ticks and grid={major at={-5,2}},ticks={minor at={-4,-3,-2,-1,1}}},
  %f1={label in data={text=$L_1$, when=x is 5}},
  %f2={label in data={text=$L_2$, when=x is 1}}
  ]
  info' {
    \draw[->] (visualization cs: x=0,y=0) -- (visualization cs: x=-3,y=2) node [midway,above] {$-\vec{w}$};
    \draw[->] (visualization cs: x=0,y=0) -- (visualization cs: x=-2,y=-5) node [midway,above left] {$\vec{z}$};
  };
\end{tikzpicture}
\end{centrado}
¿Cuáles son las coordenadas de $\left(\vec{w}-3\vec{z}\right)$?
\begin{vertical}
\alternativa $\left(-11,\,1\right)$
\alternativa $\left(7,\,-11\right)$
\alternativa $\left(9,\,-7\right)$
\alternativa $\left(9,\,13\right)$
\end{vertical}

\pregunta Considera un cuadrado $ABCD$ de lado $AB$, en el plano cartesiano,
cuyos vértices A y B tienen coordenadas $\left(1,\,1\right)$ y
$\left(3,\,1\right)$, respectivamente.
Si al cuadrado $ABCD$ se le aplica una homotecia de razón $-4$ con
respecto al punto $B$ formándose el cuadrado $A'B'C'D'$, ¿cuál es el
perímetro del hexágono $AC'D'A'CD$, en unidades?
\begin{vertical}
\alternativa 48
\alternativa $44\sqrt{17}$
\alternativa $40 + 4\sqrt{17}$
\alternativa $20 + 4\sqrt{17}$
\end{vertical}

\pregunta En el plano cartesiano se ubican los triángulos rectángulos $P$ y $Q$, tal como se representa en la figura adjunta.
\begin{centrado}
 \begin{tikzpicture}[declare function={f(\x)=(\x)^2;}]
  \datavisualization [school book axes, visualize as smooth line,
  x axis={length=5cm,label=$x$,min value=-3.5,max value=3.5,ticks and grid={major at={-3,-1,1,3}}},
  y axis={length=6cm,label=$y$,min value=-4.5,max value=4.5,ticks and grid={major at={-4,-1,1,4}}},
  %f1={label in data={text=$L_1$, when=x is 5}},
  %f2={label in data={text=$L_2$, when=x is 1}}
  ]
  info' {
    \draw[] (visualization cs: x=-1,y=1) -- (visualization cs: x=-1,y=4) -- (visualization cs: x=-3,y=1) --  (visualization cs: x=-1,y=1) node [above left=0.8em] {$p$};
    \draw[] (visualization cs: x=1,y=-4) -- (visualization cs: x=3,y=-4)  node [above left=0.8em] {$Q$} -- (visualization cs: x=3,y=-1) --  (visualization cs: x=1,y=-4);
  };
\end{tikzpicture}
\end{centrado}
¿Con cuál de las siguientes transformaciones isométricas en el plano,
se NO obtiene un cuadrilátero como resultado de la unión de ambas figuras?
\begin{vertical}
\alternativa Con una simetría a la figura $P$ respecto al eje $y$, seguida de una simetría respecto al \mbox{eje $x$}.
\alternativa Con una simetría central a la figura $P$ respecto al origen del sistema coordenado.
\alternativa Con dos rotaciones de 90° a la figura $Q$ con centro en $(0,\,0)$, seguida de una traslación según el vector $(0,\,-3)$.
\alternativa Con una simetría de la figura $Q$ respecto al eje $x$, seguida de una traslación según el vector $(-4,\,-3)$.
\alternativa Con una rotación de 180° de la figura $Q$ con respecto al origen, seguida de una traslación según el vector $(2,\,0)$.
\end{vertical}

\pregunta En un parque acuático se instala una escalera perpendicular al suelo y en ella, dos trampolines de forma paralela, tal como se muestra la figura.
\begin{centrado}
\begin{tikzpicture}[]
  \fill[pattern={north east lines}] (0,0) rectangle (0.3,5);
  \draw[] (-3.5,4.8) rectangle (0,5);
  \draw (-3.5,4.8) [dashed] -- node [midway,pin={[pin edge={<-,solid,thick}]-180:Línea visual}] {} (0,0) coordinate [pos=0.6] (A);
  \draw (A) rectangle ($(A -| 0,0) +(0,0.2)$);
\end{tikzpicture}
\end{centrado}
Si los puntos extremos de los trampolines están en la misma línea visual con la base. Y sabiendo que el trampolín más pequeño se ubica a 1,8 metros del suelo y que los trampolines miden 2,5 y 3 metros respectivamente. ¿A qué altura se encuentra el trampolín más largo, en metros?
\begin{vertical}
\alternativa 0,36
\alternativa 2,16
\alternativa 1,2
\alternativa 2,3
\end{vertical}

\pregunta En la figura adjunta se representa una homotecia de centro O y razón k, que transforma al trapecio ABCD en el trapecio EFGH.
\begin{centrado}
\vspace*{3ex}
\begin{tikzpicture}
  \node[shape=trapezium,inner sep=1cm,rotate=60,draw,name=b] at (-2,3) {};
  \node[shape=trapezium,inner sep=0.7cm,rotate=240,draw,name=c] at (3,1) {};
  \draw[dashed,shorten <=-3em,shorten >=-3em] (b.top right corner) node [above=3pt] {$B$} -- (c.top right corner) node [below=3pt] {$F$};
  \draw[dashed,shorten <=-3em,shorten >=-3em] (b.top left corner) node [below left=2pt] {$C$} -- (c.top left corner) node [above right=2pt] {$G$};
  \draw[dashed,shorten <=-3em,shorten >=-3em,name path=AE,line cap=rect] (b.bottom right corner) node [right=3pt] {$A$} -- (c.bottom right corner) node [left=3pt] {$E$};
  \draw[dashed,shorten <=-3em,shorten >=-3em,name path=DH] (b.bottom left corner) node [below=3pt] {$D$} -- (c.bottom left corner) node [above=3pt] {$H$};
  \fill[name intersections={of=AE and DH,by=O}] (O) circle [radius=3pt] node [yshift=12pt,xshift=4pt] {$O$};
\end{tikzpicture}
\vspace*{3ex}
\end{centrado}
Si AB > EF, ¿cuál de las siguientes relaciones es correcta con respecto a los valores que puede tomar la razón k?
\begin{vertical}
\alternativa $k > 1$
\alternativa $k < -1$
\alternativa $0 < k < 1$
\alternativa $-1 < k < 0$
\end{vertical}

\pregunta A un cuadrado se le realiza una homotecia con centro de homotecia desconocido, obteniéndose un cuadrado homotético cuya área es la cuarta parte de la figura original, tal como se representa en la figura adjunta.
\begin{centrado}
\begin{tikzpicture}
  \node [draw,shape=rectangle,inner sep=1cm] (A) at (0,0) {};
  \node [draw,shape=rectangle,inner sep=0.5cm,right=1cm of A] {};
\end{tikzpicture}
\end{centrado}
¿Cuál de las siguientes afirmaciones es verdadera?
\begin{vertical}
\alternativa Si el lado del cuadrado original mide 8 cm, el lado del cuadrado homotético mide 2 cm.
\alternativa Si la diagonal del cuadrado homotético mide $8\sqrt{2}$ cm, la medida de la diagonal del cuadrado original es $16\sqrt{2}$ cm.
\alternativa La razón de homotecia solo puede tomar valores entre $0 < k < 1$.
\alternativa La razón de homotecia $k$ es siempre positiva.
\end{vertical}

\pregunta Considera el triángulo $ABC$ rectángulo en $C$ de la figura adjunta.
\begin{centrado}
\begin{tikzpicture}
  \draw [] (0,0) coordinate[label=below left:$C$] (C) -- (3,0) coordinate [label=below right:$A$] (A) -- (0,3) coordinate [label=above left:$B$] (B) -- (0,0);
  \draw[] pic ["$\alpha$",draw,angle eccentricity=0.7,angle radius=1.2cm] {angle = C--B--A};
  \draw[] pic ["$\beta$",draw,angle eccentricity=0.7,angle radius=1.2cm] {angle = B--A--C};
  \draw (C) -- ($(C)!10pt!(A)$) -- ([turn]90:10pt) -- ([turn]90:10pt);
  \node at ($(C)!0.5!(A)$) [below] {$12p$};
  \node at ($(C)!0.5!(B)$) [left] {$5p$};
\end{tikzpicture}
\end{centrado}
¿Cuál es el valor de $\tan(\alpha) - \sin(\beta)$ para $p>0$?
\begin{vertical}
\alternativa $\dfrac{96}{65}$
\alternativa $\dfrac{131}{65}$
\alternativa $\dfrac{144}{65}$
\alternativa $\dfrac{181}{65}$
\end{vertical}

\pregunta Un estudiante construye en el plano cartesiano una
circunferencia con centro en el origen $O$ y de cierto radio con un compás,
tal como se representa en la figura adjunta.
\begin{centrado}
\begin{tikzpicture}
\draw[dashed] (0,0) circle [radius=2cm];
\draw[->] (0,0) -- (2.5,0) node [right] {$X$};
\draw[->] (0,0) -- (0,2.5) node [above] {$Y$};
\node at (0,0) [below left] {$O$};
\fill (30:2cm) coordinate (P) circle [radius=3pt] node [above right] {$P$};
\draw[->,shorten >=4pt] (0,0) -- (P) node [midway,sloped,above] {Radio};
\draw[->] (0,0) -- (2.5,0);
\fill (P |- 0,0) coordinate (Q) circle [radius=3pt] node [below left] {$Q$};
\draw[] ($(Q)!2!(P)$) node [above] {$L$} -- ($(P)!2.5!(Q)$);
\end{tikzpicture}
\end{centrado}
Si el punto $P$ tiene coordenadas $(m,n)$, con $P$ en el primer cuadrante,
desde el cual se construye una recta $L$ perpendicular al eje $X$ formándose
el punto $Q$, ¿cuál de las siguientes razones representa el coseno del
ángulo $OPQ$?
\begin{vertical}
\alternativa $\dfrac{m}{\sqrt{m^2+n^2}}$
\alternativa $\dfrac{n}{\sqrt{m^2+n^2}}$
\alternativa $\dfrac{\sqrt{m^2+n^2}}{m}$
\alternativa $\dfrac{\sqrt{m^2+n^2}}{n}$
\end{vertical}

\pregunta Un bloque se encuentra suspendido al techo por dos cuerdas, tal como se muestra en la figura adjunta.
\begin{centrado}
\begin{tikzpicture}
\fill[pattern={north east lines}] (0,0) rectangle (8,0.2);
\draw (0.5,0) coordinate (P) -- (2.5,-2) coordinate (R) -- (7.5,0) coordinate (Q);
\node [shape=rectangle,anchor=north,inner xsep=1cm,inner ysep=0.6cm,draw,fill,pattern={north east lines}] at (R) {};
\draw[] pic ["$60^\circ$",draw,angle eccentricity=0.7,angle radius=1.2cm] {angle = R--P--Q};
\draw[] pic ["$30^\circ$",draw,angle eccentricity=0.8,angle radius=2cm] {angle = P--Q--R};
\node at ($(P)!0.5!(R)$) [below left] {2,5 m};
\node at (P) [below left] {$P$};
\node at (R) [above=4pt] {$R$};
\node at (Q) [below right] {$Q$};
\end{tikzpicture}
\end{centrado}
Considerando que P y Q son los puntos en donde se encuentran sujetas las cuerdas y que la distancia entre P y R es de 2,5 metros, ¿cuál es la distancia entre dichos puntos en metros?
\begin{vertical}
\alternativa $\dfrac{5\sqrt{3}}{3}$
\alternativa $5\sqrt{3}$
\alternativa 5
\alternativa $\dfrac{5}{2}$
\end{vertical}

\pregunta En el gráfico de la figura adjunta se representa una encuesta acerca de la cantidad de mascotas en un grupo de 200 estudiantes de un colegio.
\begin{centrado}
\begin{tikzpicture}
  \pie[text=inside,color=white]{25/0,30/1,25/2,12/3,8/4}
\end{tikzpicture}
\end{centrado}
¿Cuál de las siguientes tablas representa la información presentada en el gráfico?
\begin{alternativasgraficas}
\alternativa \begin{tblr}{colspec={|X|X|},row{1}={font=\raggedright},row{2-Z}={halign=c},width=6cm} \hline Cantidad de mascotas & Cantidad de Estudiantes \\ \hline 0 & 50 \\ \hline 1 & 60 \\ \hline 2 & 50 \\ \hline 3 & 8 \\ \hline 4 & 12 \\ \hline \end{tblr}
\alternativa \begin{tblr}{colspec={|X|X|},row{1}={font=\raggedright},row{2-Z}={halign=c},width=6cm} \hline Cantidad de mascotas & Cantidad de Estudiantes \\ \hline 0 & 50 \\ \hline 1 & 60 \\ \hline 2 & 50 \\ \hline 3 & 24 \\ \hline 4 & 16 \\ \hline \end{tblr}
\alternativa \begin{tblr}{colspec={|X|X|},row{1}={font=\raggedright},row{2-Z}={halign=c},width=6cm} \hline Cantidad de mascotas & Cantidad de Estudiantes \\ \hline 0 & 25 \\ \hline 1 & 30 \\ \hline 2 & 25 \\ \hline 3 & 12 \\ \hline 4 & 8 \\ \hline \end{tblr}
\alternativa \begin{tblr}{colspec={|X|X|},row{1}={font=\raggedright},row{2-Z}={halign=c},width=6cm} \hline Cantidad de mascotas & Cantidad de Estudiantes \\ \hline 0 & 50 \\ \hline 1 & 110 \\ \hline 2 & 160 \\ \hline 3 & 184 \\ \hline 4 & 200 \\ \hline \end{tblr}
\end{alternativasgraficas}

\pregunta La tabla adjunta muestra, en intervalos, la edad a la que un grupo de 400 hombres son abuelos por primera vez.
\begin{centrado}
\begin{tblr}{|c|c|}
\hline
Edad en años & Frecuencia Relativa \\
\hline
{[45,50[}     & 0,245 \\
\hline
{[50,55[}     & 0,205 \\
\hline
{[55,60[}     & 0,05 \\
\hline
{[60,65[}     & 0,05 \\
\hline
{[65,70[}     & 0,205 \\
\hline
{[70,75]}     & 0,245 \\
\hline
\end{tblr}
\end{centrado}
Según los datos de la tabla, ¿cuál de las siguientes afirmaciones verdadera?
\begin{vertical}
\alternativa El 45\% de los hombres tiene más de 65 años y a lo más 75 años.
\alternativa El 75,5\% de los hombres fue abuelo por primera vez al menos a los 70 años.
\alternativa La mediana de las edades de los hombres es un valor entre 55 y 65 años.
\alternativa Una de las modas se encuentra en el intervalo [45,50[.
\end{vertical}

\pregunta Considera el conjunto de datos finito y ordenado en forma creciente $\{V_1, V_2, V_3, \dots, V_k\}$, cuyo promedio es $v_t$. Al multiplicar todos estos datos por $j^2$, y luego sumarles a todos esos datos el número $u$, con $j \neq 0$ y $u \neq 0$. ¿Cuál de las siguientes expresiones representa el nuevo promedio de los datos?
\begin{vertical}
\alternativa $u \cdot v_t + j^2$
\alternativa $j^2 \cdot v_t + u$
\alternativa $j^2 \cdot v_t$
\alternativa $v_t + u$
\end{vertical}

\pregunta Los diagramas de cajón adjuntos muestran la distribución de las edades de los trabajadores de una empresa en dos plantas diferentes A y B.

\vspace*{10pt}
\begin{tikzpicture}[]
    \begin{axis}[
      %title={Comparando número de amig@s por grupo},
      ytick={1,2},
      yticklabels={Planta A,Planta B},
      %boxplot/draw direction=y,
      xtick={20,23,26,35,45,52,56,58,63},
      xmajorgrids=true,
      xlabel={Edades (en años)},
      boxplot/box extend=0.5,
      width=.85\textwidth,
      height=4cm,
      enlarge y limits=0.5,
      y dir=reverse,
    ]
      \addplot+ [boxplot prepared={
        lower whisker=23,
        lower quartile=45,
        median=56,
        upper quartile=58,
        upper whisker=63,
      },solid]  coordinates {};
      \addplot+ [boxplot prepared={
        lower whisker=20,
        lower quartile=26,
        median=35,
        upper quartile=52,
        upper whisker=58,
      },solid]  coordinates {};
    \end{axis}
  \end{tikzpicture}
¿Cuál de las siguientes afirmaciones es siempre verdadera?
\begin{vertical}
\alternativa Entre el primer y segundo cuartil se tiene la misma cantidad de trabajadores en la planta A y en la planta B.
\alternativa La edad promedio de los trabajadores de la planta B es 35 años.
\alternativa Exactamente el 25\% de los trabajadores de la planta A tienen una edad menor que 45 años.
\alternativa El rango de edad de los trabajadores de la planta B es menor que en la planta A.
\end{vertical}

\pregunta Se tiene una muestra de cuatro datos cuantitativos A, B, C y D.
El profesor pide a un estudiante calcular el coeficiente de variación
de la muestra. Al realizar los cálculos, el estudiante, efectúa el
siguiente procedimiento:\\[5pt]
\textbf{Paso 1} Calcula el promedio de la muestra.
\begin{equation*} \bar{X} = \dfrac{A+B+C+D}{4} \end{equation*}
\textbf{Paso 2} Calcula la varianza.
\begin{equation*} \dfrac{(\bar{X}+A)^2+(\bar{X}+B)^2+(\bar{X}+C)^2+(\bar{X}+D)^2}{4} \end{equation*}
\textbf{Paso 3} Calcula la desviación estándar.
\begin{equation*} \sqrt{\dfrac{(\bar{X}+A)^2+(\bar{X}+B)^2+(\bar{X}+C)^2+(\bar{X}+D)^2}{4}} \end{equation*}
\textbf{Paso 4} Calcula el coeficiente de variación.
\begin{equation*} \dfrac{1}{\bar{X}}\sqrt{\dfrac{(\bar{X}+A)^2+(\bar{X}+B)^2+(\bar{X}+C)^2+(\bar{X}+D)^2}{4}} \end{equation*}
Pero el profesor le comenta que posiblemente cometió un error. ¿Cuál es el primer paso en que el estudiante comete el error?
\begin{vertical}
\alternativa En el paso 1
\alternativa En el paso 2
\alternativa En el paso 3
\alternativa En el paso 4
\end{vertical}

\pregunta Considera la siguiente tabla de frecuencias.
\begin{centrado}
\begin{tblr}{|c|c|}
\hline
Dato & Frecuencia \\
\hline
1    & 10 \\
\hline
2    & 25 \\
\hline
3    & 30 \\
\hline
4    & 25 \\
\hline
5    & 10 \\
\hline
\end{tblr}
\end{centrado}
¿Cuál es valor de la varianza?
\begin{vertical}
\alternativa $\sqrt{1,3}$
\alternativa 1,2
\alternativa 1,3
\alternativa $\sqrt{2}$
\alternativa 2
\end{vertical}

\pregunta Considera una distribución de 10 datos cuya varianza es P. Si cada uno de los datos se aumentan en cinco unidades, ¿cuál es la desviación estándar de la nueva distribución?
\begin{vertical}
\alternativa P
\alternativa $P+\sqrt{5}$
\alternativa $\sqrt{P}+5$
\alternativa $P+5$
\alternativa $\sqrt{P}$
\end{vertical}

\pregunta En una encuesta se consulta a un grupo de personas acerca de la comida típica favorita para las fiestas patrias. Los resultados de la encuesta se registran en la siguiente tabla.
\begin{centrado}
\begin{tblr}{|l|c|c|}
\hline
 & Hombres & Mujeres \\
\hline
Empanadas de Horno & 150     & 200 \\
\hline
Anticuchos         & 90      & 110 \\
\hline
Carne de Vacuno    & 250     & 100 \\
\hline
Pastel de Choclo   & 35      & 55 \\
\hline
\end{tblr}
\end{centrado}
Si se elige una persona al azar del grupo, ¿cuál es la probabilidad de que sea mujer sabiendo que no le gustan los anticuchos ni el pastel de choclo?
\begin{vertical}
\alternativa $\dfrac{300}{990}$
\alternativa $\dfrac{400}{700}$
\alternativa $\dfrac{300}{700}$
\alternativa $\dfrac{400}{990}$
\end{vertical}

\pregunta Una empresa que fabrica baterías para automóviles tiene dos sucursales A y B. En la sucursal A el 7\% de las baterías que vende son devueltas por tener algún tipo de desperfecto, en cambio en la sucursal B el 85\% de las baterías vendidas no han presentado fallas. Considerando una muestra de 500 baterías vendidas, 175 de la sucursal A y el resto de la sucursal B, extrayendo una de ellas al azar resultando ser defectuosa, ¿cuál es la probabilidad de que provenga de la sucursal B?
\begin{vertical}
\alternativa $\dfrac{\dfrac{325}{500} \cdot \dfrac{15}{100}}{\dfrac{175}{500} \cdot \dfrac{7}{100} + \dfrac{325}{500} \cdot \dfrac{15}{100}}$
\alternativa $\dfrac{\dfrac{175}{500} \cdot \dfrac{7}{100}}{\dfrac{175}{500} \cdot \dfrac{7}{100} + \dfrac{325}{500} \cdot \dfrac{15}{100}}$
\alternativa $\dfrac{325}{500} \cdot \dfrac{85}{100}$
\alternativa $\dfrac{325}{500} \cdot \dfrac{15}{100}$
\end{vertical}

\pregunta En un hospital se ha realizado un estudio a un grupo de pacientes que padecen daltonismo con el propósito de clasificarlos según el tipo: acromático, monocromático, dicromático y tricromático, obteniéndose los resultados que se resumen en la tabla adjunta.
\begin{centrado}
\begin{tblr}{|l|c|c|}
\hline
Daltonismo      & Pacientes Hombres & Pacientes Mujeres \\
\hline
Acromático      & 60                & 80 \\
\hline
Monocromático   & 53                & 51 \\
\hline
Dicromático     & 12                & 22 \\
\hline
Tricromático    & 45                & 55 \\
\hline
\end{tblr}
\end{centrado}
Si se elige un paciente al azar, ¿cuál es la probabilidad de elegir una mujer del grupo sabiendo que padece un daltonismo dicromático?
\begin{vertical}
\alternativa $\dfrac{11}{104}$
\alternativa $\dfrac{17}{104}$
\alternativa $\dfrac{11}{189}$
\alternativa $\dfrac{17}{189}$
\alternativa $\dfrac{11}{17}$
\end{vertical}

\pregunta Una urna contiene 5 bolitas del mismo tipo, pero todas de diferente color. Se requiere extraer todas las muestras posibles sin reposición de 3 elementos cada una. ¿Cuál es la diferencia positiva entre el número de muestras con orden y sin orden que se pueden extraer?
\begin{vertical}
\alternativa $\left(1-\dfrac{1}{3!}\right)\cdot\dfrac{5!}{2!}$
\alternativa $\dfrac{1}{3!}\cdot\dfrac{5!}{3!}$
\alternativa $\left(1-\dfrac{1}{2!}\right)\cdot\dfrac{5!}{3!}$
\alternativa $\dfrac{5!}{2!}$
\alternativa $\dfrac{5!}{2!} - \dfrac{5!}{3!}$
\end{vertical}

\pregunta Considera un conjunto de T cuerpos geométricos agrupados en forma de cono, cilindro y esfera, tal como se detalla en la tabla adjunta.
\begin{centrado}
\begin{tblr}{|l|c|}
\hline
Forma geométrica & Cantidad \\
\hline
Cono             & $n$ \\
\hline
Cilindro         & $m$ \\
\hline
Esfera           & $k$ \\
\hline
\end{tblr}
\end{centrado}
Si en cada categoría los elementos son de diferente color, ¿de cuántas maneras distintas se pueden ordenar los T cuerpos en una fila sin que se mezclen las formas geométricas?
\begin{vertical}
\alternativa $\displaystyle\binom{T}{3}\cdot n!\cdot m!\cdot k!$
\alternativa $\displaystyle\binom{T}{n}\cdot\binom{T}{m}\cdot\binom{T}{k}\cdot 3!$
\alternativa $T!$
\alternativa $n!\cdot m!\cdot k!\cdot 3!$
\end{vertical}

\pregunta Sean $a$ y $b$ dos números reales tales que $a<1<b$. Es posible establecer que $\log_b (b-a)=1$, si se sabe que:
\begin{verticaln}
\alternativa $a=0$
\alternativa $b-a=10$
\end{verticaln}
\begin{vertical}
\alternativa (1) por sí sola
\alternativa (2) por sí sola
\alternativa Ambas juntas, (1) y (2)
\alternativa Cada una por sí sola, (1) ó (2)
\alternativa Se requiere información adicional
\end{vertical}

\pregunta Se puede afirmar que la gráfica de la función cuadrática $f(x)=ax^2+bx+c$ tiene ramas con orientación positiva, si se sabe que:
\begin{verticaln}
\alternativa Las soluciones de la ecuación cuadrática $ax^2+bx+c=0$ asociada a $f(x)$ corresponden a valores positivos.
\alternativa El discriminante de la función $f$ es mayor que cero.
\end{verticaln}
\begin{vertical}
\alternativa (1) por sí sola
\alternativa (2) por sí sola
\alternativa Ambas juntas, (1) y (2)
\alternativa Cada una por sí sola, (1) ó (2)
\alternativa Se requiere información adicional
\end{vertical}

\pregunta Si $m$ es un número real, es posible determinar la distancia del segmento AB de coordenadas $A(4,-1)$ y $B(1,m)$, si se sabe que:
\begin{verticaln}
\alternativa B es el resultado de rotar 90° sentido horario el punto A.
\alternativa A es el resultado de una traslación del punto B según el vector $(3,0)$.
\end{verticaln}
\begin{vertical}
\alternativa (1) por sí sola
\alternativa (2) por sí sola
\alternativa Ambas juntas, (1) y (2)
\alternativa Cada una por sí sola, (1) ó (2)
\alternativa Se requiere información adicional
\end{vertical}

\pregunta Si un evento A y un evento B son independientes, se puede determinar la probabilidad de que suceda A o B, si se sabe que:
\begin{verticaln}
\alternativa $P(B)=0,7$
\alternativa $P(A)=0,3$
\end{verticaln}
\begin{vertical}
\alternativa (1) por sí sola
\alternativa (2) por sí sola
\alternativa Ambas juntas, (1) y (2)
\alternativa Cada una por sí sola, (1) ó (2)
\alternativa Se requiere información adicional
\end{vertical}

\end{preguntas}

\end{document}