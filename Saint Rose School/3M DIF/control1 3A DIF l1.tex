\def\curso{Tercero medio A}
\def\puntaje{7}
\def\titulo{Control}
\def\subtitulo{Técnicas de conteo}
\documentclass[]{srs}

\begin{document}

\section*{Objetivo}
  Utilizar técnicas de conteo para la resolución de problemas.

\section*{Instrucciones}
  Tiene 45 minutos para completar la evaluación.
  Incluya desarrollo en todas sus respuestas, y recuerde marcar o señalizar
  el resultado final en cada pregunta.

\section*{Criterios de evaluación}
  En la corrección, se asignará el puntaje a cada pregunta según los siguientes criterios.
\begin{center}
  \begin{tblr}{width=\linewidth,colspec={X[1,c]|X[6]}, hline{1,Z} = {1}{-}{}, hline{1,Z} = {2}{-}{},
      hlines, cells={valign=m}, row{1} = {bg=black!15}}
      Puntaje asignado & \SetCell{c} Criterios o indicadores \\
      +50\% & Señala clara y correctamente cuál es la solución o el resultado de la pregunta hecha
      en el enunciado.\\
      +50\% & Incluye un desarrollo que relata de manera clara y ordenada los procedimientos
      \mbox{necesarios} para solucionar la problemática. En caso de estar incompleto o con
      errores el desarrollo, se asignará puntaje parcial si se muestra dominio de los
       contenidos y conceptos involucrados.\\
      0\% &  La respuesta es incorrecta. De haber desarrollo, este tiene errores conceptuales.\\
  \end{tblr}
\end{center}
\separador[2mm]

Encuentre la solución de los siguientes problemas.
\begin{preguntas}(1)
  \pregunta ¿De cuántas maneras Carlos podrá seleccionar los sabores de helado, si hay 7
  helados disponibles y desea escoger por lo menos 3 sabores? [2 puntos]
  \begin{malla}[height=8cm]
  \end{malla}
  \pregunta ¿De cuántas formas pueden 10 objetos dividirse en dos grupos de 4 y 6 objetos
  respectivamente? \mbox{[2 puntos]}
  \begin{malla}[height=8cm]
  \end{malla}
  \pregunta De un grupo de 12 personas se busca escoger un grupo de 9 personas para abordar
  un bote con 8 remos y con un timón. sabiendo que de las 12 personas solo 4 pueden llevar
  el timón, ¿de cuántas maneras diferentes se puede formar la tripulación?, y ¿de cuántas
  maneras distintas se pueden ubicar dentro del bote? [3 puntos]
  \begin{malla}[height=15cm]
  \end{malla}


\end{preguntas}

\end{document}