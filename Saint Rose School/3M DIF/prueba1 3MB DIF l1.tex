\def\curso{Tercero medio B}
\def\puntaje{22}
\def\titulo{Prueba}
\def\subtitulo{Técnicas de conteo}
\documentclass[]{srs}

\begin{document}

\section*{Objetivo}
  Utilizar permutaciones y combinatoria para la resolución de problemas.

\section*{Instrucciones generales}
  Tiene 1 hora y 30 minutos para responder la evaluación. Esta es individual y debe
  usar solo sus materiales personales para trabajar durante este periodo, no los solicite
  a un compañero durante la evaluación.

\section{Opciones múltiples}

\section*{Instrucciones}
Lee con atención y escoge la alternativa que responde la pregunta en cada enunciado.

\section*{Criterios de evaluación}
En la corrección de esta sección, se asignará 2 puntos al marcar la alternativa correcta.
Las alternativas corregidas serán consideradas incorrectas, es decir, marque solo una
alternativa por enunciado.

\separador[2mm]

\begin{preguntas}[after-item-skip=2cm]
  \pregunta El valor de $\dfrac{11!-10!}{11!+10!}$ es: \\
  \begin{vertical}
    \alternativa $1/21$
    \alternativa $0$
    \alternativa $5/6$
    \alternativa $1$
    \alternativa No se puede determinar.

  \end{vertical}

  \pregunta Con los dígitos 1, 2, 3, 4, 5, 6 y 7. ¿Cuántos números de tres cifras distintas
  se pueden formar?: \\
  \begin{vertical}
    \alternativa $210$
    \alternativa $168$
    \alternativa $180$
    \alternativa $294$
    \alternativa $343$
  \end{vertical}

  \pregunta ¿De cuántas maneras se pueden ordenar una niña y tres niños en una fila?: \\
  \begin{vertical}
    \alternativa $8$
    \alternativa $12$
    \alternativa $16$
    \alternativa $24$
    \alternativa $4$
  \end{vertical}

  \pregunta ¿De cuántas maneras pueden sentarse en una fila 3 chilenos, 2 argentinos y
  4 brasileños si los de una misma nacionalidad deben quedar juntos? \\
  \begin{vertical}
    \alternativa $9!$
    \alternativa $3!\cdot 2!\cdot 4!$
    \alternativa $9$
    \alternativa $(3+2+4)!$
    \alternativa $3!\cdot 2!\cdot 4!\cdot 3!$
  \end{vertical}

  \pregunta ¿De cuántas maneras pueden sentarse 7 personas alrededor de una mesa, si
  el abuelo de la familia ya tiene su lugar asignado?: \\
  \begin{vertical}
    \alternativa $360$
    \alternativa $720$
    \alternativa $5040$
    \alternativa $2520$
    \alternativa $1440$
  \end{vertical}

  \pregunta ¿Cuántas palabras con o sin sentido, se pueden formar con todas las
  letras de la palabra \\\mbox{PARALELEPIPEDO}? \\
  \begin{vertical}
    \alternativa $14!$
    \alternativa $14\cdot 2\cdot 2\cdot 3$
    \alternativa $\dfrac{14!}{3!\cdot 2!\cdot 2!\cdot 3!}$
    \alternativa $\dfrac{14!}{3!+2!+2!+3!}$
    \alternativa $14 + 2 + 2 + 3$
  \end{vertical}

  \pregunta Un curso está formado por 16 niñas y 14 niños, si se quiere formar una
  comisión de cuatro estudiantes, en la que debe haber a lo menos tres niños, ¿cuántas
  combinaciones posibles hay?: \\
  \begin{vertical}
    \alternativa $C^4_{30}$
    \alternativa $C^3_{14}\cdot C^1_{16}$
    \alternativa $C^3_{14}\cdot C^1_{16} + C^4_{14}$
    \alternativa $C^3_{30} + C^1_{29}$
    \alternativa $C^4_{14}$
  \end{vertical}

  \pregunta ¿Cuántas diagonales se pueden trazar en un decágono regular (polígono de 10
  lados)?: \\
  \begin{vertical}
    \alternativa $C^{2}_{10} - 10$
    \alternativa $C^{2}_{10}$
    \alternativa $2!$
    \alternativa $10!$
    \alternativa $C^{2}_{8} - 10$
  \end{vertical}

\end{preguntas}

\section{Preguntas abiertas}

\section*{Instrucciones}
Lea atentamente el enunciado de cada pregunta, considere los datos entregados y
responda a la problemática planteada, explicando y detallando claramente
tanto su proceso como sus resultados.

\section*{Criterios de evaluación}
  En la corrección de esta sección, cada pregunta tiene 3 puntos y se asignará
  el puntaje de cada una según los siguientes criterios:
\begin{center}
  \begin{tblr}{width=\linewidth,colspec={X[1,c]|X[6]}, hline{1,Z} = {1}{-}{}, hline{1,Z} = {2}{-}{},
      hlines, cells={valign=m}, row{1} = {bg=black!15}}
      Puntaje asignado & \SetCell{c} Criterios o indicadores \\
      +50\% & Señala clara y correctamente cuál es la solución o el resultado de la pregunta hecha
      en el enunciado.\\
      +50\% & Incluye un desarrollo que relata de manera clara y ordenada los procedimientos
      \mbox{necesarios} para solucionar la problemática. En caso de estar incompleto o con
      errores el desarrollo, se asignará puntaje parcial si se muestra dominio de los
       contenidos y conceptos involucrados.\\
      0\% &  La respuesta es incorrecta. De haber desarrollo, este tiene errores conceptuales.\\
  \end{tblr}
\end{center}
\separador[2mm]

\begin{preguntas}(1)
  \pregunta Un club tiene 15 miembros, 10 hombres y 5 mujeres. ¿Cuántos comités de 8
  miembros se pueden formar si cada uno de ellos debe tener por lo menos 3 mujeres?
  \begin{malla}[height=10cm]
  \end{malla}
  \pregunta Un barco posee 2 astas de bandera, y cada asta tiene tres posiciones distintas
  en las que se puede colocar una bandera. Suponiendo que el barco lleva 10 banderas
  diferentes para hacer señales. ¿Cuántas señales distintas se pueden hacer si cada
  asta tiene a los más 1 bandera colgada?
  \begin{malla}[height=14cm]
  \end{malla}
\end{preguntas}





\end{document}