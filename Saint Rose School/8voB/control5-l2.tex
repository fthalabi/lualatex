\def\curso{Octavo básico B}
\def\puntaje{12}
\def\titulo{Control}
\def\subtitulo{Potencias}
\def\fecha{25 de Abril, 2025}
\documentclass[]{srs}

\begin{document}

\section*{Objetivo}
  Realizar cálculos que involucren el uso de propiedades de potencias.

\section*{Instrucciones}
  Cada pregunta tiene 2 puntos y cuenta con 40 minutos para completar
  la evaluación. Incluya desarrollo en todas sus respuestas, y recuerde marcar o señalizar
  el resultado final en cada pregunta.

\section*{Criterios de evaluación}
  En la corrección, se asignará el puntaje a cada pregunta según los siguientes criterios.
\begin{center}
  \begin{tblr}{width=\linewidth,colspec={X[1,c]|X[6]}, hline{1,Z} = {1}{-}{}, hline{1,Z} = {2}{-}{},
      hlines, cells={valign=m}, row{1} = {bg=black!15}}
      Puntaje asignado & \SetCell{c} Criterios o indicadores \\
      +50\% & Señala clara y correctamente cuál es la solución o el resultado de la pregunta hecha
      en el enunciado.\\
      +50\% & Incluye un desarrollo que relata de manera clara y ordenada los procedimientos
      \mbox{necesarios} para solucionar la problemática. En caso de estar incompleto o con
      errores el desarrollo, se asignará puntaje parcial si se muestra dominio de los
       contenidos y conceptos involucrados.\\
      0\% &  La respuesta es incorrecta. De haber desarrollo, este tiene errores conceptuales.\\
  \end{tblr}
\end{center}
\separador[2mm]


\begin{preguntas}(1)
  \pregunta ¿Cuál es el valor de la siguiente expresión?
  \begin{mcaja}
    \left(1,\overline{6}\right)^0
  \end{mcaja}
  \begin{malla}[height=5cm]
  \end{malla}
  \pregunta Reduce la siguiente expresión a una sola potencia
  \begin{mcaja}
    5^{38}\cdot 5^{23}
  \end{mcaja}
  \begin{malla}[height=5cm]
  \end{malla}
  \pregunta Reduce la siguiente expresión a una sola potencia:
  \begin{mcaja}
    3^{24}\div 3^{18}
  \end{mcaja}
  \begin{malla}[height=5cm]
  \end{malla}

  \pregunta Determina los valores de $a$ y $b$ para que se cumpla la siguiente igualdad.
  \begin{mcaja}
    \dfrac{2^3 \cdot 5^4 \cdot 2^3}{2^4 \cdot 5^2} = 2^{a}\cdot 5^{b}
  \end{mcaja}
  \begin{malla}[height=10cm]
  \end{malla}

  \pregunta Determina los valores de $a$ y $b$ para que se cumpla la siguiente igualdad.
  \begin{mcaja}
    \dfrac{3^{5} \cdot 4^6 \cdot 12^3}{3^3 \cdot 4^2 \cdot 12^2} = 3^{a}\cdot 4^{b}
  \end{mcaja}
  \begin{malla}[height=10cm]
  \end{malla}

  \pregunta Determina los valores de $a$ y $b$ para que se cumpla la siguiente igualdad.
  \begin{mcaja}
    16 \cdot 10^2 \cdot 25 \cdot 10^6 \cdot 40 \cdot 10^3 = 2^{a}\cdot 5^{b}
  \end{mcaja}
  \begin{malla}[height=10cm]
  \end{malla}

\end{preguntas}

\end{document}