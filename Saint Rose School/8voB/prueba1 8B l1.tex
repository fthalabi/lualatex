\def\curso{Octavo básico B}
\def\puntaje{20}
\def\titulo{Prueba}
\def\subtitulo{Operatoria de números}
\documentclass[]{srs}

\begin{document}

\section*{Objetivo}

Mostrar que comprenden como hacer cálculos con números enteros o racionales,
en la resolución expresiones aritméticas y/o en la solución problemas contextualizados.

\section*{Instrucciones generales}
  Tiene 1 hora y 30 minutos para responder la evaluación. Esta es individual y debe
  usar solo sus materiales personales para trabajar durante este periodo, no los solicite
  a un compañero durante la evaluación.

\section{Opciones múltiples}

\section*{Instrucciones}
Lea atentamente cada enunciado y escoja la alternativa correcta en cada caso.

\section*{Criterios de evaluación}
En la corrección de esta sección, se asignará 2 puntos al marcar la alternativa correcta.
Las alternativas corregidas serán consideradas incorrectas, es decir, marque solo una
alternativa por enunciado.

\separador[2mm]

\def\casilla{\tcbox[on line, boxsep=0pt, top=7pt, bottom=7pt, left=7pt, right=7pt,
box align=center,baseline=5pt,colback=white]{}}

\begin{preguntas}[after-item-skip=2cm]
  \pregunta ¿Cuál es el resultado de la siguiente operación?
  \begin{equation*}
    -\left\{-60\cdot 2\div\left[6\div\left(-2\right)\right]\right\}
  \end{equation*}
  \begin{vertical}
    \alternativa $40$
    \alternativa $-40$
    \alternativa $10$
    \alternativa $-10$
  \end{vertical}

  \pregunta Fernando está aprendiendo a caminar y cada tres pasos que da hacia adelante,
  da dos hacia atrás. Si Fernando ha dado 12 pasos hacia adelante,
  ¿cuántos dio hacia atrás? \\
  \begin{vertical}
    \alternativa 2 pasos.
    \alternativa 8 pasos.
    \alternativa 12 pasos.
    \alternativa 10 pasos.
  \end{vertical}

  \pregunta ¿Cuál(es) de los pares de valores puede(n) ir en los casilleros de la siguiente
  operación?\\
  \begin{equation*}
  \left(3\cdot \casilla \cdot(-2)\right)\div\casilla =4
  \end{equation*}
  \begin{vertical*}
    \alternativa 4 y $-6$
    \alternativa 10 y 15
    \alternativa $-2$ y 3
  \end{vertical*}
  \begin{vertical}
    \alternativa Solo I.
    \alternativa Solo II.
    \alternativa Solo III.
    \alternativa Solo I y III.
  \end{vertical}


  \pregunta ¿Cuál es el resultado de la siguiente operación?
  \begin{equation*}
    18 -(-45) \div 9 + (-2)\cdot(-1)
  \end{equation*}
  \begin{vertical}
    \alternativa $-9$
    \alternativa $-5$
    \alternativa $9$
    \alternativa $25$
  \end{vertical}

  \pregunta ¿Cuál de las siguientes expresiones \underline{\textbf{No}}
  es equivalente a $-3,2$?: \\
  \begin{vertical}
    \alternativa $\dfrac{3}{2}-4,7$
    \alternativa $2\dfrac{4}{5}-6$
    \alternativa $-\,\dfrac{3}{4}-2,45$
    \alternativa $\left(-\,\dfrac{5}{4}\right)+5,05$
  \end{vertical}

  \pregunta Paulina quiere utilizar $9/20$ de su patio como área verde.
  De esta, quiere destinar $4/3$ solo para
  flores. ¿Qué fracción de área verde usará solo para flores? \\
  \begin{vertical}
    \alternativa $\dfrac{5}{3}$
    \alternativa $\dfrac{3}{5}$
    \alternativa $\dfrac{27}{80}$
    \alternativa $\dfrac{4}{3}$
  \end{vertical}

  \pregunta La señora Carmen compró papas en la feria y las distribuyó en 3 bolsas,
  cada una de las cuales quedó con 2,8 kilogramos.
  ¿Cuántos kilogramos de papa compró la señora Carmen? \\
  \begin{vertical}
    \alternativa $8\dfrac{5}{2}$
    \alternativa $6\dfrac{5}{2}$
    \alternativa $6\dfrac{2}{5}$
    \alternativa $8\dfrac{2}{5}$
  \end{vertical}

\end{preguntas}

\section{Preguntas abiertas}

\section*{Instrucciones}
Lea atentamente el enunciado de cada pregunta, considere los datos entregados y
responda a la problemática planteada, explicando y detallando claramente
tanto su proceso como sus resultados.

\section*{Criterios de evaluación}
  En la corrección de esta sección, cada pregunta tiene 3 puntos y se asignará
  el puntaje de cada una según los siguientes criterios:
\begin{center}
  \begin{tblr}{width=\linewidth,colspec={X[1,c]|X[6]}, hline{1,Z} = {1}{-}{}, hline{1,Z} = {2}{-}{},
      hlines, cells={valign=m}, row{1} = {bg=black!15}}
      Puntaje asignado & \SetCell{c} Criterios o indicadores \\
      +50\% & Señala clara y correctamente cuál es la solución o el resultado de la pregunta hecha
      en el enunciado.\\
      +50\% & Incluye un desarrollo que relata de manera clara y ordenada los procedimientos
      \mbox{necesarios} para solucionar la problemática. En caso de estar incompleto o con
      errores el desarrollo, se asignará puntaje parcial si se muestra dominio de los
       contenidos y conceptos involucrados.\\
      0\% &  La respuesta es incorrecta. De haber desarrollo, este tiene errores conceptuales.\\
  \end{tblr}
\end{center}
\separador[2mm]

\begin{preguntas}(1)
  \pregunta Calcule el valor de:\\[-5pt]
  \begin{equation*}
   -\,\dfrac{1}{3} - \left(-\,\dfrac{5}{6}\right) -\,2\dfrac{3}{4}
  \end{equation*}\\[-20pt]
  \begin{malla}[height=9cm]
  \end{malla}
  \pregunta Calcule el valor de:\\[-5pt]
  \begin{equation*}
    -\left[-\,\dfrac{2}{5}-\left(\dfrac{2}{3} + 3\dfrac{2}{6} - 0,1\right)\right] - 2,5
  \end{equation*}\\[-20pt]
  \begin{malla}[height=13cm]
  \end{malla}
\end{preguntas}





\end{document}