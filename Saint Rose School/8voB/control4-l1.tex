\def\curso{Octavo básico B}
\def\puntaje{10}
\def\titulo{Control}
\def\subtitulo{Operatoria de números racionales}
\def\fecha{11 de Abril, 2025}
\documentclass[]{srs}

\begin{document}

\section*{Objetivo}
  Realizar cálculos y solucionar problemas contextualizados que involucren sumar, restar,
  multiplicar y dividir fracciones y/o número decimales.

\section*{Instrucciones}
  Cada pregunta tiene 2 puntos y cuenta con 30 minutos para completar
  la evaluación. Incluya desarrollo en todas sus respuestas, y recuerde marcar o señalizar
  el resultado final en cada pregunta.

\section*{Criterios de evaluación}
  En la corrección, se asignará el puntaje a cada pregunta según los siguientes criterios.
\begin{center}
  \begin{tblr}{width=\linewidth,colspec={X[1,c]|X[6]}, hline{1,Z} = {1}{-}{}, hline{1,Z} = {2}{-}{},
      hlines, cells={valign=m}, row{1} = {bg=black!15}}
      Puntaje asignado & \SetCell{c} Criterios o indicadores \\
      +50\% & Señala clara y correctamente cuál es la solución o el resultado de la pregunta hecha
      en el enunciado.\\
      +50\% & Incluye un desarrollo que relata de manera clara y ordenada los procedimientos
      \mbox{necesarios} para solucionar la problemática. En caso de estar incompleto o con
      errores el desarrollo, se asignará puntaje parcial si se muestra dominio de los
       contenidos y conceptos involucrados.\\
      0\% &  La respuesta es incorrecta. De haber desarrollo, este tiene errores conceptuales.\\
  \end{tblr}
\end{center}
\separador[2mm]


\begin{preguntas}(1)
  \pregunta Transforme el siguiente número a fracción:
  \begin{mcaja}
    5,42\overline{31}
  \end{mcaja}
  \begin{malla}[height=7cm]
  \end{malla}
  \pregunta Determine el valor de la siguiente expresión:
  \begin{mcaja}
    \left(1-\,\dfrac{1}{2}\right)\div\left(\dfrac{3}{4}-\,\dfrac{5}{8}\right)
  \end{mcaja}
  \begin{malla}[height=10cm]
  \end{malla}
  \pregunta Determine el valor de la siguiente expresión:
  \begin{mcaja}
    \dfrac{3}{2}\cdot\left(-2\dfrac{4}{9} + 1,\overline{6}\right)
  \end{mcaja}
  \begin{malla}[height=10cm]
  \end{malla}

  \pregunta María ha regado cinco octavos de su jardín, lo cual corresponde a 35 [$m^2$].
  ¿Cuántos metros cuadrados de su jardín faltan por regar?
  \begin{malla}[height=10cm]
  \end{malla}

  \pregunta Un rectángulo mide 23,4 centímetros de alto y 36,72 centímetros de ancho.
  ¿Cuánto mide el área del rectángulo? Determine exactamente el valor del área, no deje
  expresado sus resultados.
  \begin{malla}[height=10cm]
  \end{malla}

\end{preguntas}

\end{document}