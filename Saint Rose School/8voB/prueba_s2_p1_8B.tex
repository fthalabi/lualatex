\def\curso{Octavo básico B}
\def\puntaje{16}
\def\titulo{Prueba}
\def\subtitulo{Ecuaciones lineales}
\def\fecha{25 de julio, 2025}
\documentclass[]{srs2}

\newsavebox{\rubricaUno}
\begin{lrbox}{\rubricaUno}
\def\stop{{\FA\char"E204}\space}
\begin{centrado}
\begin{tblr}{width=176mm,colspec={X[6]X[1,c]X[1,c]}, hline{2,Z} = {1}{-}{}, hline{2,Z} = {2}{-}{},
    hlines, cells={valign=m}, row{2} = {bg=black!15},vlines}
    \SetCell[c=3]{c} \stop No rayar la tabla, es solo para uso del docente \stop &  &\\
    \SetCell{c} Criterios para la corrección & Puntaje & Asignación \\
    Encuentra la solución de la ecuación. & $2$ & \\
    El resultado está completamente simplificado (irreducible). & $1$ & \\
    Señala claramente los resultados, e incluye ordenadamente los
    procedimientos necesarios para solucionar la problemática. & $1$ & \\
\end{tblr}
\end{centrado}
\end{lrbox}


\begin{document}

\subsection*{Objetivo}
  Solucionar ecuaciones lineales con coeficientes enteros y racionales.

\subsection*{Instrucciones generales}
  Cuenta con 40 minutos para completar la evaluación. Esta es individual y debe usar solo
  sus materiales personales para trabajar durante este periodo, no los solicite a un compañero
  durante la evaluación.

  Lea atentamente cada enunciado, siga las instrucciones, y responda cada
  pregunta siguiendo los criterios descritos en su correspondiente rúbrica. Cumplir
  con todos estos criterios, es necesario para obtener el puntaje completo de cada pregunta.
  En algunos casos, se puede asignar puntaje parcial por un criterio medianamente logrado.

\separador[2mm]

  Solucione cada una de las siguientes ecuaciones y determine el valor de la incógnita
  en cada caso. Recuerde incluir un desarrollo ordenado y consistente con los resultados
  obtenidos.

\begin{preguntas}
  \pregunta $-8-2a=-5a-26$
  \begin{malla}[7]
  \end{malla}
  \usebox{\rubricaUno}

  \pregunta $-2\left(-2b+1\right)+3\left(-3+2b\right)=-3\left(-3-2b\right)-b$
  \begin{malla}[7]
  \end{malla}
  \usebox{\rubricaUno}

  \pregunta $\dfrac{c}{2}+ \dfrac{c}{4} + \dfrac{11}{6}= -\dfrac{c}{4} + \dfrac{1}{2} + \dfrac{2}{3}$
  \begin{malla}[8]
  \end{malla}
  \usebox{\rubricaUno}

  \pregunta $-\dfrac{2}{3}\left(-2d+\dfrac{9}{4}\right) + 2\left(-\dfrac{1}{16}+\dfrac{d}{4}\right)= \dfrac{1}{2}\left(-\dfrac{7d}{3}-\dfrac{5}{4}\right)$
  \begin{malla}[12]
  \end{malla}
  \usebox{\rubricaUno}

\end{preguntas}


\end{document}