\def\titulo{Guía}
\def\subtitulo{Función afín}
\def\curso{Octavo básico B}
\documentclass[sin nombre]{srs2}
\usepackage{xfrac}

\tcbset{raster before skip=10pt,raster after skip=30pt}

\begin{document}
\vspace*{0.5cm}
Encuentre la ecuación de la recta en cada uno de los siguientes casos.
\begin{preguntas}
\pregunta La pendiente de la recta es igual a $-3$ y pasa por
el punto $\left(2,\,3\right)$.
\begin{malla}[3]
\end{malla}
\pregunta La recta pasa por los puntos $\left(-1,\,2\right)$ y
$\left(-3,\,10\right)$.
\begin{malla}[4]
\end{malla}
\end{preguntas}

\begin{aviso}
  \underline{Recordatorio}:
  \begin{itemize}
  \item La intersección con el eje $x$ es en $\left(?,\,0\right)$.
  \item La intersección con el eje $y$ es en $\left(0,\,?\right)$.
  \end{itemize}
\end{aviso}
Encuentre la intersección con el eje $x$ e $y$ en cada caso.
\begin{preguntas}
\pregunta La ecuación de la recta es $y=-2x+4$.
\begin{malla}[4]
\end{malla}
\pregunta La recta pasa por los puntos $\left(-10,\,3\right)$ y
$\left(5,\,-6\right)$.
\begin{malla}[5]
\end{malla}
\end{preguntas}

Represente gráficamente cada recta en su respectivo plano cartesiano. %
\def\largo{2.5}
\begin{preguntas}[raster columns=2]
\pregunta \newline
\begin{tikzpicture}
  \node at (-2,2) [anchor=east] {$y=-\dfrac{3}{2}x+3$};
  \draw[->] (-\largo,0) -- (\largo,0) node [right] {$x$};
  \draw[->] (0,-\largo) -- (0,\largo) node [above] {$y$};
\end{tikzpicture}
\pregunta \newline
\begin{tikzpicture}
  \node at (-2,2) [anchor=east] {$y=\dfrac{3}{4}x+3$};
  \draw[->] (-\largo,0) -- (\largo,0) node [right] {$x$};
  \draw[->] (0,-\largo) -- (0,\largo) node [above] {$y$};
\end{tikzpicture}
\pregunta \newline
\begin{tikzpicture}
  \node at (-2,2) [anchor=east] {$y=-\dfrac{2}{3}x-2$};
  \draw[->] (-\largo,0) -- (\largo,0) node [right] {$x$};
  \draw[->] (0,-\largo) -- (0,\largo) node [above] {$y$};
\end{tikzpicture}
\pregunta \newline
\begin{tikzpicture}
  \node at (-2,2) [anchor=east] {$y=\dfrac{2}{3}x-2$};
  \draw[->] (-\largo,0) -- (\largo,0) node [right] {$x$};
  \draw[->] (0,-\largo) -- (0,\largo) node [above] {$y$};
\end{tikzpicture}

\end{preguntas}

Determine cuál es la ecuación de la recta a partir de
la gráfica en cada caso. \def\espacio{\contorno[4pt][7pt]{\phantom{22}}}
\begin{preguntas}[raster columns=2]
\pregunta \newline
\begin{tikzpicture}
  \node at (-1,2) [anchor=east] {$y=\espacio\cdot x+\espacio$};
  \draw[->] (-\largo,0) -- (\largo,0) node [right] {$x$};
  \draw[->] (0,-\largo) -- (0,\largo) node [above] {$y$};
  \draw[shorten >=-20pt, shorten <=-20pt] (-1,0) node [above left] {$-2$} -- (0,2) node [right] {$4$};

\end{tikzpicture}
\pregunta \newline
\begin{tikzpicture}
  \node at (-1,2) [anchor=east] {$y=\espacio\cdot x+\espacio$};
  \draw[->] (-\largo,0) -- (\largo,0) node [right] {$x$};
  \draw[->] (0,-\largo) -- (0,\largo) node [above] {$y$};
    \draw[shorten >=-20pt, shorten <=-20pt] (1,0) node [above right] {$2$} -- (0,2) node [right] {$4$};

\end{tikzpicture}
\pregunta \newline
\begin{tikzpicture}
  \node at (-1,2) [anchor=east] {$y=\espacio\cdot x+\espacio$};
  \draw[->] (-\largo,0) -- (\largo,0) node [right] {$x$};
  \draw[->] (0,-\largo) -- (0,\largo) node [above] {$y$};
  \draw[shorten >=-20pt, shorten <=-20pt] (-1,0) node [below left] {$-2$} -- (0,-2) node [right] {$-4$};

\end{tikzpicture}
\pregunta \newline
\begin{tikzpicture}
  \node at (-1,2) [anchor=east] {$y=\espacio\cdot x+\espacio$};
  \draw[->] (-\largo,0) -- (\largo,0) node [right] {$x$};
  \draw[->] (0,-\largo) -- (0,\largo) node [above] {$y$};
  \coordinate (A) at (0,-1.3);
  \coordinate (B) at (1.5,0.8);
  \draw[shorten >=-30pt, shorten <=-30pt] (A) node [left] {$-3$} -- (B);
  \draw[dashed] (B) -- (B -| 0,0) node [left] {$1$};
  \draw[dashed] (B) -- (B |- 0,0) node [below] {$2$};
  \fill (B) circle[radius=2pt];
\end{tikzpicture}
\end{preguntas}

Encuentre la intersección de las rectas en cada caso.
\begin{preguntas}[raster row skip=25pt]
\pregunta \begin{columnas}[0.4]
\begin{equation*}
\begin{+array}{ll}
\text{Recta 1} :& y=-2x+1 \\
\text{Recta 2} :& y=3x-4
\end{+array}
\end{equation*}
\siguiente
\begin{malla}[5]
\end{malla}
\end{columnas}

\pregunta \begin{columnas}[0.4]
\begin{center}
\begin{tikzpicture}
  \draw[->,name path=X] (-2,0) -- (3,0) node [right] {$x$};
  \draw[->] (0,-3) -- (0,3) node [above] {$y$};
  \draw[shorten >=-20pt, shorten <=-20pt,name path=A] (0,-1) node [left] {$-2$} -- (1.5,1.5);
  \draw[shorten >=-20pt, shorten <=-20pt,name path=B] (0,2) node [left] {$4$} -- (2,0) node [below] {$4$};
  \fill[name intersections={of=A and B,by=P}] (P) circle[radius=2pt];
  \node[name intersections={of=X and A,by=Q}] at (Q) [below] {$1$};
\end{tikzpicture}
\end{center} \siguiente
\begin{malla}[7]
\end{malla}
\end{columnas}
\pregunta La primera recta es la que pasa por los puntos
$\left(1/2,\,1/8\right)$ y
$\left(-2/5,\,4/5\right)$. Por otro lado, la segunda recta
es la que pasa por los puntos $\left(-3/2,\,-3/4\right)$ y
$\left(6/5,\,3/20\right)$.
\begin{malla}[9]
\end{malla}
\end{preguntas}

Utilice la función afín para describir cada situación y encontrar
la solución de cada problema.

\begin{preguntas}[raster row skip=10pt]
\pregunta Jorge caminará hoy 3 km, y decide que cada vez que cumpla
una semana, a partir de hoy, agregará a
su recorrido 1,5 km. ¿En cuántas semanas más habrá caminado 22,5 km?
\begin{malla}[5]
\end{malla}
\pregunta Se tiene una caja con bloques de madera, todos de igual masa. Si
la caja con 16 bloques tiene una masa total de 15 kg, y la misma caja con
21 bloques tiene una masa total de 19 kg, ¿cuál es la masa de 3 bloques?
\begin{malla}[5]
\end{malla}
\pregunta En la escala Fahrenheit, el agua se congela a $32^{\circ}$ F y su
temperatura de ebullición es de $212^{\circ}$ F. Escriba una fórmula para
convertir de grados Celsius ($\text{C}^\circ$) a Fahrenheit ($\text{F}^\circ$);
otra para convertir de Fahrenheit ($\text{F}^\circ$) a Celsius ($\text{C}^\circ$);
y finalmente, encuentre a qué temperatura ambas escalas marcan los mismos grados.
\begin{malla}[9]
\end{malla}
\end{preguntas}

\end{document}