\def\titulo{Prueba}
\def\subtitulo{Función afín (parte 1)}
\def\curso{Octavo básico B}
\def\fecha{Jueves 28 de agosto}
\documentclass[]{srs2}

\begin{document}

\subsection*{Objetivo}
  Determinar la ecuación de una recta, y utilizar dicha relación para
  encontrar las coordenadas de un puntos especifico.

\subsection*{Instrucciones generales}
  Cuenta con 40 minutos para completar la evaluación. Esta es individual y debe usar solo
  sus materiales personales para trabajar durante este periodo, no los solicite a un compañero
  durante la evaluación.

  Lea atentamente cada enunciado, siga las instrucciones, y responda cada
  pregunta siguiendo los criterios descritos en su correspondiente rúbrica. Cumplir
  con todos estos criterios, es necesario para obtener el puntaje completo de cada pregunta.
  En algunos casos, se puede asignar puntaje parcial por un criterio medianamente logrado.

\separador[2mm]

\begin{preguntas}
  \pregunta Determine cual es la ecuación de la recta que tiene
  pendiente $-3$ y pasa por el punto $\left(-5,\,-4\right)$.
  \begin{malla}[9]
  \end{malla}
  \begin{pauta*}
    Determina el coeficiente de posición de la recta. & $1$ & \\
    Establece cual es la ecuación de la recta. & $1$ & \\
    Señala claramente los resultados, e incluye ordenadamente los
    procedimientos necesarios para solucionar la problemática. & $1$ & \\
  \end{pauta*}

  \pregunta Determine cual es la ecuación de la recta que pasa por los puntos (x,y)
  y (2,1).
  \begin{malla}[20]
  \end{malla}
  \begin{pauta}
    Determina la pendiente de la recta. & $1$ & \\
    Determina el coeficiente de posición de la recta. & $1$ & \\
    Establece cual es la ecuación de la recta. & $1$ & \\
    Señala claramente los resultados, e incluye ordenadamente los
    procedimientos necesarios para solucionar la problemática. & $1$ & \\
  \end{pauta}


  \pregunta para la recta y=mx+n, encuentre la intersección con los ejes x e y.
  \begin{malla}[8]
  \end{malla}
  \begin{pauta}
    Determina la pendiente de la recta. & $2$ & \\
    Determina el coeficiente de posición de la recta. & $2$ & \\
    Determina cual es la ecuación de la recta. & $1$ & \\
    Señala claramente los resultados, e incluye ordenadamente los
    procedimientos necesarios para solucionar la problemática. & $1$ & \\
  \end{pauta}

  \pregunta Represente graficamente la recta  y=mx+n, en el siguiente plano cartesiano.
\begin{columnas}[0.3]
  \def\largo{2.5}\centering
  \begin{tikzpicture}
    \draw[->] (-\largo,0) -- (\largo,0) node [right] {$x$};
    \draw[->] (0,-\largo) -- (0,\largo) node [above] {$y$};
  \end{tikzpicture}
  \siguiente
  \begin{malla}[7]
  \end{malla}
\end{columnas}


  \pregunta Determine el punto donde las rectas se intersectan.
  \begin{malla}[7]
  \end{malla}


  \pregunta problema de modelado
  \begin{malla}[5]
  \end{malla}


\end{preguntas}

\end{document}