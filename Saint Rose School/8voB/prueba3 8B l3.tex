\def\curso{Octavo básico B}
\def\puntaje{20}
\def\titulo{Prueba}
\def\subtitulo{Variación porcentual}
\def\fecha{28 de mayo, 2025}
\documentclass[]{srs}

\begin{document}

\subsection*{Objetivo}
Realizar cálculos y solucionar problemas que involucran variación porcentual y
operatoria de números racionales.

\subsection*{Instrucciones generales}
Tiene 1 hora y 30 minutos para responder la evaluación. Esta es individual y debe
usar solo sus materiales personales para trabajar durante este periodo, no los solicite
a un compañero durante la evaluación.

Para cada pregunta, lea con atención el enunciado y escoja la alternativa que lo
responde correctamente. Solo hay una alternativa correcta por cada pregunta.

\subsection*{Criterios de evaluación}
En la corrección de esta sección, se asignará 2 puntos al marcar la alternativa correcta.
Las alternativas corregidas serán consideradas incorrectas, es decir, marque solo una
alternativa por enunciado.

\separador[2mm]

\begin{preguntas}



\pregunta El $30\%$ de un número es $45$, ¿cuál es su $12\%$?
\begin{vertical}
\alternativa $25$
\alternativa $18$
\alternativa $16$
\alternativa $12$
\alternativa $8$
\end{vertical}

\pregunta El $15\%$ de $1\dfrac{2}{3}$ es:
\begin{vertical}
\alternativa $0,125$
\alternativa $0,75$
\alternativa $0,5$
\alternativa $0,45$
\alternativa $0,25$
\end{vertical}


\pregunta El $50\%$ de la mitad de un número es $20$, entonces el número es:
\begin{vertical}
\alternativa $5$
\alternativa $10$
\alternativa $20$
\alternativa $40$
\alternativa $80$
\end{vertical}

\pregunta $103$ es el $10\%$ de:
\begin{vertical}
\alternativa $1\,000$
\alternativa $1\,020$
\alternativa $1\,030$
\alternativa $1\,040$
\alternativa $1\,050$
\end{vertical}

%\pregunta ¿Qué porcentaje es $0,4\overline{2}$ de $0,\overline{76}$ ?
%\begin{vertical}
%\alternativa $32,41\%$
%\alternativa $50\%$
%\alternativa $55\%$
%\alternativa $60,8\%$
%\alternativa $181,81\%$
%\end{vertical}

\pregunta Una camisa con un $20\%$ de descuento cuesta $\$4\,000$. ¿Cuánto costaría sin la rebaja?
\begin{vertical}
\alternativa $\$4\,800$
\alternativa $\$5\,000$
\alternativa $\$5\,200$
\alternativa $\$5\,400$
\alternativa $\$5\,500$
\end{vertical}

\pregunta La cuarta parte de $0,\overline{2}$ es:
\begin{vertical}
\alternativa $0,0\overline{4}$
\alternativa $0,05$
\alternativa $0,0\overline{5}$
\alternativa $0,\overline{5}$
\alternativa $0,\overline{8}$
\end{vertical}

\pregunta En un curso hay una mujer cada $4$ hombres. ¿Qué $\%$ del curso son mujeres?
\begin{vertical}
\alternativa $20\%$
\alternativa $25\%$
\alternativa $30\%$
\alternativa $40\%$
\alternativa $80\%$
\end{vertical}

\pregunta Se ha cancelado $\$42\,000$, que corresponde al $60\%$ de una deuda. ¿Cuánto falta por pagar?
\begin{vertical}
\alternativa $\$14\,000$
\alternativa $\$28\,000$
\alternativa $\$30\,000$
\alternativa $\$70\,000$
\alternativa $\$112\,000$
\end{vertical}



\pregunta Solo $12$ alumnas, de un curso de $30$, han pagado una cuota para un paseo. ¿Qué $\%$ del curso falta por pagar?
\begin{vertical}
\alternativa $40\%$
\alternativa $45\%$
\alternativa $55\%$
\alternativa $60\%$
\alternativa $65\%$
\end{vertical}

\pregunta El estadio A de una ciudad tiene capacidad para 40.000 personas sentadas y otro B para 18.000. Se hacen eventos simultáneos; el A se ocupa hasta el 25\% de su capacidad y el B llena sólo el 50\%. ¿Cuál(es) de las siguientes afirmaciones es(son) verdadera(s) ?
\begin{verticali}
\alternativa El estadio A registró mayor asistencia de público que el B.
\alternativa Si se hubiese llevado a los asistentes de ambos estadios al A, habría quedado en éste, menos del 50\% de sus asientos vacíos.
\alternativa Los espectadores que asistieron en conjunto a los dos estadios superan en 1.000 a la capacidad de B.
\end{verticali}
\begin{vertical}
\alternativa Sólo I
\alternativa Sólo II
\alternativa Sólo III
\alternativa Sólo I y II
\alternativa Sólo I y III
\end{vertical}

\end{preguntas}

\end{document}


