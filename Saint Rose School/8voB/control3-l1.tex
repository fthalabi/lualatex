\def\curso{Octavo básico B}
\def\puntaje{8}
\def\titulo{Control}
\def\subtitulo{Operatoria de números racionales}
\documentclass[]{srs}

\begin{document}

\section*{Objetivo}
  Realizar cálculos que involucren sumas y restas de fracciones y/o número decimales.

\section*{Instrucciones}
  Cada pregunta tiene 2 puntos y cuenta con 30 minutos para completar
  la evaluación. Incluya desarrollo en todas sus respuestas, y recuerde marcar o señalizar
  el resultado final en cada pregunta.

\section*{Criterios de evaluación}
  En la corrección, se asignará el puntaje a cada pregunta según los siguientes criterios.
\begin{center}
  \begin{tblr}{width=\linewidth,colspec={X[1,c]|X[6]}, hline{1,Z} = {1}{-}{}, hline{1,Z} = {2}{-}{},
      hlines, cells={valign=m}, row{1} = {bg=black!15}}
      Puntaje asignado & \SetCell{c} Criterios o indicadores \\
      +50\% & Señala clara y correctamente cuál es la solución o el resultado de la pregunta hecha
      en el enunciado.\\
      +50\% & Incluye un desarrollo que relata de manera clara y ordenada los procedimientos
      \mbox{necesarios} para solucionar la problemática. En caso de estar incompleto o con
      errores el desarrollo, se asignará puntaje parcial si se muestra dominio de los
       contenidos y conceptos involucrados.\\
      0\% &  La respuesta es incorrecta. De haber desarrollo, este tiene errores conceptuales.\\
  \end{tblr}
\end{center}
\separador[2mm]

Calcule el resultado de las siguientes operaciones.
\begin{preguntas}(1)
  \pregunta $\dfrac{3}{4}+\dfrac{5}{12}$
  \begin{malla}[height=7cm]
  \end{malla}
  \pregunta $-\,\dfrac{3}{5} - \left(-\,\dfrac{7}{10}\right) + \dfrac{1}{2}$
  \begin{malla}[height=7cm]
  \end{malla}
  \pregunta $6\dfrac{2}{7} - 0,36$
  \begin{malla}[height=7cm]
  \end{malla}
  \pregunta $-\left[\dfrac{1}{3}-\left(0,3 - 2\dfrac{3}{5}\right)\right] - 0,2$
  \begin{malla}[height=8.5cm]
  \end{malla}
\end{preguntas}

\end{document}