\def\curso{Octavo básico B}
\def\puntaje{10}
\def\titulo{Control}
\def\subtitulo{Números racionales y variación porcentual}
\def\fecha{14 de mayo, 2025}
\documentclass[]{srs}

\begin{document}

\section*{Objetivo}
  Resolver problemáticas que involucran el uso de números racionales y/o variación porcentual.

\section*{Instrucciones}
  Cada pregunta tiene 2 puntos y cuenta con 40 minutos para completar
  la evaluación. Incluya desarrollo en todas sus respuestas, y recuerde marcar o señalizar
  el resultado final en cada pregunta.

\section*{Criterios de evaluación}
  En la corrección, se asignará el puntaje a cada pregunta según los siguientes criterios.
\begin{center}
  \begin{tblr}{width=\linewidth,colspec={X[1,c]|X[6]}, hline{1,Z} = {1}{-}{}, hline{1,Z} = {2}{-}{},
      hlines, cells={valign=m}, row{1} = {bg=black!15}}
      Puntaje asignado & \SetCell{c} Criterios o indicadores \\
      +50\% & Señala clara y correctamente cuál es la solución o el resultado de la pregunta hecha
      en el enunciado.\\
      +50\% & Incluye un desarrollo que relata de manera clara y ordenada los procedimientos
      \mbox{necesarios} para solucionar la problemática. En caso de estar incompleto o con
      errores el desarrollo, se asignará puntaje parcial si se muestra dominio de los
       contenidos y conceptos involucrados.\\
      0\% &  La respuesta es incorrecta. De haber desarrollo, este tiene errores conceptuales.\\
  \end{tblr}
\end{center}
\separador[2mm]


\begin{preguntas}(1)
  \pregunta ¿Cuántas botellas de dos quintos de litro se llenan con 28 litros de agua?
  \begin{malla}[height=7cm]
  \end{malla}
  \pregunta Si Rosita se ha comido 16 dulces, lo cual corresponde a dos novenos de todos sus dulces.
  ¿Cuántos dulces le quedan todavía a Rosita?
  \begin{malla}[height=7cm]
  \end{malla}
  \pregunta Una tienda de aparatos electrónicos decide dar 30\% de descuento en toda su mercancía;
  si el precio normal de un televisor es de cuatrocientos mil pesos,
  ¿cuánto se deberá pagar en caja por este televisor?
  \begin{malla}[height=7cm]
  \end{malla}

  \pregunta Sofía se compró un vestido en una tienda extranjera, y al llegar su paquete
  a Chile, aduanas le hace un sobre cargo del 19\% por concepto de IVA y otro 30\% adicional
  por sobre el total anterior en impuestos específicos.
  ¿Cuánto varió porcentualmente el vestido de Sofía,
  si originalmente le costó veinte mil pesos?
  \begin{malla}[height=7cm]
  \end{malla}

  \pregunta Determine el valor de la siguiente expresión:
  \begin{mcaja}
   \left(2,\overline{3}+5\right)\div\left(4\dfrac{2}{5}-0,2\right)
  \end{mcaja}
  \begin{malla}[height=14cm]
  \end{malla}


\end{preguntas}

\end{document}