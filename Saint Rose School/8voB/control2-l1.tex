\def\curso{Octavo básico B}
\def\puntaje{10}
\def\titulo{Control}
\def\subtitulo{Operatoria de números racionales}
\documentclass[]{srs}

\begin{document}

\section*{Objetivo}
  Realizar cálculos que involucren sumas y restar de fracciones y/o número decimales.

\section*{Instrucciones}
  Cada pregunta tiene 2 puntos y cuenta con 30 minutos para completar
  la evaluación. Incluya desarrollo en todas sus respuestas, y recuerde marcar o señalizar
  el resultado final en cada pregunta.

\section*{Criterios de evaluación}
  En la corrección, se asignará el puntaje a cada pregunta según los siguientes criterios.
\begin{center}
  \begin{tblr}{width=\linewidth,colspec={X[1,c]|X[6]}, hline{1,Z} = {1}{-}{}, hline{1,Z} = {2}{-}{},
      hlines, cells={valign=m}, row{1} = {bg=black!15}}
      Puntaje asignado & \SetCell{c} Criterios o indicadores \\
      +50\% & Señala clara y correctamente cuál es la solución o el resultado de la pregunta hecha
      en el enunciado.\\
      +50\% & Incluye un desarrollo que relata de manera clara y ordenada los procedimientos
      \mbox{necesarios} para solucionar la problemática. En caso de estar incompleto o con
      errores el desarrollo, se asignará puntaje parcial si se muestra dominio de los
       contenidos y conceptos involucrados.\\
      0\% &  La respuesta es incorrecta. De haber desarrollo, este tiene errores conceptuales.\\
  \end{tblr}
\end{center}
\separador[2mm]

Calcula el resultado de las siguientes operaciones.
\begin{preguntas}(2)
  \pregunta $-\dfrac{6}{5}+\dfrac{38}{25}$
  \begin{malla}[height=7cm]
  \end{malla}
  \pregunta $-2,4-\dfrac{18}{5}$
  \begin{malla}[height=7cm]
  \end{malla}
  \pregunta! $-\dfrac{105}{32}-\left(-9,41\right)$
  \begin{malla}[height=7cm]
  \end{malla}
  \pregunta! $4,002-2\dfrac{4}{5}+\left(-0,02\right)$
  \begin{malla}[height=7cm]
  \end{malla}
  \pregunta! $\left(\dfrac{16}{5}-\dfrac{63}{8}\right)+\left(-1,31\right)-\left(5+\left(-\dfrac{15}{2}\right)\right)$
  \begin{malla}[height=7cm]
  \end{malla}
\end{preguntas}

\end{document}