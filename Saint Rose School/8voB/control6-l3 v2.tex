\def\curso{Octavo básico B}
\def\puntaje{10}
\def\titulo{Control}
\def\subtitulo{Números racionales y variación porcentual}
\def\fecha{22 de mayo, 2025}
\documentclass[]{srs}

\begin{document}

\subsection*{Objetivo}
  Resolver problemáticas que involucran el uso de números racionales y/o variación porcentual.

\subsection*{Instrucciones}
  Cada pregunta tiene 2 puntos y cuenta con 40 minutos para completar
  la evaluación. Incluya desarrollo en todas sus respuestas, y recuerde marcar o señalizar
  el resultado final en cada pregunta.

\subsection*{Criterios de evaluación}
  En la corrección, se asignará el puntaje a cada pregunta según los siguientes criterios.
\begin{center}
  \begin{tblr}{width=\linewidth,colspec={X[1,c]|X[6]}, hline{1,Z} = {1}{-}{}, hline{1,Z} = {2}{-}{},
      hlines, cells={valign=m}, row{1} = {bg=black!15}}
      Puntaje asignado & \SetCell{c} Criterios o indicadores \\
      +50\% & Señala clara y correctamente cuál es la solución o el resultado de la pregunta hecha
      en el enunciado.\\
      +50\% & Incluye un desarrollo que relata de manera clara y ordenada los procedimientos
      \mbox{necesarios} para solucionar la problemática. En caso de estar incompleto o con
      errores el desarrollo, se asignará puntaje parcial si se muestra dominio de los
       contenidos y conceptos involucrados.\\
      0\% &  La respuesta es incorrecta. De haber desarrollo, este tiene errores conceptuales.\\
  \end{tblr}
\end{center}
\separador[2mm]


\begin{preguntas}(1)
  \pregunta ¿Cuántos vasos de bebida se pueden llenar si cada vaso tiene una
  capacidad de un cuarto de litro y se disponen de 7 botellas, cada
  una con 3,5 litros de bebida?
  \begin{malla}[height=7cm]
  \end{malla}
  \pregunta Si a Rosita le quedan 15 dulces, y se ha comido 2/5 de todos los dulces que tenía,
  ¿cuántos dulces tenía Rosita originalmente?
  \begin{malla}[height=7cm]
  \end{malla}
  \pregunta Una tienda de aparatos electrónicos decide dar un 35\% de descuento en toda su mercancía.
  Si el precio normal de un lector digital de libros es $\$62\, 250$ pesos,
  ¿cuánto se deberá pagar en caja por este lector?
  \begin{malla}[height=7cm]
  \end{malla}

  \pregunta Sofía se compró un vestido en una tienda extranjera, y al llegar su paquete
  a Chile, la aduana le aplicó un sobrecargo del 19\% por concepto de IVA y otro 30\% adicional
  por sobre el total anterior en impuestos específicos.
  ¿Cuánto varió porcentualmente el vestido de Sofía,
  si originalmente le costó treinta y dos mil pesos?
  \begin{malla}[height=7cm]
  \end{malla}

  \pregunta Determine el valor de la siguiente expresión:
  \begin{doteado}
  \(  -\dfrac{5}{6} +\left(1,\overline{5}+\dfrac{1}{3}\right)\div\left(2\dfrac{3}{4}-\dfrac{1}{12}\right) \)
  \end{doteado}
  \begin{malla}[height=14cm]
  \end{malla}


\end{preguntas}

\end{document}