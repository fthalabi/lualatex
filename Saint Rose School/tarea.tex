\def\curso{Octavo básico B}
\def\puntaje{18}
\def\titulo{Prueba}
\def\subtitulo{Potencias y raíces cuadradas}
\def\fecha{6 de mayo, 2025}
\documentclass[]{srs}

\begin{document}

\section*{Objetivo}

Mostrar que comprenden como utilizar propiedades de potencias en la resolución
de expresiones aritméticas. Además de, como calcular o estimar el valor de
una raíz cuadrada.

\section*{Instrucciones generales}
  Tiene 1 hora y 30 minutos para responder la evaluación. Esta es individual y debe
  usar solo sus materiales personales para trabajar durante este periodo, no los solicite
  a un compañero durante la evaluación.

\section{Opciones múltiples}

\section*{Instrucciones}
Lea atentamente cada enunciado y escoja la alternativa correcta en cada caso.

\section*{Criterios de evaluación}
En la corrección de esta sección, se asignará 2 puntos al marcar la alternativa correcta.
Las alternativas corregidas serán consideradas incorrectas, es decir, marque solo una
alternativa por enunciado.

\separador[2mm]

\begin{preguntas}[after-item-skip=2cm]
  \pregunta La siguiente operación es equivalente a:
  \begin{mcaja}
    \left(2^3\cdot 3^2\div 6\right)^0
  \end{mcaja}
  \begin{vertical}
    \alternativa $2^3\cdot 3^2\div 6$
    \alternativa $2^2\cdot 3$
    \alternativa $2^2\div 3$
    \alternativa $12^2\div 12^2$
  \end{vertical}

  \pregunta ¿En cuál de los siguientes intervalos se ubica $\sqrt{85}$?\\
  \begin{vertical}
    \alternativa Entre 7 y 8.
    \alternativa Entre 8 y 9.
    \alternativa Entre 9 y 10.
    \alternativa Entre 10 y 11.
  \end{vertical}

  \pregunta ¿Cuál(es) de las siguientes igualdades es (son) verdadera(s)?\\
  \begin{vertical*}
    \alternativa $5^3\cdot 5^2 = 5^5$
    \alternativa $7^2\div 2^2 = (3,5)^2$
    \alternativa $6^2\cdot 6^2 = 36^2$
  \end{vertical*}
  \begin{vertical}
    \alternativa Solo I y II.
    \alternativa Solo II y III.
    \alternativa Solo I y III.
    \alternativa I, II y III.
  \end{vertical}

  \pregunta ¿Cuál es el resultado de la operación
  $\left(12^3\div 12\right)\div\left(3^2\cdot 2^2\right)$?\\
  \begin{vertical}
    \alternativa $2^2$
    \alternativa $3^2$
    \alternativa $6^2$
    \alternativa $10^2$
  \end{vertical}

  \pregunta ¿Cuál es el resultado de la operación
  $\dfrac{2^3\cdot 5^3 \cdot 6^3 \cdot 6^0}{3^3}$?\\
  \begin{vertical}
    \alternativa $1$
    \alternativa $5^3$
    \alternativa $10^3$
    \alternativa $20^3$
  \end{vertical}

  \pregunta Felipe ha heredado un terreno cuadrado de área 225 $\text{m}^2$. Si desea poner
  una malla para cercar el terreno, ¿cuál debe ser la longitud de la malla que va a poner
  Felipe?\\
  \begin{vertical}
    \alternativa 15 m
    \alternativa 30 m
    \alternativa 45 m
    \alternativa 60 m
  \end{vertical}

\end{preguntas}

\newpage

\section{Preguntas abiertas}

\section*{Instrucciones}
Lea atentamente el enunciado de cada pregunta, considere los datos entregados y
responda a la problemática planteada, explicando y detallando claramente
tanto su proceso como sus resultados.

\section*{Criterios de evaluación}
  En la corrección de esta sección, cada pregunta tiene 3 puntos y se asignará
  el puntaje de cada una según los siguientes criterios:
\begin{center}
  \begin{tblr}{width=\linewidth,colspec={X[1,c]|X[6]}, hline{1,Z} = {1}{-}{}, hline{1,Z} = {2}{-}{},
      hlines, cells={valign=m}, row{1} = {bg=black!15}}
      Puntaje asignado & \SetCell{c} Criterios o indicadores \\
      +50\% & Señala clara y correctamente cuál es la solución o el resultado de la pregunta hecha
      en el enunciado.\\
      +50\% & Incluye un desarrollo que relata de manera clara y ordenada los procedimientos
      \mbox{necesarios} para solucionar la problemática. En caso de estar incompleto o con
      errores el desarrollo, se asignará puntaje parcial si se muestra dominio de los
       contenidos y conceptos involucrados.\\
      0\% &  La respuesta es incorrecta. De haber desarrollo, este tiene errores conceptuales.\\
  \end{tblr}
\end{center}
\separador[2mm]

\begin{preguntas}(1)
  \pregunta Calcule el valor de:
  \begin{mcaja}
    \sqrt{441}
  \end{mcaja}
  \begin{malla}[height=9cm]
  \end{malla}

  \pregunta Determine los valores de $a$ y $b$ para que se cumpla la siguiente igualdad.
  \begin{mcaja}
    \dfrac{80^3 \cdot 3^4 \cdot 12^2\cdot 6^4 \cdot 54}{10^3 \cdot 2^3} = 2^a \cdot 3^b
  \end{mcaja}
  \begin{malla}[height=15cm]
  \end{malla}
\end{preguntas}





\end{document}