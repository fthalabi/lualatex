\documentclass[pagina vacia]{srs}
\title{Repositorio de problemas: Eje de números}

\begin{document}
\maketitle
\section{Conjuntos numéricos}

\begin{preguntas}[after-item-skip=1cm]
\pregunta \( \left(-3\right)^2 - \left(-2\right)^2 - \left(-1\right)^2 = \)
\begin{vertical}
\alternativa $-14$
\alternativa 4
\alternativa 6
\alternativa 12
\alternativa 14
\end{vertical}

\pregunta \( \dfrac{\dfrac{3}{5} - \dfrac{1}{3}}{2 + \dfrac{2}{5}} = \)
\begin{vertical}
\alternativa 9
\alternativa \( \dfrac{16}{25} \)
\alternativa \( \dfrac{1}{18} \)
\alternativa \( \dfrac{1}{3} \)
\alternativa \( \dfrac{1}{9} \)
\end{vertical}

\pregunta La cuarta parte de $0,\overline{2}$ es :
\begin{vertical}
\alternativa $0,0\overline{4}$
\alternativa $0,05$
\alternativa $0,0\overline{5}$
\alternativa $0,\overline{5}$
\alternativa $0,\overline{8}$
\end{vertical}

\pregunta \( \left(\dfrac{1}{2}\right)^{-2} + \left(\dfrac{1}{2}\right)^{-1} = \)
\begin{vertical}
\alternativa $-6$
\alternativa $-3$
\alternativa 4
\alternativa 6
\alternativa 8
\end{vertical}

\pregunta \( (0,\overline{2})^{-1} = \)
\begin{vertical}
\alternativa $4$
\alternativa $4,5$
\alternativa $4,\overline{5}$
\alternativa $4,9$
\alternativa $5$
\end{vertical}

\pregunta Si \(x\) = 0,24, \(y\) = \(5 \cdot 10^4\) y \(z\) = \(12\,000\), entonces
\( \dfrac{xy}{z} = \)
\begin{vertical}
\alternativa \( 10^{-8} \)
\alternativa \( 10^{-6} \)
\alternativa \( 10^{-4} \)
\alternativa \( 10^{-2} \)
\alternativa \( 10^{0} \)
\end{vertical}

\pregunta De los siguientes números reales, ¿cuál es el menor?
\begin{vertical}
\alternativa \( 38 \cdot 10^{-3} \)
\alternativa \( 390 \cdot 10^{-4} \)
\alternativa \( 4\,200 \cdot 10^{-6} \)
\alternativa \( 0,4 \cdot 10^{-3} \)
\alternativa \( 0,41 \cdot 10^{-2} \)
\end{vertical}

\pregunta Las cinco milésimas partes de 62,5 aproximado por defecto a las milésimas es:
\begin{vertical}
\alternativa 0,312
\alternativa 0,313
\alternativa 0,310
\alternativa 3,125
\alternativa 3,124
\end{vertical}

\pregunta \( \left(\dfrac{0,05}{5}\right)^{-1} = \)
\begin{vertical}
\alternativa $-100$
\alternativa 10
\alternativa 100
\alternativa 1000
\alternativa 10\,000
\end{vertical}

\pregunta \( \dfrac{\left(\dfrac{1}{3}\right)^{-2} + \left(\dfrac{2}{3}\right)^{-1}}{3,5} \)
\begin{vertical}
\alternativa Es un número entero.
\alternativa Es un número decimal infinito no periódico.
\alternativa Es un número decimal periódico.
\alternativa Es un número decimal semiperiódico.
\alternativa Es un número irracional.
\end{vertical}

\pregunta En la recta numérica, ¿cuál de los siguientes números está más cerca del cero?
\begin{vertical}
\alternativa $-0,2\overline{1}$
\alternativa $-0,3$
\alternativa $0,23$
\alternativa $0,25$
\alternativa $0,\overline{2}$
\end{vertical}

\pregunta Si el producto \(0,24 \cdot 0,75\) se divide con \(\dfrac{2}{3}\) resulta:
\begin{vertical}
\alternativa $0,03$
\alternativa $0,12$
\alternativa $0,27$
\alternativa $0,42$
\alternativa $0,5$
\end{vertical}

\pregunta Se puede determinar que la expresión \(\dfrac{a+b}{c}\), con $a$, $b$ y $c$ números enteros
y \(c \neq 0\), representa un número entero positivo, si:
\begin{verticaln}
\alternativa \(c\left(a + b\right)>0\)
\alternativa \(a + b\) es múltiplo de c.
\end{verticaln}
\begin{vertical}
\alternativa (1) por sí sola
\alternativa (2) por sí sola
\alternativa Ambas juntas, (1) y (2)
\alternativa Cada una por sí sola, (1) ó (2)
\alternativa Se requiere información adicional
\end{vertical}

\pregunta ¿Cuál de las siguientes expresiones corresponde a un número racional NO entero?
\begin{vertical}
\alternativa \( (0,2)^{-3} \)
\alternativa \(3,\overline{9}\)
\alternativa \( \dfrac{0,0\overline{2}}{0,2} \)
\alternativa \( \dfrac{0,\overline{32}}{0,\overline{04}} \)
\alternativa \( \dfrac{0,8\overline{3}}{0,1\overline{6}} \)
\end{vertical}

\pregunta ¿Cuál de los siguientes números NO está entre $1,0\overline{6}$ y $1,1$?
\begin{vertical}
\alternativa \( \dfrac{49}{45} \)
\alternativa \( \dfrac{27}{25} \)
\alternativa \( 1\dfrac{1}{11} \)
\alternativa \( \dfrac{267}{250} \)
\alternativa \( 1\dfrac{4}{33} \)
\end{vertical}

\pregunta Los tres primeros atletas en una carrera de 100 metros planos, fueron Pedro, Felipe y Andrés los cuales obtuvieron las siguientes marcas: \(12,2''\), \(12,02''\) y \(13,1''\) respectivamente. ¿Cuál(es) de las siguientes afirmaciones es (son) verdadera(s)?
\begin{verticali}
\alternativa Felipe fue el vencedor.
\alternativa Pedro llegó después de Andrés.
\alternativa Felipe llegó 18 centésimas de segundo antes que Pedro.
\end{verticali}
\begin{vertical}
\alternativa Solo I
\alternativa Solo III
\alternativa Solo I y II
\alternativa Solo I y III
\alternativa I, II y III
\end{vertical}

\pregunta La superficie de nuestro país, sin considerar el territorio antártico, es aproximadamente de \(755\,000\) km\(^2\). Sabiendo que 1 km\(^2\) equivale a \(10^6\) m\(^2\) y que 1 hectárea corresponden a \(10\,000\) m\(^2\). ¿Cuántos millones de hectáreas tiene la superficie de nuestro país?
\begin{vertical}
\alternativa \(7\,550\)
\alternativa 755
\alternativa 75,5
\alternativa 7,55
\alternativa 0,755
\end{vertical}

\pregunta ¿Cuál(es) de las siguientes expresiones da(n) como resultado un número entero?
\begin{verticali}
\alternativa \( \left(10^{-1} + 10^{-2}\right)^{-1} \)
\alternativa \( \dfrac{10^{-2} + 1}{10^{-4} + 10^{-2}} \)
\alternativa \( \dfrac{10^{-4} + 10^{-3}}{10^{-5}} \)
\end{verticali}
\begin{vertical}
\alternativa Solo I
\alternativa Solo II
\alternativa Solo I y II
\alternativa Solo II y III
\alternativa I, II y III
\end{vertical}

\pregunta Considerando que \(a = 0,23 \cdot 10^{-3}\); \(b = 5 \cdot 10^3\); \(c = 0,3 \cdot 10^{-5}\), entonces $ab + bc$ redondeado a la centésima es:
\begin{vertical}
\alternativa $1,16$
\alternativa $1,17$
\alternativa $11,51$
\alternativa $11,52$
\alternativa $0,17$
\end{vertical}

\pregunta La siguiente tabla muestra la población de algunos países de América del Sur:
\begin{tcolorbox}[blank,halign=center]
\begin{tblr}{
  colspec = {Q[l] Q[r]},
  hlines,
  vlines,
  row{1} = {font=\bfseries}
}
País & Población \\
Argentina & \(43\,823\,000\) \\
Bolivia & \(11\,066\,000\) \\
Chile & \(18\,286\,000\) \\
Perú & \(31\,660\,000\) \\
\end{tblr}
\end{tcolorbox}
¿Cuál(es) de las siguientes afirmaciones es(son) verdadera(s)?
\begin{verticali}
\alternativa La diferencia entre los dos países más poblados es superior a los \(1,215 \cdot 10^7\) habitantes.
\alternativa La suma entre los dos países menos poblados es superior a \(29,4 \cdot 10^6\) habitantes.
\alternativa Entre todos superan los \(1,048 \cdot 10^8\) habitantes.
\end{verticali}
\begin{vertical}
\alternativa Solo I
\alternativa Solo II
\alternativa Solo I y II
\alternativa Solo I y III
\alternativa I, II y III
\end{vertical}

\pregunta En una cierta mina se extrajeron en cierto mes \(3,7 \cdot 10^4\) kilogramos de mineral y al siguiente se extrajeron \(4,2 \cdot 10^5\) kilogramos. Si una tonelada son \(1\,000\) kilogramos, ¿cuál es la diferencia, en toneladas, de la extracción en ambos meses?
\begin{vertical}
\alternativa 5
\alternativa 38,3
\alternativa 383
\alternativa 416,3
\alternativa \(383\,000\)
\end{vertical}

\pregunta En su viaje de vacaciones, una persona recorrerá un trayecto en tres días. Si el primer día recorrió \(\dfrac{2}{5}\) del trayecto y el segundo día los \(\dfrac{3}{4}\) de lo que recorrió primer día, entonces ¿cuál(es) de las siguientes afirmaciones es (son) verdadera(s)?
\begin{verticali}
\alternativa El tercer día anduvo más que en el primero.
\alternativa Entre el primer y segundo día recorrió el 70\% del trayecto.
\alternativa El segundo y tercer día anduvo lo mismo.
\end{verticali}
\begin{vertical}
\alternativa Solo I
\alternativa Solo II
\alternativa Solo I y II
\alternativa Solo II y III
\alternativa I, II y III
\end{vertical}

\pregunta ¿Para cuál(es) de los siguientes números reales, su raíz cuadrada es un número racional?
\begin{verticali}
\alternativa \(16,9 \cdot 10^{-5}\)
\alternativa \(1\,960\,000\)
\alternativa \(\dfrac{196 \cdot 10^{-3}}{169 \cdot 10^{-7}}\)
\end{verticali}
\begin{vertical}
\alternativa Solo I
\alternativa Solo II
\alternativa Solo I y II
\alternativa Solo II y III
\alternativa I, II y III
\end{vertical}

\pregunta Si el producto \(0,22 \cdot 0,16\) se redondea a dos decimales resulta:
\begin{vertical}
\alternativa $0,02$
\alternativa $0,03$
\alternativa $0,04$
\alternativa $0,05$
\alternativa $0,35$
\end{vertical}

\pregunta Si el producto \(0,22 \cdot 0,16\) se trunca a dos decimales resulta:
\begin{vertical}
\alternativa $0,02$
\alternativa $0,03$
\alternativa $0,04$
\alternativa $0,05$
\alternativa $0,35$
\end{vertical}

\pregunta ¿Cuál de las siguientes afirmaciones es FALSA?
\begin{vertical}
\alternativa 5 es un número entero.
\alternativa 0,2 es un número racional.
\alternativa 3 es un número real.
\alternativa 0 es un número entero no negativo.
\alternativa \( \sqrt{-2} \) es un número racional.
\end{vertical}

\pregunta Si \(a = 0,2 \cdot 10^{-3}\) y \(b = 15 \cdot 10^{-7}\). ¿Cuál(es) de las siguientes afirmaciones es (son) FALSA(S)?
\begin{verticali}
\alternativa \( a \cdot b = 3 \cdot 10^{-10} \)
\alternativa \( a^2 + b = 1,54 \cdot 10^{-6} \)
\alternativa \( \dfrac{b}{a} = 7,5 \cdot 10^{-11} \)
\end{verticali}
\begin{vertical}
\alternativa Solo I
\alternativa Solo II
\alternativa Solo III
\alternativa Solo I y II
\alternativa Solo II y III
\end{vertical}

\pregunta Si $P = 0,\overline{24}$, \(Q = \dfrac{121}{500} \) y \( R = \dfrac{11}{45} \), entonces al ordenarlos en forma creciente, resulta:
\begin{vertical}
\alternativa $P < Q < R$
\alternativa $P < R < Q$
\alternativa $Q < R < P$
\alternativa $R < P < Q$
\alternativa $Q < P < R$
\end{vertical}

\pregunta ¿Cuál(es) de las siguientes afirmaciones es (son) verdadera(s)?
\begin{verticali}
\alternativa \( 0,\overline{32} + 0,\overline{8} = 1,\overline{12} \)
\alternativa \( 0,\overline{2} - 0,1\overline{5} = 0,0\overline{6} \)
\alternativa \( 0,3\overline{6} \cdot 0,\overline{45} = 0,1\overline{6} \)
\end{verticali}
\begin{vertical}
\alternativa Solo I
\alternativa Solo II
\alternativa Solo I y II
\alternativa Solo II y III
\alternativa I, II y III
\end{vertical}

\pregunta ¿Cuál(es) de las siguientes afirmaciones es (son) verdadera(s)?
\begin{verticali}
\alternativa El doble de $0,0\overline{5}$ es $0,\overline{1}$.
\alternativa El inverso multiplicativo de \(0,\overline{6}\) es $1,5$.
\alternativa El triple de \(0,2\overline{3}\) es $0,7$.
\end{verticali}
\begin{vertical}
\alternativa Solo I
\alternativa Solo II
\alternativa Solo I y II
\alternativa Solo II y III
\alternativa I, II y III
\end{vertical}

\pregunta Sea $p = 6t$, con $p$ y $t$ números enteros positivos. Entonces es siempre correcto afirmar que:
\begin{verticali}
\alternativa $p$ es múltiplo de 3.
\alternativa $t$ es divisor de $p$.
\alternativa $(p + t)$ es múltiplo de $t$.
\end{verticali}
\begin{vertical}
\alternativa Solo I
\alternativa Solo I y II
\alternativa Solo I y III
\alternativa Solo II y III
\alternativa I, II y III
\end{vertical}

\pregunta Si $a$ y $b$ son dígitos, con \(b \neq 0\), entonces \( \dfrac{0,a\overline{b}}{0,\overline{b}} \) es igual a:
\begin{vertical}
\alternativa \( \dfrac{a - b}{10} \)
\alternativa \( \dfrac{10a - b}{10b} \)
\alternativa \( \dfrac{10a + b}{10b} \)
\alternativa \( \dfrac{11a + b}{100b} \)
\alternativa \( \dfrac{9a + b}{10b} \)
\end{vertical}

\pregunta Si $a$ y $b$ son dígitos, ¿cuál de las siguientes fracciones es siempre igual al resultado de \(0,a\overline{b} - 0,\overline{a}\)?
\begin{vertical}
\alternativa \( \dfrac{b - a}{9} \)
\alternativa \( \dfrac{b - a}{90} \)
\alternativa \( \dfrac{a - b}{90} \)
\alternativa \( \dfrac{a - b}{9} \)
\alternativa \( \dfrac{b - a}{900} \)
\end{vertical}

\pregunta En la figura, el punto A se ubica en el decimal 0,27 y el B en el 0,32. Si el trazo \(\overline{\text{AB}}\) se ha dividido en cuatro partes iguales por los puntos P, Q y R. ¿Cuál de las siguientes afirmaciones es (son) verdadera(s)? \\ (La figura muestra los puntos A, P, Q, R, B en ese orden sobre un segmento de recta)
\begin{verticali}
\alternativa P se ubica en el número real \(2,825 \cdot 10^{-1}\).
\alternativa Q se ubica en el número real \(2,95 \cdot 10^{-1}\).
\alternativa R se ubica en el número real \(3,075 \cdot 10^{-1}\).
\end{verticali}
\begin{vertical}
\alternativa Solo I
\alternativa Solo II
\alternativa Solo I y II
\alternativa Solo II y III
\alternativa I, II y III
\end{vertical}

\pregunta Sean x = \( \dfrac{0,0025}{200} \); y = \( \dfrac{25 \cdot 10^{-3}}{2 \cdot 10^3} \); z = \( \dfrac{0,25}{20\,000} \). ¿cuál de las siguientes afirmaciones es verdadera?
\begin{vertical}
\alternativa x \(<\) z \(<\) y
\alternativa y \(<\) z \(<\) x
\alternativa z \(<\) y \(<\) x
\alternativa x = y = z
\alternativa Ninguna de ellas.
\end{vertical}

\pregunta Se tienen los números reales: x = \( \dfrac{1}{\sqrt{2}} \); y = \( \dfrac{2}{\sqrt{2}-1} \); z = \( \dfrac{4}{\sqrt{2}+1} \); w = \( \dfrac{\sqrt{2}}{\sqrt{2}-1} \). ¿Cuál de las siguientes afirmaciones es (son) verdadera(s)?
\begin{verticali}
\alternativa El mayor es y.
\alternativa y \(>\) z \(>\) x.
\alternativa w \(>\) z \(>\) x.
\end{verticali}
\begin{vertical}
\alternativa Solo I
\alternativa Solo II
\alternativa Solo I y II
\alternativa Solo II y III
\alternativa I, II y III
\end{vertical}

\pregunta Sea a un número real, se puede determinar que a es racional, sabiendo que:
\begin{verticaln}
\alternativa \( \left(a+2\right)^2 - \left(a-2\right)^2 \) es racional.
\alternativa \( \dfrac{a+2}{a-2} \) es un racional distinto de 1.
\end{verticaln}
\begin{vertical}
\alternativa (1) por sí sola
\alternativa (2) por sí sola
\alternativa Ambas juntas, (1) y (2)
\alternativa Cada una por sí sola, (1) ó (2)
\alternativa Se requiere información adicional
\end{vertical}

\pregunta Si n es un número entero, ¿cuál(es) de las siguientes expresiones corresponden a números racionales?
\begin{verticali}
\alternativa \( \left(\sqrt{5} + \sqrt{3}\right)\left(\sqrt{5} - \sqrt{3}\right) \)
\alternativa \( \left(\left(\sqrt{2} + \sqrt{3}\right)^2 - 2\sqrt{6}\right)^n \)
\alternativa \( \left(\sqrt{2}-1\right)^{2n} - \left(3 + 2\sqrt{2}\right)^n \)
\end{verticali}
\begin{vertical}
\alternativa Solo I
\alternativa Solo II
\alternativa Solo I y II
\alternativa Solo I y III
\alternativa I, II y III
\end{vertical}

\pregunta ¿Cuál(es) de los siguientes números corresponden a números racionales?
\begin{verticali}
\alternativa \( \dfrac{\sqrt{50}}{\sqrt{8}} \)
\alternativa \( \left(1 + \sqrt{2}\right)^2 \)
\alternativa \( \sqrt{\sqrt{\dfrac{1}{16}}} \)
\end{verticali}
\begin{vertical}
\alternativa Solo I
\alternativa Solo II
\alternativa Solo I y III
\alternativa Solo II y III
\alternativa I, II y III
\end{vertical}

\pregunta Si m y n son números enteros, se puede determinar que m + n es par, sabiendo que:
\begin{verticaln}
\alternativa m - n es par.
\alternativa m\(^2\) + 2mn + n\(^2\) es par.
\end{verticaln}
\begin{vertical}
\alternativa (1) por sí sola
\alternativa (2) por sí sola
\alternativa Ambas juntas, (1) y (2)
\alternativa Cada una por sí sola, (1) ó (2)
\alternativa Se requiere información adicional
\end{vertical}

\pregunta ¿Cuál(es) de las siguientes afirmaciones es (son) siempre verdadera(s)?
\begin{verticali}
\alternativa Si el perímetro de un triángulo equilátero es racional, entonces las medidas de sus lados son racionales.
\alternativa Siempre el área de una circunferencia es irracional.
\alternativa Si la longitud de el lado de un cuadrado es irracional, entonces su área es racional.
\end{verticali}
\begin{vertical}
\alternativa Solo I
\alternativa Solo II
\alternativa Solo I y II
\alternativa Solo I y III
\alternativa Ninguna de ellas.
\end{vertical}

\pregunta Si A = \(0,2 \cdot 10^{-2}\); B = \(200 \cdot 10^{-4}\) y C = \(2\,000 \cdot 10^{-5}\), ¿cuál(es) de las siguientes afirmaciones es (son) verdadera(s)?
\begin{verticali}
\alternativa B = 10A
\alternativa B = C
\alternativa \( \dfrac{\text{A}}{\text{B}} = \dfrac{\text{C}}{\text{A}} \)
\end{verticali}
\begin{vertical}
\alternativa Solo I
\alternativa Solo II
\alternativa Solo I y II
\alternativa Solo II y III
\alternativa I, II y III
\end{vertical}

\pregunta ¿Cuál(es) de las siguientes afirmaciones es (son) siempre verdadera(s)?
\begin{verticali}
\alternativa El promedio entre dos irracionales es irracional.
\alternativa La diferencia entre dos racionales es racional.
\alternativa Si la suma de dos números es racional, la diferencia también.
\end{verticali}
\begin{vertical}
\alternativa Solo I
\alternativa Solo II
\alternativa Solo I y II
\alternativa Solo II y III
\alternativa I, II y III
\end{vertical}

\pregunta Si P = \(\sqrt{8}\), Q = \(\sqrt{32}\) y R = \(\sqrt{2}\), ¿cuál(es) de las siguientes expresiones corresponde(n) a números racionales?
\begin{verticali}
\alternativa \( \dfrac{\text{P+Q}}{\text{R}} \)
\alternativa \( \dfrac{\text{PQ}}{\text{R}} \)
\alternativa PQR
\end{verticali}
\begin{vertical}
\alternativa Solo I
\alternativa Solo II
\alternativa Solo I y II
\alternativa Solo II y III
\alternativa I, II y III
\end{vertical}

\pregunta Si q y r son múltiplos de p, con q \( \neq \) r y p \( \neq \) 0, entonces ¿cuál(es) de las siguientes expresiones es (son) siempre números enteros?
\begin{verticali}
\alternativa \( \dfrac{\text{q}-\text{r}}{\text{p}} \)
\alternativa \( \dfrac{\text{q}^2 + \text{r}^2}{\text{p}} \)
\alternativa \( \dfrac{\text{q}+\text{r}}{\text{q}-\text{r}} \)
\end{verticali}
\begin{vertical}
\alternativa Solo I
\alternativa Solo II
\alternativa Solo I y II
\alternativa Solo II y III
\alternativa I, II y III
\end{vertical}

\pregunta a, b y c son números racionales, cuya ubicación en la recta numérica se muestra en la siguiente figura: \\ (Esquema: recta horizontal con marcas a, b, c, 0 en ese orden de izquierda a derecha, donde a, b, c están a la izquierda del 0, es decir, a \(<\) b \(<\) c \(<\) 0). \\ ¿Cuál(es) de las siguientes afirmaciones es (son) verdadera(s)?
\begin{verticali}
\alternativa \( \text{a}^2 < \text{b}^2 < \text{c}^2 \)
\alternativa \( \dfrac{1}{\text{a}} > \dfrac{1}{\text{b}} > \dfrac{1}{\text{c}} \)
\alternativa \( \dfrac{1}{\text{a}^2} < \dfrac{1}{\text{b}^2} < \dfrac{1}{\text{c}^2} \)
\end{verticali}
\begin{vertical}
\alternativa Solo I
\alternativa Solo II
\alternativa Solo I y II
\alternativa Solo II y III
\alternativa I, II y III
\end{vertical}

\pregunta Sean a y b dos números enteros y distintos, se puede determinar que la solución de la ecuación en x, ax - bx = a + b es un número entero negativo, sabiendo que:
\begin{verticaln}
\alternativa (a - b) es un divisor de (a + b).
\alternativa a\(^2\) - b\(^2\) \(<\) 0
\end{verticaln}
\begin{vertical}
\alternativa (1) por sí sola
\alternativa (2) por sí sola
\alternativa Ambas juntas, (1) y (2)
\alternativa Cada una por sí sola, (1) ó (2)
\alternativa Se requiere información adicional
\end{vertical}

\pregunta Si \( \alpha = \dfrac{x+y}{x-y} \), ¿cuál(es) de las siguientes afirmaciones es (son) verdadera(s)?
\begin{verticali}
\alternativa Si x \( \neq \) y, entonces \( \alpha > 1 \).
\alternativa Si y \(<\) 0 \(<\) x, entonces \( \alpha < 1 \).
\alternativa Si x \(<\) y \(<\) 0, entonces \( \alpha > 1 \).
\end{verticali}
\begin{vertical}
\alternativa Solo I
\alternativa Solo II
\alternativa Solo III
\alternativa Solo II y III
\alternativa I, II y III
\end{vertical}

\end{preguntas}

\section{Porcentajes}

\begin{preguntas}[after-item-skip=2cm]
\pregunta El 30\% de un número es 45, ¿cuál es su 12\%?
\begin{vertical}
\alternativa 25
\alternativa 18
\alternativa 16
\alternativa 12
\alternativa 8
\end{vertical}

\pregunta El 15\% de \(1\dfrac{2}{3}\) es:
\begin{vertical}
\alternativa 0,125
\alternativa 0,75
\alternativa 0,5
\alternativa 0,45
\alternativa 0,25
\end{vertical}

\pregunta El 50\% de la mitad de un número es 20, entonces el número es:
\begin{vertical}
\alternativa 5
\alternativa 10
\alternativa 20
\alternativa 40
\alternativa 80
\end{vertical}

\pregunta ¿Qué porcentaje es 0,42 de 0,76 ?
\begin{vertical}
\alternativa 32,41\%
\alternativa 50\%
\alternativa 55\%
\alternativa 60,8\%
\alternativa 181,81\%
\end{vertical}

\pregunta a es el 10\% de b y b es el 10\% de c. Si c = 10, entonces a =
\begin{vertical}
\alternativa 0,01
\alternativa 0,1
\alternativa 1
\alternativa 10
\alternativa 100
\end{vertical}

\pregunta Una camisa con un 20\% de descuento cuesta \$4000. ¿Cuánto costaría sin la rebaja?
\begin{vertical}
\alternativa \$4 800
\alternativa \$5 000
\alternativa \$5 200
\alternativa \$5 400
\alternativa \$5 500
\end{vertical}

\pregunta En un curso hay una mujer cada 4 hombres. ¿Qué \% del curso son mujeres?
\begin{vertical}
\alternativa 20\%
\alternativa 25\%
\alternativa 30\%
\alternativa 40\%
\alternativa 80\%
\end{vertical}

\pregunta El 12\% de 50, es equivalente con:
\begin{verticali}
\alternativa 20\% de 30
\alternativa 30\% de 20
\alternativa 15\% de 40
\end{verticali}
\begin{vertical}
\alternativa Solo I
\alternativa Solo II
\alternativa Solo I y II
\alternativa Solo I y III
\alternativa I, II y III
\end{vertical}

\pregunta Se ha cancelado \$42 000, que corresponde al 60\% de una deuda. ¿Cuánto falta por pagar?
\begin{vertical}
\alternativa \$14 000
\alternativa \$28 000
\alternativa \$30 000
\alternativa \$70 000
\alternativa \$112 000
\end{vertical}

\pregunta Si 12 es el 40\% de un número, ¿cuál es el número?
\begin{vertical}
\alternativa 3
\alternativa 30
\alternativa 40
\alternativa 48
\alternativa 300
\end{vertical}

\pregunta El 20\% del área de un cuadrado es \(20 \text{ cm}^2\), ¿Cuál es su perímetro?
\begin{vertical}
\alternativa 100 cm
\alternativa 40 cm
\alternativa 25 cm
\alternativa 20 cm
\alternativa 10 cm
\end{vertical}

\pregunta ¿Cuál de las siguientes alternativas equivale al 40\% de \(2xy\)?
\begin{vertical}
\alternativa \(\dfrac{4}{5}xy\)
\alternativa \(\dfrac{4}{25}xy\)
\alternativa \(\dfrac{8}{25}xy\)
\alternativa \(\dfrac{16}{5}xy\)
\alternativa \(\dfrac{3}{4}xy\)
\end{vertical}

\pregunta ¿Qué \% es \(\dfrac{6}{25}\) de \(\dfrac{3}{5}\)?
\begin{vertical}
\alternativa 20\%
\alternativa 25\%
\alternativa 40\%
\alternativa 45\%
\alternativa 60\%
\end{vertical}

\pregunta De un libro de 120 páginas, he leído 96, ¿qué \% me queda por leer?
\begin{vertical}
\alternativa \(\dfrac{1}{5}\)\%
\alternativa 5\%
\alternativa 20\%
\alternativa 25\%
\alternativa 40\%
\end{vertical}

\pregunta ¿Qué \% de la superficie del círculo de centro O está sombreada, si \(\angle A0B = 72^\circ\)?
\includegraphics[width=0.3\textwidth]{example-image-a}
\begin{vertical}
\alternativa 0,2\%
\alternativa 5\%
\alternativa 20\%
\alternativa 25\%
\alternativa 40\%
\end{vertical}

\pregunta La figura está formada por 9 cuadrados congruentes. ¿Aproximadamente, que \% del cuadrado ABCD está sombreado?
\includegraphics[width=0.3\textwidth]{example-image-b}
\begin{vertical}
\alternativa 50\%
\alternativa 56\%
\alternativa 60\%
\alternativa 65\%
\alternativa 70\%
\end{vertical}

\pregunta El 20\% de \((x + y)\) equivale a los \(\dfrac{4}{5}\) de \((x - y)\), entonces \(\dfrac{x}{y} =\)
\begin{vertical}
\alternativa \(\dfrac{3}{4}\)
\alternativa \(\dfrac{3}{5}\)
\alternativa \(\dfrac{4}{3}\)
\alternativa 1
\alternativa \(\dfrac{5}{3}\)
\end{vertical}

\pregunta Un hotel con capacidad para 800 pasajeros está completo; si un día se va un 30\% de los pasajeros y llega un 15\% de la capacidad. ¿Cuántos pasajeros faltan para que el hotel esté nuevamente completo?
\begin{vertical}
\alternativa 680
\alternativa 634
\alternativa 560
\alternativa 240
\alternativa 120
\end{vertical}

\pregunta Una emisora transmite 16 horas al día. Si su programación consiste en un 65\% de música popular, 25\% de música folclórica y el resto corresponde a música selecta, entonces, ¿cuántas horas dedica la emisora a música selecta?
\begin{vertical}
\alternativa 1h 6 min
\alternativa 1h 10 min
\alternativa 1h 36 min
\alternativa 2 horas
\alternativa Ninguna de las anteriores
\end{vertical}

\pregunta El 30\% de a equivale al 20\% de b. Si b = 150, ¿qué parte es a de b?
\begin{vertical}
\alternativa \(\dfrac{2}{3}\)
\alternativa \(\dfrac{3}{2}\)
\alternativa \(\dfrac{1}{2}\)
\alternativa \(\dfrac{2}{5}\)
\alternativa \(\dfrac{1}{4}\)
\end{vertical}

\pregunta El pago mínimo de una tarjeta de crédito es el 5\% de la deuda. Si en un estado de cuenta figura como pago mínimo \$12 000, ¿cuál es el total de la deuda?
\begin{vertical}
\alternativa \$228 000
\alternativa \$240 000
\alternativa \$252 000
\alternativa \$300 000
\alternativa \$600 000
\end{vertical}

\pregunta El precio de una radio ha sido rebajado en \$1 200, lo que corresponde al 5\% de su valor. ¿Cuánto costará durante la oferta?
\begin{vertical}
\alternativa \$21 500
\alternativa \$22 800
\alternativa \$23 800
\alternativa \$24 000
\alternativa \$25 200
\end{vertical}

\pregunta 103 es el 10\% de:
\begin{vertical}
\alternativa 1 000
\alternativa 1 020
\alternativa 1 030
\alternativa 1 040
\alternativa 1 050
\end{vertical}

\pregunta Un poste tiene enterrado el 20\% de su longitud total. Si la parte no enterrada mide 12 m. ¿Cuál es la longitud total del poste?
\begin{vertical}
\alternativa 2,4 m
\alternativa 9,6 m
\alternativa 15 m
\alternativa 18 m
\alternativa 27 m
\end{vertical}

\pregunta El 25\% de la edad del padre es la del hijo, y el 30\% de la edad del hijo es 3. ¿Qué edad tiene el padre?
\begin{vertical}
\alternativa 30
\alternativa 40
\alternativa 50
\alternativa 60
\alternativa 70
\end{vertical}

\pregunta a sumado con el 30\% de 6 resulta el 40\% de 8. Entonces el 10\% de a es:
\begin{vertical}
\alternativa 0,05
\alternativa 0,5
\alternativa 0,14
\alternativa 1,4
\alternativa 14
\end{vertical}

\pregunta Solo 12 alumnas, de un curso de 30, han pagado una cuota para un paseo. ¿Qué \% del curso falta por pagar?
\begin{vertical}
\alternativa 40\%
\alternativa 45\%
\alternativa 55\%
\alternativa 60\%
\alternativa 65\%
\end{vertical}

\pregunta El 12\% del 5\% de 10 000 es:
\begin{vertical}
\alternativa 0,6
\alternativa 6
\alternativa 60
\alternativa 600
\alternativa 6 000
\end{vertical}

\pregunta El 10\% de la quinta parte de \((x + y)\) es uno. Si x = 35 entonces y =
\begin{vertical}
\alternativa -15
\alternativa 10
\alternativa 15
\alternativa 25
\alternativa 50
\end{vertical}

\pregunta A equivale al 40\% de B y B equivale al 30\% de C. Si C = 100, entonces A + B =
\begin{vertical}
\alternativa 12
\alternativa 30
\alternativa 42
\alternativa 142
\alternativa 150
\end{vertical}

\pregunta Después de efectuar un 18\% descuento de su sueldo, una persona recibe \$328 000. ¿Cuánto habría recibido sin el descuento?
\begin{vertical}
\alternativa \$59 040
\alternativa \$72 000
\alternativa \$387 040
\alternativa \$400 000
\alternativa Más de \$400 000
\end{vertical}

\pregunta En un cierto día, el \% de asistencia de un curso fue de un 70\%, si los asistentes eran 28. ¿Cuántos alumnos en total tiene el curso?
\begin{vertical}
\alternativa 12
\alternativa 36
\alternativa 38
\alternativa 40
\alternativa 52
\end{vertical}

\pregunta ¿Cuál(es) de las siguientes afirmaciones es (son) verdadera(s)?
\begin{verticali}
\alternativa El 60\% de 0,03 es 0,02.
\alternativa El 66,6\% de 0,16 es 0,1.
\alternativa El 16,6\% de 30 es 5.
\end{verticali}
\begin{vertical}
\alternativa Solo I
\alternativa Solo II
\alternativa Solo I y II
\alternativa Solo II y III
\alternativa I, II y III
\end{vertical}

\pregunta Un curso tiene 40 alumnos y en el siguiente gráfico de barras se ilustra la asistencia en una cierta semana: ¿Cuál(es) de las siguiente(s) afirmaciones es (son) verdadera(s)?
\includegraphics[width=0.5\textwidth]{example-image-c}
\begin{verticali}
\alternativa La menor inasistencia en la semana fue de un 5\%.
\alternativa La menor asistencia diaria fue de un 80\%.
\alternativa El promedio de asistencia diaria en esos 5 días fue un 87,5\%.
\end{verticali}
\begin{vertical}
\alternativa Solo I
\alternativa Solo II
\alternativa Solo I y II
\alternativa Solo II y III
\alternativa I, II y III
\end{vertical}


\pregunta Un artículo tiene un costo de \$A y se vende en \$B (B > A), ¿cuál es el porcentaje de ganancia?
\begin{vertical}
\alternativa \( \left( \dfrac{A - B}{A} \right) \cdot 100\% \)
\alternativa \( \left( \dfrac{B - A}{A} \right) \cdot 100\% \)
\alternativa \( \left( \dfrac{B - A}{B} \right) \cdot 100\% \)
\alternativa \( \left( \dfrac{B - A}{A + B} \right) \cdot 100\% \)
\alternativa \( (AB) \cdot 100\% \)
\end{vertical}


\pregunta Durante el primer semestre, en una automotora, las ventas de un cierto modelo según su color, se muestran en el siguiente gráfico circular. Si la cantidad de vehículos azules vendidos fue 54, ¿cuál(es) de las siguientes afirmaciones es (son) verdadera(s)?
%\begin{center}
%\begin{tikzpicture}[scale=0.9]
%    \def\radius{1.3cm}
%    \coordinate (O) at (0,0);
%    % Blancos (40%) from -72 to 72 degrees (144 deg)
%    \fill[gray!30] (O) -- (-72:\radius) arc (-72:72:\radius) -- cycle;
%    \node[text width=1.2cm, align=center, font=\small] at (0:0.6*\radius) {40\% blancos};
%    % Rojos (15%) from -72 to -72-54 = -126 degrees (54 deg)
%    \fill[red!50] (O) -- (-72:\radius) arc (-72:-126:\radius) -- cycle;
%    \node[text width=1cm, align=center, font=\small] at (-99:0.65*\radius) {15\% rojos};
%    % Azules (45%) from 72 to 72+162 = 234 degrees (or -126) (162 deg)
%    \fill[blue!50] (O) -- (72:\radius) arc (72:234:\radius) -- cycle;
%    \node[text width=1.2cm, align=center, font=\small] at (153:0.6*\radius) {45\% azules};
%\end{tikzpicture}
%\end{center}
\begin{verticali}
\alternativa La cantidad de vehículos blancos vendidos fue de 48 unidades.
\alternativa La diferencia entre azules y blancos fue de 6 unidades.
\alternativa El total de vehículos vendidos durante ese semestre fueron 120.
\end{verticali}
\begin{vertical}
\alternativa Solo I
\alternativa Solo II
\alternativa Solo I y II
\alternativa Solo II y III
\alternativa I, II y III
\end{vertical}

\pregunta Por efectos de la evaporación, la altura del agua de un estanque disminuye en un 5\% por día. Si a los 30 días la altura era de 120 metros, entonces la altura inicial era de:
\begin{vertical}
\alternativa \( 120 \cdot (0.95)^{30} \)
\alternativa \( 120 \cdot (0.95)^{-30} \)
\alternativa \( 120 \cdot (1.05)^{30} \)
\alternativa \( \dfrac{(0.95)^{30}}{120} \)
\alternativa \( 120 \cdot (1.05 - 30) \)
\end{vertical}

\pregunta Se puede determinar que \% es a de b sabiendo que
\begin{verticaln}
\alternativa \( a = 1,2b \)
\alternativa \( \dfrac{A - B}{A} = \)
\end{verticaln}
\begin{vertical}
\alternativa (1) por sí sola
\alternativa (2) por sí sola
\alternativa Ambas juntas, (1) y (2)
\alternativa Cada una por sí sola, (1) ó (2)
\alternativa Se requiere información adicional
\end{vertical}

\pregunta En una caja hay n bolitas, A son de color verde, B son rojas y las restantes son C son azules. Si A > B > C, ¿cuál de las siguientes afirmaciones es FALSA?
\begin{vertical}
\alternativa El \% de bolitas verdes de la caja es \( \left( \dfrac{A}{n} \right) \cdot 100\% \)
\alternativa El \% de bolitas que no son rojas de la caja es \( \left( \dfrac{n-B}{n} \right) \cdot 100\% \)
\alternativa El \% en que las verdes exceden a las rojas es \( \left( \dfrac{A-B}{n} \right) \cdot 100\% \)
\alternativa El \% en que las rojas exceden a las azules es \( \left( \dfrac{B-C}{n} \right) \cdot 100\% \)
\alternativa El \% en que las azules exceden a las verdes es \( \left( \dfrac{C-A}{n} \right) \cdot 100\% \)
\end{vertical}

\pregunta Un automóvil vale \$A y se vende con un B\% de ganancia, ¿cuál es su precio de venta?
\begin{vertical}
\alternativa \( \$ \dfrac{B}{100} \)
\alternativa \( \$ \dfrac{AB}{100} \)
\alternativa \( \$ \left( A + \dfrac{AB}{100} \right) \)
\alternativa \( \$ \left( A + \dfrac{B}{100} \right) A \)
\alternativa \( \$ \left( A - \dfrac{B}{100} \right) A \)
\end{vertical}

\pregunta Un artículo tiene un A\% de descuento, con lo que su nuevo precio es \$C, ¿cuál era su precio original?
\begin{vertical}
\alternativa \( C + \dfrac{B}{100} \)
\alternativa \( C + \dfrac{A}{100} \cdot C \)
\alternativa \( \dfrac{100C}{A + 100} \)
\alternativa \( \dfrac{100C}{100 - A} \)
\alternativa \( AC \)
\end{vertical}

\pregunta En la siguiente tabla, se muestra la distribución de ausentes/presentes por género en un día de clases, siendo n el total de alumnos:
\begin{centrado}
  \begin{tblr}{
    colspec={Q[c,m] Q[c,m] Q[c,m]},
    row{1}={font=\bfseries},
    hlines, vlines
  }
            & Presentes & Ausentes \\
  Hombres   & a         & c        \\
  Mujeres   & b         & d        \\
  \end{tblr}
\end{centrado}
¿Cuál de las siguientes afirmaciones es FALSA?
\begin{vertical}
\alternativa El porcentaje de presentes ese día fue \( \left( \dfrac{a+b}{n} \right) \cdot 100\% \)
\alternativa El porcentaje de mujeres del curso es \( \left( \dfrac{b+d}{n} \right) \cdot 100\% \)
\alternativa De las mujeres, el porcentaje que asistió ese día fue \( \left( \dfrac{b}{b+d} \right) \cdot 100\% \)
\alternativa Del curso, el porcentaje de los hombres ausentes ese día fue \( \left( \dfrac{c}{n} \right) \cdot 100\% \)
\end{vertical}

\pregunta Un capital de $\$10^6$ se deposita a un interés compuesto trimestral de un 2\% durante tres años, entonces el capital final que obtendrá al cabo de ese tiempo será de:
\begin{vertical}
\alternativa \( \$10^6 \cdot \left( 1 + \dfrac{2}{100} \right)^9 \)
\alternativa \( \$10^6 \cdot \left( 1 + \dfrac{2}{100} \right)^{12} \)
\alternativa \( \$10^6 \cdot \left( 1 + \dfrac{2}{100} \right)^3 \)
\alternativa \( \$10^6 \cdot \left( 1 + \dfrac{2}{100} \right)^4 \)
\end{vertical}

\pregunta Se depositan \$C a un interés compuesto de un i \% anual durante t años. Al final del período el porcentaje de aumento del capital con respecto al capital inicial es de
\begin{vertical}
\alternativa \( \left( 1 + \dfrac{i}{100} \right)^t \cdot 100\% \)
\alternativa \( \left( 1 + \dfrac{i}{100} \right) \cdot 100\% \)
\alternativa \( \left( \left( 1 + \dfrac{i}{100} \right)^t - 1 \right) \cdot 100\% \)
\alternativa \( \left( \left( 1 + \dfrac{it}{100} \right) - 1 \right) \cdot 100\% \)
\alternativa Otra expresión.
\end{vertical}

\pregunta Se tienen tres artículos cuyos precios son \$A, \$B y \$C. Si A es el 20\% de B, se puede determinar qué \% es A de C, sabiendo que:
\begin{verticaln}
\alternativa El 10\% de B equivale al 40\% de C.
\alternativa B = 4C.
\end{verticaln}
\begin{vertical}
\alternativa (1) por sí sola
\alternativa (2) por sí sola
\alternativa Ambas juntas, (1) y (2)
\alternativa Cada una por sí sola, (1) ó (2)
\alternativa Se requiere información adicional
\end{vertical}

\pregunta En un rectángulo, el largo aumenta un 30\% y el ancho disminuye un 30\%, entonces su área
\begin{vertical}
\alternativa queda igual.
\alternativa aumenta un 3\%.
\alternativa disminuye en un 9\%.
\alternativa sube en un 10\%.
\alternativa disminuye en un 10\%.
\end{vertical}

\pregunta En el paralelepípedo recto de la figura, las aristas basales a y b aumentan un 10\% y la altura c disminuye un 20\%, ¿qué sucede con su volumen?
%\begin{center}
%\begin{tikzpicture}[scale=0.6, every node/.style={color=black}]
%    \def\widthA{3}
%    \def\heightC{2.5}
%    \def\depthB{1.5}
%    \def\angle{-35}
%    \coordinate (V_FBL) at (0,0);
%    \coordinate (V_FBR) at (\widthA,0);
%    \coordinate (V_FTL) at (0,\heightC);
%    \coordinate (V_FTR) at (\widthA,\heightC);
%    \coordinate (V_BBL) at ($(V_FBL) + (\angle:\depthB)$);
%    \coordinate (V_BBR) at ($(V_FBR) + (\angle:\depthB)$);
%    \coordinate (V_BTL) at ($(V_FTL) + (\angle:\depthB)$);
%    \coordinate (V_BTR) at ($(V_FTR) + (\angle:\depthB)$);
%    \draw (V_BBL) -- (V_BBR);
%    \draw (V_BBL) -- (V_BTL);
%    \draw (V_BTL) -- (V_BTR);
%    \draw (V_FBL) -- (V_BBL);
%    \draw (V_FBR) -- (V_BBR) node[pos=0.4, below right, xshift=0pt, yshift=-2pt] {b};
%    \draw (V_FTR) -- (V_BTR);
%    \draw (V_FBL) -- (V_FBR) node[midway, below] {a};
%    \draw (V_FBR) -- (V_FTR) node[midway, right] {c};
%    \draw (V_FTR) -- (V_FTL);
%    \draw (V_FTL) -- (V_FBL);
%    \draw (V_BBR) -- (V_BTR);
%\end{tikzpicture}
%\end{center}
\begin{vertical}
\alternativa Aumenta un 10\%
\alternativa Disminuye un 3,2\%
\alternativa Disminuye un 2,2\%
\alternativa Aumenta un 4,2\% a
\alternativa Permanece igual.
\end{vertical}

\pregunta Sean f, a y b tres variables que se relacionan de modo que \( \dfrac{1}{f} = \dfrac{1}{a} + \dfrac{1}{b} \). Si a y b disminuyen en un 20\%, entonces f
\begin{vertical}
\alternativa aumenta en un 20\%.
\alternativa aumenta en un 25\%.
\alternativa disminuye un 20\%.
\alternativa disminuye un 25\%.
\alternativa disminuye un 40\%.
\end{vertical}

\pregunta El volumen de un cilindro es \( \pi r^2 h \) donde r es la longitud del radio basal y h es su altura. Si el radio aumenta en un 10\% y su altura disminuye en un 10\%, entonces su volumen
\begin{vertical}
\alternativa aumentó en un 10\%.
\alternativa aumentó en menos de un 1\%.
\alternativa aumentó en un 8,9\%.
\alternativa aumento en un 89\%.
\alternativa queda igual.
\end{vertical}

\pregunta En un rectángulo, el largo aumenta en un 10\% y el ancho disminuye en un 20\%, obteniéndose un rectángulo de área 22 cm\(^2\), ¿cuál era el área del rectángulo original?
\begin{vertical}
\alternativa 25 cm\(^2\)
\alternativa 24,2 cm\(^2\)
\alternativa 23,4 cm\(^2\)
\alternativa 23,2 cm\(^2\)
\alternativa 24,64 cm\(^2\)
\end{vertical}

\pregunta Las variables P, A, B y C son tales que \( P = \dfrac{AB}{C} \). Si A y B aumentan en un 20\% y C disminuye en un 10\%, entonces P
\begin{vertical}
\alternativa aumenta en un 50\%.
\alternativa disminuye en un 30\%.
\alternativa aumenta en un 60\%.
\alternativa aumenta en un 40\%
\alternativa aumenta en un 160\%.
\end{vertical}

\pregunta Sebastián deposita \$C en el banco, durante los dos primeros años la tasa de interés fue de un 2\% y en el tercer año subió a un 3\%. Si el capital se reajusta anualmente, ¿cuál será el nuevo capital a fines del tercer año?
\begin{vertical}
\alternativa \$C \(\cdot\) 1,07
\alternativa \$C \(\cdot\) 0,04 \(\cdot\) 0,03
\alternativa \$C \(\cdot\) (0,02)\(^2\) \(\cdot\) 0,03
\alternativa \$C \(\cdot\) (1,02)\(^2\) \(\cdot\) 1,03
\alternativa \$C \(\cdot\) (1,02)\(^3\) \(\cdot\) 1,03
\end{vertical}

\pregunta En una reserva forestal, la cantidad de hectáreas de árboles disminuye a una tasa de un 20\% anual. ¿Cuál de las siguientes ecuaciones nos permite determinar la cantidad de años t que deben transcurrir para que la cantidad de hectáreas iniciales C se haya reducido a un 1\%?
\begin{vertical}
\alternativa \( C \cdot (0.8)^t = 0.01 C \)
\alternativa \( C \cdot (0.8)^t = 0.99 C \)
\alternativa \( C \cdot (0.2)^t = 0.99 C \)
\alternativa \( C \cdot (1.2)^t = 0.99 C \)
\alternativa \( C \cdot (1.2)^t = 1+0.99 C \)
\end{vertical}

\end{preguntas}

\section{Potencias y raíces}

\begin{preguntas}
\pregunta \(3^3 + 3^3 + 3^3 =\)
\begin{vertical}
\alternativa \(3^4\)
\alternativa \(3^5\)
\alternativa \(3^9\)
\alternativa \(9^3\)
\alternativa \(9^9\)
\end{vertical}

\pregunta \(\sqrt{50} - \sqrt{18} - \sqrt{8} =\)
\begin{vertical}
\alternativa \(0\)
\alternativa \(\sqrt{24}\)
\alternativa \(6\sqrt{2}\)
\alternativa \(\sqrt{40}\)
\alternativa \(\sqrt{60}\)
\end{vertical}

\pregunta \(2^{10} + 2^{11} =\)
\begin{vertical}
\alternativa \(2^{21}\)
\alternativa \(2^{22}\)
\alternativa \(4^{21}\)
\alternativa \(6^{10}\)
\alternativa \(3 \cdot 2^{10}\)
\end{vertical}

\pregunta \(\dfrac{\sqrt{8} \cdot \sqrt{6}}{\sqrt{3}} =\)
\begin{vertical}
\alternativa \(2\)
\alternativa \(4\)
\alternativa \(6\)
\alternativa \(8\)
\alternativa \(16\)
\end{vertical}

\pregunta \(\dfrac{\sqrt{2020}}{\sqrt{0,2020}} =\)
\begin{vertical}
\alternativa \(10^4\)
\alternativa \(10^2\)
\alternativa \(10^{-2}\)
\alternativa \(10^{-1}\)
\alternativa \(10\)
\end{vertical}

\pregunta \((0,00036)^{-3} : (6000)^{-3} =\)
\begin{vertical}
\alternativa \(6^{-3} \cdot 10^6\)
\alternativa \(6^{-3} \cdot 10^{12}\)
\alternativa \(6^{-3} \cdot 10^{-24}\)
\alternativa \(6^{-3} \cdot 10^{24}\)
\alternativa \(6^{-9} \cdot 10^{-24}\)
\end{vertical}

\pregunta Sean los números: \(a = \sqrt{2}\) ; \(b = \dfrac{1}{\sqrt{2}}\) ; \(c = 1,4\).
Al ordenarlos de menor a mayor, resulta:
\begin{vertical}
\alternativa c - b - a
\alternativa a - b - c
\alternativa a - c - b
\alternativa b - a - c
\alternativa b - c - a
\end{vertical}

\pregunta \((\sqrt{2} - 1)^2 - (1 + \sqrt{2})^2 =\)
\begin{vertical}
\alternativa \(-4\sqrt{2}\)
\alternativa \(2\sqrt{2}\)
\alternativa \(\sqrt{2}\)
\alternativa \(2\)
\alternativa \(0\)
\end{vertical}

\pregunta \(\dfrac{\sqrt{20} + \sqrt{45}}{\sqrt{5}} =\)
\begin{vertical}
\alternativa \(5\)
\alternativa \(7\)
\alternativa \(\sqrt{5}\)
\alternativa \(\sqrt{13}\)
\alternativa \(2 + 3\sqrt{5}\)
\end{vertical}

\pregunta \(\dfrac{2^4 + 2^5}{2^6 + 2^7} =\)
\begin{vertical}
\alternativa \(2^{-4}\)
\alternativa \(2^{-2}\)
\alternativa \(2^{-1}\)
\alternativa \(2^2\)
\alternativa \(2^3\)
\end{vertical}

\pregunta Se puede determinar la potencia \(a^n\), con a y n racionales y \(a \neq 0\), si se sabe que:
\begin{verticaln}
\alternativa \(a^{-2n} = 9\)
\alternativa \(a^{3n} = -\dfrac{1}{27}\)
\end{verticaln}
\begin{vertical}
\alternativa (1) por sí sola
\alternativa (2) por sí sola
\alternativa Ambas juntas, (1) y (2)
\alternativa Cada una por sí sola, (1) ó (2)
\alternativa Se requiere información adicional
\end{vertical}

\pregunta \(\dfrac{1}{\sqrt{2}-1} - \dfrac{1}{\sqrt{2}} =\)
\begin{vertical}
\alternativa \(1 + \sqrt{2}\)
\alternativa \(\dfrac{1}{2}\)
\alternativa \(\dfrac{1}{3}\)
\alternativa \(\dfrac{2 + \sqrt{2}}{2}\)
\alternativa \(-\dfrac{2 + \sqrt{2}}{2}\)
\end{vertical}

\pregunta El resultado de \(\dfrac{1 + \dfrac{1}{\sqrt{2}}}{\sqrt{2}-1}\) es un número real que está entre:
\begin{vertical}
\alternativa 1 y 2
\alternativa 2 y 3
\alternativa 3 y 4
\alternativa 4 y 5
\alternativa 5 y 6
\end{vertical}

\pregunta \((\sqrt{3}+\sqrt{2})^3 \cdot (\sqrt{2}-\sqrt{3})^4 =\)
\begin{vertical}
\alternativa \(3\sqrt{2}-2\sqrt{3}\)
\alternativa \(\sqrt{2}+\sqrt{3}\)
\alternativa \(\sqrt{30}\)
\alternativa \(\sqrt{2}-\sqrt{3}\)
\alternativa \(\sqrt{3}-\sqrt{2}\)
\end{vertical}

\pregunta Si \(P = \sqrt{4+\sqrt{7}} + \sqrt{4-\sqrt{7}}\), entonces \(P^2 =\)
\begin{vertical}
\alternativa 4
\alternativa 8
\alternativa 14
\alternativa 16
\alternativa \(2\sqrt{2}\)
\end{vertical}

\pregunta \(\left(2^n - 2^{n-1}\right)^2 =\)
\begin{vertical}
\alternativa \(2^{2n-1}\)
\alternativa \(4^{n-2}\)
\alternativa \((0,25)^{2-n}\)
\alternativa \((0,25)^{1-n}\)
\alternativa \((0,5)^{2+2n}\)
\end{vertical}

\pregunta La expresión \((2^{12} - 1)\) NO es divisible por:
\begin{vertical}
\alternativa 7
\alternativa 9
\alternativa 5
\alternativa 13
\alternativa 25
\end{vertical}

\pregunta El resultado de \(2^{40} + 2^{39} + 2^{38}\) es divisible por:
\begin{verticali}
\alternativa 8
\alternativa 10
\alternativa 100
\end{verticali}
Es (son) correcta(s):
\begin{vertical}
\alternativa Solo I
\alternativa Solo II
\alternativa Solo I y II
\alternativa Solo II y III
\alternativa I, II y III
\end{vertical}

\pregunta Si \(\dfrac{2^{x+1} + 2^x}{3^x - 3^{x-2}} = \dfrac{4}{9}\), entonces el valor de \(2x+1\) es:
\begin{vertical}
\alternativa 5
\alternativa 15
\alternativa 14
\alternativa 13
\alternativa 11
\end{vertical}

\pregunta ¿Cuál(es) de las siguientes expresiones es (son) equivalente(s) al cuociente \(\dfrac{3^8 + 3^7}{2^{10} - 2^8}\)?
\begin{verticali}
\alternativa \((1,5)^6\)
\alternativa \((0,6)^{-6}\)
\alternativa \(\dfrac{3^6}{2^6}\)
\end{verticali}
\begin{vertical}
\alternativa Solo I
\alternativa Solo II
\alternativa Solo I y II
\alternativa Solo II y III
\alternativa I, II y III
\end{vertical}

\pregunta \(\sqrt{\dfrac{\sqrt{75} + \sqrt{48}}{\sqrt{3}}} =\)
\begin{vertical}
\alternativa 3
\alternativa 9
\alternativa \(\sqrt{3}\)
\alternativa \(2\sqrt{3}\)
\alternativa \(4\sqrt{3}\)
\end{vertical}

\pregunta Sean los números: \(x = \sqrt{3} - \sqrt{2}\) ; \(y = \sqrt{3} + \sqrt{2}\) ; \(z = \dfrac{\sqrt{3}}{\sqrt{2}}\), entonces \(xyz =\)
\begin{vertical}
\alternativa \(1 + \sqrt{6}\)
\alternativa \(\sqrt{3} + \sqrt{2}\)
\alternativa \(\sqrt{3}\)
\alternativa \(\dfrac{\sqrt{6}}{2}\)
\alternativa \(\sqrt{6}\)
\end{vertical}

\pregunta Si \(ab = \sqrt{3}\) y \(b = \sqrt{3} - \sqrt{2}\), entonces \(a:\)
\begin{vertical}
\alternativa \(3 + \sqrt{6}\)
\alternativa \(3 + \sqrt{3}\)
\alternativa \(\sqrt{3} + \sqrt{2}\)
\alternativa \(-(1 + \sqrt{2})\)
\alternativa \(-\sqrt{2}\)
\end{vertical}

\pregunta ¿Cuál de las siguientes expresiones NO es equivalente a \(2\sqrt{6}\)?
\begin{vertical}
\alternativa \(\dfrac{\sqrt{72}}{\sqrt{3}}\)
\alternativa \(\sqrt{12} \cdot \sqrt{4}\sqrt{3}\)
\alternativa \(\sqrt{2\sqrt{7}+2} \cdot \sqrt{2\sqrt{7}-2}\)
\alternativa \((\sqrt{2}+\sqrt{3}+\sqrt{5}) \cdot (\sqrt{2}+\sqrt{3}-\sqrt{5})\)
\alternativa \((\sqrt{3}+\sqrt{2})^2 - (\sqrt{3}-\sqrt{2})^2\)
\end{vertical}

\pregunta \((\sqrt{2})^{20} \cdot \left(1 + \dfrac{1}{\sqrt{2}}\right)^{10} \cdot \left(1 - \dfrac{1}{\sqrt{2}}\right)^{10} =\)
\begin{vertical}
\alternativa 1
\alternativa \(\dfrac{1}{4}\)
\alternativa \(\dfrac{9}{4}\)
\alternativa \(\dfrac{3}{4}\)
\alternativa \(\dfrac{9}{16}\)
\end{vertical}

\pregunta Si \(A = 2^x + 2^{-x}\), entonces \(4^x + 4^{-x} =\)
\begin{vertical}
\alternativa \(A^2 + 4\)
\alternativa \(A^2 - 4\)
\alternativa \(A^2 + 2\)
\alternativa \(A^2 - 2\)
\alternativa \(A^2\)
\end{vertical}

\pregunta \((\sqrt{3}-\sqrt{2})\cdot\sqrt{5 + 2\sqrt{6}} =\)
\begin{vertical}
\alternativa 1
\alternativa 2
\alternativa \(\sqrt{6}\)
\alternativa \(2\sqrt{6}\)
\alternativa 7
\end{vertical}

\pregunta Si \(x > 0\), entonces \(\sqrt{\sqrt{x+9}+\sqrt{x}} \cdot \sqrt{\sqrt{x+9}-\sqrt{x}} =\)
\begin{vertical}
\alternativa 3
\alternativa 9
\alternativa \(\sqrt{3}\)
\alternativa \(2\sqrt{3}\)
\alternativa \(3\sqrt{3}\)
\end{vertical}

\pregunta Si \(a > 0\), entonces \(\dfrac{\sqrt{a^2}}{\sqrt{a}} =\)
\begin{vertical}
\alternativa \(\sqrt[3]{a^2}\)
\alternativa \(\sqrt{a^3}\)
\alternativa \(\sqrt{a}\)
\alternativa \(\sqrt[3]{a}\)
\alternativa \(\sqrt[6]{a}\)
\end{vertical}

\pregunta ¿Cuál(es) de las siguientes igualdades es (son) verdadera(s)?
\begin{verticali}
\alternativa \(\sqrt{3} \cdot \sqrt[3]{3^2} = 3\)
\alternativa \(\dfrac{\sqrt[3]{3}}{\sqrt{3}} = \dfrac{1}{\sqrt[6]{3}}\)
\alternativa \(\sqrt[3]{3} \cdot \sqrt[3]{3} = \sqrt[3]{3}\)
\end{verticali}
\begin{vertical}
\alternativa Solo I
\alternativa Solo II
\alternativa Solo I y II
\alternativa Solo II y III
\alternativa I, II y III
\end{vertical}

\pregunta Sean a y b números reales y n un número entero. Se puede determinar que \(a^n > b^n\), sabiendo que:
\begin{verticaln}
\alternativa \(a > b\)
\alternativa a y b son positivos.
\end{verticaln}
\begin{vertical}
\alternativa (1) por sí sola
\alternativa (2) por sí sola
\alternativa Ambas juntas, (1) y (2)
\alternativa Cada una por sí sola, (1) ó (2)
\alternativa Se requiere información adicional
\end{vertical}

\pregunta ¿Cuál(es) de las siguientes expresiones es (son) equivalente(s) a la expresión \(\sqrt{4^{2n+3} + 4^{2n+2} + 4^{2n}}\)?
\begin{verticali}
\alternativa \(9 \cdot 4^n\)
\alternativa \(18^n\)
\alternativa \((0,25)^{-n} \cdot 9\)
\end{verticali}
\begin{vertical}
\alternativa Solo I
\alternativa Solo II
\alternativa Solo I y III
\alternativa Solo II y III
\alternativa I, II y III
\end{vertical}

\pregunta Si \(0 < a < 2\), entonces \(\sqrt{a^2-4a+4} + \sqrt{a^2+4a+4} =\)
\begin{vertical}
\alternativa \(2a\)
\alternativa \(4a\)
\alternativa \(2\)
\alternativa \(4\)
\alternativa \(-2\)
\end{vertical}

\pregunta Si \(m>n>0\), ¿cuál(es) de las siguientes expresiones es (son) equivalentes a: \(\dfrac{\sqrt{4n^2 - 12mn + 9m^2}}{\sqrt{9m^2 - 4n^2}}\)?
\begin{verticali}
\alternativa \(\dfrac{2n - 3m}{\sqrt{(3m+2n)(3m-2n)}}\)
\alternativa \(\dfrac{\sqrt{9m^2 - 4n^2}}{2n + 3m}\)
\alternativa \(\sqrt{\dfrac{3m - 2n}{3m + 2n}}\)
\end{verticali}
\begin{vertical}
\alternativa Solo I
\alternativa Solo II
\alternativa Solo I y II
\alternativa Solo II y III
\alternativa I, II y III
\end{vertical}

\pregunta La expresión \(\sqrt{4(m^2+n^2-2mn)} - \sqrt{9(m-n)^2}\) con \(n>m\) es equivalente a:
\begin{vertical}
\alternativa \(5(m+n)\)
\alternativa \(n-m\)
\alternativa \(m-n\)
\alternativa \(7(m+n)\)
\alternativa \(5(m-n)\)
\end{vertical}

\pregunta Si \(x = \dfrac{1}{2\sqrt{3}}\), \(y = \dfrac{\sqrt{7}}{3}\), \(z = \dfrac{\sqrt{10}}{4}\), \(w = \dfrac{\sqrt{18}}{5}\), entonces:
\begin{vertical}
\alternativa \(z<x<w<y\)
\alternativa \(z<w<y<x\)
\alternativa \(z<w<x<y\)
\alternativa \(w<z<x<y\)
\alternativa \(y<x<w<z\)
\end{vertical}

\pregunta ¿Cuál(es) de las siguientes afirmaciones es (son) verdadera(s)?
\begin{verticali}
\alternativa \(\dfrac{\sqrt{3}-\sqrt{2}}{2} < \dfrac{1}{\sqrt{3}+\sqrt{2}}\)
\alternativa \(\dfrac{\sqrt{5}+2}{4} < \dfrac{\sqrt{3}+\sqrt{2}}{3}\)
\alternativa \(\dfrac{3\sqrt{2}+2}{2\sqrt{5}+1} < \dfrac{\sqrt{3}}{2}\)
\end{verticali}
\begin{vertical}
\alternativa Solo I
\alternativa Solo II
\alternativa Solo I y II
\alternativa Solo II y III
\alternativa I, II y III
\end{vertical}

\pregunta Si \(A=2\sqrt{3}+2\), \(B=2\sqrt{7}-1\) y \(C=\sqrt{38}\), entonces:
\begin{vertical}
\alternativa \(A<B<C\)
\alternativa \(A<C<B\)
\alternativa \(B<A<C\)
\alternativa \(B<C<A\)
\alternativa \(C<A<B\)
\end{vertical}

\end{preguntas}

\section{Razones y proporciones}

\begin{preguntas}
%\pregunta Dada la siguiente tabla:
%\begin{center}
%\begin{tblr}{colspec={|c|c|c|c|},hlines,vlines}
%A & 10 & 15 & 20 \\
%B & 3  & X  & 1,5 \\
%\end{tblr}
%\end{center}
%¿Cuál(es) de las siguientes afirmaciones es(son) verdadera(s)?:
%\begin{verticali}
%\alternativa A y B son directamente proporcionales.
%\alternativa El valor de x es 2.
%\alternativa La constante de proporcionalidad inversa es 30.
%\end{verticali}
%\begin{vertical}
%\alternativa Sólo I
%\alternativa Sólo I y II
%\alternativa Sólo I y III
%\alternativa Sólo II y III
%\alternativa I, II y III
%\end{vertical}

\pregunta 2 electricistas hacen un trabajo en 6 días, trabajando 8 horas diarias.
¿Cuál (es) de las siguientes afirmaciones es(son) verdadera(s)?
\begin{verticali}
\alternativa 4 electricistas harán el trabajo en 3 días, trabajando 8 horas diarias.
\alternativa Los electricistas y las horas son directamente proporcionales.
\alternativa La constante de proporcionalidad es 3.
\end{verticali}
\begin{vertical}
\alternativa Sólo I
\alternativa Sólo I y II
\alternativa Sólo I y III
\alternativa Sólo II y III
\alternativa I, II y III
\end{vertical}

\pregunta En una quinta hay naranjos, manzanos y duraznos que suman en total 300 árboles.
Si hay 120 naranjos y la razón entre los duraznos y manzanos es 7: 3, entonces
¿cuántos duraznos hay en la quinta?
\begin{vertical}
\alternativa 54
\alternativa 77
\alternativa 84
\alternativa 126
\alternativa 210
\end{vertical}

\pregunta $y$ es inversamente proporcional al cuadrado de $x$, cuando $y=16$, $x=1$.
Si $x=8$, entonces $y=$
\begin{vertical}
\alternativa $\dfrac{1}{2}$
\alternativa $\dfrac{1}{4}$
\alternativa 2
\alternativa 4
\alternativa 9
\end{vertical}

\pregunta Se desea cortar un alambre de 720 mm en tres trozos de modo que la razón de sus
longitudes sea 8: 6: 4. ¿Cuánto mide cada trozo de alambre, de acuerdo al orden de
las razones dadas?
\begin{vertical}
\alternativa 180 mm \hspace*{10pt} 120 mm \hspace*{10pt} 90 mm
\alternativa 420 mm \hspace*{10pt} 180 mm \hspace*{10pt} 120 mm
\alternativa 320 mm \hspace*{10pt} 240 mm \hspace*{10pt} 160 mm
\alternativa 510 mm \hspace*{10pt} 120 mm \hspace*{10pt} 90 mm
\alternativa Ninguna de las medidas anteriores
\end{vertical}

\pregunta Se sabe que $a$ es directamente proporcional al número $\dfrac{1}{b}$ y cuando
$a$ toma el valor 15, el valor de $b$ es 4. Si $a$ toma el valor 6, entonces
el valor de $b$ es:
\begin{vertical}
\alternativa 10
\alternativa $\dfrac{8}{5}$
\alternativa $\dfrac{1}{10}$
\alternativa $\dfrac{15}{4}$
\alternativa Ninguno de los valores anteriores
\end{vertical}

\pregunta En un mapa (a escala) se tiene que 2 cm en él corresponden a 25 km en la realidad.
Si la distancia en el mapa entre dos ciudades es 5,4 cm, entonces la distancia real es
\begin{vertical}
\alternativa 50 km
\alternativa 65 km
\alternativa 67,5 km
\alternativa 62,5 km
\alternativa ninguno de los valores anteriores.
\end{vertical}

\pregunta Dos variables $N$ y $M$ son inversamente proporcionales entre sí.
Para mantener el valor de la constante de proporcionalidad, si $M$ aumenta al
doble, entonces $N$
\begin{vertical}
\alternativa aumenta al doble.
\alternativa disminuye a la mitad.
\alternativa aumenta en dos unidades.
\alternativa disminuye en dos unidades.
\alternativa se mantiene constante.
\end{vertical}
%
%\pregunta En la tabla adjunta $z$ es directamente proporcional a $\dfrac{1}{y}$
%según los datos registrados, el valor de $\dfrac{a}{c}$, es:
%\begin{center}
%\begin{tblr}{
%  colspec={|c|c|},
%  hlines,
%  vlines
%}
%Z & y \\
%8 & 2 \\
%$a$ & 4 \\
%1 & $\dfrac{1}{4}$ \\
%16 & $b$ \\
%$c$ & $\dfrac{1}{2}$ \\
%\end{tblr}
%\end{center}
%\begin{tblr}{
%  colspec={|c|c|},
%  hlines,
%  vlines
%}
%Z & y \\ \hline
%8 & 2 \\
%$a$ & 4 \\
%1 & $\dfrac{1}{4}$ \\
%16 & $b$ \\
%\end{tblr}
%\begin{vertical}
%\alternativa 256
%\alternativa 16
%\alternativa $\dfrac{1}{16}$
%\alternativa 64
%\alternativa $\dfrac{1}{64}$
%\end{vertical}
%
\pregunta La escala de un mapa es 1: 500.000. Si en el mapa la distancia entre dos ciudades
 es 3,5 cm, ¿cuál es la distancia real entre ellas?
\begin{vertical}
\alternativa 1,75 km
\alternativa 17,5 km
\alternativa 175 km
\alternativa 1.750 km
\alternativa 17.500 km
\end{vertical}

\pregunta Los cajones $M$ y $S$ pesan juntos $K$ kilogramos. Si la razón entre los pesos
de $M$ y $S$ es 3: 4, entonces $S: K=$
\begin{vertical}
\alternativa 4: 7
\alternativa 4: 3
\alternativa 7: 4
\alternativa 3: 7
\alternativa 3: 4
\end{vertical}

\pregunta La ley combinada que rige el comportamiento ideal de un gas es
$\dfrac{P \cdot V}{T} = \text{constante}$, donde $P$ es la presión del gas,
$V$ su volumen y $T$ su temperatura absoluta. ¿Cuál (es) de las siguientes
afirmaciones es(son) verdadera(s)?
\begin{verticali}
\alternativa A volumen constante la presión es directamente proporcional a la temperatura
\alternativa A temperatura constante la presión es inversamente proporcional al volumen
\alternativa A presión constante el volumen es inversamente proporcional a la temperatura
\end{verticali}
\begin{vertical}
\alternativa Solo I
\alternativa Solo II
\alternativa Solo I y II
\alternativa Solo I y III
\alternativa I, II y III
\end{vertical}

\pregunta Una nutricionista mezcla tres tipos de jugos de fruta de modo que sus
volúmenes están en la razón 1: 2: 3. Si el volumen del segundo tipo es de 4
litros, ¿cuántos litros tiene la mezcla total?
\begin{vertical}
\alternativa 6 litros
\alternativa 10 litros
\alternativa 12 litros
\alternativa 14 litros
\alternativa 16 litros
\end{vertical}

\pregunta En un curso de 40 estudiantes, la razón entre mujeres y hombres
es $m:h$. ¿Cuál es la expresión que representa el número de mujeres?
\begin{vertical}
\alternativa $\dfrac{40m}{m+h}$
\alternativa $\dfrac{40(m+h)}{m}$
\alternativa $\dfrac{40(m+h)}{h}$
\alternativa $\dfrac{40h}{m+h}$
\alternativa $\dfrac{40m}{h}$
\end{vertical}

%\pregunta El gráfico de la figura, representa a una proporcionalidad inversa
%entre las magnitudes $m$ y $t$. ¿Cuál(es) de las siguientes afirmaciones
%es (son) verdadera(s)?
%\begin{center}
%\includegraphics[width=0.4\textwidth]{example-image-a} % Placeholder for graph from page 5
%\end{center}
%\begin{verticali}
%\alternativa La constante de proporcionalidad es 36
%\alternativa El valor de $t_1$ es 9
%\alternativa El valor de $m_1$ es 36
%\end{verticali}
%\begin{vertical}
%\alternativa Solo I
%\alternativa Solo I y II
%\alternativa Solo I y III
%\alternativa I, II y III
%\alternativa Ninguna de ellas
%\end{vertical}

\pregunta A un evento asistieron 56 personas. Si había 4 mujeres por cada 3
hombres, ¿cuántas mujeres asistieron al evento?
\begin{vertical}
\alternativa 8
\alternativa 21
\alternativa 24
\alternativa 28
\alternativa 32
\end{vertical}

\pregunta Si $h$ hombres pueden fabricar 50 artículos en un día,
¿cuántos hombres se necesitan para fabricar $x$ artículos en un día?
\begin{vertical}
\alternativa $\dfrac{hx}{50}$
\alternativa $\dfrac{50x}{h}$
\alternativa $\dfrac{x}{50h}$
\alternativa $\dfrac{h}{50x}$
\alternativa Ninguno de los valores anteriores
\end{vertical}

\pregunta En un balneario, hay 2.500 residentes permanentes. En el
mes de febrero, de cada seis personas solo una es residente permanente,
¿cuántas personas hay en febrero?
\begin{vertical}
\alternativa 416
\alternativa 4.000
\alternativa 12.500
\alternativa 15.000
\alternativa 17.500
\end{vertical}

\pregunta Las variables $x, w, u, v$ son tales que: $x$ es directamente
 proporcional a $u$, con constante de proporcionalidad 2, y $w$ es
 inversamente proporcional a $v$, con constante de proporcionalidad 8.
 ¿Cuáles de las siguientes relaciones entre dichas variables representan este hecho?
\begin{vertical}
\alternativa $\dfrac{x}{u}=2 \quad y \quad w \cdot v=8$
\alternativa $x-u=2 \quad y \quad w+v=8$
\alternativa $x \cdot u=2 \quad y \quad \dfrac{w}{v}=8$
\alternativa $x+u=2 \quad y \quad w-v=8$
\alternativa $x+w=10$
\end{vertical}

\pregunta Un trabajador $X$, trabajando solo se demora $t$ días en hacer
un jardín, otro trabajador $Y$ se demora $t+15$ días en hacer el mismo
jardín, y si ambos trabajan juntos se demoran 10 días. ¿Cuántos días se
demorará $Y$ trabajando solo?
\begin{vertical}
\alternativa 30
\alternativa 28
\alternativa 25
\alternativa 20
\alternativa 15
\end{vertical}

\pregunta Si el índice de crecimiento $C$ de una población es inversamente
proporcional al índice $D$ de desempleo y en un instante en que $C=0,5$ se
tiene que $D=0,25$, entonces entre ambos índices se cumple:
\begin{vertical}
\alternativa $D=0,5C$
\alternativa $D=C^2$
\alternativa $D=\dfrac{0,5}{C}$
\alternativa $D=0,125C$
\alternativa $D=\dfrac{0,125}{C}$
\end{vertical}

\pregunta Para hacer arreglos en un edificio se contratará un cierto
número de electricistas. Si se contratara 2 electricistas, ellos se demorarían
6 días, trabajando 8 horas diarias, ¿cuál (es) de las siguientes aseveraciones
es(son) verdadera(s)?
\begin{verticali}
\alternativa Si se contrataran 4 electricistas, se demorarían 3 días, trabajando 8 horas diarias
\alternativa El número de electricistas y el número de días son variables directamente proporcionales
\alternativa La constante de proporcionalidad entre las variables es 3
\end{verticali}
\begin{vertical}
\alternativa Solo I
\alternativa Solo III
\alternativa Solo I y II
\alternativa Solo II y III
\alternativa I, II y III
\end{vertical}

\pregunta Un trabajador hace un trabajo en 60 días, mientras que cinco
trabajadores hacen el mismo trabajo en 12 días. ¿Cuál de los siguientes
gráficos representa mejor la relación trabajadores - días?
\begin{alternativasgraficas}
\alternativa \includegraphics[width=0.2\textwidth]{example-image-a}
\alternativa \includegraphics[width=0.2\textwidth]{example-image-a}
\alternativa \includegraphics[width=0.2\textwidth]{example-image-a}
\alternativa \includegraphics[width=0.2\textwidth]{example-image-a}
\alternativa \includegraphics[width=0.2\textwidth]{example-image-a}
\end{alternativasgraficas}

%\pregunta Según el grafico obreros versus el tiempo que demoran en construir una casa
%del tipo M se puede afirmar correctamente que:
%\begin{center}
%\includegraphics[width=0.5\textwidth]{example-image-b} % Placeholder for graph from page 9
%\end{center}
%\begin{vertical}
%\alternativa Dos trabajadores construyen una casa del tipo M en un año
%\alternativa Tres trabajadores construyen una casa del tipo M en cinco meses
%\alternativa $b$ trabajadores construyen más casas del tipo M que $c$ trabajadores en un año
%\alternativa $(c-b)$ trabajadores construyen una casa del tipo M en ocho meses
%\alternativa $d$ trabajadores construyen dos casas del tipo M en un año
%\end{vertical}

\pregunta La mitad de una parcela de $10.000 \text{m}^2$, está dividida en
dos partes que están en la razón 1: 4. La parte menor será utilizada para
cultivo, ¿cuántos metros cuadrados serán usados para este fin?
\begin{vertical}
\alternativa 625
\alternativa 2.000
\alternativa 400
\alternativa 1.250
\alternativa 1.000
\end{vertical}

\pregunta Entre tres hermanos compran un número de rifa que cuesta \$ 1.000.
Juan aporta con \$ 240, Luis con \$ 360 y Rosa aporta el resto. El premio es
de \$ 60.000 Deciden, en caso de ganarlo repartirlo en forma directamente
proporcional al aporte de cada uno, ¿Qué cantidad de dinero le correspondería a Rosa?
\begin{vertical}
\alternativa \$ 30.000
\alternativa \$ 18.000
\alternativa \$ 24.000
\alternativa \$ 20.000
\alternativa \$ 40.000
\end{vertical}

\pregunta Don Julio tiene 42 años de edad y Rubén 18, ¿en qué razón están las edades de Rubén y don Julio?
\begin{vertical}
\alternativa 3:4
\alternativa 7:3
\alternativa 7:4
\alternativa 3:7
\alternativa 3:8
\end{vertical}

\pregunta Sea la proporción $3n:4=n:x$. Entonces, $x=$
\begin{vertical}
\alternativa $1,\overline{3}$
\alternativa $1,\overline{3} n$
\alternativa $0,75$
\alternativa $0,75 n$
\alternativa $0,5$
\end{vertical}

\pregunta Una docena de botones cuesta \$ 240. ¿Cuánto hay que pagar si se compran 54 botones?
\begin{vertical}
\alternativa \$ 648
\alternativa \$ 864
\alternativa \$ 1.080
\alternativa \$ 1.188
\alternativa \$ 1.296
\end{vertical}

\pregunta A una fiesta asisten 12 hombres. Si asistieron mujeres y hombres en la razón 2:3, respectivamente, ¿cuántas personas asistieron a la fiesta?
\begin{vertical}
\alternativa 8
\alternativa 16
\alternativa 18
\alternativa 20
\alternativa 24
\end{vertical}

\pregunta La diferencia entre los números es 48 y están en la razón 5: 9. ¿Cuál es el menor de ellos?
\begin{vertical}
\alternativa 5
\alternativa 9
\alternativa 12
\alternativa 60
\alternativa 108
\end{vertical}

\pregunta 4,2 horas equivalen a
\begin{vertical}
\alternativa 4 horas y 2 minutos.
\alternativa 4 horas y 12 minutos.
\alternativa 4 horas y 16 minutos.
\alternativa 4 horas y 20 minutos.
\alternativa ninguna de las anteriores.
\end{vertical}

\pregunta Con \$ 4.000 se pueden comprar $x$ kilogramos de dulce. ¿Cuántos kilogramos de dulce se pueden comprar con \$ 10.000?
\begin{vertical}
\alternativa $25x$
\alternativa $2,5x$
\alternativa $2,25x$
\alternativa $1,25x$
\alternativa $0,25x$
\end{vertical}

\pregunta Tres kilogramos de papas cuestan $m$ pesos y 6 kilogramos de papas cuestan \$ $(m+300)$. ¿Cuánto vale un kilogramo de papas?
\begin{vertical}
\alternativa \$ 100
\alternativa \$ 300
\alternativa \$ 500
\alternativa \$ 600
\alternativa \$ 1.000
\end{vertical}

\pregunta ¿Qué gráfico(s) representa (n) mejor a dos cantidades directamente proporcionales?
\begin{centrado}
(I) \includegraphics[width=0.2\textwidth]{example-image-a} \\
(II) \includegraphics[width=0.2\textwidth]{example-image-b} \\
(III) \includegraphics[width=0.2\textwidth]{example-image-c}
\end{centrado}
\begin{vertical}
\alternativa Solo I
\alternativa Solo II
\alternativa Solo III
\alternativa Solo I y II
\alternativa I, II y III
\end{vertical}

\pregunta De acuerdo al gráfico adjunto, de dos cantidades inversamente proporcionales, el valor de $a$ es
\begin{centrado}
\includegraphics[width=0.3\textwidth]{example-image-a}
\end{centrado}
\begin{vertical}
\alternativa $\dfrac{1}{3}$
\alternativa $\dfrac{2}{3}$
\alternativa 2
\alternativa 3
\alternativa 6
\end{vertical}

\pregunta Una secretaria digita en un computador una página de 54 líneas a doble espacio. ¿Cuántas líneas escribirá en la misma página a triple espacio?
\begin{vertical}
\alternativa 32
\alternativa 33
\alternativa 35
\alternativa 36
\alternativa 81
\end{vertical}

\pregunta Se tiene que limpiar una siembra de ají en una semana, para lo cual necesitan 19 obreros con jornada normal de trabajo (8 horas). Si sólo se dispone de 16 hombres, ¿cuántas horas diarias tendrán que trabajar?
\begin{vertical}
\alternativa 6,7 horas
\alternativa 9 horas
\alternativa 9,3 horas
\alternativa 9,5 horas
\alternativa 12 horas
\end{vertical}

\pregunta A un cordel de 2,4 metros de longitud, se le hacen dos marcas de modo que éste queda dividido en tres partes cuyas longitudes quedan en la razón 3:4:5. ¿Cuál es la longitud del segmento mayor?
\begin{vertical}
\alternativa 60 cm
\alternativa 80 cm
\alternativa 100 cm
\alternativa 120 cm
\alternativa 140 cm
\end{vertical}

\pregunta En pintar los dos tercios de una pared se ocupa un quinto de un tarro de pintura. ¿Qué parte del tarro se ocupará en pintar toda la pared?
\begin{vertical}
\alternativa Diez tercios
\alternativa Dos quinceavos
\alternativa Dos cuarenta y cincoavos
\alternativa Tres quintos
\alternativa Tres décimos
\end{vertical}

\pregunta En un corredor hay 12 hileras de baldosas de 0,20 m de lado. ¿Cuántas corridas de baldosas de 0,15 m por lado podrían colocarse?
\begin{vertical}
\alternativa 12
\alternativa 13
\alternativa 14
\alternativa 15
\alternativa 16
\end{vertical}

\pregunta Con un jarro de jugo se pueden llenar 36 vasos. ¿Cuántos vasos de la misma capacidad se podrán servir, si sólo son llenados hasta tres cuartos de su capacidad?
\begin{vertical}
\alternativa 27
\alternativa 35
\alternativa 45
\alternativa 48
\alternativa 50
\end{vertical}

\pregunta Jorge tiene 10 años, Andrés 15 y Pedro 5 años. Si se reparten cierta suma de dinero en proporción directa a sus edades, recibiendo Andrés \$ 1.500, ¿cuánto dinero se repartió?
\begin{vertical}
\alternativa \$ 4.000
\alternativa \$ 3.000
\alternativa \$ 2.800
\alternativa \$ 2.500
\alternativa \$ 2.000
\end{vertical}

\pregunta Si en $H$ horas se llena la cuarta parte de un estanque, entonces ¿en cuánto tiempo se llenará la tercera parte del estanque?
\begin{vertical}
\alternativa $\dfrac{H}{12}$
\alternativa $\dfrac{3}{4}H$
\alternativa $\dfrac{4}{3}H$
\alternativa $\dfrac{7}{12}H$
\alternativa H
\end{vertical}

\pregunta ¿Qué número debe sumarse a 7 y sustraerse de 3 para obtener dos números cuya razón sea 3: 1?
\begin{vertical}
\alternativa $\dfrac{1}{2}$
\alternativa 1
\alternativa 2
\alternativa $-\dfrac{1}{2}$
\alternativa -2
\end{vertical}

\pregunta En la fórmula $\dfrac{A\cdot B}{C}=K$; $A$, $B$ y $C$ son variables. Si $K$ es una constante, ¿cuál(es) de las siguientes afirmaciones es (son) verdadera(s)?
\begin{verticali}
\alternativa $A$ y $C$ son directamente proporcionales.
\alternativa $A$ y $B$ son inversamente proporcionales.
\alternativa $B$ y $C$ son directamente proporcionales.
\end{verticali}
\begin{vertical}
\alternativa Solo I
\alternativa Solo I y II
\alternativa Solo I y III
\alternativa I, II y III
\alternativa Ninguna de ellas
\end{vertical}

\end{preguntas}


\section{Logaritmos}

\begin{preguntas}
\pregunta Si $\log_{2}8=x,$ entonces $x=$
\begin{vertical}
\alternativa -3
\alternativa $2\sqrt{2}$
\alternativa 3
\alternativa 4
\alternativa 5
\end{vertical}

\pregunta Si $\log_{3}x=-2,$ entonces $x=$
\begin{vertical}
\alternativa -9
\alternativa -6
\alternativa $0,\overline{1}$
\alternativa $0,\overline{3}$
\alternativa 9
\end{vertical}

\pregunta $\log_{2}\left(0,25\right)+\log_{3}9=$
\begin{vertical}
\alternativa -1
\alternativa 0
\alternativa 1
\alternativa 3
\alternativa 4
\end{vertical}

\pregunta $\log_{3}\sqrt{0,\overline{1}}=$
\begin{vertical}
\alternativa -1
\alternativa 1
\alternativa 2
\alternativa -2
\alternativa $\dfrac{2}{3}$
\end{vertical}

\pregunta Si a es un número real mayor que uno, entonces $\log_{a}\left(\dfrac{\sqrt[3]{a^{2}}}{\sqrt{a}}\right)=$
\begin{vertical}
\alternativa -6
\alternativa 6
\alternativa $-\dfrac{1}{6}$
\alternativa $\dfrac{1}{6}$
\alternativa $\dfrac{7}{6}$
\end{vertical}

\pregunta Si $\log\left(x+1\right)=2$, entonces $x=$
\begin{vertical}
\alternativa 19
\alternativa 21
\alternativa 99
\alternativa 101
\alternativa 1023
\end{vertical}

\pregunta Si $\log\left(x+2\right)=1$ entonces $\log_{2}x=$
\begin{vertical}
\alternativa 2
\alternativa 3
\alternativa 4
\alternativa 0,25
\alternativa 0,125
\end{vertical}

\pregunta Sean $P=\log_{2}\sqrt[3]{4},$ $Q=\log_{4}\sqrt[3]{4} \text{ y } R=\log_{8}\sqrt[3]{4},$ ¿cuál(es) de las siguientes afirmaciones es (son) verdadera(s)?
\begin{verticali}
\alternativa $Q=\dfrac{P}{2}$
\alternativa $R=\dfrac{P}{3}$
\alternativa $PQ=R$
\end{verticali}
\begin{vertical}
\alternativa Solo I
\alternativa Solo II
\alternativa Solo I y II
\alternativa Solo II y III
\alternativa I, II y III
\end{vertical}

\pregunta $\log_{5}\left(\dfrac{5-x}{2}\right)=2,$ entonces $x=$
\begin{vertical}
\alternativa -45
\alternativa -5
\alternativa 25
\alternativa 55
\alternativa $-\dfrac{15}{2}$
\end{vertical}

\pregunta Si $p=\log_{4}\sqrt{2}$, $4=\log_{q}16 \text{ y } 2=\log_{4}r,$ entonces ¿cuál(es) de las siguientes afirmaciones es (son) verdadera(s)?
\begin{verticali}
\alternativa $pr=2q$
\alternativa $pqr=8$
\alternativa $r^{p}=q$
\end{verticali}
\begin{vertical}
\alternativa Solo I
\alternativa Solo II
\alternativa Solo I y II
\alternativa Solo II y III
\alternativa I, II y III
\end{vertical}

\pregunta $\log 2+\log 8-\log 4=$
\begin{vertical}
\alternativa $\log 4$
\alternativa $\log 6$
\alternativa $\log 8$
\alternativa $\log 12$
\alternativa $\log\left(\dfrac{5}{2}\right)$
\end{vertical}

\pregunta Si $x>1$, $\log\left(x+1\right)+\log\left(x-1\right)=$
\begin{vertical}
\alternativa $2 \log x$
\alternativa $2 \log\left(x-1\right)$
\alternativa $2 \log x-1$
\alternativa $\log\left(x^{2}-1\right)$
\alternativa $\log x+\log 2$
\end{vertical}

\pregunta $\log_{2}\left(\log_{9}\left(\log_{5}125\right)\right)=$
\begin{vertical}
\alternativa 2
\alternativa -2
\alternativa 1
\alternativa -1
\alternativa 0
\end{vertical}

\pregunta $\log_{2}\left(\log_{4}\left(\log_{2}\sqrt[3]{4^{6}}\right)\right)=$
\begin{vertical}
\alternativa -1
\alternativa 1
\alternativa 0
\alternativa 2
\alternativa $\log 2$
\end{vertical}

\pregunta Si $\log\sqrt{m}=0,24 \text{ y } \log n^{3}=0.69$, entonces $\log\left(\dfrac{m}{n}\right)=$
\begin{vertical}
\alternativa -0,11
\alternativa 0,16
\alternativa 0,25
\alternativa 0,35
\alternativa 0,71
\end{vertical}

\pregunta ¿Cuál (es) de las siguientes expresiones es (son) equivalente(s) a la expresión: $\log\left(\dfrac{ba^{2}}{c^{2}}\right)?$
\begin{verticali}
\alternativa $2 \log a+\log b-2 \log c$
\alternativa $\log b+2 \log\left(\dfrac{a}{c}\right)$
\alternativa $2 \log\left(ab\right)-2 \log c$
\end{verticali}
\begin{vertical}
\alternativa Solo I
\alternativa Solo II
\alternativa Solo I y II
\alternativa Solo I y III
\alternativa I, II y III
\end{vertical}

\pregunta Si a, b y c son números reales positivos, entonces $\log a-\log b-2 \log c=$
\begin{vertical}
\alternativa $\log\left(\dfrac{a}{bc^{2}}\right)$
\alternativa $\log\left(\dfrac{ab}{c^{2}}\right)$
\alternativa $\log\left(\dfrac{ac^{2}}{b}\right)$
\alternativa $\log\left(\dfrac{c^{2}}{ab}\right)$
\alternativa $\log\left(\dfrac{b}{ac^{2}}\right)$
\end{vertical}

\pregunta ¿Cuál (es) de las siguientes afirmaciones es (son) verdadera(s)?
\begin{verticali}
\alternativa $\log_{2}\left(\dfrac{1}{4}\right)=-2$
\alternativa Si $\log_{x}25=2$ entonces $x=5$
\alternativa Si $\log_{4}x=8,$ entonces $x=32$
\end{verticali}
\begin{vertical}
\alternativa Solo I
\alternativa Solo II
\alternativa Solo I y II
\alternativa Solo I y III
\alternativa I, II y III
\end{vertical}

\pregunta Si a y b son números positivos, se puede determinar que $a=b^{2}$, si:
\begin{verticaln}
\alternativa $\log a=2 \log b$
\alternativa $\log\left(\dfrac{a}{b^{2}}\right)=0$
\end{verticaln}
\begin{vertical}
\alternativa (1) por sí sola
\alternativa (2) por sí sola
\alternativa Ambas juntas, (1) y (2)
\alternativa Cada una por sí sola, (1) ó (2)
\alternativa Se requiere información adicional
\end{vertical}

\pregunta $\log\left(\dfrac{\sqrt{6}+3}{\sqrt{2}+\sqrt{3}}\right)=$
\begin{vertical}
\alternativa $\dfrac{1}{2}\log 3$
\alternativa $\log 3$
\alternativa $2 \log 3$
\alternativa $\log 6$
\alternativa $\log 2$
\end{vertical}

\pregunta Si $3 \log a=2 \log b$, ¿cuál(es) de las siguientes afirmaciones es (son) verdadera(s)?
\begin{verticali}
\alternativa $b=a\sqrt{a}$
\alternativa $\log_{a}b=\dfrac{2}{3}$
\alternativa $\log\left(\dfrac{a^{3}}{b^{2}}\right)=0$
\end{verticali}
\begin{vertical}
\alternativa Solo I
\alternativa Solo III
\alternativa Solo I y II
\alternativa Solo I y III
\alternativa I, II y III
\end{vertical}

\pregunta $\log x^{2}+\log\left(2xy\right)+\log y^{2}=$
\begin{vertical}
\alternativa $3\left(\log x+\log y\right)+\log 100$
\alternativa $\log\left(\dfrac{xy}{100}\right)$
\alternativa $2 \log\left(x+y\right)$
\alternativa $\log\left(\dfrac{xy}{3}\right)$
\alternativa $3 \log\left(xy\right)+\log 2$
\end{vertical}

\pregunta ¿Cuál (es) de las siguientes afirmaciones es (son) verdadera(s)?
\begin{verticali}
\alternativa $\log\left(0,5\right)<0$
\alternativa $\left(\log 2^{-3}\right)\cdot\left(\log 2^{3}\right)\ge0$
\alternativa $\log 2\cdot \log\left(0,3\right)<0$
\end{verticali}
\begin{vertical}
\alternativa Solo I
\alternativa Solo II
\alternativa Solo I y II
\alternativa Solo I y III
\alternativa I, II, III
\end{vertical}

\pregunta Si $\log\left(\dfrac{3}{2}\right)=0,18$ ¿cuál (es) de las siguientes igualdades es (son) verdadera(s)?
\begin{verticali}
\alternativa $\log\left(\dfrac{9}{4}\right)=0,36$
\alternativa $\log\left(0,\overline{6}\right)=\dfrac{50}{9}$
\alternativa $\log\sqrt{1,5}=0,9$
\end{verticali}
\begin{vertical}
\alternativa Solo I
\alternativa Solo II
\alternativa Solo I y Il
\alternativa Solo I y III
\alternativa Ninguna de ellas.
\end{vertical}

\pregunta Si $\log 2=a,$ ¿cuál (es) de las siguientes igualdades es (son) verdadera(s)?
\begin{verticali}
\alternativa $\log\left(0,25\right)=-2a$
\alternativa $\log 8=4a$
\alternativa $\log\left(0,5\right)=\dfrac{1}{a}$
\end{verticali}
\begin{vertical}
\alternativa Solo I
\alternativa Solo II
\alternativa Solo I y II
\alternativa Solo I y III
\alternativa I, II, III
\end{vertical}

\pregunta Si $x>y>0,$ ¿Cuál (es) de las siguientes expresiones es (son) equivalente(s) a $\log\left(x^{2}-y^{2}\right)$?
\begin{verticali}
\alternativa $2 \log x-2 \log y$
\alternativa $\log\left(x+y\right)+\log\left(x-y\right)$
\alternativa $\dfrac{\log x^{2}}{\log y^{2}}$
\end{verticali}
\begin{vertical}
\alternativa Solo I
\alternativa Solo II
\alternativa Solo I y III
\alternativa Solo II y III
\alternativa I, II, III
\end{vertical}

\pregunta Sia, byc son números reales positivos, la expresión: $2 \log a+2 \log b-2 \log c \text{ es equivalente a}$
\begin{verticali}
\alternativa $\log\left(\left(\dfrac{ab}{c}\right)^{2}\right)$
\alternativa $2 \log\left(\dfrac{ab}{c}\right)$
\alternativa $\log\left(\left(ab\right)^{2}\right)-\log c^{2}$
\end{verticali}
\begin{vertical}
\alternativa Solo I y II
\alternativa Solo III
\alternativa Solo I y III
\alternativa Solo II y III
\alternativa I, II, III
\end{vertical}

\pregunta Si $\log 2=u \text{ y } \log 3=v$ entonces $\log 18$ en términos de u y v es:
\begin{vertical}
\alternativa $2u+v$
\alternativa $2v+u$
\alternativa $uv^{2}$
\alternativa $3v+u$
\alternativa $2uv$
\end{vertical}

\pregunta Si $\log m^{2}=n \text{ y } \dfrac{1}{2}\log\sqrt{p}=q$, entonces $\log\left(\dfrac{m}{p}\right)=$
\begin{vertical}
\alternativa $\dfrac{n}{2}+4q$
\alternativa $\dfrac{n}{2}-4q$
\alternativa $\dfrac{n}{2}-\dfrac{q}{2}$
\alternativa $\dfrac{n}{2}-\dfrac{q}{4}$
\alternativa $2n-4q$
\end{vertical}

\pregunta Si $\log_{a}b=3,$ entonces $\log_{b^{2}}a=$
\begin{vertical}
\alternativa 6
\alternativa $\dfrac{3}{2}$
\alternativa $\dfrac{1}{2}$
\alternativa $\dfrac{1}{6}$
\alternativa $\dfrac{1}{3}$
\end{vertical}

\pregunta La masa de un material radioactivo medida en kilogramos, está dada por la expresión $m\left(t\right)=4\cdot\left(0,2\right)^{t},$ donde t es el tiempo medido en años. ¿Cuántos años deben transcurrir para que la masa del material quede reducida a dos kilogramos?
\begin{vertical}
\alternativa $\log 2,5$
\alternativa $\dfrac{\log 5}{\log 2}$
\alternativa $\log 5-\log 2$
\alternativa $\dfrac{\log 2}{1-\log 2}$
\alternativa Todas las anteriores.
\end{vertical}

\pregunta Si $\log 2=a \text{ y } \log 3=b$, ¿cuál de las siguientes afirmaciones es FALSA?
\begin{vertical}
\alternativa $\log 144=4a+2b$
\alternativa $\log\left(4,5\right)=2b-a$
\alternativa $\log\left(0,8\right)=3a-2b$
\alternativa $\log\sqrt{12}=a+\dfrac{b}{2}$
\alternativa $\log\left(1,5\right)=\dfrac{b}{a}$
\end{vertical}

\pregunta ¿Cuál de las siguientes afirmaciones es (son) verdadera(s)?
\begin{verticali}
\alternativa $2 \log\sqrt{2}=\dfrac{1}{4}\log 2$
\alternativa $2 \log\left(\sqrt{2}-1\right)=\log\left(3-2\sqrt{2}\right)$
\alternativa $\log\left(\sqrt{3}+\sqrt{2}\right)+\log\left(\sqrt{3}-\sqrt{2}\right)=0$
\end{verticali}
\begin{vertical}
\alternativa Solo I
\alternativa Solo III
\alternativa Solo I y Il
\alternativa Solo II y III
\alternativa I, II y III
\end{vertical}

\pregunta Si $\log x^{3}=2$, ¿cuál de las siguientes afirmaciones es (son) verdadera(s)?
\begin{verticali}
\alternativa $x=\sqrt[3]{100}$
\alternativa $\log x^{12}=8$
\alternativa $\log\left(\dfrac{10}{x}\right)=\dfrac{1}{3}$
\end{verticali}
\begin{vertical}
\alternativa Solo I
\alternativa Solo III
\alternativa Solo I y Il
\alternativa Solo I y III
\alternativa I, II y III
\end{vertical}

\pregunta Si $\log_{2}5=a,$ entonces $\log 2=$
\begin{vertical}
\alternativa $\dfrac{1}{a}$
\alternativa $\dfrac{2}{a}$
\alternativa $\dfrac{a+1}{a}$
\alternativa $\dfrac{1}{a-1}$
\alternativa $\dfrac{1}{a+1}$
\end{vertical}

\pregunta Si a es un número real positivo, se puede determinar $\log a$ sabiendo que:
\begin{verticaln}
\alternativa $\log\left(10a\right)-\log a=1$
\alternativa $\log\left(10a\right)+\log a=3$
\end{verticaln}
\begin{vertical}
\alternativa (1) por sí sola
\alternativa (2) por sí sola
\alternativa Ambas juntas, (1) y (2)
\alternativa Cada una por sí sola, (1) ó (2)
\alternativa Se requiere información adicional
\end{vertical}

\pregunta Si a y b son números reales positivos, se puede determinar ab sabiendo:
\begin{verticaln}
\alternativa $\log a+\log b=1$
\alternativa $\log a+\log b=2-\log\left(ab\right)$
\end{verticaln}
\begin{vertical}
\alternativa (1) por sí sola
\alternativa (2) por sí sola
\alternativa Ambas juntas, (1) y (2)
\alternativa Cada una por sí sola, (1) ó (2)
\alternativa Se requiere información adicional
\end{vertical}
\end{preguntas}

\end{document}


