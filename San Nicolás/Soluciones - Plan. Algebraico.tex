\providecommand{\tituloDocumento}{Solucionario}
\providecommand{\subtituloDocumento}{Planteamiento Algebraico}
\documentclass{sn-guia}

\begin{document} 
\raggedright
\setcounter{section}{11}
\subsection{Estrategias para resolver problemas de planteamiento}
\begin{enumerate}
    \item \begin{equation*}
        [4(w-1) + (w+2)] -3w = 2w -2 
    \end{equation*}
    \item \begin{equation*}
        p^2 - 5p^2 = -4p^2 
    \end{equation*}
    \item \begin{equation*}
        \frac{m+n}{2} - \frac{m-n}{2} = n
    \end{equation*}
    \item \begin{align*}
        x + 40 &= 2\left( x-1 \right) \\
        x      &= 42
    \end{align*}
    El sucesor de 42 es 43.
    \item \begin{align*}
        4x - 35 &= 35 - 3x \\
        x &= 10
    \end{align*}
    \item \begin{align*}
        \left( 4x+8 \right) + \left( x \right) &= 28 \\
        x &= 4
    \end{align*}
    \item \begin{align*}
        \left( x \right) + \left( 2x+30000 \right) &= 147000 \\
        x &= 39000
    \end{align*}
    Iván tiene $2\cdot\$ 39000 + \$ 30000 = \$ 108000$.
    \item \begin{equation*}
        a^2 - \left( a-t \right)^2 = 2at - t^2
    \end{equation*}
\end{enumerate}

\subsection{Problemas con fracciones}
\begin{enumerate}
    \item \begin{equation*}
        90 - \frac{1}{3}\cdot 90 - \frac{5}{9}\cdot 90 = 10
    \end{equation*}
    \item \begin{equation*}
        \left( M - \frac{3}{4}M \right) = \frac{3}{4}M
    \end{equation*}
    \item \begin{equation*}
        \left( 5B - \frac{5B}{6} \right): 25 = \frac{B}{6}
    \end{equation*}
    \item \begin{align*}
        \frac{2}{3}x - 5 &= 25 \\
        x &= 45
    \end{align*}
    La quinta parte de 45 es 9.
    \item Andrea y Marcela juntas aportaron 
    \begin{equation*}
        \frac{1}{3} + \frac{2}{5} = \frac{11}{15} 
    \end{equation*}
    del capital inicial. Eso significa que Pedro con sus $\$200000$ aporto $\frac{4}{15}$ del capital. Así, 
    el capital inicial total es
    \begin{equation*}
        \dfrac{15}{4}\cdot \$200000 = \$750000
    \end{equation*}
    pesos. 
    La diferencia positiva entre el aporte de Andrea y Marcela es $\frac{1}{15}$ del capital, lo cual 
    corresponde a 50000 pesos.
    \item \begin{align*}
        \frac{B}{4} + \frac{1}{2}\left( B - \frac{B}{4} \right) + 21 &= B\\
        B &= 56
    \end{align*}
    Al segundo día recorrió $56\cdot3/8 = 21$ kilómetros.
\end{enumerate}
\subsection{Problemas de dígitos}
\begin{enumerate}
    \item \begin{equation*}
        3407 = 3\cdot 10^3 + 4\cdot 10^2 + 7\cdot 10^0
    \end{equation*}
    \item \begin{equation*}
        867,93 = 8\cdot 10^2 + 6\cdot 10^1 + 7\cdot 10^0 + 9\cdot 10^{-1} + 3\cdot 10^{-2}
    \end{equation*}
    \item \begin{equation*}
        4000 + 700 + 3 + 0,2 + 0,06
    \end{equation*}
    \item \begin{equation*}
        \left( 10\cdot z + w \right) - 2 
    \end{equation*}
    \item 
    \begin{equation*}
        100\cdot c + 10\cdot 2c + 1\cdot \left( 6 -c -2c \right) = 117c + 6
    \end{equation*}
    \item \begin{align*}
        \frac{10\cdot(x+5) + 1\cdot x}{(x+5) + x} &= 8 \\
        x &= 2
    \end{align*}
    Así, el número buscado es
    \begin{equation*}
        10\cdot(2+5) + 2 = 72 \,,
    \end{equation*}
    y su sucesor es 73.
    \item \begin{align*}
        x + 2m + m &= 20 \\
        x &= 20 - 3m
    \end{align*}
\end{enumerate}
\subsection{Problemas de edades}
\begin{enumerate}
    \item \begin{align*}
        3x - (x+8) &= 28 \\
        x &= 18
    \end{align*}
    \item \hphantom{}
    \begin{table}[h]
        \centering
        \begin{tblr}{vlines,hlines}
            Edad pasada & Edad actual & Edad futura \\
            $c-a-b$ & $c-a$ & $c$ \\
        \end{tblr}        
    \end{table}
    \item \begin{equation*}
        3x = 63 \quad \Rightarrow \quad \frac{x}{3} + 1 = 8
    \end{equation*}
    \item \hphantom{}
    \begin{table}[h]
        \centering
        \begin{tblr}{vlines,hlines}
            Edad actual & Edad futura $(+y)$ \\
            $x-y$ & $x$ \\
        \end{tblr}        
    \end{table}
    \item \begin{align*}
        x + (3x - 37) &= 91 \\
        x &= 32
    \end{align*}
    Si Mauricio hubiera nacido 6 años antes, tendría 38 años.
    \item \hphantom{}
    \begin{table}[h]
        \centering
        \begin{tblr}{vlines,hlines}
            Edad pasada $(-12)$ & Edad actual \\
            $(p-12)=4\cdot(m-12)$ & $p=3m$\\
        \end{tblr}        
    \end{table}\\
    Esto implica que
    \begin{align*}
        3m-12 &= 4m - 48 \\
        m &= 36
    \end{align*}
    y por lo tanto María en el futuro tendría $36 +12 = 48$ años.
    \item \begin{align*}
        x + z &= 4(y + z) \\
        \frac{x-4y}{3} &= z
    \end{align*}
\end{enumerate}
\subsection{Problemas de trabajo simultáneo}
\begin{enumerate}
    \item \begin{align*}
        \frac{1}{x} &= \frac{1}{2} + \frac{1}{3} \\
        x &= \frac{6}{5}
    \end{align*}
    Las máquinas se demoraron $\frac{6}{5}$ horas o $\frac{6}{5}\cdot 60 = 72$ minutos. 
\item El desagüe realiza trabajo negativo, por lo tanto se cumple que
\begin{align*}
    \frac{1}{x} &= \frac{1}{2} + \frac{1}{6} - \frac{1}{3}\\
    x &= 3 \,,
\end{align*}
Así, el estanque se llenara en 3 horas.
\item \begin{align*}
    \frac{1}{x} &= \frac{1}{15} + \frac{1}{3x} \\
    \frac{1}{x} &= \frac{x+5}{15x} \\
    x &= 10 
\end{align*}
Si ambos trabajan juntos se demoraran 10 días, y si Nelson trabaja solo le tomara $3\cdot 10=30$ días.
\end{enumerate}
\subsection{Problemas de móviles}
\begin{enumerate}
    \item Ambos móviles se acercan a una velocidad de $220$ $(120+100)$ [km/h], y las 12 pm han transcurrido 4 
    horas desde que partieron. Por lo tanto, han recorrido $220\cdot 4=880$ [km] y los separan $1120-880=240$ [km]. 
    \item En una hora, un móvil va a recorrer 20 kilómetros más que el otro. Por lo tanto, en 3 horas los separarán 
    $20\cdot 3 = 60$ kilómetros.
    \item \begin{align*}
        v &= x + (x-10) \\
        v &= 2x - 10
    \end{align*}
    La velocidad $(v)$ por el tiempo (5 horas) es igual a la distancia recorrida (1050 kilómetros), continua que
    \begin{align*}
        (2x-10)\cdot 5 &= 1050 \\
        x = 110 \,.
    \end{align*}
    Concluyendo que los buses se movían a 110 [km/h] y $110-10=100$ [km/h] respectivamente.
\end{enumerate}

\subsection{Problemas de mezclas}
\begin{enumerate}
    \item 
    \begin{align*}
        x\cdot 100 + (85-x)\cdot 500 &= 22500 \\ 
        42500 - 22500 &= 500x - 100x\\
        50 &= x
    \end{align*}
    Se ahorró $50\cdot \$ 100=\$ 5000$. 
    \item \begin{align*}
        (1000-x)\cdot 2000 + x\cdot 3000 &= 2650000\\
        x &= 650
    \end{align*}
    \item \begin{align*}
        32x + 18(x+700) &= 52600 \\
        x &= 800
    \end{align*}
    Cada bebida costo $\$800$ y cada pizza costo $\$800 + \$700 = \$1500$.
    \item \begin{align*}
        (x+70)\cdot 4000 + x\cdot 1500 &= 555000 \\
        x &= 50
    \end{align*}
    Así, asistieron al bingo $(50+70) + (50) = 170$ personas.   
\end{enumerate}


\end{document}