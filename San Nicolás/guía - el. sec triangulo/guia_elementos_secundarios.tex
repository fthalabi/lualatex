\providecommand{\tituloDocumento}{Guía}
\providecommand{\subtituloDocumento}{Elementos secundarios del triángulo}
\documentclass{sn-guia}

\begin{document}
 
\raggedright
\begin{tcbraster}[enhanced,raster columns=2,raster width=\linewidth,raster column skip=3pt,raster force size=false]
    \begin{caja}[title={\sffamily\scshape\bfseries Nombre},height=30pt,add to width=4cm]
    \end{caja}
    \begin{caja}[title={\sffamily\scshape\bfseries Curso},height=30pt,add to width=-4cm]
    \end{caja}    
    \begin{caja}[title={\sffamily\scshape\bfseries Fecha},height=30pt]
    \end{caja}                
    \begin{caja}[title={\sffamily\scshape\bfseries Profesor},height=30pt]
    \end{caja}
\end{tcbraster}
\vspace*{5mm}


\section*{Transversal de gravedad}
Es la recta que une un vértice con el punto medio del lado opuesto. Se denomina ta, tb y tc, donde el
subíndice indica el vértice por el cual pasa. Las tres transversales de gravedad se interceptan en un
mismo punto llamado Centro de Gravedad (o Baricentro).


\begin{tcolorbox}[blanker,sidebyside]
    \begin{center}
        \begin{tikzpicture}[ampersand replacement=\&, line width=1pt]
            \draw (0,0) coordinate[label=below:$A$] (A) -- (7,0) coordinate[label=below:$B$] (B) %
            -- ([turn]150:6) coordinate[label=above:$C$] (C) -- cycle; 
            \draw[name path=ta] (A) -- ($(B)!0.5!(C)$) node [above,pos=0.3] {$t_a$} node [right,yshift=3pt,pos=1] {$D$};
            \draw[name path=tb] (B) -- ($(C)!0.5!(A)$) node [above,pos=0.3] {$t_b$} node [left,pos=1] {$E$}; 
            \draw (C) -- ($(A)!0.5!(B)$) node [right,pos=0.3] {$t_c$} node [below,pos=1] {$F$};
            \draw[name intersections={of=ta and tb, by=G},fill] (G) circle (1pt) node [below=3pt,xshift=-3pt] {$G$};
        \end{tikzpicture}
    \end{center}
    \tcblower
    \begin{itemize}
        \item D, E, F: Puntos medios de los lados.
        \item $\overline{AD} = t_a\;$; $\overline{BE} = t_b\;$; $\overline{CF} = t_c\,$.
        \item $t_a \cap t_b \cap t_c = \left\{G\right\}$.
        \item $G$: Centro de Gravedad (o Baricentro).
    \end{itemize}
\end{tcolorbox}

\subsection*{Propiedad} 
El baricentro divide cada transversal de gravedad en dos segmentos que están en la razón 2:1. El 
segmento que va desde el centro al baricentro mide el doble que el segmento que va del baricentro 
al lado.

\section*{Altura}

\begin{tcolorbox}[blanker,sidebyside]
    \begin{center}
        \begin{tikzpicture}[ampersand replacement=\&, line width=1pt]
            \draw (0,0) coordinate[label=below:$A$] (A) -- (6,0) coordinate[label=below:$B$] (B) %
            -- ([turn]120:7) coordinate[label=above:$C$] (C) -- cycle; 
            \draw[name path=ha] (A) -- ($(B)!0.5!(C)$) node [above left,pos=0.4] {$h_a$} node [right,yshift=3pt,pos=1] {$D$};
            \draw[name path=hb] (B) -- ($(C)!0.5!(A)$) node [above right,pos=0.4] {$h_b$} node [left,pos=1] {$E$}; 
            \draw (C) -- ($(A)!0.5!(B)$) node [right,pos=0.4] {$h_c$} node [below,pos=1] {$F$};
            \draw[name intersections={of=ha and hb, by=H},fill] (H) circle (1pt) node [below right=3pt,xshift=-4pt,yshift=-2pt] {$H$};
        \end{tikzpicture}
        \end{center}
    \tcblower
    \begin{itemize}
        \item $\overline{AD} \perp \overline{BC}\;$ ; $\overline{BE} \perp \overline{CA}\;$ ; $\overline{CF} \perp \overline{AB}\,$.
        \item $\overline{AD} = h_a\;$; $\overline{BE} = h_b\;$; $\overline{CF} = h_c$.
        \item $h_a \cap h_b \cap h_c = \left\{H\right\}$.
        \item $H$: Ortocentro.
    \end{itemize}
\end{tcolorbox}


hola asdc oasdm awd

\end{document}