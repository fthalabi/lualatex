\providecommand{\tituloDocumento}{Evaluación}
\providecommand{\subtituloDocumento}{Operatoria algebraica y productos notables}
\documentclass{sn-guia}

\begin{document} 
\raggedright
\begin{tcbraster}[enhanced,raster columns=4,raster width=\linewidth,raster column skip=3pt,raster force size=false]
    \begin{caja}[title={\sffamily\scshape\bfseries Nombre},height=35pt,add to width=4.5cm]
    \end{caja}
    \begin{caja}[title={\sffamily\scshape\bfseries Curso},height=35pt,add to width=-1.5cm]
    \end{caja}    
    \begin{caja}[title={\sffamily\scshape\bfseries Puntaje},height=35pt,add to width=-1.5cm]
    \end{caja}
    \begin{caja}[title={\sffamily\scshape\bfseries Nota},height=35pt,add to width=-1.5cm]
    \end{caja}      
\end{tcbraster}
\vspace*{5mm}

\begin{preguntas}

    \pregunta ¿Cuál(es) de las siguientes expresiones al ser simplificada(s) 
    resulta(n) 1?
    \begin{tcenter} 
    \begin{alternativas*}[label=(\Roman*)]
        \item $\dfrac{2a+3}{3+2a}$
        \item $\dfrac{a^2-b^2}{(a-b)^2}$
        \item $\dfrac{(b-a)^2}{a^2+b^2-2ab}$
    \end{alternativas*}
    \end{tcenter}
    \begin{alternativas}[]
        \item Solo I
        \item Solo I y II
        \item Solo I y III
        \item I, II y III
    \end{alternativas}

    \pregunta ¿Cuál(es) de las siguientes igualdades es (son) verdadera(s)?
    \begin{tcenter}
        \begin{alternativas*}[label=(\Roman*)]
            \item $5x \cdot -x \cdot -x = -5x^3$
            \item $-4x \cdot 3x^2 = -12x^3$    
            \item $-3y \cdot -x \cdot -7xy = -21x^2y^2$
        \end{alternativas*}
    \end{tcenter}
    \begin{alternativas}[]
        \item Solo II
        \item Solo III
        \item Solo I y III
        \item Solo II y III
    \end{alternativas}

    \pregunta Si $t-7=8$, entonces la diferencia entre $t^2$ y $4^2$, en ese orden, es igual a
    \begin{alternativas}[]
        \item $-15$
        \item $209$
        \item $22$
        \item $121$
    \end{alternativas}

    \pregunta $101^2 + 100^2 - 99^2 =$
    \begin{alternativas}[]
        \item $102^2$
        \item $104^2$
        \item $10004$
        \item $10400$
    \end{alternativas}

    \pregunta $(3w-2)^2-2(2w-3)(2w+3)=$
    \begin{alternativas}[]
        \item $w^2-12w-14$
        \item $w^2-12w+22$
        \item $w^2-12w+13$
        \item $w^2-12w+14$
    \end{alternativas}

    \pregunta ¿Cuál de las siguientes expresiones es un factor de $k^2+k-6$?
    \begin{alternativas}[]
        \item $k+2$
        \item $k-6$
        \item $k-3$
        \item $k-2$
    \end{alternativas}

    \pregunta La expresión $\dfrac{xy-x}{y}\div \dfrac{ay-a}{y^2}$ es igual a
    \begin{alternativas}[]
        \item $\dfrac{a}{xy}$
        \item $\dfrac{ax}{y}$
        \item $\dfrac{xa(y-1)^2}{y^3}$
        \item $\dfrac{xy}{a}$
    \end{alternativas}

    \pregunta Si $x$ es distinto de: $a$, $-a$ y $0$, entonces 
    $\dfrac{x^2-a^2}{x^2-ax}\div\dfrac{x-a}{x+a}$ es igual a
    \begin{alternativas}[]
        \item $\dfrac{x(x-a)}{(x+a)^2}$
        \item $\dfrac{x-a}{x}$
        \item $\dfrac{x+a}{x}$
        \item $\dfrac{(x+a)^2}{x(x-a)}$
    \end{alternativas}
    
    \pregunta Si $a = \dfrac{1}{2x}$, $b=\dfrac{1}{4x}\;$ y $\;c=\dfrac{1}{6x}$, entonces 
    $x-(a+b+c)$ es
    \begin{alternativas}[]
        \item $\dfrac{12x-11}{12}$
        \item $\dfrac{12x^2-11}{12x}$
        \item $\dfrac{x-11}{12x}$
        \item Ninguna de las expresiones anteriores.
    \end{alternativas}

    \pregunta ¿Cuál de las siguientes expresiones es igual que $\left( a + (b+c) \right)\cdot\left( a + (b-c) \right)$?
    \begin{alternativas}
        \item $a^2 + b^2 - c^2$
        \item $a^2 + 2ab + b^2 - c^2$
        \item $a^2 + a^2b^2 + b^2 - c^2$
        \item $a^2 + (b-c)^2$
    \end{alternativas}    

    \pregunta Para $x \neq 0$, la expresión $1 + \dfrac{1}{x} + \dfrac{1}{x^2}$ es igual a
    \begin{alternativas}[]
        \item $\dfrac{x^2+x+1}{x^2}$
        \item $\dfrac{3}{1+x+x^2}$
        \item $1+\dfrac{2}{x^2}$
        \item $\dfrac{(x+1)^2}{x^2}$
    \end{alternativas}

    \pregunta La expresión $(a+1)^2 + (a+1)(a-3)$ se factoriza como el producto de dos factores, tal 
    que uno de ellos es $(a+1)$.
    ¿Cuál de las siguientes expresiones corresponde al otro factor de la expresión?
    \begin{alternativas}
        \item $\left(a^2 + 3a -2\right)$
        \item $\left(a-2\right)$
        \item $\left(a^2 -a -2\right)$
        \item $\left(2a -2\right)$
    \end{alternativas}

    \pregunta Si $P = x^2 + 4ax + a^2$, ¿cuál(es) de las siguientes expresiones se puede(n) factorizar
    como un cuadrado del binomio perfecto?
    \begin{tcenter}
        \begin{alternativas*}[label=(\Roman*)]
            \item $P + 3x^2$
            \item $P - a^2$
            \item $P -6ax$
        \end{alternativas*}             
    \end{tcenter}
    \begin{alternativas}[]
        \item Solo III
        \item Solo I y III
        \item Solo II y III
        \item I, II y III
    \end{alternativas}

    \pregunta Si $a+b = 8$ y $ab=10$, entonces el valor de $(a^2 + 6ab + b^2)$ es
    \begin{alternativas}[]
        \item 76
        \item 104
        \item 124
        \item Indeterminable con los datos dados.
    \end{alternativas}

    \pregunta Si $H=\sqrt{x+\sqrt{2x-1}}+\sqrt{x-\sqrt{2x-1}}$, con $x \geq 1$, ¿cuál de las 
    siguientes expresiones es igual a $H^2$?
    \begin{alternativas}[]
        \item $2x$
        \item $4x -2$
        \item $2x + 2\sqrt{x^2-2x-1}$
        \item $2x +\sqrt{x^2-2x-1}$
    \end{alternativas}

    \pregunta Si a y b son números reales positivos, $P=a^2+b^2$, $Q=(a+b)^2\;$ y
    $\;R = (a^3+b^3)/(a+b)$, ¿cuál de las siguientes relaciones es verdadera?
    \begin{alternativas}[]
        \item $R < P = Q$
        \item $R = P < Q$
        \item $R < P < Q$
        \item $P < Q < R$
    \end{alternativas}


    \pregunta Se tienen dos números reales positivos, tal que $\,x^2+y^2=6xy\,$, con $x>y$, 
    ¿cuál es el valor de la expresión $(x+y)/(x-y)$?
    \begin{alternativas}[]
        \item $2\sqrt{2}$
        \item $\sqrt{2}$
        \item $\dfrac{\sqrt{2}}{2}$
        \item $2$
    \end{alternativas} 

    \pregunta Dada la expresion $x^2y^2 + x^2y + xy + x$, ¿cuál(es) de las siguientes 
    expresiones es (son) factor(es) de ella?
    \begin{center} 
    \begin{alternativas*}[label=(\Roman*)]
        \item $xy+1$
        \item $x+1$
        \item $y+1$
    \end{alternativas*}
    \end{center}
    \begin{alternativas}[]
        \item Solo I
        \item Solo III
        \item Solo I y III
        \item Solo II y III
    \end{alternativas}

    \begin{tcolorbox}[boxrule=1pt,colback=white,leftrule=3mm,grow to left by=-1cm,
        grow to right by=-1cm, enlarge top by=5mm, enlarge bottom by=0mm]
        De las siguientes preguntas, debe responder solo dos a libre elección.
    \end{tcolorbox}

    \pregunta Si $x=\sqrt{2}$, entonces el valor de la expresión 
    $(x-2)^2(x-1)^2(x+1)^2(x+2)^2$ es
    \begin{alternativas}[]
        \item 5
        \item 4
        \item 3
        \item 2
    \end{alternativas}

    \pregunta $\dfrac{p^2-q^2}{pq}-\dfrac{pq-q^2}{pq-p^2} =$
    \begin{alternativas}[]
        \item $p^2$
        \item $q^2$
        \item $\dfrac{p}{q}$
        \item $\dfrac{pq-2q^2}{pq}$
    \end{alternativas}

    \pregunta Si $T \neq \pm 2$ y $T \neq 0$, entonces 
    $\dfrac{T-4+\dfrac{4}{T}}{T - \dfrac{4}{T}}$ es igual a
    \begin{alternativas}[]
        \item $-1$
        \item $4$
        \item $\dfrac{T+2}{T-2}$
        \item $1 - \dfrac{4}{T+2}$
    \end{alternativas}

    \pregunta Si $\left( n + \dfrac{1}{n} \right)^2 = 3$, entonces $n^3 + \dfrac{1}{n^3}$
    es igual a
    \begin{alternativas}[]
        \item 6
        \item 3
        \item 1
        \item 0
    \end{alternativas}
\end{preguntas}

\end{document}