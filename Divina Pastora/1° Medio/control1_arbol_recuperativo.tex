\providecommand{\curso}{Primero Medio}
\providecommand{\colegio}{Colegio Divina Pastora}
\providecommand{\tituloDocumento}{Control recuperativo}
\providecommand{\subtituloDocumento}{Reglas multiplicativa y aditiva de la probabilidad}

\documentclass{cdplf-prueba}

\begin{document}
%
\begin{tcbraster}[enhanced,raster columns=3,raster width=\linewidth,raster column skip=3pt,raster force size=false]
    \begin{caja}[title={\sffamily\scshape\bfseries Nombre},height=30pt,add to width=4cm]
    \end{caja}
    \begin{caja}[title={\sffamily\scshape\bfseries Puntaje},height=30pt,add to width=-2cm]
    \end{caja}
    \begin{caja}[title={\sffamily\scshape\bfseries Nota},height=30pt,add to width=-2cm]
    \end{caja}                    
\end{tcbraster}
%
\vspace*{10pt}
\begin{tcolorbox}[boxrule=1pt,colback=white,leftrule=3mm]
    \raggedright Resuelva el problema que se encuentra a continuación. Para esto, no olvide 
    incluir un desarrollo pertinente y la respuesta al enunciado en los espacios señalizados.        
\end{tcolorbox}
%
%\settasks{debug}
Una caja contiene 2 bolas rojas y 3 azules. Hallar la probabilidad de que si 
dos bolas se extraen aleatoriamente (sin reemplazo) (a) ambas sean azules, 
(b) ambas sean rojas, (c) una sea roja y la otra azul.
\begin{desarrollo}[height=11cm]
\end{desarrollo}
\begin{respuesta}[height=3cm]
\end{respuesta}
{\setlength{\parskip}{0pt}
Puntaje:\begin{itemize}[nosep,topsep=0pt]
    \item 2 puntos por dibujar un árbol que represente la situación expuesta en el enunciado.
    \item 1 punto por rellenar cada una de las ramas del árbol con su probabilidad asociada.
    \item 1 punto por cada pregunta contestada (3 en total).
\end{itemize}}
\end{document}