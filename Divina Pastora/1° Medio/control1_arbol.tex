\providecommand{\curso}{Primero Medio}
\providecommand{\colegio}{Colegio Divina Pastora}
\providecommand{\tituloDocumento}{Control 1}
\providecommand{\subtituloDocumento}{Reglas multiplicativa y aditiva de la probabilidad}

\documentclass{cdplf-prueba}

\begin{document}
%
\begin{tcbraster}[enhanced,raster columns=3,raster width=\linewidth,raster column skip=3pt,raster force size=false]
    \begin{caja}[title={\sffamily\scshape\bfseries Nombre},height=30pt,add to width=4cm]
    \end{caja}
    \begin{caja}[title={\sffamily\scshape\bfseries Puntaje},height=30pt,add to width=-2cm]
    \end{caja}
    \begin{caja}[title={\sffamily\scshape\bfseries Nota},height=30pt,add to width=-2cm]
    \end{caja}                    
\end{tcbraster}
%
\vspace*{10pt}
\begin{tcolorbox}[boxrule=1pt,colback=white,leftrule=3mm]
    \raggedright Resuelva el problema que se encuentra a continuación. Para esto, no olvide 
    incluir un desarrollo pertinente y la respuesta al enunciado en los espacios señalizados.        
\end{tcolorbox}
%
%\settasks{debug}
En México, las ventas de automóviles se distribuyen de la siguiente manera:
\begin{itemize}
    \item 20\% son Nissan
    \item 14\% son Volkswagen
    \item El resto es de otras marcas
\end{itemize}
Además, cuando un auto es Nissan, tiene una probabilidad del 73\% de ser robado; 
si es Volkswagen, tiene una probabilidad del 9\% de ser robado, y 18\% de 
ser robado si es de otras marcas. Entonces, ¿Cuál es la probabilidad de que 
se roben un auto?
\begin{desarrollo}[height=10cm]
\end{desarrollo}
\begin{respuesta}[height=2cm]
\end{respuesta}
\end{document}