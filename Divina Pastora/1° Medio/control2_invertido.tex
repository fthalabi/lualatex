\providecommand{\curso}{Primero Medio}
\providecommand{\colegio}{Colegio Divina Pastora}
\providecommand{\tituloDocumento}{Control 2}
\providecommand{\subtituloDocumento}{Probabilidad condicional}
 
\documentclass{cdplf-prueba}

\begin{document}
%
\begin{tcbraster}[enhanced,raster columns=3,raster width=\linewidth,raster column skip=3pt,raster force size=false]
    \begin{caja}[title={\sffamily\scshape\bfseries Nombre},height=30pt,add to width=4cm]
    \end{caja}
    \begin{caja}[title={\sffamily\scshape\bfseries Puntaje},height=30pt,add to width=-2cm]
    \end{caja}
    \begin{caja}[title={\sffamily\scshape\bfseries Nota},height=30pt,add to width=-2cm]
    \end{caja}                    
\end{tcbraster}
%
\vspace*{10pt}
\begin{tcolorbox}[boxrule=1pt,colback=white,leftrule=3mm]
    \raggedright Resuelva el problema que se encuentra a continuación. Para esto, no olvide 
    incluir un desarrollo pertinente y la respuesta al enunciado en los espacios señalizados.        
\end{tcolorbox}
%
%\settasks{debug}
Supongamos que la ciencia médica, ha desarrollado una prueba para 
el diagnóstico del cáncer, que detecta de igual manera si se tiene o no 
cáncer con un 95\% de exactitud. Si 0.5\% de la población
realmente tiene cáncer, calcular la probabilidad que un determinado individuo tenga
cáncer, si la prueba dice que tiene.

\begin{desarrollo}[height=14cm]
\end{desarrollo}
\begin{respuesta}[height=1.5cm]
\end{respuesta}
\end{document}