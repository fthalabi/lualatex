\providecommand{\curso}{Primero Medio}
\providecommand{\colegio}{Colegio Divina Pastora}
\providecommand{\tituloDocumento}{Control recuperativo}
\providecommand{\subtituloDocumento}{Probabilidad condicional}

\documentclass{cdplf-prueba}

\begin{document}
%
\begin{tcbraster}[enhanced,raster columns=3,raster width=\linewidth,raster column skip=3pt,raster force size=false]
    \begin{caja}[title={\sffamily\scshape\bfseries Nombre},height=30pt,add to width=4cm]
    \end{caja}
    \begin{caja}[title={\sffamily\scshape\bfseries Puntaje},height=30pt,add to width=-2cm]
    \end{caja}
    \begin{caja}[title={\sffamily\scshape\bfseries Nota},height=30pt,add to width=-2cm]
    \end{caja}                    
\end{tcbraster}
%
\vspace*{10pt}
\begin{tcolorbox}[boxrule=1pt,colback=white,leftrule=3mm]
    \raggedright Resuelva el problema que se encuentra a continuación. Para esto, no olvide 
    incluir un desarrollo pertinente y la respuesta al enunciado en los espacios señalizados.        
\end{tcolorbox}
%
%\settasks{debug}
Una agencia distribuidora de automóviles tiene tres vendedores: El señor Pérez 
que vende un 20\% del total de las unidades vendidas al mes, el señor Hernández 
con un 30\%, y por último el señor Fernández con un 50\%. Además, 
el 1\% de los autos que vende el señor Pérez es de lujo, el 4\% de los que vende 
el señor Hernández son de lujo y un 2\% de los que vende el señor Fernández son
de lujo.

Con estos datos, si se vende un auto de lujo, ¿Cuál es la probabilidad de que haya sido el señor 
Hernández el vendedor?
\begin{desarrollo}[height=10.5cm]
\end{desarrollo}
\begin{respuesta}[height=3cm]
\end{respuesta}
\end{document}