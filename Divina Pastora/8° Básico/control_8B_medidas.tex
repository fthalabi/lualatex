\providecommand{\curso}{Octavo Básico B}
\providecommand{\colegio}{Colegio Divina Pastora}
\providecommand{\tituloDocumento}{Control 2}
\providecommand{\subtituloDocumento}{Percentiles y otras medidas}
\documentclass{cdplf-prueba}

\begin{document}
%
\begin{tcbraster}[enhanced,raster columns=3,raster width=\linewidth,raster column skip=3pt,raster force size=false]
    \begin{caja}[title={\sffamily\scshape\bfseries Nombre},height=30pt,add to width=4cm]
    \end{caja}
    \begin{caja}[title={\sffamily\scshape\bfseries Puntaje},height=30pt,add to width=-2cm]
    \end{caja}
    \begin{caja}[title={\sffamily\scshape\bfseries Nota},height=30pt,add to width=-2cm]
    \end{caja}                    
\end{tcbraster}
%
\vspace*{10pt}
\underline{Datos:} \hspace{4pt} 4 \hspace{4pt}\textbullet\hspace{4pt} 8 \hspace{4pt}\textbullet\hspace{4pt} 12 \hspace{4pt}\textbullet\hspace{4pt} 15 \hspace{4pt}\textbullet\hspace{4pt} 15 \hspace{4pt}\textbullet\hspace{4pt} 18 \hspace{4pt}\textbullet\hspace{4pt} 12 \hspace{4pt}\textbullet\hspace{4pt} 16 \hspace{4pt}\textbullet\hspace{4pt} 10 \hspace{4pt}\textbullet\hspace{4pt} 9 \hspace{4pt}\textbullet\hspace{4pt} 11 \hspace{4pt}\textbullet\hspace{4pt} 9 \hspace{4pt}\textbullet\hspace{4pt} 4 \hspace{4pt}\textbullet\hspace{4pt} 1 \hspace{4pt}\textbullet\hspace{4pt} 8 \hspace{4pt}\textbullet\hspace{4pt} 14 \hspace{4pt}\textbullet\hspace{4pt} 7 \hspace{4pt}\textbullet\hspace{4pt} 12 \hspace{4pt}\textbullet\hspace{4pt} 4
\begin{center}\begin{tblr}{colspec={ccccc},hlines,vlines,hline{2,Z} = {1}{-}{},hline{2,Z} = {2}{-}{},row{even}={black!10}}
    .&Frecuencia&Probabilidad&Frecuencia Acumulada&Probabilidad Acumulada \\
   1&1&0.053&1&0.053 \\
   4&3&0.158&4&0.211 \\
   7&1&0.053&5&0.264 \\
   8&2&0.105&7&0.369 \\
   9&2&0.105&9&0.474 \\
   10&1&0.053&10&0.527 \\
   11&1&0.053&11&0.58 \\
   12&3&0.158&14&0.738 \\
   14&1&0.053&15&0.791 \\
   15&2&0.105&17&0.896 \\
   16&1&0.053&18&0.949 \\
   18&1&0.053&19&1.002 \\
\end{tblr}\end{center}

Utilice los datos y la tabla anterior, para calcular cuanto vale cada una de las siguientes medidas.
Además, para cada medida {\bfseries describa que significa cada uno de los valores obtenidos}.  
\begin{ejercicios}
    \task Media (Promedio) \vspace*{2pt}\begin{lineas}[height=2cm]\end{lineas}
    \task Rango \vspace*{2pt}\begin{lineas}[height=2cm]\end{lineas}
    \task Primer cuartil $\left(Q_1\right)$ \vspace*{2pt}\begin{lineas}[height=2cm]\end{lineas}
    \task Mediana $\left(Q_2\right)$ \vspace*{2pt}\begin{lineas}[height=2cm]\end{lineas}
    \task Tercer cuartil $\left(Q_3\right)$ \vspace*{2pt}\begin{lineas}[height=2cm]\end{lineas}
\end{ejercicios}

\end{document}