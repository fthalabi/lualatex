\providecommand{\curso}{Séptimo Básico}
\providecommand{\tituloDocumento}{Rubrica de corrección}
\providecommand{\subtituloDocumento}{La geometría del círculo}
\documentclass{cdplf-pauta}
\begin{document}
\raggedright
\begin{multicols}{3}

\parte{Explicar definiciones y propiedades}
\pregunta
\destacado
[3 puntos] Hace referencia a que el borde de la circunferencia se encuentra 
a una distancia "radio" del centro.
\competente
[1-2 puntos] Describe partes de la circunferencia, pero, la información es 
insuficiente para definir una.
\insuficiente
[0 puntos] No se describe el concepto.
\pregunta
\destacado
[3 puntos] Se refiere al perímetro del círculo como el largo total de su borde, y
se puede estimar como: 
\begin{equation*}
    P = 2\cdot\pi\cdot r \quad \textrm{ó} \quad P = \pi\cdot d
\end{equation*}
\competente
[1-2 puntos] No asocia el perímetro con el largo del borde, o bien, no describe 
como estimarlo. 
\insuficiente
[0 puntos] No describe el concepto.
\pregunta
\destacado
[3 puntos] Se refiere al área del círculo como la superficie (espacio plano) que ocupa, y que 
se puede estimar como: 
\begin{equation*}
    A = \pi \cdot r^2
\end{equation*}
\competente
[1-2 puntos] No asocia el área del círculo con la superficie de este, o bien, no describe como 
estimarla.
\insuficiente
[0 puntos] No describe el concepto.
\parte{Resolución de problemas}
\pregunta
\destacado
\begin{center}
\begin{tikzpicture}[ampersand replacement=\&,scale=0.7]
    \draw[pattern={crosshatch},pattern color=black!15] (0,0) circle (2cm);
    \draw[fill=white] (0,0) circle (1cm);
    \draw[line width=1pt] (0,0) -- (30:2cm);
    \draw[line width=1pt] (0,0) -- (-50:1cm);
    \node (A) at (4,1) {Mide 6 [cm]};
    \node (B) at (4,-1) {Mide 0.5 [cm]};
    \draw[->,dashed] (A.south) to[bend left=30] (30:1.5cm);
    \draw[->,dashed] (B) to[bend right=30] (-50:0.5cm);
\end{tikzpicture}
\end{center}
[4-5 puntos] Estima el área como:
\setlength{\belowdisplayskip}{0pt}
\setlength{\belowdisplayshortskip}{0pt}
\begin{align*}
    \textrm{Área total} &= \mathrm{A}(\raisebox{-2.5pt}{\tikz[]{\draw (0,0) circle (5pt);}}) - \mathrm{A}(\raisebox{-0.5pt}{\tikz{\draw (0,0) circle (2pt);}}) \\
                        &= \pi\cdot 6^2 - \pi\cdot 0.5^2 \\
                        &\approx 3\cdot 36 - 3\cdot 0.25 \\
                        &= 108 - 0.75 \\
                        &= 107.25 \; [\textrm{cm}^2] \\
\end{align*}
\competente
[1-3 puntos] Estima el área como:
%\setlength{\abovedisplayskip}{0pt}
%\setlength{\abovedisplayshortskip}{0pt}
\setlength{\belowdisplayskip}{0pt}
\setlength{\belowdisplayshortskip}{0pt}
\begin{align*}
    \mathrm{A}(\raisebox{-2.5pt}{\tikz[]{\draw (0,0) circle (5pt);}}) &= \pi\cdot 6^2\\
                        &\approx 3\cdot 36 \\
                        &= 108 \; [\textrm{cm}^2] \\
\end{align*}
\insuficiente
[0 puntos] No identifica el radio del círculo y como calcular su área de superficie.
\pregunta
\destacado
[4-5 puntos] Reconoce que el nuevo radio es:
\setlength{\abovedisplayskip}{4pt}
\setlength{\abovedisplayshortskip}{4pt}
\setlength{\belowdisplayskip}{4pt}
\setlength{\belowdisplayshortskip}{4pt}
\begin{equation*}
    6\cdot 200 \;\textrm{[cm]} = 1200 \;\textrm{[cm]} = 12 \;\textrm{[m]}
\end{equation*}
Estima el perímetro como:
\setlength{\abovedisplayskip}{4pt}
\setlength{\abovedisplayshortskip}{4pt}
\setlength{\belowdisplayskip}{4pt}
\setlength{\belowdisplayshortskip}{4pt}
\begin{equation*}
    \mathrm{P} = 2\cdot \pi \cdot r = 2\cdot 3 \cdot 12 = 72 \;\textrm{[m]}
\end{equation*}
Aproxima el espacio que ocupa una persona y determina: Si cada persona en el borde 
ocupa 1 [m], entonces caben 72 personas en el borde del CD gigante; Por otro lado, si 
cada persona ocupa medio metro, entonces caben 144 personas en el borde del CD gigante. 
\competente
[1-3 puntos] Reconoce que el nuevo radio del CD es 1200 [cm] ó 12 [m]. 

Estima el perímetro como 7200 [cm] ó 72 [m], pero, no lo identifica 
como el espacio disponible para ubicar a las personas y/o cuantas
personas podrían ubicarse en este espacio.
\insuficiente
[0 puntos] No asocia la pregunta con el perímetro del círculo.
\pregunta
\destacado
[4-5 puntos] Reconoce que las nuevas medidas del CD son:
\begin{itemize}[nosep,label=\textbullet]
    \item Radio grande = 1200 [cm] = 12 [m]
    \item Radio chico = 0.5 $\cdot$ 200 = 100 [cm] = 1 [m]
\end{itemize}
Estima el área de superficie como:
%\setlength{\abovedisplayskip}{0pt}
%\setlength{\abovedisplayshortskip}{0pt}
\setlength{\belowdisplayskip}{0pt}
\setlength{\belowdisplayshortskip}{0pt}
\begin{align*}
    \textrm{Área total} &= \mathrm{A}(\raisebox{-2.5pt}{\tikz[]{\draw (0,0) circle (5pt);}}) - \mathrm{A}(\raisebox{-0.5pt}{\tikz{\draw (0,0) circle (2pt);}}) \\
                        &= \pi\cdot 12^2 - \pi\cdot 1^2 \\
                        &\approx 3\cdot 144 - 3\cdot 1 \\
                        &= 429 \; [\textrm{m}^2] \\
\end{align*}
Aproxima el espacio que ocupa cada persona y determina: Si cada persona ocupa 1 [$\textrm{m}^2$], entonces caben 
429 personas sobre el CD gigante; Por otro lado, si las personas se encuentran más apretadas y cada una ocupa
0.25 [$\textrm{m}^2$], entonces caben 1716 personas sobre el CD.
\competente
[1-3 puntos] Reconoce el nuevo radio del CD como 1200 [cm] ó 12 [m], y 
estima el área como:
\setlength{\belowdisplayskip}{0pt}
\setlength{\belowdisplayshortskip}{0pt}
\begin{align*}
    \textrm{Área}       &=  \pi\cdot 12^2\\
                        &\approx 3\cdot 144\\
                        &= 432 \; [\textrm{m}^2] \\
\end{align*}
No reconoce el área como el espacio disponible para ubicar 
a las personas y/o cuantas personas podrían ubicarse en este 
espacio.
\insuficiente
[0 puntos] No asocia la pregunta con el área del CD.

\end{multicols}

\Large\vspace*{2cm}\hspace*{\fill}
\Caja[Puntaje][0.2\textwidth][40pt]
\Caja[Nota][0.2\textwidth][40pt]


\end{document}