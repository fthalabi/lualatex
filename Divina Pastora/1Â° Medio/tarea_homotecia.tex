\providecommand{\curso}{Primero Medio}
\providecommand{\colegio}{Colegio Divina Pastora}
\providecommand{\tituloDocumento}{Tarea 1}
\providecommand{\subtituloDocumento}{Homotecia y semejanza}

\documentclass{cdplf-prueba}

\begin{document}
%
\begin{tcbraster}[enhanced,raster columns=3,raster width=\linewidth,raster column skip=3pt,raster force size=false]
    \begin{caja}[title={\sffamily\scshape\bfseries Nombre},height=30pt,add to width=4cm]
    \end{caja}
    \begin{caja}[title={\sffamily\scshape\bfseries Puntaje},height=30pt,add to width=-2cm]
    \end{caja}
    \begin{caja}[title={\sffamily\scshape\bfseries Nota},height=30pt,add to width=-2cm]
    \end{caja}                    
\end{tcbraster}

\subsection*{Objetivos de la evaluación}
\begin{itemize}[]
    \item Aplican propiedades de la homotecia y semejanza para solucionar una \mbox{problemática}.
    \item Realizan homotecias en el plano cartesiano, conjeturan sobre el factor de la homotecia y las 
    \\\mbox{propiedades} de la imagen resultante.
    \item Explicar y fundamentar: 
    \begin{itemize}[]
        \item   Soluciones propias y los procedimientos utilizados.
        \item   Resultados mediante definiciones, axiomas, propiedades y teoremas.
    \end{itemize}
    %\item Explica soluciones propias y los procedimientos utilizados.
    %\item Elegir o elaborar representaciones de acuerdo a las necesidades de la actividad, identificando sus limitaciones y validez de éstas.
\end{itemize}

\subsection*{Instrucciones generales}

Esta tarea, abarca los contenidos trabajados en clases, en preparación para la 
evaluación sumativa de cierre de unidad. Esta es individual y con nota al libro.

La entrega de la tarea es para el día 14 de octubre, al comienzo de la clase de matemáticas.
Atrasos en la entrega deben ser debidamente justificados.

\subsection*{Pauta de cotejo}

En la corrección de la tarea, se le asignará puntaje a cada respuesta según
los criterios que se encuentran detallados en la tabla a continuación.

\begin{center}
    \begin{tblr}{width=\linewidth,colspec={X[1,c]|X[6]}, hline{1,Z} = {1}{-}{}, hline{1,Z} = {2}{-}{}, 
        hlines, cells={valign=m}, row{1} = {bg=black!15}}
        Puntaje asignado & \SetCell{c} Criterios o indicadores \\
        +50\% & Señala clara y correctamente cuál es la solución o el resultado de la pregunta hecha
        en el enunciado. \\ 
        +50\% & Incluye un desarrollo que relata de manera clara y ordenada los procedimientos 
         \mbox{necesarios} para solucionar la problemática. En caso de estar incompleto o con 
         \mbox{errores} el desarrollo, se asignará puntaje parcial si se muestra dominio de los 
         con\-tenidos y conceptos involucrados. \\
        0\% &  La respuesta es incorrecta. De haber desarrollo, este tiene errores conceptuales.\\
    \end{tblr}    
\end{center}
    
\vspace*{\fill}
\begin{center}
    \begin{tikzpicture}[ampersand replacement=\&,]
        %\node (A) [opacity=0.4] {\includegraphics[width=2cm]{../flork3.jpg}};
        \node (B) [font=\slshape,text width=12cm]
        {``Cree en ti mismo y en lo que eres. Sé consciente de que hay algo en tu interior %
        que es más grande que cualquier obstáculo''};
        \node [left=0mm of B,opacity=0.4] {\pgfornament[width=2cm]{37}};
        \node [right=0mm of B,opacity=0.4] {\pgfornament[width=2cm]{38}};
    \end{tikzpicture}
\end{center}
\vspace*{\fill}
\newpage

%\parte 
\begin{tcolorbox}[boxrule=1pt,colback=white,leftrule=3mm]
    \raggedright Resuelva los problemas que se encuentran a continuación. Para esto, no olvide 
    incluir un desarrollo pertinente y la respuesta al enunciado en los espacios señalizados.        
\end{tcolorbox}

\subsection{} Calcular los valores de $a$, $b$ y $c$ [1 punto c/u], si las rectas $L_1$, $L_2$, $L_3$ y $L_4$ son paralelas.

\begin{center}
\begin{tikzpicture}[line width=1pt,y=1cm]
    \coordinate (A) at (0,0);
    \coordinate (B) at (5,6);
    \coordinate (C) at (6,0);

    \draw [<->,shorten >=-1cm, shorten <=-1cm, name path=A--B] (A) -- (B);
    \draw [<->,shorten <=-1cm, shorten >=-1cm, name path=B--C] (B) -- (C);
    \draw [<->] ($(B) - (1cm,0)$) -- ($(B)+ (1cm,0)$) node[pos=-0.2] {$L_1$};
    \path [name path=path1] ($(B)!0.3!(C)$) --+(-180:5cm);
    \draw [<->,name intersections={of=path1 and A--B, by=x},
        shorten <=-1cm, shorten >=-1cm,
    ] ($(B)!0.3!(C)$) -- (x) node[pos=1.2,xshift=-1cm] {$L_2$};
    \path [name path=path2] ($(B)!0.7!(C)$) -- +(-180:6cm);
    \draw [<->,name intersections={of=path2 and A--B, by=y},
        shorten <=-1cm, shorten >=-1cm
    ]
        ($(B)!0.7!(C)$) -- (y) node[pos=1.2,xshift=-0.6cm] {$L_3$};
    \path [name path=path3] ($(B)!0.9!(C)$) -- +(-180:6cm);
    \draw [<->,name intersections={of=path3 and A--B, by=z},
        shorten <=-1cm, shorten >=-1cm
    ]
        ($(B)!0.9!(C)$) -- (z) node[pos=1.2,xshift=-0.5cm] {$L_4$};
    
    \node [right=10pt] at ($(B)!0.15!(C)$) {8};
    \node [right=10pt] at ($(B)!0.5!(C)$) {12};
    \node [right=10pt] at ($(B)!0.8!(C)$) {$c$};
    \node [left=10pt] at ($(B)!0.5!(x)$) {$a$};
    \node [left=10pt] at ($(x)!0.5!(y)$) {15};
    \node [left=10pt] at ($(y)!0.5!(z)$) {8};
    \node [below=6pt] at ($(B)!0.9!(C)!0.5!(z)$) {20};
    \node [below=6pt] at ($(B)!0.3!(C)!0.5!(x)$) {$b$};
    
    % \draw[line width=0.4pt,decorate, decoration={brace,raise=0pt,amplitude=10pt}]
    %     (B) -- ($(B)!0.3!(C)$) node[midway,right=15pt] {8};
\end{tikzpicture}
\end{center}

\begin{desarrollo}[height=9cm]
\end{desarrollo}
\begin{respuesta}[height=2cm]
\end{respuesta}

\subsection{} 
Ubica en el plano el trapecio de coordenadas $A(0,2)$, $B(4,0)$, $C(4,1)$ y
$D(1,3)$, y dibuja su imagen homotética con centro en el origen y razón
de homotecia $k=2.5$ [2 puntos]. Por último, determina como cambia el perímetro 
y área del trapecio resultante con respecto al original [2 puntos]. 

\begin{center}
    \plano[15][10][0][0][0.8cm]    
\end{center}
\begin{desarrollo}[height=8cm]
\end{desarrollo}
\begin{respuesta}[height=2cm]
\end{respuesta}

\subsection{} En la figura a continuación, determine usando homotecia el largo de los lados 
$\overline{AB}$ y $\overline{DE}$ [1 punto c/u] si se sabe que: 
$\overline{AB} = x - 1$, $\overline{BC} = 12$, $\overline{DE} = x$, $\overline{CE} = 15$, y 
$L_1 \parallel L_2 \parallel L_3$. Además, justifique por qué se puede usar homotecia para 
resolver este problema [1 punto].

\begin{center}
    \begin{tikzpicture}[
        x=0.6cm,y=0.6cm,
        rotate=20, line width=1pt,
        extended line/.style={shorten >=-#1,shorten <=-#1},
        extended line/.default=1cm,
    ]
        \coordinate (A) at (0,0);
        \coordinate (C) at (10,0);
        \coordinate (B) at ($(A)!0.3!(C)$);
        \coordinate (D) at (0,-5);
        \coordinate (E) at ($(D)!0.3!(C)$);
    
        \draw [<->,extended line=1cm] (A) -- (C);
        \draw [<->,extended line=1cm] (D) -- (C);
        \draw [<->,extended line=1cm] (A) -- (D) node [pos=-0.5] {$L_1$};
        \draw [<->,extended line=1cm] (B) -- (E) node [pos=-0.67] {$L_2$};
        \draw [<->,shorten <=-1cm, shorten >=1cm] (C) -- (C |- E) node [pos=-0.67] {$L_3$};
    
        \fill (A) circle [radius=3pt] node [below left,yshift=-3mm,xshift=0mm] {$A$}; 
        \fill (B) circle [radius=3pt] node [below left,yshift=-3mm,xshift=0mm] {$B$}; 
        \fill (C) circle [radius=3pt] node [below left,yshift=-3mm,xshift=1mm] {$C$}; 
        \fill (D) circle [radius=3pt] node [below left,yshift=-3mm,xshift=1mm] {$D$}; 
        \fill (E) circle [radius=3pt] node [below left,yshift=-3mm,xshift=1mm] {$E$}; 
    \end{tikzpicture}
\end{center}
\begin{desarrollo}[height=9cm]
\end{desarrollo}
\begin{respuesta}[height=3cm]
\end{respuesta}

\subsection{} 
Calcule la medida de los segmentos $\overline{AD}$, $\overline{BD}$ y $\overline{CD}$ 
[1 punto c/u]

\begin{center}
\begin{tikzpicture}[ampersand replacement=\&]
    \draw[->,line width=1pt,shorten >=-1cm] (0,0) -- (9,0);
    \draw[->,line width=1pt,shorten >=-1cm] (0,0) -- (0,7);
    \foreach \x in {0,1,...,9} {
        \draw (\x,5pt) -- (\x,-5pt) node [below] {$\x$};
    }
    \foreach \y in {1,2,...,7} {
        \draw (5pt,\y) -- (-5pt,\y) node [left] {$\y$};
    }
    \draw[help lines] (0,0) grid (9,7);
    \draw[line width=1pt] (8,6) node [circle,inner sep=2pt,fill=black,label=above:$A$] {} -- (1,0) node [circle,inner sep=2pt,fill=black,label=above:$B$] {}-- (8,0) node [circle,inner sep=2pt,fill=black,label=above right:$C$] {} -- cycle;
    \draw[line width=1pt] (8,0) -- ($(1,0)!(8,0)!(8,6)$) node [circle,inner sep=2pt,fill=black,label=above left:$D$] {};
    \draw[line width=1pt] ($(1,0)!(8,0)!(8,6)$) -- ($(1,0)!(8,0)!(8,6)!10pt!(8,6)$) -- ([turn]-90:10pt) -- ([turn]-90:10pt) -- ([turn]-90:10pt);
\end{tikzpicture}
\end{center}
\begin{desarrollo}[height=10cm]
\end{desarrollo}
\begin{respuesta}[height=2cm]
\end{respuesta}

\newpage
\renewcommand{\hrulefill}{%
  \leavevmode\leaders\hrule height 1pt\hfill\kern0pt }

\begin{itemize}
    \item hola \hrulefill \\ .\hrulefill
    \item hola
    \item hola
    \item hola
    \item hola
\end{itemize}

\end{document}