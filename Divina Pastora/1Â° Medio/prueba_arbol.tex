\providecommand{\curso}{Primero Medio}
\providecommand{\colegio}{Colegio Divina Pastora}
\providecommand{\tituloDocumento}{Prueba}
\providecommand{\subtituloDocumento}{Diagramas de árbol y Probabilidad}
\documentclass{cdplf-prueba}

\begin{document}
%
\begin{tcbraster}[enhanced,raster columns=3,raster width=\linewidth,raster column skip=3pt,raster force size=false]
    \begin{caja}[title={\sffamily\scshape\bfseries Nombre},height=30pt,add to width=4cm]
    \end{caja}
    \begin{caja}[title={\sffamily\scshape\bfseries Puntaje},height=30pt,add to width=-2cm]
    \end{caja}
    \begin{caja}[title={\sffamily\scshape\bfseries Nota},height=30pt,add to width=-2cm]
    \end{caja}                    
\end{tcbraster}

\subsection*{Objetivos de la evaluación}
\begin{itemize}[]
    \item Modela situaciones y eventos probabilísticos usando diagramas de árbol.
    \item Aplica las reglas aditiva y multiplicativa de la probabilidad en la resolución de problemas.
    \item Toma decisiones en situaciones de incerteza que involucren probabilidades condicionales.
    \item Explicar y fundamentar: 
    \begin{itemize}[]
        \item   Soluciones propias y los procedimientos utilizados.
        \item   Resultados mediante definiciones, axiomas, propiedades y teoremas.
    \end{itemize}
\end{itemize}

\subsection*{Instrucciones generales}

Esta evaluación, de cierre de unidad, abarca todos los contenidos trabajados en la unidad 
de Estadística y Probabilidad. Esta es individual y con nota al libro. Si lo estima conveniente 
puede usar calculadora para su desarrollo.

Por otro lado, cualquier acto de deshonestidad durante la evaluación, será sancionado según 
el reglamento del colegio.

\subsection*{Pauta de cotejo}

En la corrección de la evaluación, se le asignará puntaje a cada respuesta según
los criterios que se encuentran detallados en la tabla a continuación.

\begin{center}
    \begin{tblr}{width=\linewidth,colspec={X[1,c]|X[6]}, hline{1,Z} = {1}{-}{}, hline{1,Z} = {2}{-}{}, 
        hlines, cells={valign=m}, row{1} = {bg=black!15}}
        Puntaje asignado & \SetCell{c} Criterios o indicadores \\
        +50\% & Señala clara y correctamente cuál es la solución o el resultado de la pregunta hecha
        en el enunciado. \\ 
        +50\% & Incluye un desarrollo que relata de manera clara y ordenada los procedimientos 
         \mbox{necesarios} para solucionar la problemática. En caso de estar incompleto o con 
         \mbox{errores} el desarrollo, se asignará puntaje parcial si se muestra dominio de los 
         con\-tenidos y conceptos involucrados. \\
        0\% &  La respuesta es incorrecta. De haber desarrollo, este tiene errores conceptuales.\\
    \end{tblr}    
\end{center}
    
\vspace*{\fill}
\begin{center}
    \begin{tikzpicture}[ampersand replacement=\&,]
        %\node (A) [opacity=0.4] {\includegraphics[width=2cm]{../flork3.jpg}};
        \node (B) [font=\slshape,text width=12cm]
        {``Cree en ti mismo y en lo que eres. Sé consciente de que hay algo en tu interior %
        que es más grande que cualquier obstáculo''};
        \node [left=0mm of B,opacity=0.4] {\pgfornament[width=2cm]{37}};
        \node [right=0mm of B,opacity=0.4] {\pgfornament[width=2cm]{38}};
    \end{tikzpicture}
\end{center}
\vspace*{\fill}
\newpage

%\parte 
\begin{tcolorbox}[boxrule=1pt,colback=white,leftrule=3mm]
    \raggedright Resuelva los problemas que se encuentran a continuación. Para esto, no olvide 
    incluir un desarrollo pertinente y la respuesta al enunciado en los espacios señalizados.        
\end{tcolorbox}

%%% PROBLEMA 1
% Tres joyeros idénticos tienen dos compartimientos. En cada compartimiento del primer joyero hay un reloj de oro. En cada compartimiento del segundo joyero hay un reloj de plata. En el tercer joyero en un compartimiento hay un reloj de oro, en tanto que en el otro hay un reloj de plata. Si seleccionamos un joyero aleatoriamente, abrimos uno de los compartimientos y hallamos un reloj de plata, ¿cuál es la probabilidad de
% que el otro compartimiento tenga un reloj de oro?

%%% PROBLEMA 2
%% 
% r$> dt %>%
%         count(Sex, Survived)
%      Sex   Survived   n
% 1 Hombre Sobrevivio 468
% 2 Hombre      Murio 109
% 3  Mujer Sobrevivio  81
% 4  Mujer      Murio 233

% r$> dt %>%
%         count(Pclass, Survived)
%   Pclass   Survived   n
% 1    1ra Sobrevivio  80
% 2    1ra      Murio 136
% 3    2da Sobrevivio  97
% 4    2da      Murio  87
% 5    3ra Sobrevivio 372
% 6    3ra      Murio 119
%

\subsection{} Tres joyeros idénticos tienen dos compartimientos. 
En cada compartimiento del primer joyero hay un reloj de oro. 
En cada compartimiento del segundo joyero hay un reloj de plata. 
En el tercer joyero, un compar\-timien\-to tiene un reloj de oro, 
mientras que el otro compartimiento tiene un reloj de plata. Si seleccionamos un joyero 
aleatoriamente, abrimos uno de los compartimientos y hallamos un reloj de plata, 
¿Cuál es la probabilidad de que el otro compartimiento tenga un reloj de oro? [4 puntos]


\begin{desarrollo}[height=14cm]
\end{desarrollo}
\begin{respuesta}[height=2cm]
\end{respuesta}

%%% PROBLEMA 2
% count(Sex, Survived)
% Sex           Survived            n
% Masculino     No Sobrevivió       468
% Masculino     Sobrevivió          109
% Femenino      No Sobrevivió       81
% Femenino      Sobrevivió          233

% r$> dt %>%
% count(Pclass, Survived)
% Pclass        Survived            n
% 1ra           No Sobrevivió       80
% 1ra              Sobrevivió       136
% 2da           No Sobrevivió       97
% 2da              Sobrevivió       87
% 3ra           No Sobrevivió       372
% 3ra              Sobrevivió       119

\subsection{} La mañana del 15 de abril de 1912, el RMS Titanic se hundió al chocar
con un iceberg en su trayecto desde Inglaterra a Nueva York. Pocos sobrevivieron y 
muchos desaparecieron. Múltiples esfuerzos se han hecho con los años para 
reconstruir el manifiesto de pasajeros, a continuación se presentan dos tablas 
que representan un resumen de la información que se tiene.

\begin{tcolorbox}[blanker,sidebyside,lefthand width=0.42\linewidth,box align=top,halign upper=center]
    \begin{center}
        \begin{talltblr}[caption={Sobrevivientes según sexo}]%
            {colspec={ccc},hlines,vlines,hline{2} = {1}{-}{},hline{2} = {2}{-}{},vline{2}={1}{-}{},vline{2}={2}{-}{},row{1}={black!10},column{1}={black!10},
            row{1}={font=},rowsep=3pt,cells={valign=m}}
                \diagbox{}{Sexo} & Masculino & Femenino \\
                Sobrevivió    &     109    &   233        \\
                No Sobrevivió &     468   &    81         \\
        \end{talltblr}
    \end{center}
    %
    \tcblower 
    \begin{center}
        \begin{talltblr}[caption={Sobrevivientes según tipo de pasaje}]%
            {colspec={cccc},hlines,vlines,hline{2} = {1}{-}{},hline{2} = {2}{-}{},row{1}={black!10},column{1}={black!10},vline{2}={1}{-}{},vline{2}={2}{-}{},
            row{1}={font=},rowsep=3pt,cells={valign=m}}
                \diagbox{}{Pasaje} & 1ra Clase & 2da Clase & 3ra Clase \\
                Sobrevivió & 136 & 87 & 119 \\
                No Sobrevivió & 80 & 97 & 372 \\
        \end{talltblr}
    \end{center}
\end{tcolorbox}
%
\begin{desarrollo}[height=3cm]
\end{desarrollo}
%
Considerando la información entregada, respondas las siguientes preguntas y no 
olvide fundamentar sus respuestas [2 puntos por pregunta].
%
\begin{tasks}[label={\tcbox[colback=black!60, colframe=black!60, coltext=white, on line, boxsep=0pt, left=3pt, right=3pt, top=2pt, bottom=2pt]{\sffamily\bfseries\Alph*}},
    item-indent=1.2cm,column-sep=20pt,label-offset=0.3cm,label-width=15pt,after-item-skip=10pt](1)
    \task \raggedright ¿Afecta el sexo de un pasajero su probabilidad de supervivencia? ¿En qué manera?
    \vspace{5pt}\begin{respuesta}[height=2cm]\end{respuesta}
    \task \raggedright ¿Afecta la clase del pasaje la probabilidad de supervivencia de un pasajero? ¿En qué manera?
    \vspace{5pt}\begin{respuesta}[height=2cm]\end{respuesta}
    \task ¿Qué afecta más la probabilidad de supervivencia de un pasajero, su sexo o la clase de su pasaje?
    \vspace{5pt}\begin{respuesta}[height=3cm]\end{respuesta}
\end{tasks}


\end{document}