\usepackage{fontspec} % For loading OpenType fonts
\usepackage{unicode-math} % For modern math font handling
%% Set the main fonts (Libertinus family)
\setmainfont{Libertinus Serif}
\setsansfont{Libertinus Sans}
\setmonofont{Libertinus Mono} % For code listings
%% Set the math font
\setmathfont{Libertinus Math} % Excellent math font that pairs with Libertinus
\usepackage{setspace}
\setstretch{0.9}

%%% visual debug
%%%\usepackage{lua-visual-debug}
\RequirePackage{expl3}
\RequirePackage{luacode}
%%\RequirePackage[spanish,es-noshorthands]{babel}
%%\selectlanguage{spanish}
\RequirePackage{polyglossia}
\setmainlanguage[spanishoperators=all]{spanish}

%%\usepackage[fontsize=9pt]{fontsize}
\RequirePackage{letltxmacro}
\RequirePackage[]{amsmath}
\RequirePackage{icomma}
\RequirePackage{calc}
\RequirePackage{ifthen}
\RequirePackage{xparse}
\RequirePackage{hyperref}
\hypersetup{
    colorlinks=true,          % true: colored links, false: boxes
    linkcolor=darkblue,       % internal links
    citecolor=darkgreen,      % citations
    filecolor=darkmagenta,    % file links
    urlcolor=darkred,         % external urls
    breaklinks=true,          % allow links to break across lines
    hidelinks,                % remove colored boxes AND colored text
    breaklinks=true,          % Enable breaking of links
}
\urlstyle{same}
%%%% Define where URLs can break
\renewcommand{\UrlBreaks}{\do\.\do\@\do\\\do\/\do\!\do\_\do\|\do\;\do\>}

\RequirePackage{lipsum}
\RequirePackage{tikz}
\RequirePackage{varwidth}
\RequirePackage{pgfplots}
\usepgfplotslibrary{statistics}
\usepgfplotslibrary{fillbetween}
\pgfplotsset{compat=newest}
\usepgflibrary {fpu}
\usetikzlibrary{positioning,calc,math,backgrounds,%
shapes.symbols,shapes.geometric,shapes.misc,shapes.callouts,%
decorations.pathreplacing,calligraphy,arrows.meta,%
cd,intersections,patterns,patterns.meta,angles,through,
graphs,graphdrawing,quotes,datavisualization,
datavisualization.formats.functions}
\usegdlibrary {force}
\usegdlibrary{trees}
\usegdlibrary{circular}
\usegdlibrary {layered}
\RequirePackage{pgfornament}

\pgfkeys{/pgf/number format/.cd,
  set decimal separator={,},set thousands separator={\,}}
%\pgfkeys{/pgf/number format/.cd,fixed,fixed zerofill,precision=3}
\pgfkeys{/pgfplots/boxplot/estimator=R1,
  /pgfplots/boxplot/every median/.style={
    line width=3pt
  }
}

\tikzset{
  >={Triangle[round]},
  %every pin/.style={pin edge={Triangle[]-}},
  /tikz/graphs/layered layout/.append style={
    level distance=2cm,
    sibling distance=1.5cm,
    edge quotes={draw, inner sep=3pt,rounded corners, dashed, fill=white},
    edge={-{Triangle[round]}}
  },
  samples=200,
}

\pgfplotsset{
  eje escolar/.style={
    axis y line=center,
    axis x line=middle,
    xlabel=$x$,ylabel=$y$,
    axis line style={-{Triangle[round]},shorten >=-10pt},
    xlabel style={
      at={(xticklabel* cs:1)},
      anchor=west,
      xshift=10pt,
    },
    ylabel style={
      at={(yticklabel* cs:1)},
      anchor=south,
      yshift=10pt,
    },
    enlargelimits=0.01,  % Extends y-axis by 10% beyond ymax
    axis on top,
    tick label style={
      fill=white,
      fill opacity=0.9,
      inner xsep=1pt,
      inner ysep=1pt,
    },
    trig format plots=rad,
    legend style={
      at={(1.05,1)},
      anchor=north west,
      cells={anchor=west},
      font=\footnotesize,
      nodes={
        inner xsep=0.1em,  % Reduce horizontal spacing between symbol and text
        inner ysep=0.1em,  % Reduce vertical spacing between legend entries
      }
    }
  },
  cycle list name=linestyles,
  every axis plot/.append style={line width=1pt}
}

\RequirePackage[most]{tcolorbox}
\tcbuselibrary{skins}
\tcbuselibrary{raster}
\tcbset{enhanced}
\RequirePackage[export]{adjustbox}
\RequirePackage{tabularray}
\RequirePackage{array}
\RequirePackage{environ}
\RequirePackage{xcolor}
\RequirePackage{varwidth}
\UseTblrLibrary{amsmath}
\UseTblrLibrary{siunitx,varwidth,diagbox,functional}
\sisetup{round-precision=3,round-mode=places,output-decimal-marker={,},
  mode = text, text-font-command = \normalsize}
\RequirePackage[inline]{enumitem}
\RequirePackage{tasks}
\RequirePackage{fontawesome5}
\RequirePackage{lastpage}
\RequirePackage{geometry}
\RequirePackage{microtype}
\RequirePackage[skip=10pt, indent=0pt]{parskip}
\RequirePackage{fancyhdr}
%\RequirePackage[big,sc,bf,sf]{titlesec}
\RequirePackage[]{titlesec}
\RequirePackage{ragged2e}
%\RequirePackage{notomath}
%\usepackage[mono,vvarbb,upint]{notomath}
%\usepackage[scaled=1.12]{nimbusmononarrow}% typewriter font
%\usepackage[sfdefault,subscriptcorrection]{notomath}
%\setmonofont{NotoMono-Regular}[
%  Path = /Users/fenho/Library/Fonts/,
%  Extension =  .ttf,
%  Ligatures = TeX,
%  Scale=MatchLowercase,
%  Scale=0.95
%]

%\RequirePackage[frac,rfrac]{mathfixs}
\let\originaldfrac\dfrac
\renewcommand*{\dfrac}[2]{\mathinner{\originaldfrac{#1}{#2}}}

\RequirePackage[executable=/Users/fenho/.mipython/bin/python]{pyluatex}
\RequirePackage{csvsimple-l3}
\RequirePackage{alphalph}

\newfontfamily\FA{Font Awesome 6 Pro}
\def\check{\FA\char"F00C}
\def\xmark{\FA\char"F00D}

\ExplSyntaxOn
\NewDocumentCommand{\lvl}{m}{
  \tl_gclear:N \tl_symbol
  \int_case:nnF {#1} {
    {1} {\tl_gset:Nn \tl_symbol {{\FA \char"F816}}}
    {2} {\tl_gset:Nn \tl_symbol {{\FA \char"F06D}}}
    {3} {\tl_gset:Nn \tl_symbol {{\FA \char"F54C}}}
  } {\tl_gset:Nn \tl_symbol {?}}
  \tl_use:N \tl_symbol
}

\ExplSyntaxOff

%%% Raíz cerrada
%\LetLtxMacro{\OldSqrt}{\sqrt}
%\newcommand{\ClosedSqrt}[1][\hphantom{3}]{\def\DHLindex{#1}\mathpalette\DHLhksqrt}
%\makeatletter
%    \newcommand*\bold@name{bold}
%    \def\DHLhksqrt#1#2{%
%        \setbox0=\hbox{$#1\OldSqrt{#2\,}$}\dimen0=\ht0\relax%
%        % Modify this value to adjust the vertical position of the closing line
%        \advance\dimen0-0.3\ht0\relax% Changed from 0.4 to 0.2 %%% ALTO DE LA RALLA
%        %\setbox2=\hbox{\vrule height\ht0 depth -\dimen0}%
%        \setbox2=\hbox{\vrule width 0.6pt height\ht0 depth -\dimen0}% ANCHO DE LA RALLA
%        {\hbox{$#1\expandafter\OldSqrt\expandafter[\DHLindex]{#2\hspace{1pt}}$}
%        % Adjust this vertical offset as needed
%        \lower\ifx\math@version\bold@name0.8pt\else0.6pt\fi\box2}% CAMBIA EL INICIO DE LA LINEA
%    }
%    % root index positioning and added space at the end
%    \renewcommand*{\sqrt}[2][\ ]{\ClosedSqrt[\leftroot{0}\uproot{1}#1]{#2}\kern0.1em}
%\makeatother
