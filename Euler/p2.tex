\documentclass{experimento} 

\pgfkeys{/pgf/number format/.cd,
  fixed,
  precision=2,
  set thousands separator={.},
  use comma
}

\ExplSyntaxOn

\cs_new:Npn \generar_fibo:Nn #1#2 {
  \seq_set_from_clist:Nn #1 {1,2}
  \int_step_inline:nnn {3} {#2} {
    \seq_log:N #1
    \int_zero_new:N \l_suma
    \int_zero_new:N \l_ultimo
    \int_set:Nn \l_ultimo {\seq_item:Nn #1 {-1}}
    \int_log:N \l_ultimo
    \int_zero_new:N \l_penultimo
    \int_set:Nn \l_penultimo {\seq_item:Nn #1 {-2}}
    \int_log:N \l_penultimo
    \int_set:Nn \l_suma {\int_eval:n {\l_penultimo + \l_ultimo}}
    \seq_gput_right:Ne #1 {\int_use:N \l_suma}
  }
}

\cs_new:Npn \generar_fibo_hasta:Nn #1#2 {
  \seq_set_from_clist:Nn #1 {1,2}
  \int_zero_new:N \l_penultimo_int
  \int_set:Nn \l_penultimo_int {1}
  \int_zero_new:N \l_ultimo_int
  \int_set:Nn \l_ultimo_int {2}
  \bool_do_while:nn {\int_compare_p:nNn {\int_use:N \l_ultimo_int} < {#2}}{
    \int_zero_new:N \l_suma_int
    \int_set:Nn \l_suma_int {\int_eval:n {\l_penultimo_int + \l_ultimo_int}}
    \seq_gput_right:Nx #1 {\int_use:N \l_suma_int}
    \int_set:Nn \l_penultimo_int {\int_use:N \l_ultimo_int}
    \int_set:Nn \l_ultimo_int {\int_use:N \l_suma_int}    
  }
  \int_compare:nNnTF {\int_use:N \l_ultimo_int} > {#2} {
    \seq_pop_right:NN #1 \l_tmp_tl
  } {}
}

\cs_new:Npn \obtener_pares:NN #1#2 {
  \seq_log:N #1
  \seq_log:N #2
  \seq_gset_filter:NNn {#1} {#2} {
    \int_compare_p:nNn {\int_mod:nn {##1} {2}} = {0}
  }
}

\cs_new:Npn \formatear_seq:NN #1#2 {
  \seq_map_inline:Nn #2 {
    \seq_gput_right:Nx #1 {\pgfmathprintnumber{##1}}
  }
}

\cs_new:Npn \sumar_seq:NN #1#2 {
  \int_gzero:N #1
  \seq_map_inline:Nn #2 {
    \int_gadd:Nn #1 {##1}
  }
}

%%% funcion para formatear

\NewDocumentCommand{\generarFibo}{m}{
  \seq_clear_new:N \l_fibo_seq
  \generar_fibo:Nn \l_fibo_seq {#1}
  \seq_clear_new:N \l_fibof_seq
  \formatear_seq:NN \l_fibof_seq \l_fibo_seq
  \seq_use:Nn \l_fibof_seq {,~}
}

\NewDocumentCommand{\generarFiboHasta}{m}{
  \seq_clear_new:N \l_fibo_seq
  \generar_fibo_hasta:Nn \l_fibo_seq {#1}
  \seq_clear_new:N \l_fibof_seq
  \formatear_seq:NN \l_fibof_seq \l_fibo_seq
  \seq_use:Nn \l_fibof_seq {,~}
}

\NewDocumentCommand{\obtenerPares}{}{
  \seq_clear_new:N \l_pares_seq
  \obtener_pares:NN \l_pares_seq \l_fibo_seq
  \seq_clear_new:N \l_paresf_seq
  \formatear_seq:NN \l_paresf_seq \l_pares_seq
  \seq_use:Nn \l_paresf_seq {,~}
}

\NewDocumentCommand{\total}{}{
  \int_zero_new:N \l_total
  \sumar_seq:NN \l_total \l_pares_seq
  \pgfmathprintnumber{\int_use:N \l_total}
}

\ExplSyntaxOff


\begin{document}

P2: Each new term in the Fibonacci sequence is generated by adding the previous two terms. 
By starting with $1$ and $2$, the first $10$ terms will be:
$$1, 2, 3, 5, 8, 13, 21, 34, 55, 89, \dots$$
By considering the terms in the Fibonacci sequence whose values do not exceed four million,
find the sum of the even-valued terms.

Answer:\par

Fibo <= 4.000.000: \generarFiboHasta{4000000} \par
Fibo <= 4.000.000 \& even: \obtenerPares \par
Total: \total \par

%\ExplSyntaxOn
%\iow_new:N \l_archivo
%\iow_open:Nn \l_archivo { datos.tmp }
%\seq_map_inline:Nn \l_valores_unicos
%  {
%    \iow_now:Nx \l_archivo { }
%  }
%\iow_close:N \l_archivo
%\ExplSyntaxOff

\ExplSyntaxOn
  \seq_clear_new:N \l_puntos_seq
  \seq_map_indexed_inline:Nn \l_fibo_seq {
    \seq_gput_right:Ne \l_puntos_seq {(#1,#2)} 
  }
  \tl_new:N \l_puntos_tl
  \tl_set:Nx \l_puntos_tl {\seq_use:Nn \l_puntos_seq {~}}
  \edef\puntos{\tl_use:N \l_puntos_tl}
\ExplSyntaxOff

\begin{tikzpicture}[baseline=(current axis.north)]
  \begin{axis}[
    legend style={at={(0.05,0.95)},anchor=north west}
  ]
      \addplot [solid] coordinates {\puntos};
      \addplot [dashed,domain=0:30] {x^4};
      \addplot [dotted,domain=0:30,update limits=false] {x^5};
      \legend{Fibo,$x^4$,$x^5$}
  \end{axis}
\end{tikzpicture} 



\end{document}