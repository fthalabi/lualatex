\documentclass{experimento}

\ExplSyntaxOn
  \cs_new:Npn \es_divisible_por:nn #1#2
  {
    \int_compare_p:nNn { \int_mod:nn {#1} {#2} } = { 0 }
  }

  \tl_new:N \l_temp_tl
  \tl_set:Nx \l_temp_tl { \es_divisible_por:nn {10} {2} }
  \tl_log:N \l_temp_tl

  \seq_new:N \l_divisibles_seq
  \int_new:N \l_suma_acumulativa_int

  \int_step_inline:nn {999} {
    \bool_lazy_any:nTF { {\es_divisible_por:nn {#1} {3}}  {\es_divisible_por:nn {#1} {5}} }
      {
        \seq_put_right:Nn \l_divisibles_seq {#1}
        \int_gadd:Nn \l_suma_acumulativa_int {#1}
      }
      {}
  }
  \seq_log:N \l_divisibles_seq

  \edef\divisibles{\seq_use:Nnnn \l_divisibles_seq {~and~} {,~} {,~and~}}
  \edef\suma{\int_use:N \l_suma_acumulativa_int}
\ExplSyntaxOff


\begin{document}
  P1: If we list all the natural numbers below $10$ that are multiples of $3$ or $5$,
  we get $3, 5, 6$ and $9$. The sum of these multiples is $23$.
  Find the sum of all the multiples of $3$ or $5$ below $1000$. \par

  the divisible numbers are: \divisibles. \par

  the total sum is: \suma. \par
\end{document}