\providecommand{\tituloDocumento}{Prueba formativa}
\providecommand{\subtituloDocumento}{Números Enteros}

\documentclass{sn-guia}

\begin{document}
\raggedright
\begin{tcbraster}[enhanced,raster columns=3,raster width=\linewidth,raster column skip=3pt,raster force size=false]
    \begin{caja}[title={\sffamily\scshape\bfseries Nombre},height=30pt,add to width=3.5cm]
    \end{caja}
    \begin{caja}[title={\sffamily\scshape\bfseries Curso},height=30pt,add to width=-1.5cm]
    \end{caja}    
    \begin{caja}[title={\sffamily\scshape\bfseries Décimas},height=30pt,add to width=-2cm]
    \end{caja}                
\end{tcbraster}
\vspace*{5mm}

\begin{ejercicios}(1)
    \task %
        Complete la siguiente tabla de operaciones. \hfill [1 décima por cada 8 correctas]\vspace*{5pt}
        \begin{tblr}{colspec={ccccccc},hlines,vlines,hline{2,Z} = {1}{-}{}, hline{2,Z} = {2}{-}{}, row{even}={black!15},rowsep=5pt,colsep=20pt}
            $a$ & $b$ & $c$ & $a\cdot b$ & $a:-b$ & $a\cdot b \cdot c$ & $-c\div a$ \\
                5   &  -1   &  200    &            &        &                    &            \\  
                -12   &  -4   &  -96    &            &        &                    &            \\  
                150   &  30   &  -1050    &            &        &                    &            \\  
                -126   &  6   &  378    &            &        &                    &            \\  
        \end{tblr}
    \task $\left(3-10\right)\cdot -2 + \left(-5-3\right)\div -2 + 6$ \hfill [1 décima]
    \begin{malla}[height=2cm,enlarge top by=10pt]
    \end{malla}
    \task 
        Agustín utiliza su bicicleta para hacer deporte. Cada día recorre 12 km en la mañana y 5 km en la tarde.
        ¿Cuántos kilómetros recorre en total al cabo de 4 días?. \hfill [1 décima]
        \begin{malla}[height=2cm,enlarge top by=10pt]
        \end{malla}
        \begin{respuesta}[height=1cm,enlarge top by=5pt]
        \end{respuesta}
    \task 
        La era de los romanos empieza en el año 754 antes de Cristo y la de los musulmanes en el año 622 después de Cristo. 
        ¿Cuántos años transcurrieron desde el comienzo de la era romana hasta el comienzo de la era musulmana?. \hfill [1 décima]
        \begin{malla}[height=2cm,enlarge top by=10pt]
        \end{malla}
        \begin{respuesta}[height=1cm,enlarge top by=5pt]
        \end{respuesta}
\end{ejercicios}

\end{document}