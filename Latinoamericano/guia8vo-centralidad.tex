
\def\colegio{Colegio Latinoamericano de Integración}
\def\titulo{Guía}
\def\subtitulo{Medidas de centralidad}
\def\curso{Octavo Básico}
%\def\puntaje{30}

\documentclass[sin nombre]{plantilla-evaluacion-v1}

\begin{python}
import numpy as np

def divisores(n):
    div = [i for i in range(1, int(n ** 0.5) + 1) if n % i == 0]
    div.extend([n // i for i in div if i != n // i])
    return np.array(sorted(div))

def n_barras(lista, ideal):
  if (lista is None or len(lista) == 0):
    return 0
  min = np.min(lista)
  max = np.max(lista) + 1
  rango = np.abs(max-min)
  div_rango = divisores(rango).astype(int)
  n_bars = (rango/div_rango).astype(int)
  distancia = np.abs(n_bars - ideal)
  return n_bars[distancia.argmin()]

def ticks(muestras):
  t = np.linspace(
    np.min(muestras),
    np.max(muestras)+1,
    n_barras(muestras,5)+1
  ).astype(int)
  return ",".join(t.astype(str))

def obtener_muestras(m,d,n,s=None):
  l = np.random.default_rng(seed=s).normal(m,d,size=n)
  return np.floor(l).astype(int)

#muestras_h = np.array([160, 162, 160, 153, 153, 154, 157, 160, 157, 159, 155, 159, 165, 163,
#  154, 165, 159, 159, 157, 156, 163, 157, 156, 155, 156, 154, 158, 159, 154, 160, 161, 153,
#  158, 159, 164, 157, 155, 155, 157, 160, 154, 152, 166, 163, 158, 162, 155, 160, 157, 159],
#  dtype=int)
#while (n_barras(muestras_m,5) != 5):
muestras_m = obtener_muestras(155,3,45)

#muestras_m = np.array([153, 156, 152, 153, 152, 153, 153, 153, 153, 152, 155, 155, 152, 158,
# 153, 157, 148, 153, 154, 152, 148, 158, 151, 156, 154, 150, 154, 157, 148, 155, 150, 157,
# 153, 156, 154, 160, 153, 151, 154, 154, 162, 151, 157, 151, 155], dtype=int)
#while (n_barras(muestras_h,5) != 5):
muestras_h = obtener_muestras(158,3,50)

@np.vectorize(excluded=['lista'])
def frecuencia(x, lista):
  filtrado = np.extract(lista == x,lista)
  return len(filtrado)

@np.vectorize(excluded=['lista'])
def frecuencia_acumulada(x, lista):
  filtrado = np.extract(lista <= x, lista)
  return len(filtrado)

@np.vectorize(excluded=['lista'])
def probabilidad(x, lista):
  return frecuencia(x, lista=lista)/len(lista)

@np.vectorize(excluded=['lista'])
def probabilidad_acumulada(x, lista):
  return frecuencia_acumulada(x, lista=lista)/len(lista)

@np.vectorize(excluded=['lista'])
def cuantil(p, lista):
  unicos = np.unique_values(lista)
  dato = 0
  for i in unicos:
    p_acumulada = probabilidad_acumulada(i, lista=lista)
    if (p_acumulada >= p):
      dato = i
      break
  return dato

def pgf_coords(x,y):
  return " ".join(f"({i},{j})" for i, j in zip(x,y))

x_h = np.unique_values(muestras_h)
y_h = frecuencia(x_h,lista=muestras_h)

x_m = np.unique_values(muestras_m)
y_m = frecuencia(x_m,lista=muestras_m)

np.savetxt("muestras_h.csv",muestras_h,delimiter=",")
np.savetxt("muestras_m.csv",muestras_m,delimiter=",")

\end{python}

\begin{document}

A continuación se encuentran las alturas, en centímetros, de un grupo de niñas y otro de
niños. \par

\begin{center}
  \begin{tikzpicture}[baseline=(current axis.north)]
    \begin{axis}[
        ybar,
        title={Gráfico de barras (Niñas)},
        ylabel={Frecuencia},
        xlabel={Altura (cm)},
        xtick distance=3,
        ymin=0,
        %xtick=data,
        %nodes~near~coords,
        %nodes~near~coords~align={vertical},
        ]
        \addplot [ybar,bar width=1,pattern={Lines[angle=-45,distance=5pt]}]
         coordinates {\py{pgf_coords(x_m,y_m)}};
    \end{axis}
  \end{tikzpicture}\hspace*{1cm}%
  \begin{tikzpicture}[baseline=(current axis.north)]
    \begin{axis}[
        ybar,
        title={Gráfico de barras (Niños)},
        ylabel={Frecuencia},
        xlabel={Altura (cm)},
        xtick distance=3,
        ymin=0,
        %xtick=data,
        %nodes~near~coords,
        %nodes~near~coords~align={vertical},
        ]
        \addplot [ybar,bar width=1,pattern={Lines[angle=-45,distance=5pt]}]
         coordinates {\py{pgf_coords(x_h,y_h)}};
    \end{axis}
  \end{tikzpicture}
\end{center}

Traspase los datos desde los gráficos de barras a un histograma. \par

\begin{center}
  \begin{tikzpicture}[baseline=(current axis.north)]
    \begin{axis}[
        ybar interval,
        title={Histograma (Niñas)},
        ylabel={Frecuencia},
        xlabel={Altura (cm)},
        xtick = {\py{ticks(muestras_m)}},
        ymin=0,
        grid=major,
        xmin={\py{np.min(muestras_m)}},
        xmax={\py{np.max(muestras_m)+1}},
        enlarge x limits,
        xticklabel= $[\pgfmathprintnumber\tick-\pgfmathprintnumber\nexttick[$,
        x tick label style={rotate=45,anchor=east},
        ]
        \addplot [draw=none,hist={bins=\py{n_barras(muestras_m,5)}}]
         table[y index=0] {muestras_m.csv};
    \end{axis}
  \end{tikzpicture}\hspace*{1cm}
  \begin{tikzpicture}[baseline=(current axis.north)]
    \begin{axis}[
        ybar interval,
        title={Histograma (Niños)},
        ylabel={Frecuencia},
        xlabel={Altura (cm)},
        ymin=0,
        grid=major,
        xmin={\py{np.min(muestras_h)}},
        xmax={\py{np.max(muestras_h)+1}},
        enlarge x limits,
        xticklabel= $[\pgfmathprintnumber\tick-\pgfmathprintnumber\nexttick[$,
        x tick label style={rotate=45,anchor=east}
        %xtick=data,
        %nodes near coords,
        %nodes near coords align={vertical},
        ]
        \addplot [draw=none,hist={bins=\py{n_barras(muestras_h,5)}}]
         table[y index=0] {muestras_h.csv};
    \end{axis}
  \end{tikzpicture}
\end{center} \par

\section*{Medidas de centralidad}

Las medidas de centralidad resumen todos los datos en un solo valor numérico e intentan
comunicar cuales son los valores más frecuentes dentro de los datos, no siempre
funcionan, y es la razón por la cual tenemos tres medidas que intentan hacer lo mismo
de tres maneras distintas. \par

\negrita{Media}: Es el promedio de los datos. Por ejemplo, para los datos: 3, 4, 4 y 5;
el promedio se calcula como $(3+4\times2+5)\div4=\pgfmathprintnumber{\py{(3+8+5)/4}}$.
¿Logra el promedio en este caso decirnos cual es valor más frecuente o más probable? \par

Un ejemplo donde el promedio no funciona bien, es con los datos: 1, 1, 7 y 7.
Aquí el promedio es $(1\times2 + 7\times2)=4$, pero no hay ningún cuatro en los datos. \par

\negrita{Moda}: Es el dato con mayor frecuencia. En el ejemplo: 3, 4, 4 y 5; la moda
es 4. Por otro lado, en: 1, 1, 7, y 7; la moda es 1 o 7, ya que tienen
la misma frecuencia.\par

\negrita{Mediana}: Es el dato más pequeño con una probabilidad acumulada mayor o igual
a 0,5. \par

En el ejemplo con los datos: 3, 4, 4 y 5. No hay datos con probabilidad acumulada igual
a 0,5 y el que está más cerca es el 4 con probabilidad acumulada 0,750. Así, en
este caso la mediana es 4. \par

\begin{python}
import numpy as np
import polars as pl
def generar_tabla(archivo,datos):
  tabla = pl.DataFrame({'valores': np.floor(datos).astype(int)})
  tabla = tabla.group_by('valores').len().sort('valores')
  tabla = tabla.rename({'len':'frecuencia'})
  tabla = tabla.with_columns(
    pl.col('frecuencia').cum_sum().alias('f_cumulativa'),
    (pl.col('frecuencia')/pl.col('frecuencia').sum()).alias('probabilidad'),
  )
  tabla = tabla.with_columns(
    pl.col('probabilidad').cum_sum().alias('p_acumulada')
  )
  tabla.write_csv(archivo,include_header=False)

generar_tabla("ejemplo1.csv",[3,4,4,5])
\end{python}

\begingroup
\csvreader[no head,
before reading=\pgfkeys{/pgf/number format/.cd,fixed,fixed zerofill,precision=3,verbatim,use comma},
centered tabularray={
cells={valign=m},
colspec={X[1,c]X[2,c]X[2,c]X[2,c]X[2,c]},
width=0.7\linewidth,
hlines,
vlines,
hline{1,2,Z}={black,1pt},
rows={rowsep+=2pt},
},
table head={Datos & Frecuencia & {Frecuencia\\Acumulada} & Probabilidad & {Probabilidad\\Acumulada} \\}
]{ejemplo1.csv}{1=\a, 2=\b, 3=\c, 4=\d, 5=\e}%
{\a & \b & \c & \pgfmathprintnumber{\d}  & \pgfmathprintnumber{\e}}
\endgroup

Para los datos: 1, 1, 7 y 7. Existe un dato con probabilidad acumulada 0,500, por lo que
la mediana en este caso es 1. \par

\begin{python}
generar_tabla("ejemplo2.csv",[1,1,7,7])
\end{python}

\begingroup
\csvreader[no head,
before reading=\pgfkeys{/pgf/number format/.cd,fixed,fixed zerofill,precision=3,verbatim,use comma},
centered tabularray={
cells={valign=m},
colspec={X[1,c]X[2,c]X[2,c]X[2,c]X[2,c]},
width=0.7\linewidth,
hlines,
vlines,
hline{1,2,Z}={black,1pt},
rows={rowsep+=2pt},
},
table head={Datos & Frecuencia & {Frecuencia\\Acumulada} & Probabilidad & {Probabilidad\\Acumulada} \\}
]{ejemplo2.csv}{1=\a, 2=\b, 3=\c, 4=\d, 5=\e}%
{\a & \b & \c & \pgfmathprintnumber{\d}  & \pgfmathprintnumber{\e}}
\endgroup

Utiliza los datos que se encuentran en los gráficos de barra para calcular las
medidas de centralidad de cada grupo.\par

\begin{center}
  \begin{tblr}{colspec={ccc},vlines,hlines,rows={rowsep+=3pt},columns={colsep+=20pt}}
          &  Niñas    &   Niños   \\
  Media   &           &           \\
  Moda    &           &           \\
  Mediana &           &           \\
  \end{tblr}
\end{center} \par

¿Qué grupo tiene la media más alta? ¿Qué sugiere que un grupo tenga una media
más alta que el otro?
\begin{respuesta}[height=2cm]
\end{respuesta}
¿Qué ocurre con la moda y la mediana de ambos grupos? ¿Sugieren lo mismo que la media?
\begin{respuesta}[height=2cm]
\end{respuesta}

\newpage

\begin{minipage}[c][\lineskip+20pt][c]{\linewidth}
  \centering
  \sffamily\Large {\scshape\bfseries Respuestas} - \subtitulo
\end{minipage}

\begin{center}
  \begin{tikzpicture}[baseline=(current axis.north)]
    \begin{axis}[
        ybar interval,
        title={Histograma (Niñas)},
        ylabel={Frecuencia},
        xlabel={Altura (cm)},
        xtick = {\py{ticks(muestras_m)}},
        ymin=0,
        grid=none,
        xmin={\py{np.min(muestras_m)}},
        xmax={\py{np.max(muestras_m)+1}},
        enlarge x limits,
        xticklabel= $[\pgfmathprintnumber\tick-\pgfmathprintnumber\nexttick[$,
        x tick label style={rotate=45,anchor=east},
        %xtick=data,
        %nodes near coords,
        %nodes near coords align={vertical},
        ]
        \addplot [hist={bins=\py{n_barras(muestras_m,5)}},pattern={Lines[angle=-45,distance=5pt]}]
         table[y index=0] {muestras_m.csv};
    \end{axis}
  \end{tikzpicture}\hspace*{1cm}
  \begin{tikzpicture}[baseline=(current axis.north)]
    \begin{axis}[
        ybar interval,
        title={Histograma (Niños)},
        ylabel={Frecuencia},
        xlabel={Altura (cm)},
        ymin=0,
        grid=none,
        xmin={\py{np.min(muestras_h)}},
        xmax={\py{np.max(muestras_h)+1}},
        enlarge x limits,
        xticklabel= $[\pgfmathprintnumber\tick-\pgfmathprintnumber\nexttick[$,
        x tick label style={rotate=45,anchor=east}
        %xtick=data,
        %nodes near coords,
        %nodes near coords align={vertical},
        ]
        \addplot [hist={bins=\py{n_barras(muestras_h,5)}},pattern={Lines[angle=-45,distance=5pt]}]
         table[y index=0] {muestras_h.csv};
    \end{axis}
  \end{tikzpicture}
\end{center} \par

\begin{center}
  \begin{tblr}{colspec={ccc},vlines,hlines,rows={rowsep+=3pt},columns={colsep+=20pt},
    cell{2-Z}{2-Z}={cmd=\pgfmathprintnumber}}
          &  Niñas    &   Niños   \\
  Media   & \py{np.mean(muestras_m)} & \py{np.mean(muestras_h)} \\
  Moda    & \py{np.bincount(muestras_m).argmax()} & \py{np.bincount(muestras_h).argmax()} \\
  Mediana & \py{cuantil(0.5,lista=muestras_m)} & \py{cuantil(0.5,lista=muestras_h)} \\
  \end{tblr} \par
\end{center}

% \begin{center}
%   \begin{tikzpicture}[baseline=(current axis.north)]
%     \begin{axis}[
%       title={Comparación de las alturas por grupo},
%       xlabel={Altura (cm)},
%       ytick={1,2},
%       yticklabels={Niñas,Niños},
%     ]
%       \addplot+ [boxplot={
%         draw position=1,
%         whisker range=10,
%       }] table[y index=0] {muestras_m.csv};
%       \addplot+ [boxplot={
%         draw position=2,
%         whisker range=10,
%       },solid] table[y index=0] {muestras_h.csv};

%     \end{axis}
%   \end{tikzpicture}
% \end{center}

\end{document}