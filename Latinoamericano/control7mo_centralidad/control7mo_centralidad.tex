\def\colegio{Colegio Latinoamericano de Integración}
\def\titulo{Control}
\providecommand{\forma}{XX}
\def\subtitulo{Medidas de centralidad (forma \forma)}
\def\curso{Séptimo Básico}
% 3*5 + 2 + 4
\def\puntaje{21}

\documentclass[sin curso]{plantilla-evaluacion-v1}

\begin{python}
import numpy as np

def divisores(n):
    div = [i for i in range(1, int(n ** 0.5) + 1) if n % i == 0]
    div.extend([n // i for i in div if i != n // i])
    return np.array(sorted(div))

def n_barras(lista, ideal):
  if (lista is None or len(lista) == 0):
    return 0
  min = np.min(lista)
  max = np.max(lista) + 1
  rango = np.abs(max-min)
  div_rango = divisores(rango).astype(int)
  n_bars = (rango/div_rango).astype(int)
  distancia = np.abs(n_bars - ideal)
  return n_bars[distancia.argmin()]

def ticks(muestras):
  t = np.linspace(
    np.min(muestras),
    np.max(muestras)+1,
    n_barras(muestras,5)+1
  ).astype(int)
  return ",".join(t.astype(str))

def obtener_muestras(m,d,n,s=None):
  l = np.random.default_rng(seed=s).normal(m,d,size=n)
  return np.floor(l).astype(int)

muestras_m = obtener_muestras(155,3,45)
muestras_h = obtener_muestras(158,3,50)

@np.vectorize(excluded=['lista'])
def frecuencia(x, lista):
  filtrado = np.extract(lista == x,lista)
  return len(filtrado)

@np.vectorize(excluded=['lista'])
def frecuencia_acumulada(x, lista):
  filtrado = np.extract(lista <= x, lista)
  return len(filtrado)

@np.vectorize(excluded=['lista'])
def probabilidad(x, lista):
  return frecuencia(x, lista=lista)/len(lista)

@np.vectorize(excluded=['lista'])
def probabilidad_acumulada(x, lista):
  return frecuencia_acumulada(x, lista=lista)/len(lista)

@np.vectorize(excluded=['lista'])
def cuantil(p, lista):
  unicos = np.unique_values(lista)
  dato = 0
  for i in unicos:
    p_acumulada = probabilidad_acumulada(i, lista=lista)
    if (p_acumulada >= p):
      dato = i
      break
  return dato

def moda(lista):
  valores, conteo = np.unique_counts(lista)
  indice_max = np.argwhere(conteo == np.max(conteo))
  return valores[indice_max].flatten()

def pgf_coords(x,y):
  return " ".join(f"({i},{j})" for i, j in zip(x,y))

def imprimir(l):
  return " • ".join(l.astype(str))

while(True):
  m1 = obtener_muestras(10,2,7)
  if (np.min(m1)>0 and len(moda(m1))==1):
    break

while(True):
  m2 = obtener_muestras(14,3,14)
  if (np.min(m2)>0 and len(moda(m2))==1):
    break

while(True):
  m3 = obtener_muestras(14,3,21)
  if (np.min(m3)>0 and len(moda(m3))==1):
    break

while(True):
  muestrasH = obtener_muestras(16,3.5,23)
  if (np.min(muestrasH)>0 and len(np.unique_values(muestrasH))<=10 and len(moda(muestrasH))==1):
    break

while(True):
  muestrasM = obtener_muestras(18,3.5,26)
  if (np.min(muestrasM)>0 and n_barras(muestrasM,5) in [4,5,6] and len(moda(muestrasM))==1):
    break

np.savetxt("tablaH.csv",np.column_stack([
  x := (np.unique_values(muestrasH)),
  frecuencia(x,lista=muestrasH),
  frecuencia_acumulada(x,lista=muestrasH),
  probabilidad(x,lista=muestrasH),
  probabilidad_acumulada(x,lista=muestrasH)
  ]),
  fmt=["%d","%d","%d","%.4f","%.4f"],
  delimiter=","
)

np.savetxt("muestrasH.csv",muestrasH,delimiter=",")
np.savetxt("muestrasM.csv",muestrasM,delimiter=",")
\end{python}

\newsavebox{\tabla}
\begin{lrbox}{\tabla}
  \begin{tblr}{colspec={ccc},vlines,hlines,rowsep=6pt,colsep=10pt,hline{1,2,Z}={black,1pt}}
    Media & Mediana & Moda \\
          &         &      \\
  \end{tblr}
\end{lrbox}

\NewDocumentCommand{\generarTabla}{m}{%
\begin{tblr}{colspec={ccc},vlines,hlines,rowsep=6pt,colsep=10pt,hline{1,2,Z}={black,1pt}}
  Media & Mediana & Moda \\
  \pgfmathprintnumber[verbatim,use comma]{\py{np.mean(#1)}} & \py{cuantil(0.5,lista=#1)} & \py{imprimir(moda(lista=#1))} \\
\end{tblr}}

\begin{document}
Utiliza los datos entregados para completar las tablas con los valores indicados. [3p c/u]
\begin{preguntas}[after-item-skip=15pt]
  \pregunta Datos: \py{imprimir(m1)}
  \begin{malla}[height=3cm]
  \end{malla}
  \usebox{\tabla}

  \pregunta Datos: \py{imprimir(m2)}
  \begin{malla}[height=3cm]
  \end{malla}
  \usebox{\tabla}

  \pregunta Datos: \py{imprimir(m3)}
  \begin{malla}[height=3cm]
  \end{malla}
  \usebox{\tabla}
\end{preguntas}

\section*{Comparando el número de amistades por género}
A continuación, se encuentran los resultados de encuestar a un grupo de estudiantes y
preguntarles a cada uno: \textbf{¿Cuántos amig@s tienes?}. \par

Los resultados se encuentran separados por género. Para los niños, los datos se encuentran
en una tabla de frecuencias; y para las niñas, los datos se encuentran en un gráfico de barras.
Utilice estos valores para completar la tabla de valores de cada género, y así finalmente
determinar que grupo generalmente tiene más amistades.

\NewDocumentCommand{\formatear}{m}{%
\pgfmathprintnumber[fixed,fixed zerofill,precision=3,verbatim,use comma]{#1}%
}
\csvreader[no head,
centered tabularray={
cells={valign=m},
colspec={X[1,c]X[2,c]X[2,c]X[2,c]X[2,c]},
width=0.85\linewidth,
hlines,
vlines,
hline{1,2,3,Z}={black,1pt},
rows={rowsep+=2pt},
cell{1}{1}={r=1,c=5}{c}
},
table head={Número de amig@s (Niños) & F & FA & P & PA \\ Datos & Frecuencia &
  {Frecuencia\\Acumulada} & Probabilidad & {Probabilidad\\Acumulada} \\}
]{tablaH.csv}{1=\a, 2=\b, 3=\c, 4=\d, 5=\e}%
{\a & \b & \c & \formatear{\d} & \formatear{\e} }

\begin{preguntas}(1)
  \pregunta Completa la tabla de valores usando las respuestas de los niños. [3p] \\[5pt]
  \usebox{\tabla}
\end{preguntas}%
%
\begin{python}
  x_m = np.unique_values(muestrasM)
  y_m = frecuencia(x_m,lista=muestrasM)
\end{python}%
%
\begin{center}
  \begin{tikzpicture}[baseline=(current axis.north)]
    \begin{axis}[
        ybar,
        title={Respuestas de las niñas},
        ylabel={Frecuencia},
        xlabel={Número de amig@s},
        xtick distance=3,
        ymin=0,
        %xtick=data,
        %nodes~near~coords,
        %nodes~near~coords~align={vertical},
        ]
        \addplot [ybar,bar width=1,pattern={Lines[angle=-45,distance=5pt]}]
         coordinates {\py{pgf_coords(x_m,y_m)}};
    \end{axis}
  \end{tikzpicture}
\end{center}

\begin{preguntas}[after-item-skip=20pt](1)
  \pregunta Rellena el siguiente histograma con \textbf{las respuestas de las niñas} y coloca
  la frecuencia arriba de cada una de las barras. [2p] \\
  \begin{tikzpicture}[baseline=(current axis.north)]
    \begin{axis}[
        ybar interval,
        %title={Histograma (Niñas)},
        ylabel={Frecuencia},
        xlabel={Número de amig@s},
        xtick = {\py{ticks(muestrasM)}},
        ymin=0,
        grid=major,
        xmin={\py{np.min(muestrasM)}},
        xmax={\py{np.max(muestrasM)+1}},
        enlarge x limits,
        xticklabel= $[\pgfmathprintnumber\tick-\pgfmathprintnumber\nexttick[$,
        x tick label style={rotate=45,anchor=east},
        ]
        \addplot [draw=none,hist={bins=\py{n_barras(muestrasM,5)}}]
        table[y index=0] {muestrasM.csv};
    \end{axis}
  \end{tikzpicture}

  \pregunta Completa la tabla de valores usando las respuestas de las niñas. [3p]
  \begin{malla}[height=2cm]
  \end{malla}
  \usebox{\tabla}

  \pregunta Comparando las medidas de centralidad para ambos grupos de niños y niñas,
  ¿Qué grupo tiene en general más amistades?. [4p]
  \begin{respuesta}[height=5cm]
  \end{respuesta}
\end{preguntas}

\newpage
\respuestas


\begin{preguntas}[after-item-skip=15pt,resume=false](1)
  \pregunta \generarTabla{m1}
  \pregunta \generarTabla{m2}
  \pregunta \generarTabla{m3}
  \pregunta \generarTabla{muestrasH}
  \pregunta
  \begin{tikzpicture}[baseline=(current axis.north)]
    \begin{axis}[
        ybar interval,
        title={Respuestas de las niñas},
        ylabel={Frecuencia},
        xlabel={Número de amig@s},
        xtick = {\py{ticks(muestrasM)}},
        ymin=0,
        grid=major,
        xmin={\py{np.min(muestrasM)}},
        xmax={\py{np.max(muestrasM)+1}},
        enlarge x limits,
        xticklabel= $[\pgfmathprintnumber\tick-\pgfmathprintnumber\nexttick[$,
        x tick label style={rotate=45,anchor=east},
        ]
        \addplot [hist={bins=\py{n_barras(muestrasM,5)}},pattern={Lines[angle=-45,distance=5pt]}]
         table[y index=0] {muestrasM.csv};
    \end{axis}
  \end{tikzpicture}
  \pregunta \generarTabla{muestrasM}

\end{preguntas}

\end{document}
