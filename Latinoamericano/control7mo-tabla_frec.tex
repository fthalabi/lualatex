\def\colegio{Colegio Latinoamericano de Integración}
\def\titulo{Evaluación Recuperativa}
\def\subtitulo{Tabla de frecuencias}
\def\curso{Séptimo Básico}
\def\puntaje{24}

\documentclass[sin curso]{plantilla-evaluacion-v1}

\begin{python}
import numpy as np
import polars as pl

def divisores(n):
    div = [i for i in range(1, int(n ** 0.5) + 1) if n % i == 0]
    div.extend([n // i for i in div if i != n // i])
    return np.array(sorted(div))

def n_barras(lista, ideal):
  if (lista is None or len(lista) == 0):
    return 0
  min = np.min(lista)
  max = np.max(lista) + 1
  rango = np.abs(max-min)
  div_rango = divisores(rango).astype(int)
  n_bars = (rango/div_rango).astype(int)
  distancia = np.abs(n_bars - ideal)
  return n_bars[distancia.argmin()]

def ticks(muestras):
  t = np.linspace(
    np.min(muestras),
    np.max(muestras)+1,
    n_barras(muestras,5)+1
  ).astype(int)
  return ",".join(t.astype(str))

def obtener_muestras(m,d,n):
  l = np.random.default_rng().normal(m,d,size=n)
  return np.floor(l).astype(int)

@np.vectorize(excluded=['lista'])
def frecuencia(x, lista):
  filtrado = np.extract(lista == x,lista)
  return len(filtrado)

@np.vectorize(excluded=['lista'])
def frecuencia_acumulada(x, lista):
  filtrado = np.extract(lista <= x, lista)
  return len(filtrado)

@np.vectorize(excluded=['lista'])
def probabilidad(x, lista):
  return frecuencia(x, lista=lista)/len(lista)

@np.vectorize(excluded=['lista'])
def probabilidad_acumulada(x, lista):
  return frecuencia_acumulada(x, lista=lista)/len(lista)

@np.vectorize(excluded=['lista'])
def cuantil(p, lista):
  unicos = np.unique_values(lista)
  dato = 0
  for i in unicos:
    p_acumulada = probabilidad_acumulada(i, lista=lista)
    if (p_acumulada >= p):
      dato = i
      break
  return dato

def pgf_coords(x,y):
  return " ".join(f"({i},{j})" for i, j in zip(x,y))


muestras = obtener_muestras(15,3,25)

def generar_tabla(archivo,datos):
  tabla = pl.DataFrame({'valores': np.floor(datos).astype(int)})
  tabla = tabla.group_by('valores').len().sort('valores')
  tabla = tabla.rename({'len':'frecuencia'})
  tabla = tabla.with_columns(
    pl.col('frecuencia').cum_sum().alias('f_cumulativa'),
    (pl.col('frecuencia')/pl.col('frecuencia').sum()).alias('probabilidad'),
  )
  tabla = tabla.with_columns(
    pl.col('probabilidad').cum_sum().alias('p_acumulada')
  )
  tabla.write_csv(archivo,include_header=False)
generar_tabla(archivo='tabla_frecuencia.csv', datos=muestras)

def muestra_aleatoria(muestras):
  rango = np.unique_values(muestras)[1:-1]
  return np.random.choice(rango)

m_f = muestra_aleatoria(muestras)
m_fa = muestra_aleatoria(muestras)
m_p = muestra_aleatoria(muestras)
m_pa = muestra_aleatoria(muestras)

#frec = frecuencia(datos,lista=muestras)
#frec_acum = frecuencia_acumulada(datos,lista=muestras)
#prob = probabilidad(datos,lista=muestras)
#prob_acum = probabilidad_acumulada(datos,lista=muestras)
#np.savetxt("tabla_frecuencias2.csv",np.column_stack([datos,frec,frec_acum,prob,prob_acum]),delimiter=",")

\end{python}

\NewDocumentCommand{\f}{m}{\pgfmathprintnumber[fixed,fixed zerofill,precision=3,verbatim,use comma]{#1}}

\begin{document}

A continuación se presentan los resultados al encuestar un grupo de estudiantes. \par
\negrita{Pregunta}: ¿Cuántas horas usas tu celular dentro de una semana? \par
\negrita{Respuestas}: \py{", ".join(muestras.astype(str))}

\begin{preguntas}(1)
  \pregunta Utilice las respuestas de los encuestados para rellenar la tabla
  de frecuencias. \newline
  \csvreader[no head,
  before reading=\pgfkeys{/pgf/number format/.cd,fixed,fixed zerofill,precision=3,verbatim,use comma},
  centered tabularray={
  cells={valign=m},
  colspec={X[1,c]X[2,c]X[2,c]X[2,c]X[2,c]},
  width=0.7\linewidth,
  hlines,
  vlines,
  hline{1,2,Z}={black,1pt},
  rows={rowsep+=2pt},
  cell{2-Z}{1-Z}={fg=white}
},
  table head={Datos & Frecuencia & {Frecuencia\\Acumulada} & Probabilidad & {Probabilidad\\Acumulada} \\}
]{tabla_frecuencia.csv}{1=\a, 2=\b, 3=\c, 4=\d, 5=\e}%
{\a & \b & \c & \pgfmathprintnumber{\d}  & \pgfmathprintnumber{\e}}
\end{preguntas}

\begin{importante}
  Contextualice y explique el significado de los siguientes valores que
  aparecen en la tabla de frecuencias.
\end{importante}

\begin{preguntas}(1)
  \pregunta ¿Qué significa que al dato \py{m_f} le corresponda la frecuencia
  \py{frecuencia(m_f,lista=muestras)}?
  \begin{respuesta}[height=4cm]
  \end{respuesta}
  \pregunta ¿Qué significa que al dato \py{m_fa} le corresponda la frecuencia
  acumulada \py{frecuencia_acumulada(m_fa,lista=muestras)}?
  \begin{respuesta}[height=4cm]
  \end{respuesta}
  \pregunta ¿Qué significa que al dato \py{m_p} le corresponda la probabilidad
  \f{\py{probabilidad(m_p,lista=muestras)}}?
  \begin{respuesta}[height=4cm]
  \end{respuesta}
  \pregunta ¿Qué significa que al dato \py{m_pa} le corresponda la probabilidad
  acumulada \f{\py{probabilidad_acumulada(m_pa,lista=muestras)}}?
  \begin{respuesta}[height=4cm]
  \end{respuesta}
\end{preguntas}

\newpage

\begin{minipage}[c][\lineskip+20pt][c]{\linewidth}
  \centering
  \sffamily\Large {\scshape\bfseries Respuestas} - \subtitulo
\end{minipage}

\begin{preguntas}[resume=false,after-item-skip=20pt]
  \pregunta \newline
  \csvreader[no head,
  before reading=\pgfkeys{/pgf/number format/.cd,fixed,fixed zerofill,precision=3,verbatim,use comma},
  centered tabularray={
  cells={valign=m},
  colspec={X[1,c]X[2,c]X[2,c]X[2,c]X[2,c]},
  width=0.7\linewidth,
  hlines,
  hline{1,2,Z}={black,1pt},
  rows={rowsep+=2pt},
  },
  table head={Datos & Frecuencia & {Frecuencia\\Acumulada} & Probabilidad & {Probabilidad\\Acumulada} \\}
  ]{tabla_frecuencia.csv}{1=\a, 2=\b, 3=\c, 4=\d, 5=\e}%
  {\a & \b & \c & \pgfmathprintnumber{\d}  & \pgfmathprintnumber{\e}}

  \pregunta Significa que hubo \py{frecuencia(m_f,lista=muestras)} estudiante(s)
  que usa(n) el celular \py{m_f} hora(s) dentro de una semana.

  \pregunta Significa que hubo \py{frecuencia_acumulada(m_fa,lista=muestras)} estudiante(s)
  que usan el celular \py{m_fa} o menos horas dentro de una semana. En otras palabras,
  \py{frecuencia_acumulada(m_fa,lista=muestras)} estudiante(s) usan el celular
  a lo más \py{m_fa} horas dentro de la semana.

  \pregunta Significa que el \py{np.int_(probabilidad(m_p,lista=muestras)*100)}\% de los
  estudiantes utiliza el celular \py{m_p} horas dentro de la semana.

  \pregunta Significa que el
  \py{np.int_(probabilidad_acumulada(m_pa,lista=muestras)*100)}\% de los
  estudiantes utiliza el celular \py{m_pa} o menos horas dentro de la semana. Es decir,
  el \py{np.int_(probabilidad_acumulada(m_pa,lista=muestras)*100)}\% de los estudiantes
  utiliza el celular a lo más \py{m_pa} horas dentro de la semana.

\end{preguntas}


\end{document}