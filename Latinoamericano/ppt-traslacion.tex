\def\colegio{Colegio Latinoamericano de Integración}
\def\titulo{Transformaciones isometricas}
\def\subtitulo{La traslación}

\documentclass[tema claro]{presentacion}

\usepackage{emoji}

\NewDocumentCommand{\corazon}{O{0}mm}{
  \addplot [
    samples=100,
    domain=-3:3,
    variable=\t,
    trig format plots=rad,
    rotate around={#1:(#2,#3)}
  ] ( {sin(t)^3 + #2},{(13*cos(t)-5*cos(2*t)-2*cos(3*t)-cos(4*t))/16 + 0.1 + #3} );
}

\NewDocumentCommand{\ticket}{mm}{\node[] at (axis cs: #1,#2) {\emoji{check-mark}};}
\NewDocumentCommand{\cruz}{mm}{\node[] at (axis cs: #1,#2) {\emoji{cross-mark}};}
\ExplSyntaxOn
\NewDocumentCommand{\flecha}{mmmm}{
  \draw[->,shorten~>=0.2cm,shorten~<=0.2cm] (#1,#2) -- (#3,#4)
    node[midway,above,fill=white,fill~opacity=1,text~opacity=1,inner~sep=1pt]
      {(\int_eval:n{#3-#1},\int_eval:n{#4-#2})};
}
\ExplSyntaxOff

\begin{document}

\begin{frame}{¿Qué es una traslación?}

\begin{center}
  \begin{tikzpicture}
    \begin{axis}[
        width = .7\linewidth,
        height = .7\linewidth,
        %axis equal,
        grid=major,
        major grid style={gray},
        xmin=0, xmax=10,
        ymin=0, ymax=10,
        xtick distance=1,         % Use tick distance instead of explicit ticks
        ytick distance=1,
        enlarge x limits=false,
        enlarge y limits=false,
        tick align=outside        % Try to align ticks differently
    ]
      \corazon{5}{5}
      \only<2->{\corazon{7}{7}}
      \only<2->{\ticket{7}{7}}
      \only<3->{\flecha{5}{5}{7}{7}}
      \only<4->{\corazon{3}{8}}
      \only<4->{\ticket{3}{8}}
      \only<5->{\flecha{5}{5}{3}{8}}
      \only<6->{\corazon[90]{2}{2}}
      \only<7->{\cruz{2}{2}}
    \end{axis}
  \end{tikzpicture}
\end{center}


\end{frame}

\end{document}