\def\colegio{Colegio Latinoamericano de Integración}
\def\titulo{Guía de trabajo}
\def\subtitulo{Resolución de ecuaciones con una incógnita}
\def\curso{Octavo Básico}

\documentclass[sin nombre]{plantilla-evaluacion-v1}

\begin{document}
\begin{partes}
  \parte Resuelve las siguientes ecuaciones.
  \begin{ejercicios}
    \ejercicio $3x-4 = -11$
    \begin{malla}[height=4cm]
    \end{malla}
    \ejercicio $3-x=3x$
    \begin{malla}[height=4cm]
    \end{malla}
    \ejercicio $4-\dfrac{x}{2} = \dfrac{18}{4}$
    \begin{malla}[height=4cm]
    \end{malla}
    \ejercicio $-x+11=-2x+6$
    \begin{malla}[height=4cm]
    \end{malla}
    \ejercicio $3(x-6)=2(9-3x)$
    \begin{malla}[height=4cm]
    \end{malla}
    \ejercicio $3x-4 = 6x+20$
    \begin{malla}[height=4cm]
    \end{malla}
    \ejercicio! $\dfrac{x}{2}= 1 -\dfrac{3x}{4}$
    \begin{malla}[height=4cm]
    \end{malla}
    \ejercicio! $3,5x+4=2,5x-5$
    \begin{malla}[height=4cm]
    \end{malla}
    \ejercicio! $2(x+7)=3(x-1)$
    \begin{malla}[height=4cm]
    \end{malla}
    \ejercicio! $\dfrac{x}{2}+\dfrac{7}{4}=\dfrac{3}{2}$
    \begin{malla}[height=4cm]
    \end{malla}
    \ejercicio! $\dfrac{3x}{10}-\dfrac{6}{5}=\dfrac{3}{5}$
    \begin{malla}[height=4cm]
    \end{malla}
  \end{ejercicios}

\end{partes}
\end{document}