\def\colegio{Colegio Latinoamericano de Integración}
\def\titulo{Guía}
%\def\subtitulo{Tabla de Frecuencias}
\def\curso{Octavo Básico}

\documentclass[sin nombre]{plantilla-evaluacion-v1}

\begin{document}

\begin{luacode}
  require('estadistica.lua')
  function generar_lista()
    repeat
        m = math.random(13,23)
        d = math.random(3,5)
        lista = Generar_muestras(30,m,d)
        v_unicos = Obtener_unicos(lista)
        divisiones = Divisor_ideal(v_unicos)
    until divisiones >2 and divisiones <7 and Rango(lista) < 12
    return lista
  end
  lista = generar_lista()
  divisiones = Divisor_ideal(Obtener_unicos(lista))
  function lista_a_clist()
    lista_str = table.concat(lista, ",")
    tex.sprint(lista_str)
  end
\end{luacode}


\ExplSyntaxOn
\clist_new:N \l_lista_lua
\clist_set:Nx \l_lista_lua {\directlua{lista_a_clist()}}
\seq_set_from_clist:NN \l_lista_seq \l_lista_lua
\def\datos{\seq_use:Nn \l_lista_seq {,~}}

%% ORDENAR LISTA
\seq_clear_new:N \l_lista_ordenada_seq
\seq_set_eq:NN \l_lista_ordenada_seq \l_lista_seq
\seq_sort:Nn \l_lista_ordenada_seq {
  \int_compare:nNnTF { #1 } > { #2 }
  { \sort_return_swapped: }
  { \sort_return_same: }
}

%% VALORES UNICOS
\seq_clear_new:N \l_valores_unicos_seq
\seq_set_eq:NN \l_valores_unicos_seq \l_lista_ordenada_seq
\seq_remove_duplicates:N \l_valores_unicos_seq

\cs_new_protected:Npn \cuantil:n #1 {
  \fp_zero_new:N \l_cd_fp_indice
  \fp_set:Nn \l_cd_fp_indice
    {\fp_eval:n {ceil(#1 * \seq_count:N \l_lista_seq ,0)}}
  \int_zero_new:N \l_cd_int_indice
  \int_set:Nn \l_cd_int_indice {\fp_to_int:N \l_cd_fp_indice}
  \tl_set:Nx \l_cd_valor {\seq_item:Nn \l_lista_ordenada_seq \l_cd_int_indice}
  \tl_use:N \l_cd_valor
}

%% PREGUNTAS AL AZAR
\int_gzero_new:N \l_p_prob_int
\int_gset:Nn \l_p_prob_int {\int_rand:nn {3} {\seq_count:N \l_valores_unicos_seq}}
\int_gzero_new:N \l_p_probAcum_int
\int_gset:Nn \l_p_probAcum_int {\int_rand:nn {3} {\seq_count:N \l_valores_unicos_seq}}
\int_gzero_new:N \l_p_frec_int
\int_gset:Nn \l_p_frec_int {\int_rand:nn {3} {\seq_count:N \l_valores_unicos_seq}}
\int_gzero_new:N \l_p_frecAcum_int
\int_gset:Nn \l_p_frecAcum_int {\seq_count:N \l_valores_unicos_seq}
\int_compare:nNnTF {\l_p_frecAcum_int} = {\seq_count:N \l_valores_unicos_seq}
{\int_gdecr:N \l_p_frecAcum_int} {}
\int_compare:nNnTF {\l_p_frecAcum_int} = {\seq_count:N \l_valores_unicos_seq}
{\int_gdecr:N \l_p_frecAcum_int} {}

\tl_clear_new:N \l_r_prob_tl
\tl_clear_new:N \l_r_probAcum_tl
\tl_clear_new:N \l_r_frec_tl
\tl_clear_new:N \l_r_frecAcum_tl


%% CREANDO LA TABLA

\int_gzero_new:N \l_frecuencia
\int_gzero_new:N \l_frecuencia_acumulada
\fp_gzero:N \l_probabilidad
\fp_gzero:N \l_probabilidad_acumulada

\iow_new:N \g_archivo_tabla
\iow_open:Nn \g_archivo_tabla { datos_tabla.tmp }
\seq_map_indexed_inline:Nn \l_valores_unicos_seq
  {
    \int_gzero:N \l_frecuencia
    \seq_map_inline:Nn \l_lista_seq {
      \int_compare:nT {#2=##1} {\int_gincr:N \l_frecuencia}
    }
    \int_gadd:Nn \l_frecuencia_acumulada {\int_use:N \l_frecuencia}
    \fp_gset:Nn \l_probabilidad {\int_use:N \l_frecuencia / \seq_count:N \l_lista_seq }
    \fp_gadd:Nn \l_probabilidad_acumulada {\fp_use:N \l_probabilidad}

    %% guardando los datos para despues
    \int_compare:nNnTF {\l_p_frec_int} = {#1}
    {\tl_gset:Nx \l_r_frec_tl {\int_use:N \l_frecuencia}} {}
    \int_compare:nNnTF {\l_p_frecAcum_int} = {#1}
    {\tl_gset:Nx \l_r_frecAcum_tl {\int_use:N \l_frecuencia_acumulada}} {}
    \int_compare:nNnTF {\l_p_prob_int} = {#1}
    {\tl_gset:Nx \l_r_prob_tl {\fp_use:N \l_probabilidad}} {}
    \int_compare:nNnTF {\l_p_probAcum_int} = {#1}
    {\tl_gset:Nx \l_r_probAcum_tl {\fp_use:N \l_probabilidad_acumulada}} {}
    %%

    \iow_now:Nx \g_archivo_tabla
      { #2 ~&~ \int_use:N \l_frecuencia ~&~ \int_use:N \l_frecuencia_acumulada ~&~
      \fp_use:N \l_probabilidad ~&~ \fp_use:N \l_probabilidad_acumulada ~\\~ }
  }
\iow_close:N \g_archivo_tabla

\iow_new:N \g_archivo_tabla_vacia
\iow_open:Nn \g_archivo_tabla_vacia { tabla_vacia.tmp }
\seq_map_indexed_inline:Nn \l_valores_unicos_seq
  {
    \tl_gclear_new:N \l_linea_tl
    \tl_gput_right:Nn \l_linea_tl {~~&~}

    \int_compare:nNnTF {\l_p_frec_int} = {#1}%
    {\tl_gput_right:Nn \l_linea_tl {~\casillabox[2pt][2pt]{\widthof{A}}{\bfseries\sffamily~A}~&}}
    {\tl_gput_right:Nn \l_linea_tl {~&~}}

    \int_compare:nNnTF {\l_p_frecAcum_int} = {#1}
    {\tl_gput_right:Nn \l_linea_tl {~\casillabox[2pt][2pt]{\widthof{A}}{\bfseries\sffamily~B}~&}}
    {\tl_gput_right:Nn \l_linea_tl {~&~}}

    \int_compare:nNnTF {\l_p_prob_int} = {#1}
    {\tl_gput_right:Nn \l_linea_tl {~\casillabox[2pt][2pt]{\widthof{A}}{\bfseries\sffamily~C}~&}}
    {\tl_gput_right:Nn \l_linea_tl {~&~}}

    \int_compare:nNnTF {\l_p_probAcum_int} = {#1}
    {\tl_gput_right:Nn \l_linea_tl {~\casillabox[2pt][2pt]{\widthof{A}}{\bfseries\sffamily~D}~\\}}
    {\tl_gput_right:Nn \l_linea_tl {~\\}}

    \iow_now:Nx \g_archivo_tabla_vacia {\tl_to_str:N \l_linea_tl}
  }
\iow_close:N \g_archivo_tabla_vacia

\ExplSyntaxOff

\NewDocumentCommand{\casillabox}{O{1pt}O{1pt}mm}{%
\tcbox[on line, boxsep=0pt, left=#1, right=#1, top=#2, bottom=#2,%
   width=1cm]{\makebox[#3][c]{#4}}
}


\begin{preguntas}(1)
  \pregunta Utiliza los datos entregados a continuación para completar la tabla de
  frecuencias.\par
  \vspace*{5pt}
  \textbf{Datos}: \datos. \par \vspace*{10pt}
  \begin{tblr}[evaluate=\fileInput]{colspec={X[1]X[2]X[2]X[2]X[2]},hlines,hline{1,2,Z}={black,1pt},
    row{1}={valign=m,c},rows={rowsep+=2pt},vlines,cell{2-Z}{1-Z}={r}}
    Datos & Frecuencia & {Frecuencia\\Acumulada} & Probabilidad & {Probabilidad\\Acumulada} \\
    \fileInput{tabla_vacia.tmp}
  \end{tblr}
\end{preguntas}

Describe e interpreta el significado de los valores numéricos encontrados en las casillas señaladas.
\begin{preguntas}(2)
  \pregunta Casilla \casillabox[2pt][2pt]{\widthof{A}}{\bfseries\sffamily A}
  \begin{respuesta}[height=3cm]
  \end{respuesta}
  \pregunta Casilla \casillabox[2pt][2pt]{\widthof{A}}{\bfseries\sffamily B}
  \begin{respuesta}[height=3cm]
  \end{respuesta}
  \pregunta Casilla \casillabox[2pt][2pt]{\widthof{A}}{\bfseries\sffamily C}
  \begin{respuesta}[height=3cm]
  \end{respuesta}
  \pregunta Casilla \casillabox[2pt][2pt]{\widthof{A}}{\bfseries\sffamily D}
  \begin{respuesta}[height=3cm]
  \end{respuesta}
\end{preguntas}


\newpage

\begin{minipage}[c][\lineskip+20pt][c]{\linewidth}
  \centering
  \sffamily\Large {\scshape\bfseries Respuestas} - \subtitulo
\end{minipage}

\ExplSyntaxOn
\edef\RespFrec{\tl_use:N \l_r_frec_tl}
\tl_clear_new:N \l_frec_dato_tl
\tl_set:Nx \l_frec_dato_tl {\seq_item:Nn \l_valores_unicos_seq {\int_use:N \l_p_frec_int}}
\edef\RespFrecDato{\tl_use:N \l_frec_dato_tl}
\edef\RespFrecAcum{\tl_use:N \l_r_frecAcum_tl}
\tl_clear_new:N \l_frecAcum_dato_tl
\tl_set:Nx \l_frecAcum_dato_tl {\seq_item:Nn \l_valores_unicos_seq {\int_use:N \l_p_frecAcum_int}}
\edef\RespFrecAcumDato{\tl_use:N \l_frecAcum_dato_tl}
\edef\RespProb{\tl_use:N \l_r_prob_tl}
\tl_clear_new:N \l_prob_dato_tl
\tl_set:Nx \l_prob_dato_tl {\seq_item:Nn \l_valores_unicos_seq {\int_use:N \l_p_prob_int}}
\edef\RespProbDato{\tl_use:N \l_prob_dato_tl}
\edef\RespProbAcum{\tl_use:N \l_r_probAcum_tl}
\tl_clear_new:N \l_probAcum_dato_tl
\tl_set:Nx \l_probAcum_dato_tl {\seq_item:Nn \l_valores_unicos_seq {\int_use:N \l_p_probAcum_int}}
\edef\RespProbAcumDato{\tl_use:N \l_probAcum_dato_tl}
\ExplSyntaxOff

\NewDocumentCommand{\porcentaje}{m}{\pgfmathtruncatemacro{\y}{round(#1*1000)/10}\y\%}

\begin{preguntas}[resume=false,after-item-skip=20pt]
  \pregunta Tabla completada \par \vspace*{10pt}
      \begingroup
      \pgfkeys{/pgf/number format/.cd,fixed,fixed zerofill,precision=3,verbatim,use comma}
      \begin{tblr}[evaluate=\fileInput]{colspec={ccccc},
        cell{2-Z}{4,5}={cmd=\pgfmathprintnumber},
        hlines,hline{1,2,Z}={black,1pt},
        row{1}={valign=m},rows={rowsep+=2pt}}
        Datos & Frecuencia & {Frecuencia\\Acumulada} & Probabilidad & {Probabilidad\\Acumulada} \\
        \fileInput{datos_tabla.tmp}
      \end{tblr}
      \endgroup
  \pregunta La Frecuencia es {\RespFrec}, lo que significa que hay {\RespFrec}
  datos con el valor {\RespFrecDato}.

  \pregunta La Frecuencia Acumulada es {\RespFrecAcum}, lo que significa que hay {\RespFrecAcum}
  datos que son menores o iguales a {\RespFrecAcumDato}.
  \pregunta La Probabilidad es
  \pgfmathprintnumber[fixed,fixed zerofill,precision=3,verbatim,use comma]{\RespProb},
  por lo tanto, el \porcentaje{\RespProb} de los datos tienen el valor {\RespProbDato}.
  \pregunta La Probabilidad Acumulada es
  \pgfmathprintnumber[fixed,fixed zerofill,precision=3,verbatim,use comma]{\RespProbAcum},
  es decir, el \porcentaje{\RespProbAcum} de los datos tiene un valor menor o igual a
  {\RespProbAcumDato}.
\end{preguntas}




\end{document}
