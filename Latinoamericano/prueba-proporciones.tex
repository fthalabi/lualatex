\def\colegio{Colegio Latinoamericano de Integración}
\def\titulo{Evaluación Sumativa}
\def\subtitulo{Proporciones}
\def\curso{Séptimo Básico}
% 2*6+2*8+2*6+7
\def\puntaje{47}

\documentclass[sin curso]{plantilla-evaluacion-v1}

\begin{document}

\section*{Objetivo de la evaluación}
(OA8) Mostrar que comprenden las proporciones directas e inversas
en sus diversas formas, y utilizan estos conceptos en la resolución de problemas
contextualizados.

\begin{center}
  \vspace*{1cm}
  \begin{tikzpicture}[ampersand replacement=\&,]
      %\node (A) [opacity=0.4] {\includegraphics[width=2cm]{../flork3.jpg}};
      \node (B) [font=\slshape,text width=12cm]
      {``Cree en ti mismo y en lo que eres. Sé consciente de que hay algo en tu interior %
      que es más grande que cualquier obstáculo''};
      \node [left=0mm of B,opacity=0.4] {\pgfornament[width=2cm]{37}};
      \node [right=0mm of B,opacity=0.4] {\pgfornament[width=2cm]{38}};
  \end{tikzpicture}
  \vspace*{1cm}
\end{center}

\section{Definir los conceptos}

Define y Describe los conceptos de proporción directa e inversa, según los criterios 
que se encuentran a continuación.

\begin{importante}
  \begin{tblr}{colspec={X[1,c,m]X[6,m]},hline{2-Y}={solid},vline{2-Y}={solid},
    cell{1}{2}={c},cell{2-Z}{2}={cmd=\raggedright}}
    Puntaje asignado & Criterios o indicadores \\
    +3 puntos & La definición deja claro como es el comportamiento de dos 
    cantidades que se dicen ser directa o inversamente proporcionales. \\
    +3 puntos & Describe como se identifica, más allá de toda duda, si dos cantidades
    son realmente proporcionales o no.\\
  \end{tblr}
\end{importante}

\begin{preguntas}
  \pregunta ¿Qué es una proporción directa? ¿Cómo se verifica que dos cantidades
  son directamente proporcionales?
  \begin{respuesta}[height=4cm]
  \end{respuesta}

  \pregunta ¿Qué es una proporción inversa? ¿Cómo se verifica que dos cantidades
  son inversamente proporcionales?
  \begin{respuesta}[height=4cm]
  \end{respuesta}
\end{preguntas}

\section{Reconocer y aplicar los conceptos}

Determina si las cantidades, descritas en las siguientes oraciones y tablas de valores, son proporcionales
o no y marca la alternativa correcta [2 puntos c/u].

\begin{preguntas}(2)
  \pregunta El valor de un kilogramo de cierto producto y la cantidad final a pagar.
  \begin{vertical}
    \alternativa Proporción directa.
    \alternativa Proporción inversa.
    \alternativa Ninguna de las anteriores.
  \end{vertical}
  \pregunta El caudal de una llave y el tiempo que se demora en llenar un estanque.
  \begin{vertical}
    \alternativa Proporción directa.
    \alternativa Proporción inversa.
    \alternativa Ninguna de las anteriores.
  \end{vertical}
  \pregunta La estatura de una persona y su masa corporal.
  \begin{vertical}
    \alternativa Proporción directa.
    \alternativa Proporción inversa.
    \alternativa Ninguna de las anteriores.
  \end{vertical}
  \pregunta La medida de los lados de un triángulo equilátero y su perímetro.
  \begin{vertical}
    \alternativa Proporción directa.
    \alternativa Proporción inversa.
    \alternativa Ninguna de las anteriores.
  \end{vertical}
  \pregunta
  \begin{tblr}{colspec={cc},row{1}={black!10},vlines,hlines,
    cells={mode=math},baseline=1}
    x & y \\
    9 & 1 \\
    3 & 3 \\
    -3 & -3 \\
  \end{tblr}
  \begin{vertical}
    \alternativa Proporción directa.
    \alternativa Proporción inversa.
    \alternativa Ninguna de las anteriores.
  \end{vertical}
  \pregunta
  \begin{tblr}{colspec={cc},row{1}={black!10},vlines,hlines,
    cells={mode=math},baseline=1}
    v & w \\
    0 & 0 \\
    5 & 100 \\
    20 & 400 \\
  \end{tblr}
  \begin{vertical}
    \alternativa Proporción directa.
    \alternativa Proporción inversa.
    \alternativa Ninguna de las anteriores.
  \end{vertical}
  \pregunta
  \begin{tblr}{colspec={cc},row{1}={black!10},vlines,hlines,
    cells={mode=math},baseline=1}
    j & k \\
    6 & 0,15 \\
    3 & 0,075 \\
    1 & 0,025 \\
  \end{tblr}
  \begin{vertical}
    \alternativa Proporción directa.
    \alternativa Proporción inversa.
    \alternativa Ninguna de las anteriores.
  \end{vertical}
  \pregunta
  \begin{tblr}{colspec={cc},row{1}={black!10},vlines,hlines,
    cells={mode=math},baseline=1}
    a & b \\
    5 & 15 \\
    10 & 10 \\
    15 & 5 \\
  \end{tblr}
  \begin{vertical}
    \alternativa Proporción directa.
    \alternativa Proporción inversa.
    \alternativa Ninguna de las anteriores.
  \end{vertical}
\end{preguntas}

Para las siguientes tablas de valores, encuentra el valor faltante en cada una y 
colócalo en la casilla correspondiente [2 punto c/u].

\begin{preguntas}(2)
  \pregunta
  \begin{tblr}{colspec={X[c]X[c]},width=0.6\linewidth,row{1}={black!10},vlines,hlines,
    row{2-Z}={mode=math},baseline=1,cell{1}{1}={c=2,r=1}{c}}
    Proporción Directa & \\
    h & 6 \\
    4 & 2 \\
  \end{tblr}
  \begin{caja}[height=35pt,title=$h$,hbox]
  \hspace*{1.5cm}
  \end{caja}
  \pregunta
  \begin{tblr}{colspec={X[c]X[c]},width=0.6\linewidth,row{1}={black!10},vlines,hlines,
    row{2-Z}={mode=math},baseline=1,cell{1}{1}={c=2,r=1}{c}}
    Proporción Directa & \\
    75 & 100 \\
    150 & p \\
  \end{tblr}
  \begin{caja}[height=35pt,title=$p$,hbox]
  \hspace*{1.5cm}
  \end{caja}
  \pregunta
  \begin{tblr}{colspec={X[c]X[c]},width=0.6\linewidth,row{1}={black!10},vlines,hlines,
    row{2-Z}={mode=math},baseline=1,cell{1}{1}={c=2,r=1}{c}}
    Proporción Directa & \\
    0,05 & 0,25 \\
    0,15 & k \\
  \end{tblr}
  \begin{caja}[height=35pt,title=$k$,hbox]
  \hspace*{1.5cm}
  \end{caja}
  \pregunta
  \begin{tblr}{colspec={X[c]X[c]},width=0.6\linewidth,row{1}={black!10},vlines,hlines,
    row{2-Z}={mode=math},baseline=1,cell{1}{1}={c=2,r=1}{c}}
    Proporción Inversa & \\
    2 & y \\
    4 & 5 \\
  \end{tblr}
  \begin{caja}[height=35pt,title=$y$,hbox]
  \hspace*{1.5cm}
  \end{caja}
  \pregunta
  \begin{tblr}{colspec={X[c]X[c]},width=0.6\linewidth,row{1}={black!10},vlines,hlines,
    row{2-Z}={mode=math},baseline=1,cell{1}{1}={c=2,r=1}{c}}
    Proporción Inversa & \\
    2159 & x \\
    127 & 34 \\
  \end{tblr}
  \begin{caja}[height=35pt,title=$x$,hbox]
  \hspace*{1.5cm}
  \end{caja}
  \pregunta
  \begin{tblr}{colspec={X[c]X[c]},width=0.6\linewidth,row{1}={black!10},vlines,hlines,
    row{2-Z}={mode=math},baseline=1,cell{1}{1}={c=2,r=1}{c}}
    Proporción Inversa & \\
    z & 0,14 \\
    0,17 & 0,21 \\
  \end{tblr}
  \begin{caja}[height=35pt,title=$z$,hbox]
  \hspace*{1.5cm}
  \end{caja}
  
\end{preguntas} % 0,17*0,21/0,14

\section{Resolución de problemas}

Lee los enunciados y encuentra la solución a cada una de las problemáticas. Cada 
pregunta tiene un puntaje diferente, pero en todos se distribuirá el puntaje 
utilizando los siguientes criterios.

\begin{importante}
  \begin{tblr}{colspec={X[1,c,m]X[6,m]},hline{2-Y}={solid},vline{2-Y}={solid},
    cell{1}{2}={c},cell{2-Z}{2}={cmd=\raggedright}}
    Puntaje asignado & Criterios o indicadores\\
    +50\% & Señala clara y correctamente cuál es la solución o el resultado de la pregunta hecha
    en el enunciado.\\
    +50\% & Incluye un desarrollo que relata de manera clara y ordenada los procedimientos 
    \mbox{necesarios} para solucionar la problemática. En caso de estar incompleto o con 
    \mbox{errores} el desarrollo, se asignará puntaje parcial si se muestra dominio de los 
    contenidos y conceptos involucrados.\\
    0\% &  La respuesta es incorrecta. De haber desarrollo, este tiene errores conceptuales.\\
  \end{tblr}
\end{importante}

\begin{preguntas}(1)

  \pregunta Una bodega se llena con 3500 sacos de 6 kg de papas cada uno y otra
  con la misma capacidad se llena con sacos de 5 kg, ¿cuántos sacos caben en la
  segunda bodega? [2 puntos]
  \begin{malla}[height=8cm]
  \end{malla}
  \begin{respuesta}[height=2cm]
  \end{respuesta}

  \pregunta Si un automóvil se demoró 9 horas durante un recorrido de 750 kilómetros,
  ¿qué tiempo empleará en recorrer 2250 kilómetros si su velocidad es constante?
  [2 puntos]
  \begin{malla}[height=8cm]
  \end{malla}
  \begin{respuesta}[height=2cm]
  \end{respuesta}

  \pregunta Si 4 hombres terminan un trabajo en 63 días, ¿cuántos más deben de
  añadirse a los primeros para concluir el mismo trabajo en 28 días?
  [2 puntos]
  \begin{malla}[height=8cm]
  \end{malla}
  \begin{respuesta}[height=2cm]
  \end{respuesta}

  \pregunta Andrea lee un libro de 500 páginas en 20 días y lee 1 hora diaria,
  ¿cuántos minutos debe leer diariamente para que en condiciones iguales lea un libro
  de 800 páginas en 15 días?
  [1 punto]
  \begin{malla}[height=8cm]
  \end{malla}
  \begin{respuesta}[height=2cm]
  \end{respuesta}

\end{preguntas}

\end{document}