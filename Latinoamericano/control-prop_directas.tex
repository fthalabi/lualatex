\def\colegio{Colegio Latinoamericano de Integración}
\def\titulo{Control}
\def\subtitulo{Proporciones directas}
\def\curso{Séptimo Básico}

\documentclass[sin curso]{plantilla-evaluacion-v1}

\begin{document}
\begin{partes}
  \parte Describe que es una proporción directa y que se debe cumplir para
  que se diga que dos cantidades son directamente proporcionales.\hfill [4 puntos]
  \begin{respuesta}[height=3.5cm]
  \end{respuesta}
  \parte Determina si las cantidades, descritas en las siguientes oraciones y 
  tablas de valores, son directamente proporcionales o no y marca la alternativa
  correcta.\hfill[2 puntos cada una]
  \begin{ejercicios}(2)
    \ejercicio La cantidad de agua que una planta recibe y el tamaño de la planta.
    \begin{vertical}
      \alternativa Son directamente proporcionales.
      \alternativa No lo son.
    \end{vertical}
    \ejercicio La cantidad de helado ingerida y las calorías consumidas.
    \begin{vertical}
      \alternativa Son directamente proporcionales.
      \alternativa No lo son.
    \end{vertical}
    \ejercicio El número de horas trabajadas y la remuneración total.
    \begin{vertical}
      \alternativa Son directamente proporcionales.
      \alternativa No lo son.
    \end{vertical}
    \ejercicio El número de libros leídos y el tamaño del vocabulario que se posee.
    \begin{vertical}
      \alternativa Son directamente proporcionales.
      \alternativa No lo son.
    \end{vertical}

    \ejercicio
      \begin{tblr}{colspec={cc},row{1}={black!10},vlines,hlines,
        cells={mode=math},baseline=1}
        v & w \\
        3 & 33 \\
        5 & 55 \\
        7 & 77 \\
      \end{tblr}
    \begin{vertical}
      \alternativa Son directamente proporcionales.
      \alternativa No lo son.
    \end{vertical}
    \ejercicio
      \begin{tblr}{colspec={cc},row{1}={black!10},vlines,hlines,
        cells={mode=math},baseline=1}
        x & y \\
        1 & 2 \\
        2 & 3 \\
        3 & 4 \\
      \end{tblr}
    \begin{vertical}
      \alternativa Son directamente proporcionales.
      \alternativa No lo son.
    \end{vertical}
    \ejercicio
      \begin{tblr}{colspec={cc},row{1}={black!10},vlines,hlines,
        cells={mode=math},baseline=1}
        x & y \\
        32 & 4 \\
        72 & 9 \\
        122 & 14 \\
      \end{tblr}
    \begin{vertical}
      \alternativa Son directamente proporcionales.
      \alternativa No lo son.
    \end{vertical}
    \ejercicio
      \begin{tblr}{colspec={cc},row{1}={black!10},vlines,hlines,
        cells={mode=math},baseline=1}
        x & y \\
        0,1 & 1 \\
        0,01 & 0,1 \\
        0,001 & 0,01 \\
      \end{tblr}
    \begin{vertical}
      \alternativa Son directamente proporcionales.
      \alternativa No lo son.
    \end{vertical}
  \end{ejercicios}
\parte Para las siguientes proporciones directas, determina el valor faltante y la 
constante de proporcionalidad (CP) para cada uno de los casos. Usa las casillas 
para responder y colocar los valores correspondientes. \\\hfill [1 punto por casilla]
  \begin{ejercicios}(2)
    \ejercicio 
    \begin{tblr}{colspec={cc},row{1}={black!10},vlines,hlines,
      cells={mode=math},baseline=1}
      x & y \\
      r & 49 \\
      5 & 35 \\
      3 & 21 \\
    \end{tblr}
    \begin{tcbraster}[raster columns=2, raster column skip=4pt,raster width=.6\linewidth]
      \begin{caja}[height=35pt,title=$r$]
      \end{caja}
      \begin{caja}[height=35pt,title=CP]
      \end{caja}
    \end{tcbraster}
    \ejercicio 
    \begin{tblr}{colspec={cc},row{1}={black!10},vlines,hlines,
      cells={mode=math},baseline=1}
      u & v \\
      0,05 & 0,1 \\
      z & 0,15 \\
      0,15 & 0,3 \\
    \end{tblr}
    \begin{tcbraster}[raster columns=2, raster column skip=4pt,raster width=.6\linewidth]
      \begin{caja}[height=35pt,title=$z$]
      \end{caja}
      \begin{caja}[height=35pt,title=CP]
      \end{caja}
    \end{tcbraster}
    \ejercicio $\dfrac{3}{2} = \dfrac{w}{5}$
    \vspace*{30pt}
    \begin{tcbraster}[raster columns=2, raster column skip=4pt,raster width=.6\linewidth]
      \begin{caja}[height=35pt,title=$w$]
      \end{caja}
      \begin{caja}[height=35pt,title=CP]
      \end{caja}
    \end{tcbraster}
    \ejercicio $\dfrac{3}{m} = \dfrac{0,001}{0,1}$
    \vspace*{30pt}
    \begin{tcbraster}[raster columns=2, raster column skip=4pt,raster width=.6\linewidth]
      \begin{caja}[height=35pt,title=$m$]
      \end{caja}
      \begin{caja}[height=35pt,title=CP]
      \end{caja}
    \end{tcbraster}
  \end{ejercicios}
\parte Usando la figura a continuación, crea una gráfica formada por puntos directamente
proporcionales y que su {\bfseries constante de proporcionalidad sea igual a 3}. 
Asegúrate de marcar los puntos en la rejilla y unirlos para formar una recta, 
además, rellena los valores a lo largo de los ejes para que tu gráfica tenga sentido. 
\hfill[Bonus sin puntaje, pero con 3 décimas para la prueba de cierre de unidad]
  
\begin{tcolorbox}[sidebyside,blankest,sidebyside align=center seam,lefthand ratio=0.6]
  \begin{tikzpicture}[line width=1pt,y=0.4cm,x=0.4cm]
  \def\xto{18}; \def\xby{3}; \def\yto{18}; \def\yby{3};
  \NewDocumentCommand{\drawpoint}{mm}{\draw[dashed] (0,0 -| #1,#2) -- (#1,#2) %
        node[shape=circle,fill,inner sep=2pt] {} -- (0,0 |- #1,#2);}
    \draw[->] (0,0) node[below left] {0} -- (\xto+\xby,0) node[pos=1,below right] {$x$};
    \draw[->] (0,0) -- (0,\yto+\yby) node[pos=1,above left] {$y$};
    \pgfmathparse{int(2*\xby)}
    \foreach \x in {\xby,\pgfmathresult,...,\xto} {
    \draw  ($(\x,0)+(0,-4pt)$) node[below=5pt,draw,dashed,rounded corners,inner sep=10pt] {} -- ($(\x,0)+(0,4pt)$);}
    \pgfmathparse{int(2*\yby)}
    \foreach \y in {\yby,\pgfmathresult,...,\yto} {
    \draw ($(0,\y)+(-4pt,0)$)  node[left=5pt,draw,dashed,rounded corners,inner sep=10pt] {} -- ($(0,\y)+(4pt,0)$);}
    \pgfmathparse{\xto+\xby/2}\let\rx=\pgfmathresult
    \pgfmathparse{\yto+\yby/2}\let\ry=\pgfmathresult
    \draw[help lines] (0,0) grid[xstep=\xby,ystep=\yby] (\rx,\ry);
  %  \pgfmathparse{(\yto+\yby)/2}
  %  \node[anchor=east,draw,dashed,inner ysep=50pt,inner xsep=10pt,xshift=-10pt,rounded corners] at (current bounding box.west |- 0,\pgfmathresult) {};
  %  \pgfmathparse{(\xto+\xby)/2}
  %  \node[anchor=north,draw,dashed,inner xsep=50pt,inner ysep=10pt,yshift=-10pt,rounded corners] at (current bounding box.south -| \pgfmathresult,0) {};
  \end{tikzpicture}
  \tcblower
  \begin{tblr}{colspec={X[1,c]X[1,c]},width=0.7\textwidth,row{1,2}={black!10},vlines,hlines,
    row{2}={mode=math},row{3-Z}={rowsep+=5pt},baseline=1,cell{1}{1}={r=1,c=2}{c}}
    {Valores con constante de\\proporcionalidad igual a 3} & hola \\
    x & y \\
      &  \\
      &  \\
      &  \\
  \end{tblr}
\end{tcolorbox}
% \begin{tcolorbox}[blankest,sidebyside,sidebyside align=center seam,lefthand ratio=0.1,left=1cm]
%     \begin{tblr}{colspec={cc},row{1}={black!10},vlines,hlines,
%       cells={mode=math},baseline=1}
%       x & y \\
%       1 & 2 \\
%       1 & 2 \\
%       1 & 2 \\
%     \end{tblr}
%   \tcblower
%     \begin{tikzpicture}[line width=1pt,y=0.5cm,x=0.5cm]
%     \def\xto{15}; \def\xby{3}; \def\yto{15}; \def\yby{3};
%     \NewDocumentCommand{\drawpoint}{mm}{\draw[dashed] (0,0 -| #1,#2) -- (#1,#2) %
%           node[shape=circle,fill,inner sep=2pt] {} -- (0,0 |- #1,#2);}
    
%       \draw[->] (0,0) node[below left] {0} -- (\xto+\xby,0);
%       \draw[->] (0,0) -- (0,\yto+\yby);
%       \pgfmathparse{int(2*\xby)}
%       \foreach \x in {\xby,\pgfmathresult,...,\xto} {
%       \draw  ($(\x,0)+(0,-4pt)$) node[below=5pt,draw,dashed,rounded corners,inner sep=10pt] {} -- ($(\x,0)+(0,4pt)$);}
    
%       \pgfmathparse{int(2*\yby)}
%       \foreach \y in {\yby,\pgfmathresult,...,\yto} {
%       \draw ($(0,\y)+(-4pt,0)$)  node[left=5pt,draw,dashed,rounded corners,inner sep=10pt] {} -- ($(0,\y)+(4pt,0)$);}
      
%       \draw[help lines] (0,0) grid[xstep=\xby,ystep=\yby] (\xto,\yto);
    
%       \pgfmathparse{(\yto+\yby)/2}
%       \node[anchor=east,draw,dashed,inner ysep=50pt,inner xsep=10pt,xshift=-10pt,rounded corners] at (current bounding box.west |- 0,\pgfmathresult) {};
%       \pgfmathparse{(\xto+\xby)/2}
%       \node[anchor=north,draw,dashed,inner xsep=50pt,inner ysep=10pt,yshift=-10pt,rounded corners] at (current bounding box.south -| \pgfmathresult,0) {};

%     \end{tikzpicture}
% \end{tcolorbox}

\end{partes}
\end{document}