\def\colegio{Colegio Latinoamericano de Integración}
\def\titulo{Control}
\def\subtitulo{Proporciones directas e inversas}
\def\curso{Séptimo Básico}
\def\puntaje{18}

\documentclass[sin curso]{plantilla-evaluacion-v1}

\begin{document}
\begin{partes}
  \parte Describe que es una {\bfseries proporción inversa} y que se debe cumplir para que se diga que dos cantidades son
  inversamente proporcionales. [4p]
  \begin{respuesta}[height=3cm]
  \end{respuesta}
  \parte Determina si las cantidades, descritas en las siguientes oraciones y tablas de valores, 
  son proporcionales o no y marca la alternativa correcta. [1p c/u]
  \begin{ejercicios}
    \ejercicio Al dividir una pizza, el tamaño de los pedazos y el número de pedazos.
    \begin{vertical}
      \alternativa Proporción directa.
      \alternativa Proporción inversa.
      \alternativa Ninguna de las anteriores.
    \end{vertical}
    \ejercicio El precio de un producto y la demanda por el producto.
    \begin{vertical}
      \alternativa Proporción directa.
      \alternativa Proporción inversa.
      \alternativa Ninguna de las anteriores.
    \end{vertical}
    \ejercicio La edad de una persona y su altura.
    \begin{vertical}
      \alternativa Proporción directa.
      \alternativa Proporción inversa.
      \alternativa Ninguna de las anteriores.
    \end{vertical}
    \ejercicio Velocidad de viaje y la distancia recorrida, en un intervalo de tiempo fijo.
    \begin{vertical}
      \alternativa Proporción directa.
      \alternativa Proporción inversa.
      \alternativa Ninguna de las anteriores.
    \end{vertical}
    \ejercicio
    \begin{tblr}{colspec={cc},row{1}={black!10},vlines,hlines,
      cells={mode=math},baseline=1}
      x & y \\
      125 & 25 \\
      100 & 20 \\
      75 & 15 \\
    \end{tblr}
    \begin{vertical}
      \alternativa Proporción directa.
      \alternativa Proporción inversa.
      \alternativa Ninguna de las anteriores.
    \end{vertical}
    \ejercicio
    \begin{tblr}{colspec={cc},row{1}={black!10},vlines,hlines,
      cells={mode=math},baseline=1}
      v & w \\
      4 & 1 \\
      2 & 2 \\
      1 & 4 \\
    \end{tblr}
    \begin{vertical}
      \alternativa Proporción directa.
      \alternativa Proporción inversa.
      \alternativa Ninguna de las anteriores.
    \end{vertical}
    \ejercicio
    \begin{tblr}{colspec={cc},row{1}={black!10},vlines,hlines,
      cells={mode=math},baseline=1}
      x & y \\
      5 & 10 \\
      15 & 20 \\
      25 & 30 \\
    \end{tblr}
    \begin{vertical}
      \alternativa Proporción directa.
      \alternativa Proporción inversa.
      \alternativa Ninguna de las anteriores.
    \end{vertical}
    \ejercicio
    \begin{tblr}{colspec={cc},row{1}={black!10},vlines,hlines,
      cells={mode=math},baseline=1}
      a & b \\
      0,25 & 4 \\
      0,10 & 10 \\
      0,04 & 25 \\
    \end{tblr}
    \begin{vertical}
      \alternativa Proporción directa.
      \alternativa Proporción inversa.
      \alternativa Ninguna de las anteriores.
    \end{vertical}
  \end{ejercicios}
  \parte Para las siguientes tablas de valores, encuentra el valor faltante en cada una
  y colócalo en la casilla correspondiente. [1p c/u]
  \begin{ejercicios}
    \ejercicio
    \begin{tblr}{colspec={X[c]X[c]},width=0.6\linewidth,row{1}={black!10},vlines,hlines,
      row{2-Z}={mode=math},baseline=1,cell{1}{1}={c=2,r=1}{c}}
      Proporción Directa & \\
      27 & 3 \\
      h & 9 \\
    \end{tblr}
    \begin{caja}[height=35pt,title=$h$,hbox]
    \hspace*{1.5cm}
    \end{caja}
    \ejercicio
    \begin{tblr}{colspec={X[c]X[c]},width=0.6\linewidth,row{1}={black!10},vlines,hlines,
      row{2-Z}={mode=math},baseline=1,cell{1}{1}={c=2,r=1}{c}}
      Proporción Directa & \\
      150 & 25 \\
      120 & k \\
    \end{tblr}
    \begin{caja}[height=35pt,title=$k$,hbox]
    \hspace*{1.5cm}
    \end{caja}
    \ejercicio
    \begin{tblr}{colspec={X[c]X[c]},width=0.6\linewidth,row{1}={black!10},vlines,hlines,
      row{2-Z}={mode=math},baseline=1,cell{1}{1}={c=2,r=1}{c}}
      Proporción Directa & \\
      0,02 & 0,1 \\
      0,1 & b \\
    \end{tblr}
    \begin{caja}[height=35pt,title=$b$,hbox]
    \hspace*{1.5cm}
    \end{caja}
    \ejercicio
    \begin{tblr}{colspec={X[c]X[c]},width=0.6\linewidth,row{1}={black!10},vlines,hlines,
      row{2-Z}={mode=math},baseline=1,cell{1}{1}={c=2,r=1}{c}}
      Proporción Inversa & \\
      a & 3 \\
      3 & 2 \\
    \end{tblr}
    \begin{caja}[height=35pt,title=$a$,hbox]
    \hspace*{1.5cm}
    \end{caja}
    \ejercicio
    \begin{tblr}{colspec={X[c]X[c]},width=0.6\linewidth,row{1}={black!10},vlines,hlines,
      row{2-Z}={mode=math},baseline=1,cell{1}{1}={c=2,r=1}{c}}
      Proporción Inversa & \\
      5 & x \\
      6 & 7 \\
    \end{tblr}
    \begin{caja}[height=35pt,title=$x$,hbox]
    \hspace*{1.5cm}
    \end{caja}
    \ejercicio
    \begin{tblr}{colspec={X[c]X[c]},width=0.6\linewidth,row{1}={black!10},vlines,hlines,
      row{2-Z}={mode=math},baseline=1,cell{1}{1}={c=2,r=1}{c}}
      Proporción Inversa & \\
      z & 0,05 \\
      0,35 & 0,4 \\
    \end{tblr}
    \begin{caja}[height=35pt,title=$z$,hbox]
    \hspace*{1.5cm}
    \end{caja}
  \end{ejercicios}
\end{partes}  
\end{document}