\def\colegio{Colegio Latinoamericano de Integración}
\def\titulo{Evaluación Recuperativa}
\def\subtitulo{Ecuaciones}
\def\curso{Octavo Básico}
\def\puntaje{36}

\documentclass[sin curso]{plantilla-evaluacion-v1}

\begin{document}

Se asignará el puntaje a cada pregunta según los criterios que se encuentran
a continuación.

\begin{importante}
  \begin{tblr}{colspec={X[1,c,m]X[6,m]},hline{2-Y}={solid},vline{2-Y}={solid},
    cell{1}{2}={c},cell{2-Z}{2}={cmd=\raggedright}}
    Puntaje asignado & Criterios o indicadores\\
    +3 puntos & Señala clara y correctamente cuál es la solución o el resultado de la pregunta hecha
    en el enunciado.\\
    +3 puntos & Incluye un desarrollo que relata de manera clara y ordenada los procedimientos
    \mbox{necesarios} para solucionar la problemática. En caso de estar incompleto o con
    \mbox{errores} el desarrollo, se asignará puntaje parcial si se muestra dominio de los
    contenidos y conceptos involucrados.\\
    0\% &  La respuesta es incorrecta. De haber desarrollo, este tiene errores conceptuales.\\
  \end{tblr}
\end{importante}

\section{Resolución de ecuaciones}
Determina el valor de la incógnita para cada una de las siguientes ecuaciones.

\begin{preguntas}(1)
  \pregunta $10x + 21 = 15 - 2x$
  \begin{malla}[height=4cm]
  \end{malla}

  \pregunta $3w + 5 - 7w + 9w - 11w + 13 = 16 - 8w$
  \begin{malla}[height=6cm]
  \end{malla}

  \pregunta $7(3x+1) + 8(2x - 3) = 4(3x - 1) - 7(x - 4)$
  \begin{malla}[height=8cm]
  \end{malla}

  \pregunta $\dfrac{4}{3}x-\dfrac{2}{5}=\dfrac{7}{5}x-\dfrac{1}{10}$
  \begin{malla}[height=8cm]
  \end{malla}
\end{preguntas}

\section{Resolución de problemas}
Modele las siguientes situaciones usando lenguaje algebraico y encuentra la solución a los
problemas planteados. Es importante que haya una respuesta a la pregunta formulada y
que esta describa el significado del valor numérico alcanzado.

\begin{preguntas}(1)
  \pregunta Andrés tiene 30 monedas, algunas son de \$5 y otras de \$10.
  Si en total dispone de \$200, ¿cuántas monedas de cada denominación tiene?
  \begin{malla}[height=7cm]
  \end{malla}
  \begin{respuesta}[height=2cm]
  \end{respuesta}

  \pregunta Sandra pagó \$66 por una pasta dental, un jabón y un champú.
  Si el costo de la pasta excede en \$15 al del jabón y
  en \$3 al del champú, determina el costo de cada uno de los artículos.
  \begin{malla}[height=7cm]
  \end{malla}
  \begin{respuesta}[height=2cm]
  \end{respuesta}

\end{preguntas}

\end{document}
