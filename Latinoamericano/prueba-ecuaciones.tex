\def\colegio{Colegio Latinoamericano de Integración}
\def\titulo{Evaluación Sumativa}
\def\subtitulo{Ecuaciones}
\def\curso{Octavo Básico}
% 2*4 + 2*3 + 2*5 + 2*3
\def\puntaje{30}

\documentclass[sin curso]{plantilla-evaluacion-v1}

\begin{document}

\section*{Objetivo de la evaluación}

(OA6) Mostrar que comprenden operatoria de expresiones algebraicas y (OA8) utilizan
ecuaciones lineales para solucionar problemas contextualizados.

\section*{Pauta de cotejo}
En la corrección de la evaluación, se asignará puntaje a cada respuesta según los criterios 
que se encuentran detallados en la tabla a continuación.

\begin{importante}
  \begin{tblr}{colspec={X[1,c,m]X[6,m]},hline{2-Y}={solid},vline{2-Y}={solid},
    cell{1}{2}={c},cell{2-Z}{2}={cmd=\raggedright}}
    Puntaje asignado & Criterios o indicadores\\
    +50\% & Señala clara y correctamente cuál es la solución o el resultado de la pregunta hecha
    en el enunciado.\\
    +50\% & Incluye un desarrollo que relata de manera clara y ordenada los procedimientos 
    \mbox{necesarios} para solucionar la problemática. En caso de estar incompleto o con 
    \mbox{errores} el desarrollo, se asignará puntaje parcial si se muestra dominio de los 
    contenidos y conceptos involucrados.\\
    0\% &  La respuesta es incorrecta. De haber desarrollo, este tiene errores conceptuales.\\
  \end{tblr}
\end{importante}

\begin{center}
  \vspace*{5mm}
  \begin{tikzpicture}[ampersand replacement=\&,]
      %\node (A) [opacity=0.4] {\includegraphics[width=2cm]{../flork3.jpg}};
      \node (B) [font=\slshape,text width=12cm]
      {``Cree en ti mismo y en lo que eres. Sé consciente de que hay algo en tu interior %
      que es más grande que cualquier obstáculo''};
      \node [left=0mm of B,opacity=0.4] {\pgfornament[width=2cm]{37}};
      \node [right=0mm of B,opacity=0.4] {\pgfornament[width=2cm]{38}};
  \end{tikzpicture}
  \vspace*{5mm}
\end{center}



\section{Operatoria algebraica}

Reduce las siguientes expresiones algebraicas [2 puntos c/u].

\begin{preguntas}(2)
  \pregunta $-7a + 3a - 16a$
  \begin{malla}[height=3cm]
  \end{malla}
  \pregunta $-12m +3n -4m -10n +5m-n$
  \begin{malla}[height=3cm]
  \end{malla}
  \pregunta $(a^2+a-1)-(a^2-a+1)$
  \begin{malla}[height=4cm]
  \end{malla}
  \pregunta $(2x-5)\cdot(3x+2)$
  \begin{malla}[height=4cm]
  \end{malla}

\end{preguntas}

Considera las siguientes igualdades y determina el valor de cada expresión [2 puntos c/u].
\vspace*{3mm}
\begin{tcbraster}[raster columns=3,raster width=.7\linewidth,raster column skip=20pt,
  raster halign=center,halign=center,borderline={1pt}{0pt}{black,dashed},
  colframe=white,colback=white]
  \begin{tcolorbox}
    $A = m + 2$
  \end{tcolorbox}
  \begin{tcolorbox}
    $B = 3m -5$
  \end{tcolorbox}
  \begin{tcolorbox}
    $C = -2m + 3$
  \end{tcolorbox}
\end{tcbraster}
\vspace*{5mm}

\begin{preguntas}(2)
  \pregunta $3A$
  \begin{malla}[height=4cm]
  \end{malla}
  \pregunta $A-B$
  \begin{malla}[height=4cm]
  \end{malla}
  \pregunta! $A-(B-C)$
  \begin{malla}[height=4cm]
  \end{malla}

\end{preguntas}

\section{Resolución de ecuaciones}
Determine el valor de la incógnita para cada una de las siguientes igualdades [2 puntos c/u].

\begin{preguntas}(2)
  \pregunta $2x-3=5$
  \begin{malla}[height=3cm]
  \end{malla}
  \pregunta $11y - 5y + 6 = -24 -9y$
  \begin{malla}[height=3cm]
  \end{malla}
  \pregunta $x-(2x+1)=8-(3x+3)$
  \begin{malla}[height=5cm]
  \end{malla}
  \pregunta $\dfrac{1}{2}x+\dfrac{6}{4}=5$
  \begin{malla}[height=5cm]
  \end{malla}
  \pregunta! $\dfrac{5}{6}x-\dfrac{7}{4}+\dfrac{2x}{3}=2x -\dfrac{5}{12}+\dfrac{x}{3}$
  \begin{malla}[height=6cm]
  \end{malla}
\end{preguntas}

\section{Resolución de problemas}
Modele las siguientes situaciones usando lenguaje algebraico y encuentra la solución a los
problemas planteados. Es importante que haya una respuesta a la pregunta formulada y 
que esta describa el significado del valor numérico alcanzado [2 puntos c/u].

\begin{preguntas}
  \pregunta Un rectángulo es dos veces más ancho que alto y su perímetro mide 60. 
  ¿Cuánto miden los lados del rectángulo? 
  \begin{malla}[height=3.5cm]
  \end{malla}
  \begin{respuesta}[height=1.5cm]
  \end{respuesta}

  \pregunta La edad de Carla excede en 3 años a la de Daniel y el doble de la edad de Carla 
  más 12 años equivale al triple de la de Daniel. Determina ambas edades.
  \begin{malla}[height=7cm]
  \end{malla}
  \begin{respuesta}[height=2.5cm]
  \end{respuesta}

  \pregunta Sofía retira del banco \$5 000, en billetes de \$500, \$200 y \$100. Si el 
  número de billetes de \$200 excede en 3 a los de \$100, y el número de billetes de \$100 
  es el doble de los de \$500, ¿cuántos billetes de cada denominación recibió Sofía?
  \begin{malla}[height=7cm]
  \end{malla}
  \begin{respuesta}[height=2.5cm]
  \end{respuesta}

\end{preguntas}

\end{document}