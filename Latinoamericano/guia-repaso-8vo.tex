\def\colegio{Colegio Latinoamericano de Integración}
\def\titulo{Repaso de unidad}
\def\subtitulo{Plantear y solucionar ecuaciones}
\def\curso{Octavo Básico}

\documentclass[sin nombre]{plantilla-evaluacion-v1}

\begin{document}

\begin{partes}

\parte Resuelva las siguientes ecuaciones.

\begin{ejercicios}(1)
  \ejercicio  $8x - 4 + 3x = 7x + x + 14$
  \begin{malla}[height=3cm]
  \end{malla}
  \ejercicio  $x -(2x + 1) = 8 -(3x + 3)$
  \begin{malla}[height=3cm]
  \end{malla}
  \ejercicio  $(5-3x)-(-4x+6)=(8x+11)-(3x-6)$
  \begin{malla}[height=3cm]
  \end{malla}
  \ejercicio  $\dfrac{x}{6}+5=\dfrac{1}{3}-x$
  \begin{malla}[height=3cm]
  \end{malla}
  \ejercicio  $\dfrac{5}{9}x-\dfrac{5}{3}=\dfrac{3}{4}x-\dfrac{1}{2}$
  \begin{malla}[height=3cm]
  \end{malla}
\end{ejercicios}


\parte Plantee una ecuación que describa la problemática y encuentre la solución.

\begin{ejercicios}(1)
    \ejercicio La suma de dos números es 106 y el mayor excede al menor en ocho. 
    Encuentra los números.
    \begin{malla}[height=3cm]
    \end{malla}

    \ejercicio La diferencia de dos números es 42 y los dos quintos del mayor 
    equivalen al menor. ¿Cuáles son los números?
    \begin{malla}[height=3cm]
    \end{malla}

    \ejercicio Un número excede en seis a otro y el doble del mayor equivale 
    al triple del menor. Encuentra los números.
    \begin{malla}[height=3cm]
    \end{malla}

    \ejercicio La suma de dos números es 60 y el mayor equivale cinco veces
     el menor aumentado en 30. Determina los números.
    \begin{malla}[height=3cm]
    \end{malla}

    \ejercicio En un número de dos dígitos, el dígito de las decenas es 3 unidades 
    menor que el de las unidades. Si el número excede en 6 al cuádruplo de la suma 
    de sus dígitos, halla el número.
    \begin{malla}[height=3cm]
    \end{malla}

\end{ejercicios}

\end{partes}

\end{document}