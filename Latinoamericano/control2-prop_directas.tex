\def\colegio{Colegio Latinoamericano de Integración}
\def\titulo{Control 2}
\def\subtitulo{Proporciones directas}
\def\curso{Séptimo Básico}

\documentclass[sin curso]{plantilla-evaluacion-v1}

\begin{document}
\begin{partes}
  \parte Describe que es una proporción directa y que se debe cumplir para
  que se diga que dos cantidades son directamente proporcionales.\hfill [4 puntos]
  \begin{respuesta}[height=3.5cm]
  \end{respuesta}
  \parte Determina si las cantidades, descritas en las siguientes oraciones y 
  tablas de valores, son directamente proporcionales o no y marca la alternativa
  correcta.\hfill[2 puntos cada una]
  \begin{ejercicios}(2)
    \ejercicio El tiempo dedicado a estudiar y la nota en la evaluación final.
    \begin{vertical}
      \alternativa Son directamente proporcionales.
      \alternativa No lo son.
    \end{vertical}
    \ejercicio La altura de una persona y el tamaño de su calzado.
    \begin{vertical}
      \alternativa Son directamente proporcionales.
      \alternativa No lo son.
    \end{vertical}
    \ejercicio La distancia recorrida por un automóvil y tiempo de viaje
    a velocidad constante.
    \begin{vertical}
      \alternativa Son directamente proporcionales.
      \alternativa No lo son.
    \end{vertical}
    \ejercicio El lado de un cuadrado y su perimetro.
    \begin{vertical}
      \alternativa Son directamente proporcionales.
      \alternativa No lo son.
    \end{vertical}

    \ejercicio
    \begin{tblr}{colspec={cc},row{1}={black!10},vlines,hlines,
      cells={mode=math},baseline=1}
      x & y \\
      8 & 3 \\
      6 & 2 \\
      4 & 1 \\
    \end{tblr}
    \begin{vertical}
      \alternativa Son directamente proporcionales.
      \alternativa No lo son.
    \end{vertical}
    \ejercicio
      \begin{tblr}{colspec={cc},row{1}={black!10},vlines,hlines,
        cells={mode=math},baseline=1}
        v & w \\
        0 & 0 \\
        4 & 12 \\
        8 & 24 \\
      \end{tblr}
    \begin{vertical}
      \alternativa Son directamente proporcionales.
      \alternativa No lo son.
    \end{vertical}
    \ejercicio
      \begin{tblr}{colspec={cc},row{1}={black!10},vlines,hlines,
        cells={mode=math},baseline=1}
        x & y \\
        1 & 1,5 \\
        2 & 3 \\
        3 & 4,5 \\
      \end{tblr}
    \begin{vertical}
      \alternativa Son directamente proporcionales.
      \alternativa No lo son.
    \end{vertical}
    \ejercicio
      \begin{tblr}{colspec={cc},row{1}={black!10},vlines,hlines,
        cells={mode=math},baseline=1}
        x & y \\
        7 & 49 \\
        8 & 56 \\
        9 & 62 \\
      \end{tblr}
    \begin{vertical}
      \alternativa Son directamente proporcionales.
      \alternativa No lo son.
    \end{vertical}
  \end{ejercicios}
\parte Para las siguientes proporciones directas, determina el valor faltante y la 
constante de proporcionalidad (CP) para cada uno de los casos. Usa las casillas 
para responder y colocar los valores correspondientes. \\\hfill [1 punto por casilla]
  \begin{ejercicios}(2)
    \ejercicio 
    \begin{tblr}{colspec={cc},row{1}={black!10},vlines,hlines,
      cells={mode=math},baseline=1}
      x & y \\
      50 & 5 \\
      80 & 8 \\
      e & 11 \\
    \end{tblr}
    \begin{tcbraster}[raster columns=2, raster column skip=4pt,raster width=.6\linewidth]
      \begin{caja}[height=35pt,title=$e$]
      \end{caja}
      \begin{caja}[height=35pt,title=CP]
      \end{caja}
    \end{tcbraster}
    \ejercicio 
    \begin{tblr}{colspec={cc},row{1}={black!10},vlines,hlines,
      cells={mode=math},baseline=1}
      u & v \\
      3 & 2 \\
      7,5 & h \\
      12 & 8 \\
    \end{tblr}
    \begin{tcbraster}[raster columns=2, raster column skip=4pt,raster width=.6\linewidth]
      \begin{caja}[height=35pt,title=$h$]
      \end{caja}
      \begin{caja}[height=35pt,title=CP]
      \end{caja}
    \end{tcbraster}
    \ejercicio $\dfrac{w}{3} = \dfrac{32}{24}$
    \vspace*{30pt}
    \begin{tcbraster}[raster columns=2, raster column skip=4pt,raster width=.6\linewidth]
      \begin{caja}[height=35pt,title=$w$]
      \end{caja}
      \begin{caja}[height=35pt,title=CP]
      \end{caja}
    \end{tcbraster}
    \ejercicio $\dfrac{150}{75} = \dfrac{m}{100}$
    \vspace*{30pt}
    \begin{tcbraster}[raster columns=2, raster column skip=4pt,raster width=.6\linewidth]
      \begin{caja}[height=35pt,title=$m$]
      \end{caja}
      \begin{caja}[height=35pt,title=CP]
      \end{caja}
    \end{tcbraster}
    \ejercicio $\dfrac{16}{5} = \dfrac{4}{k}$
    \vspace*{30pt}
    \begin{tcbraster}[raster columns=2, raster column skip=4pt,raster width=.6\linewidth]
      \begin{caja}[height=35pt,title=$k$]
      \end{caja}
      \begin{caja}[height=35pt,title=CP]
      \end{caja}
    \end{tcbraster}
    \ejercicio $\dfrac{5}{m} = \dfrac{0,03}{0,09}$
    \vspace*{30pt}
    \begin{tcbraster}[raster columns=2, raster column skip=4pt,raster width=.6\linewidth]
      \begin{caja}[height=35pt,title=$m$]
      \end{caja}
      \begin{caja}[height=35pt,title=CP]
      \end{caja}
    \end{tcbraster}
  \end{ejercicios}
\parte ¿Qué es la constante de proporcionalidad? ¿Qué significa que la constante 
de proporcionalidad sea 5 entre dos cantidades directamente proporcionales?
\hfill[Bonus con 2 décimas para la prueba]
\begin{respuesta}[height=5 cm]
\end{respuesta}
\end{partes}

\end{document}