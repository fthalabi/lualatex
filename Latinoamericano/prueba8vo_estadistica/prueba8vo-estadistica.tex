\def\colegio{Colegio Latinoamericano de Integración}
\def\titulo{Evaluación Sumativa}
\providecommand{\forma}{XX}
\def\subtitulo{Unidad de Estadística (forma \forma)}
\def\curso{Octavo Básico}
% 7*2 + 5 + 7 + 3 + 7 + 2 + 4
\def\puntaje{42}

\documentclass[sin curso]{plantilla-evaluacion-v1}

\begin{python}
import numpy as np

def divisores(n):
    div = [i for i in range(1, int(n ** 0.5) + 1) if n % i == 0]
    div.extend([n // i for i in div if i != n // i])
    return np.array(sorted(div))

def n_barras(lista, ideal):
  if (lista is None or len(lista) == 0):
    return 0
  min = np.min(lista)
  max = np.max(lista) + 1
  rango = np.abs(max-min)
  div_rango = divisores(rango).astype(int)
  n_bars = (rango/div_rango).astype(int)
  distancia = np.abs(n_bars - ideal)
  return n_bars[distancia.argmin()]

def ticks(muestras):
  t = np.linspace(
    np.min(muestras),
    np.max(muestras)+1,
    n_barras(muestras,5)+1
  ).astype(int)
  return ",".join(t.astype(str))

def obtener_muestras(m,d,n,s=None):
  l = np.random.default_rng(seed=s).normal(m,d,size=n)
  return np.floor(l).astype(int)

muestras_m = obtener_muestras(155,3,45)
muestras_h = obtener_muestras(158,3,50)

@np.vectorize(excluded=['lista'])
def frecuencia(x, lista):
  filtrado = np.extract(lista == x,lista)
  return len(filtrado)

@np.vectorize(excluded=['lista'])
def frecuencia_acumulada(x, lista):
  filtrado = np.extract(lista <= x, lista)
  return len(filtrado)

@np.vectorize(excluded=['lista'])
def probabilidad(x, lista):
  return frecuencia(x, lista=lista)/len(lista)

@np.vectorize(excluded=['lista'])
def probabilidad_acumulada(x, lista):
  return frecuencia_acumulada(x, lista=lista)/len(lista)

@np.vectorize(excluded=['lista'])
def cuantil(p, lista):
  unicos = np.unique_values(lista)
  dato = 0
  for i in unicos:
    p_acumulada = probabilidad_acumulada(i, lista=lista)
    if (p_acumulada >= p):
      dato = i
      break
  return dato

def moda(lista):
  valores, conteo = np.unique_counts(lista)
  indice_max = np.argwhere(conteo == np.max(conteo))
  return valores[indice_max].flatten()

def pgf_coords(x,y):
  return " ".join(f"({i},{j})" for i, j in zip(x,y))

def imprimir(l):
  return " • ".join(l.astype(str))

while(True):
  m1 = obtener_muestras(10,2,13)
  if (np.min(m1)>0 and len(moda(m1))==1):
    break

while(True):
  m2 = obtener_muestras(14,3,20)
  if (np.min(m2)>0 and len(moda(m2))==1):
    break

while(True):
  muestrasH = obtener_muestras(16,3.5,23)
  if (np.min(muestrasH)>0 and len(np.unique_values(muestrasH))<=10 and len(moda(muestrasH))==1):
    break

while(True):
  muestrasM = obtener_muestras(18,3.5,26)
  if (np.min(muestrasM)>0 and n_barras(muestrasM,5) in [4,5,6] and len(moda(muestrasM))==1):
    break

np.savetxt("tablaH.csv",np.column_stack([
  x := (np.unique_values(muestrasH)),
  frecuencia(x,lista=muestrasH),
  frecuencia_acumulada(x,lista=muestrasH),
  probabilidad(x,lista=muestrasH),
  probabilidad_acumulada(x,lista=muestrasH)
  ]),
  fmt=["%d","%d","%d","%.4f","%.4f"],
  delimiter=","
)

np.savetxt("muestrasH.csv",muestrasH,delimiter=",")
np.savetxt("muestrasM.csv",muestrasM,delimiter=",")
\end{python}

\newsavebox{\tablaQ}
\begin{lrbox}{\tablaQ}
  \begin{tblr}{colspec={ccccc},vlines,hlines,rowsep=6pt,colsep=10pt,column{2,3,4}={wd=1.3cm},
  hline{1,2,Z}={black,1pt}}
    Mínimo & $Q_1$  & $Q_2$ & $Q_3$ & Máximo \\
           &        &       &       &        \\
  \end{tblr}
\end{lrbox}
\newsavebox{\tablaM}
\begin{lrbox}{\tablaM}
  \begin{tblr}{colspec={cc},vlines,hlines,rowsep=6pt,colsep=10pt,cells={wd=1.3cm},
  hline{1,2,Z}={black,1pt}}
    Media &  Moda \\
          &       \\
  \end{tblr}
\end{lrbox}

\NewDocumentCommand{\generarTabla}{m}{%
\begin{tblr}{colspec={ccccc},vlines,hlines,rowsep=6pt,colsep=10pt,column{2,3,4}={wd=1.3cm},
  hline{1,2,Z}={black,1pt}}
  Mínimo & $Q_1$ & $Q_2$ & $Q_3$ & Máximo \\
  \py{np.min(#1)} & \py{cuantil(0.25,lista=#1)} &
  \py{cuantil(0.5,lista=#1)} & \py{cuantil(0.75,lista=#1)} &
  \py{np.max(#1)} \\
\end{tblr}\hspace{15pt}%
\begin{tblr}{colspec={ccc},vlines,hlines,rowsep=6pt,colsep=10pt,cells={wd=1.3cm},
  hline{1,2,Z}={black,1pt}}
  Media & Moda \\
  \pgfmathprintnumber[verbatim,use comma]{\py{np.mean(#1)}} & \py{imprimir(moda(lista=#1))} \\
\end{tblr}}


\begin{document}

\section*{Objetivo de la evaluación}

\begin{tasks}[style=itemize](1)
  \task {\sffamily\bfseries 7B|OA15}: Estimar el porcentaje de algunas características de una
  población desconocida por medio del muestreo.
  \task {\sffamily\bfseries 7B|OA16}: Representar datos obtenidos en una muestra mediante
  tablas de frecuencias absolutas y relativas; o con gráficos apropiados.
  \task {\sffamily\bfseries 7B|OA17}: Mostrar que comprenden las medidas de tendencia central
  y el rango.
  \task {\sffamily\bfseries 8B|OA15}: Mostrar que comprenden las medidas de posición,
  percentiles y cuartiles.
\end{tasks}

\section{Síntesis de datos}

Utiliza los datos entregados para rellenar las tablas con los valores indicados. [7p c/u]

\begin{preguntas}[after-item-skip=15pt,resume=false]
  \pregunta Datos: \py{imprimir(m1)}
  \begin{malla}[height=3cm]
  \end{malla}
  \usebox{\tablaQ}\hspace{15pt}\usebox{\tablaM}

  \pregunta Datos: \py{imprimir(m2)}
  \begin{malla}[height=3cm]
  \end{malla}
  \usebox{\tablaQ}\hspace{15pt}\usebox{\tablaM}
\end{preguntas}

\section{Comparando resultados de una encuesta}

A continuación, se encuentran los resultados de encuestar a un grupo de estudiantes y
preguntarles a cada uno: \textbf{¿Cuántas prendas de vestir tienes?}
(poleras, pantalones, etc). Utiliza los datos entregados para contestar las preguntas
respecto a cada grupo. \par
Respuestas de los niños:\\[5pt]
\hspace*{20pt}\py{imprimir(muestrasH)}.

\NewDocumentCommand{\formatear}{m}{%
\pgfmathprintnumber[fixed,fixed zerofill,precision=3,verbatim,use comma]{#1}%
}
\begin{preguntas}[after-item-skip=15pt](1)
  \pregunta Complete la tabla de frecuencias con los datos de los niños encuestados. [5p]\\[5pt]
  \csvreader[no head,
  tabularray={
  cells={valign=m},
  colspec={X[1,c]X[2,c]X[2,c]X[2,c]X[2,c]},
  width=0.85\linewidth,
  hlines,
  vlines,
  hline{1,2,3,Z}={black,1pt},
  rows={rowsep+=2pt},
  cell{1}{1}={r=1,c=5}{c},
  cell{3-Z}{1-Z}={fg=white}
  },
  table head={Número de prendas de vestir (Niños) & F & FA & P & PA \\ Datos & Frecuencia &
    {Frecuencia\\Acumulada} & Probabilidad & {Probabilidad\\Acumulada} \\}
  ]{tablaH.csv}{1=\a, 2=\b, 3=\c, 4=\d, 5=\e}%
  {\a & \b & \c & \formatear{\d} & \formatear{\e} }

  \pregunta Utiliza las respuestas de los niños para completar la tabla de valores. [7p]\\[5pt]
  \usebox{\tablaQ}\hspace{15pt}\usebox{\tablaM}
\end{preguntas}
%
\begin{python}
  x_m = np.unique_values(muestrasM)
  y_m = frecuencia(x_m,lista=muestrasM)
\end{python}%
%
\begin{tikzpicture}[baseline=(current axis.north)]
  \begin{axis}[
      ybar,
      title={Respuestas de las niñas},
      ylabel={Frecuencia},
      xlabel={Número de prendas de vestir},
      xtick distance=3,
      ymin=0,
      %xtick=data,
      %nodes~near~coords,
      %nodes~near~coords~align={vertical},
      ]
      \addplot [ybar,bar width=1,pattern={Lines[angle=-45,distance=5pt]}]
        coordinates {\py{pgf_coords(x_m,y_m)}};
  \end{axis}
\end{tikzpicture}

\begin{preguntas}[after-item-skip=15pt](1)
  \pregunta Rellena el siguiente histograma con las respuestas de las niñas y coloca
  la frecuencia arriba de cada una de las barras. [3p] \\
    \begin{tikzpicture}[baseline=(current axis.north)]
      \begin{axis}[
          ybar interval,
          %title={Histograma (Niñas)},
          ylabel={Frecuencia},
          xlabel={Número de prendas de vestir},
          xtick = {\py{ticks(muestrasM)}},
          ymin=0,
          grid=major,
          xmin={\py{np.min(muestrasM)}},
          xmax={\py{np.max(muestrasM)+1}},
          enlarge x limits,
          xticklabel= $[\pgfmathprintnumber\tick-\pgfmathprintnumber\nexttick[$,
          x tick label style={rotate=45,anchor=east},
          ]
          \addplot [draw=none,hist={bins=\py{n_barras(muestrasM,5)}}]
          table[y index=0] {muestrasM.csv};
      \end{axis}
    \end{tikzpicture}

  \pregunta Utiliza las respuestas de las niñas para completar la tabla de valores. [7p]\\[10pt]
  \usebox{\tablaQ}\hspace{15pt}\usebox{\tablaM}

  \pregunta Utiliza las respuestas de ambos grupos y dibuja un diagrama de caja para
  representar los datos de cada grupo. [2p]\\
  \begin{tikzpicture}[baseline=(current axis.north)]
    \begin{axis}[
      %title={Comparando número de amig@s por grupo},
      xlabel={Número de prendas de vestir},
      ytick={1,2},
      yticklabels={Niñas,Niños},
      width=0.85\linewidth,
      height=5cm,
    ]
      \addplot+ [boxplot={
        draw position=1,
        whisker range=10,
      },draw=white] table[y index=0] {muestrasM.csv};
      \addplot+ [boxplot={
        draw position=2,
        whisker range=10,
      },draw=white] table[y index=0] {muestrasH.csv};
    \end{axis}
  \end{tikzpicture}

  \pregunta ¿Qué grupo tiene más prendas? Justifique su respuesta usando
  las medidas de centralidad y/o los cuantiles de cada grupo. [4p]
  \begin{respuesta}[height=3cm]
  \end{respuesta}
\end{preguntas}

\newpage
\respuestas

\begin{preguntas}[after-item-skip=15pt,resume=false](1)
  \pregunta \generarTabla{m1}
  \pregunta \generarTabla{m2}
  \pregunta
  \csvreader[no head,
  tabularray={
  cells={valign=m},
  colspec={X[1,c]X[2,c]X[2,c]X[2,c]X[2,c]},
  width=0.85\linewidth,
  hlines,
  vlines,
  hline{1,2,3,Z}={black,1pt},
  rows={rowsep+=2pt},
  cell{1}{1}={r=1,c=5}{c},
  %cell{3-Z}{1-Z}={fg=white}
  },
  table head={Número de prendas de vestir (Niños) & F & FA & P & PA \\ Datos & Frecuencia &
    {Frecuencia\\Acumulada} & Probabilidad & {Probabilidad\\Acumulada} \\}
  ]{tablaH.csv}{1=\a, 2=\b, 3=\c, 4=\d, 5=\e}%
  {\a & \b & \c & \formatear{\d} & \formatear{\e} }

  \pregunta \generarTabla{muestrasH}
  \pregunta
  \begin{tikzpicture}[baseline=(current axis.north)]
    \begin{axis}[
        ybar interval,
        title={Respuestas de las niñas},
        ylabel={Frecuencia},
        xlabel={Número de prendas de vestir},
        xtick = {\py{ticks(muestrasM)}},
        ymin=0,
        grid=major,
        xmin={\py{np.min(muestrasM)}},
        xmax={\py{np.max(muestrasM)+1}},
        enlarge x limits,
        xticklabel= $[\pgfmathprintnumber\tick-\pgfmathprintnumber\nexttick[$,
        x tick label style={rotate=45,anchor=east},
        ]
        \addplot [hist={bins=\py{n_barras(muestrasM,5)}},pattern={Lines[angle=-45,distance=5pt]}]
         table[y index=0] {muestrasM.csv};
    \end{axis}
  \end{tikzpicture}
  \pregunta \generarTabla{muestrasM}
  \pregunta
  \begin{tikzpicture}[baseline=(current axis.north)]
    \begin{axis}[
      %title={Comparando número de amig@s por grupo},
      xlabel={Número de prendas de vestir},
      ytick={1,2},
      yticklabels={Niñas,Niños},
      width=0.85\linewidth,
      height=5cm,
    ]
      \addplot+ [boxplot={
        draw position=1,
        whisker range=10
      },solid] table[y index=0] {muestrasM.csv};
      \addplot+ [boxplot={
        draw position=2,
        whisker range=10
      },solid] table[y index=0] {muestrasH.csv};
    \end{axis}
  \end{tikzpicture}
\end{preguntas}

\end{document}