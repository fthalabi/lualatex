\def\colegio{Colegio Latinoamericano de Integración}
\def\titulo{Evaluación Recuperativa}
\def\subtitulo{Proporciones}
\def\curso{Séptimo Básico}
\def\puntaje{24}

\documentclass[sin curso]{plantilla-evaluacion-v1}

\begin{document}

Lee los enunciados y encuentra la solución a cada una de las problemáticas.
Se asignará el puntaje a cada pregunta según los criterios que se encuentran
a continuación.

\begin{importante}
  \begin{tblr}{colspec={X[1,c,m]X[6,m]},hline{2-Y}={solid},vline{2-Y}={solid},
    cell{1}{2}={c},cell{2-Z}{2}={cmd=\raggedright}}
    Puntaje asignado & Criterios o indicadores\\
    +3 puntos & Señala clara y correctamente cuál es la solución o el resultado de la pregunta hecha
    en el enunciado.\\
    +3 puntos & Incluye un desarrollo que relata de manera clara y ordenada los procedimientos
    \mbox{necesarios} para solucionar la problemática. En caso de estar incompleto o con
    \mbox{errores} el desarrollo, se asignará puntaje parcial si se muestra dominio de los
    contenidos y conceptos involucrados.\\
    0\% &  La respuesta es incorrecta. De haber desarrollo, este tiene errores conceptuales.\\
  \end{tblr}
\end{importante}

\begin{preguntas}(1)
  \pregunta Un automóvil gasta 9 litros de gasolina cada 120 km. Si quedan en el depósito
  6 litros, ¿cuántos kilómetros podrá recorrer?
  \begin{malla}[height=7cm]
  \end{malla}
  \begin{respuesta}[height=2cm]
  \end{respuesta}

  \pregunta Una piscina se llena en 10 horas con una llave que arroja 120 litros de agua por
  minuto, ¿cuántos minutos tardará para llenarse si esta llave arrojara 80 litros del líquido?
  \begin{malla}[height=7cm]
  \end{malla}
  \begin{respuesta}[height=2cm]
  \end{respuesta}

  \pregunta Teresa tiene en su tienda varios sacos de harina de 18 kg y va a vender cada uno
  en \$108, pero como nadie quiere comprar sacos enteros y decide venderla por kilo.
  Su primer cliente le pidió 4 kg, ¿Cuánto debe cobrarle?
  \begin{malla}[height=7cm]
  \end{malla}
  \begin{respuesta}[height=2cm]
  \end{respuesta}

  \pregunta Si 15 hombres hacen una obra de construcción en 60 días, ¿cuánto tiempo emplearán
  20 hombres para realizar la misma obra?
  \begin{malla}[height=7cm]
  \end{malla}
  \begin{respuesta}[height=2cm]
  \end{respuesta}

\end{preguntas}

\end{document}
