\def\colegio{Colegio Latinoamericano de Integración}
\def\titulo{Estudio}
\def\subtitulo{¿Son mejores los establecimientos privados?}
\def\curso{Octavo Básico}
\documentclass[sin nombre,con autor]{plantilla-evaluacion-v1}

\begin{document}
\section*{Motivación}
Este breve estudio busca demostrar cómo los contenidos trabajados en clases, se pueden
utilizar para estudiar y modelar el mundo que nos rodea; y es que una persona puede
preguntarse: ¿Para qué necesito este tipo de habilidades matemáticas? Si no busco
trabajar en las ciencias o en la ingeniería a futuro. Por el simple hecho que nos
dan herramientas necesarias para ser ciudadanos críticos e informados. \par

Una persona que no desarrolla estas competencias matemáticas básicas, no es capaz de
distinguir entre realidad o ficción, tampoco es capaz de argumentar sus respuestas o
distinguir cuando está siendo manipulado por cifras mal usadas.\par

\section*{Introducción}

En enero, se liberan los resultados de la prueba de acceso a la educación superior; y
cada año se destaca en los medios de comunicación que: 198 de los 200 establecimientos
con mejor desempeño en las pruebas son particulares sin subvención.  Creando una
narrativa de que estos establecimientos imparten una mejor educación; una educación
de calidad. \par

¿Es esto cierto? ¿Es mejor la educación privada que la pública? Usaremos los resultados
de la prueba PAES del año 2024 para responder esta pregunta. \par

\section*{Resultados en la prueba de Matemática (M1)}
\includegraphics{/Users/fenho/Documents/r/paes_2024/mate_dep.pdf}

Se puede observar que los puntajes van aumentando a medida que nos alejamos del
sistema público; y es que, un establecimiento particular sin subvención con 651 puntos,
supera al 98\% de los establecimientos públicos y al 91\% de los subvencionados. Por
otro lado, este
establecimiento con 651 puntos se encuentra en la posición 333 de 444,
superando solo al 25\% de los establecimientos de su mismo tipo.


  \tcbsidebyside[sidebyside adapt=both,sidebyside align=center,
  colframe=white,colback=white,coltitle=black,sidebyside gap=5mm
  ]{%
  \includegraphics{/Users/fenho/Documents/r/paes_2024/media_dep.pdf}
  }{%
  \csvreader[
    head to column names,
    tabularray={
    cells={valign=m},
    colspec={lcc},
    hline{1,2,Z}={black,1pt},
    rows={rowsep+=2pt},
    },
    table head={{Tipo de\\Establecimiento} & Media & Mediana \\}
    ]{/Users/fenho/Documents/r/paes_2024/mat_dep.csv}{}%
    {\dep & \pgfmathprintnumber[verbatim,int trunc]{\media}
      & \pgfmathprintnumber[verbatim,int trunc]{\mediana}}
  }

Esta es una gráfica que intenta comunicar lo mismo que la anterior, pero de manera
más sencilla, resumiendo los datos en un solo punto usando la media.
La tendencia creciente se observa más claramente, pero se pierden ciertos datos,
como por ejemplo: que en establecimientos públicos los puntajes van desde 417 a 855
puntos; o que en dichos establecimientos el 50\% se encuentra entre los 497 y 548
puntos. Así, cada gráfica tiene sus ventajas. \par

Las gráficas anteriores, establecen que hay una gran brecha de
desempeño  entre los distintos tipos de
establecimientos, pero, ¿a qué se debe esta diferencia? Para entender esto,
debemos contextualizar los datos y mirar a los alumnos que asisten a estos
establecimientos.\par

%\includegraphics{/Users/fenho/Documents/r/paes_2024/n_dep_nse.pdf}
\includegraphics{/Users/fenho/Documents/r/paes_2024/n_dep_nse_alt.pdf}

\begin{tasks}[style=itemize](1)
  \task
  En el sector público se concentran los niveles socioeconómicos más bajos y están ausentes
  los niveles de mayores recursos.
  \task
  Establecimientos privados con subvención tienen una composición un poco más equilibrada
  que el sector público, notablemente con una mayor presencia de nivel medio y medio alto.
  \task
  Establecimientos privados sin subvención están compuestos casi
  exclusivamente de los sectores con mayores recursos.
\end{tasks}

Esto nos muestra lo altamente segregado que está el sistema educacional chileno
por niveles socioeconómicos. Veamos que pasa con el desempeño por tipo
de establecimiento, cuando tomamos en cuenta la vulnerabilidad de los alumnos
que asisten a estos establecimientos. \par

\includegraphics{/Users/fenho/Documents/r/paes_2024/mate_dep_nse.pdf}

\begin{tasks}[style=itemize](1)
  \task
  Se puede observar la misma tendencia en cada sector educacional, y es que, a medida que
  incrementa el nivel socioeconómico los puntajes también aumentan.
  \task
  Cuando se comparan diagramas de caja de un mismo nivel socioeconómico, el
  desempeño es similar. Sugiriendo que el desempeño del establecimiento depende mucho más de
  la vulnerabilidad de sus estudiantes que del tipo de dependencia.
\end{tasks}

\tcbsidebyside[sidebyside adapt=both,colframe=white,colback=white,
coltitle=black,lower separated=false,sidebyside gap=3mm
]{%
\csvreader[
head to column names,
tabularray={
cells={valign=m},
colspec={llc},
hline{1,2,Z}={black,1pt},
rows={rowsep+=2pt},
},
table head={{Nivel\\Socioeconómico} & {Tipo de\\Establecimiento} & Media \\}
]{/Users/fenho/Documents/r/paes_2024/mat_dep_nse.csv}{}%
{\nse & \dep & \pgfmathprintnumber[verbatim,int trunc]{\media}}
}{%
\includegraphics{/Users/fenho/Documents/r/paes_2024/media_dep_nse.pdf}
}

Para concluir que los establecimientos privados entregan una mejor educación, primero
hay que solucionar el problema de la segregación; para que la comparación sea justa. Porque
por lo observado acá, es un mejor predictor de buenos resultados el nivel
socioeconómico que el tipo de establecimiento. Ya que todos se desempeñan de manera
similar cuando se compara dentro de un mismo estrato social. \par

Para profundizar en esta temática, recomiendo leer: Formas de Capital (Cap. IV) en
Poder, Derecho y Clases sociales por Pierre Bourdieu. \par

\section*{Fuentes}
\begin{itemize}
  \item Los puntajes por establecimiento fueron recopilados desde la base de datos del DEMRE
  en la página \url{https://colegios.demre.cl/estadistica-resultados-puntajes}
  \item Para la clasificación socioeconómica de los establecimientos se utilizó los datos
  liberados por la Agencia de Calidad de la Educación en
  \url{https://informacionestadistica.agenciaeducacion.cl}
\end{itemize}

\end{document}