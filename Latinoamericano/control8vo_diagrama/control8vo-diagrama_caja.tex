\def\colegio{Colegio Latinoamericano de Integración}
\def\titulo{Control}
%\def\subtitulo{Cuantiles y el diagrama de caja}
\def\curso{Octavo Básico}
\def\puntaje{14}

\documentclass[sin curso]{plantilla-evaluacion-v1}

\begin{python}
  import numpy as np

  def divisores(n):
      div = [i for i in range(1, int(n ** 0.5) + 1) if n % i == 0]
      div.extend([n // i for i in div if i != n // i])
      return np.array(sorted(div))

  def n_barras(lista, ideal):
    if (lista is None or len(lista) == 0):
      return 0
    min = np.min(lista)
    max = np.max(lista) + 1
    rango = np.abs(max-min)
    div_rango = divisores(rango).astype(int)
    n_bars = (rango/div_rango).astype(int)
    distancia = np.abs(n_bars - ideal)
    return n_bars[distancia.argmin()]

  def ticks(muestras):
    t = np.linspace(
      np.min(muestras),
      np.max(muestras)+1,
      n_barras(muestras,5)+1
    ).astype(int)
    return ",".join(t.astype(str))

  def obtener_muestras(m,d,n,s=None):
    l = np.random.default_rng(seed=s).normal(m,d,size=n)
    return np.floor(l).astype(int)

  @np.vectorize(excluded=['lista'])
  def frecuencia(x, lista):
    filtrado = np.extract(lista == x,lista)
    return len(filtrado)

  @np.vectorize(excluded=['lista'])
  def frecuencia_acumulada(x, lista):
    filtrado = np.extract(lista <= x, lista)
    return len(filtrado)

  @np.vectorize(excluded=['lista'])
  def probabilidad(x, lista):
    return frecuencia(x, lista=lista)/len(lista)

  @np.vectorize(excluded=['lista'])
  def probabilidad_acumulada(x, lista):
    return frecuencia_acumulada(x, lista=lista)/len(lista)

  @np.vectorize(excluded=['lista'])
  def cuantil(p, lista):
    unicos = np.unique_values(lista)
    dato = 0
    for i in unicos:
      p_acumulada = probabilidad_acumulada(i, lista=lista)
      if (p_acumulada >= p):
        dato = i
        break
    return dato

  def pgf_coords(x,y):
    return " ".join(f"({i},{j})" for i, j in zip(x,y))

  mH = np.random.default_rng().integers(low=10,high=14,dtype=int)
  dH = np.random.default_rng().integers(low=2,high=4,dtype=int)
  nH = np.random.default_rng().integers(low=15,high=19,dtype=int)
  nAmigosH = obtener_muestras(m=mH,d=dH,n=nH)
  mM = np.random.default_rng().integers(low=14,high=20,dtype=int)
  dM = np.random.default_rng().integers(low=3,high=5,dtype=int)
  nM = np.random.default_rng().integers(low=22,high=26,dtype=int)
  nAmigosM = obtener_muestras(m=mM,d=dM,n=nM)

  np.savetxt("nAmigosH.csv",np.column_stack([
    x := (np.unique_values(nAmigosH)),
    frecuencia(x,lista=nAmigosH),
    frecuencia_acumulada(x,lista=nAmigosH),
    probabilidad(x,lista=nAmigosH),
    probabilidad_acumulada(x,lista=nAmigosH)
    ]),
    fmt=["%d","%d","%d","%.4f","%.4f"],
    delimiter=","
  )

  np.savetxt("muestrasH.csv",nAmigosH,delimiter=",")
  np.savetxt("muestrasM.csv",nAmigosM,delimiter=",")

  def imprimir(l):
    return ", ".join(l.astype(str))

\end{python}%

\NewDocumentCommand{\generarTabla}{m}{%
\begin{tblr}{colspec={ccccc},vlines,hlines,rowsep=6pt,colsep=10pt}
  Mínimo & 1er Cuartil ($Q_1$) & Mediana ($Q_2$) & 3er Cuartil ($Q_3$) & Máximo \\
  \py{np.min(#1)} & \py{cuantil(0.25,lista=#1)} &
  \py{cuantil(0.5,lista=#1)} & \py{cuantil(0.75,lista=#1)} &
  \py{np.max(#1)} \\
\end{tblr}}

\newsavebox{\tabla}
\begin{lrbox}{\tabla}
  \begin{tblr}{colspec={ccccc},vlines,hlines,rowsep=6pt,colsep=10pt}
    Mínimo & 1er Cuartil ($Q_1$) & Mediana ($Q_2$) & 3er Cuartil ($Q_3$) & Máximo \\
           &                     &                 &                     &        \\
  \end{tblr}
\end{lrbox}%

\begin{document}

A continuación, se encuentran los resultados de encuestar a un grupo de estudiantes y
preguntarles a cada uno: \textbf{¿Cuántos amig@s tienes?}. \par

Los resultados se encuentran separados por género. Para los niños, los datos se encuentran
en una tabla de frecuencias; y para las niñas, los datos se encuentran en un gráfico de barras.
Utilice estos valores para completar la tabla de valores de cada género, y así finalmente
dibujar dos diagramas de caja comparando el número de amistades por género.

\NewDocumentCommand{\formatear}{m}{\pgfmathprintnumber[fixed,fixed zerofill,precision=3,verbatim,use comma]{#1}}
\csvreader[no head,
%before reading=\pgfkeys{/pgf/number format/.cd,fixed,fixed zerofill,precision=3,verbatim,use comma},
centered tabularray={
cells={valign=m},
colspec={X[1,c]X[2,c]X[2,c]X[2,c]X[2,c]},
width=0.75\linewidth,
hlines,
vlines,
hline{1,2,3,Z}={black,1pt},
rows={rowsep+=2pt},
cell{1}{1}={r=1,c=5}{c}
},
table head={Número de amig@s (Niños) & F & FA & P & PA \\ Datos & Frecuencia &
  {Frecuencia\\Acumulada} & Probabilidad & {Probabilidad\\Acumulada} \\}
]{nAmigosH.csv}{1=\a, 2=\b, 3=\c, 4=\d, 5=\e}%
{\a & \b & \c & \formatear{\d} & \formatear{\e} }

\begin{preguntas}[after-item-skip=15pt](1)
  \pregunta Complete la tabla de valores usando las respuestas de los niños. [5 puntos] \\[5pt]
  \usebox{\tabla}
  \pregunta ¿Qué nos dice sobre los datos el valor de la Mediana ($Q_2$)? Contextualice
  su respuesta usando la pregunta que aborda la encuesta. [2 puntos]
  \begin{respuesta}[height=4cm]
  \end{respuesta}
\end{preguntas}

\begin{python}
  x_m = np.unique_values(nAmigosM)
  y_m = frecuencia(x_m,lista=nAmigosM)
\end{python}%
%
\begin{center}
  \begin{tikzpicture}[baseline=(current axis.north)]
    \begin{axis}[
        ybar,
        title={Número de amig@s (Niñas)},
        ylabel={Frecuencia},
        xlabel={Número de amig@s},
        xtick distance=3,
        ymin=0,
        %xtick=data,
        %nodes~near~coords,
        %nodes~near~coords~align={vertical},
        ]
        \addplot [ybar,bar width=1,pattern={Lines[angle=-45,distance=5pt]}]
         coordinates {\py{pgf_coords(x_m,y_m)}};
    \end{axis}
  \end{tikzpicture}
\end{center}

\begin{preguntas}[after-item-skip=20pt](1)
  \pregunta Complete la tabla de valores usando las respuestas de las niñas. [5 puntos]\\[5pt]
  \begin{malla}[height=3cm]
  \end{malla}
  \usebox{\tabla}
  \pregunta Usa los datos de ambos grupos para dibujar los diagramas en el espacio
  señalizado. [2 puntos] \\[5pt]
  \begin{center}
    \begin{tikzpicture}[baseline=(current axis.north)]
      \begin{axis}[
        title={Comparando número de amig@s por grupo},
        xlabel={Número de amig@s},
        ytick={1,2},
        yticklabels={Niñas,Niños},
        width=0.85\linewidth,
        height=5cm,
      ]
        \addplot+ [boxplot={
          draw position=1,
          whisker range=10,
        },draw=white] table[y index=0] {muestrasM.csv};
        \addplot+ [boxplot={
          draw position=2,
          whisker range=10,
        },draw=white] table[y index=0] {muestrasH.csv};
      \end{axis}
    \end{tikzpicture}
  \end{center}
\end{preguntas}

\newpage
\begin{minipage}[c][\lineskip+20pt][c]{\linewidth}
  \centering
  \sffamily\Large {\scshape\bfseries Respuestas} - \subtitulo
\end{minipage}

\begin{preguntas}[resume=false,after-item-skip=10pt]
  \pregunta \generarTabla{nAmigosH}
  \pregunta La mediana de los datos es \py{cuantil(0.5,lista=nAmigosH)}, esto nos dice que
  el 50\% de los niños encuestados tiene \py{cuantil(0.5,lista=nAmigosH)} o menos amig@s. En
  otras palabras, el 50\% de los niños tiene a lo más \py{cuantil(0.5,lista=nAmigosH)}
  amig@s.
  \pregunta \generarTabla{nAmigosM}
  \pregunta
  \begin{center}
    \begin{tikzpicture}[baseline=(current axis.north)]
      \begin{axis}[
        title={Comparando número de amig@s por grupo},
        xlabel={Número de amig@s},
        ytick={1,2},
        yticklabels={Niñas,Niños},
        width=0.8\linewidth,
        height=5cm,
      ]
        \addplot+ [boxplot={
          draw position=1,
          whisker range=10,
        }] table[y index=0] {muestrasM.csv};
        \addplot+ [boxplot={
          draw position=2,
          whisker range=10,
        },solid] table[y index=0] {muestrasH.csv};
      \end{axis}
    \end{tikzpicture}
  \end{center}
\end{preguntas}


\end{document}