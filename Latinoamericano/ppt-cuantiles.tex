\documentclass[tema claro]{presentacion}
\usepackage{graphicx}

\begin{python}
import numpy as np
import polars as pl

np.random.seed()
muestras = np.random.default_rng().normal(10,3,size=99)

def generar_tabla(archivo,datos):
  tabla = pl.DataFrame({'valores': np.floor(datos).astype(int)})
  tabla = tabla.group_by('valores').len().sort('valores')
  tabla = tabla.rename({'len':'frecuencia'})
  tabla = tabla.with_columns(
    pl.col('frecuencia').cum_sum().alias('f_cumulativa'),
    (pl.col('frecuencia')/pl.col('frecuencia').sum()).alias('probabilidad'),
  )
  tabla = tabla.with_columns(
    pl.col('probabilidad').cum_sum().alias('p_acumulada')
  )
  tabla.write_csv(archivo,include_header=False)
generar_tabla(archivo='tabla.csv', datos=muestras)
\end{python}


\begin{document}

\begin{frame}

\includegraphics[width=\linewidth]{/Users/fenho/Documents/r/paes_2024/mate_dep.pdf}%

\end{frame}


\begin{frame}
\frametitle{ESTE en un titulo}
\framesubtitle{En cambio este es el subtitulo}
\begin{equation*}
  \frac{1}{2\pi i}\int_\gamma\! f = \sum_{k=1}^m
  n(\gamma;a_k)\,\text{Res}(f;a_k)\,.
\end{equation*}
\end{frame}

\begin{frame}{hola}

\begin{itemize}
  \item hola
  \item hola
  \item hola
  \begin{itemize}
    \item hola
    \item hola
    \item hola

  \end{itemize}

\end{itemize}

\end{frame}

\begin{frame}

\begin{itemize}
\item 2 is prime (two divisors: 1 and 2).
\pause
\item 3 is prime (two divisors: 1 and 3).
\pause
\item 4 is not prime (\alert{three} divisors: 1, 2, and 4).
\end{itemize}

\end{frame}

\begin{frame}
  \frametitle{There Is No Largest Prime Number}

  \begin{enumerate}
  \item<1-> Suppose $p$ were the largest prime number.
  \item<2-> Let $q$ be the product of the first $p$ numbers.
  \item<3-> Then $q + 1$ is not divisible by any of them.
  \item<1-> But $q + 1$ is greater than $1$, thus divisible by some prime
  number not in the first $p$ numbers. $\square$
  \end{enumerate}
  \uncover<4->{The proof used \textit{reductio ad absurdum}.}
\end{frame}

\begin{frame}
  \begin{tikzpicture}
    \begin{axis}[eje escolar]
      \addplot {x^2};
    \end{axis}
  \end{tikzpicture}%
  \begin{uncoverenv}<2>
    \begin{tikzpicture}
      \begin{axis}[eje escolar]
        \addplot {x^3};
      \end{axis}
    \end{tikzpicture}
  \end{uncoverenv}
\end{frame}

\begin{frame}
  \begin{columns}
    \begin{column}[t]{5cm}
    Two\\lines.
    \end{column}
    \begin{column}[t]{5cm}
    One line (but aligned).
    \end{column}
  \end{columns}
\end{frame}

\begin{frame}
  \begin{columns}[t]
    \column{.2\linewidth}
    Two\\lines.
    \column[c]{.5\linewidth}
    \begin{tikzpicture}
      \begin{axis}[eje escolar]
        \addplot {x^3};
      \end{axis}
    \end{tikzpicture}
    \end{columns}
\end{frame}

\begin{frame}

\csvreader[no head,
  before reading=\pgfkeys{/pgf/number format/.cd,fixed,fixed zerofill,precision=3,verbatim,use comma},
  centered tabularray={
  cells={valign=m},
  colspec={X[1,c]X[2,c]X[2,c]X[2,c]X[2,c]},
  hlines,
  hline{1,2,Z}={1pt},
  rows={rowsep=2pt},
  %vline{1,2,4,6} = {1pt,solid},
  %hline{1,2,Z} = {1pt,solid},
  % hline{2}={1}{-}{solid},
  % hline{2}={2}{-}{solid},
},
  table head={Datos & Frecuencia & {Frecuencia\\Acumulada} & Probabilidad & {Probabilidad\\Acumulada} \\}
]{tabla.csv}{1=\a, 2=\b, 3=\c, 4=\d, 5=\e}%
{\a & \b & \c & \pgfmathprintnumber{\d}  & \pgfmathprintnumber{\e}}

\end{frame}

\end{document}