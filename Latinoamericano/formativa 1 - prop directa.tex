\def\colegio{Colegio Latinoamericano de Integración}
\def\titulo{Evaluación Formativa}
\def\subtitulo{Proporción directa}
\def\curso{Séptimo Básico}
\def\puntaje{20}

\documentclass[sin curso]{plantilla-guia-v1}

\begin{document}

\begin{enumerar}
  \item Analizar las tablas y determinar si las variables son directamente proporcionales.
    \begin{ejercicios}(2)
      \ejercicio \\
      \begin{minipage}[b][3cm][t]{0.5\linewidth}
        ¿Son directamente\\ proporcionales?
        \begin{enumerate}
          \item Verdadero
          \item Falso
        \end{enumerate}
      \end{minipage}%
      \begin{minipage}[b][3cm][t]{0.3\linewidth}
        \centering
        \begin{tblr}{colspec={cc},row{1}={black!10},vlines,hlines,cells={mode=math}}
          a & b \\
          6 & 8 \\
          12 & 4 \\
          18 & 2 \\
        \end{tblr}
      \end{minipage}
      \ejercicio \\
      \begin{minipage}[b][3cm][t]{0.5\linewidth}
        ¿Son directamente\\ proporcionales?
        \begin{enumerate}
          \item Verdadero
          \item Falso
        \end{enumerate}
      \end{minipage}%
      \begin{minipage}[b][3cm][t]{0.3\linewidth}
        \centering
        \begin{tblr}{colspec={cc},row{1}={black!10},vlines,hlines,cells={mode=math}}
          a & b \\
          6 & 8 \\
          12 & 4 \\
          18 & 2 \\
        \end{tblr}
      \end{minipage}
    \end{ejercicios}
  \item Encontrar los valores faltantes para las siguientes proporciones directas.
  \item A partir de la grafica, cuanto vale la constante de proporcionalidad. encuentre 
  el valor de easds
\end{enumerar}


\end{document}