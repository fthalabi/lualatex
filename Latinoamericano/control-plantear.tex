\def\colegio{Colegio Latinoamericano de Integración}
\def\titulo{Control}
\def\subtitulo{Plantear y resolver ecuaciones}
\def\curso{Octavo Básico}
\def\puntaje{12}

\documentclass[sin curso]{plantilla-evaluacion-v1}

\begin{document}

\begin{center}
  \vspace*{5pt}
  \begin{tcolorbox}[borderline={1pt}{0pt}{black,dashed},colframe=white,colback=white]
    Asignación de puntaje: 1 punto por cada pregunta con alternativas y 2 puntos por cada pregunta abiertas.
  \end{tcolorbox}
  \vspace*{5pt}
\end{center}

\begin{partes}
  \parte Al resolver la ecuación $3x-2x-7=8x+14$, ¿cuál es el valor de x?
  \begin{vertical}
    \alternativa $-7$
    \alternativa $-3$
    \alternativa $3$
    \alternativa $14$
  \end{vertical}

  \parte La mitad de un número, aumentada en 5 unidades es equivalente al mismo
  número disminuido en dos unidades. ¿Cuál es el número?
  \begin{vertical}
    \alternativa $-14$
    \alternativa $-10$
    \alternativa $10$
    \alternativa $14$
  \end{vertical}

  \parte Si $n$ globos cuestan $x$ pesos, ¿cuánto cuesta un globo?
  \begin{vertical}
    \alternativa $x\div n$
    \alternativa $n\div x$
    \alternativa $n-x$
    \alternativa $x-n$
  \end{vertical}

  \parte La edad de Luis es el doble que la de Margarita, y la diferencia es de 
  12 años. ¿Cuál es la edad de Luis? 
  \begin{vertical}
    \alternativa 6 años
    \alternativa 10 años
    \alternativa 12 años
    \alternativa 24 años
  \end{vertical}

  \parte Sofía quiere comprar un juego para su consola. Para esto, ha ahorrado
  \$17000. ¿Cuánto más tendrá que ahorrar si el juego que desea cuesta \$35000?
  ¿Qué ecuación representa esta situación? 
  \begin{vertical}
    \alternativa $17000 +x=35000$
    \alternativa $35000 +x=17000$
    \alternativa $17000 +35000=x$
    \alternativa $17000 -35000=x$
  \end{vertical}

  \parte Si se cumple que: $\dfrac{1}{2}x-2=\dfrac{3}{4}$. ¿Qué valor tiene $x$? 
  \begin{vertical}[itemsep=5pt]
    \alternativa $-\dfrac{5}{2}$
    \alternativa $\dfrac{11}{2}$
    \alternativa $11$
    \alternativa $22$
  \end{vertical}

  \parte El padre de Sandra tiene 43 años, 4 años más que el triple de la edad de 
  Sandra. ¿Cuál es la edad de Sandra? 
  \begin{malla}[height=3cm]
  \end{malla}
  \begin{respuesta}[height=1.5cm]
  \end{respuesta}
  \parte La suma de 4 números es 90. El segundo número es el doble del primero; 
  el tercero es el doble del segundo y el cuarto, el doble del tercero. ¿Cuáles
  son los números?
  \begin{malla}[height=3cm]
  \end{malla}
  \begin{respuesta}[height=1.5cm]
  \end{respuesta}
  \parte La suma de 3 números consecutivos es 66. ¿Cuáles son los números?
  \begin{malla}[height=3cm]
  \end{malla}
  \begin{respuesta}[height=1.5cm]
  \end{respuesta}

\end{partes}
\end{document}