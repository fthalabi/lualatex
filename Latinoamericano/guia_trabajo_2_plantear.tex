\def\colegio{Colegio Latinoamericano de Integración}
\def\titulo{Guía de trabajo}
\def\subtitulo{Plantear y resolver usando ecuaciones}
\def\curso{Octavo Básico}

\documentclass[sin nombre]{plantilla-evaluacion-v1}

\begin{document}
\begin{partes}
  \parte Escribe una expresión algebraica que represente cada caso.
  \begin{ejercicios}
    \ejercicio Un número aumenta en 2.
    \begin{malla}[height=2cm]
    \end{malla}
    \ejercicio El quíntuple de un número.
    \begin{malla}[height=2cm]
    \end{malla}
    \ejercicio El sucesor del doble de un número.
    \begin{malla}[height=2cm]
    \end{malla}
    \ejercicio El triple de un número es igual al número más 8.
    \begin{malla}[height=2cm]
    \end{malla}
  \end{ejercicios}

  \parte Representa cada enunciado con una ecuación.
  \begin{ejercicios}(1)
    \ejercicio La suma de dos números consecutivos aumentada en 10 unidades equivale
    al mayor de ellos aumentado en 9 unidades.
    \begin{malla}[height=2cm]
    \end{malla}
    \ejercicio Un número equivale a la cuarta parte del número disminuido en 3 unidades.
    \begin{malla}[height=2cm]
    \end{malla}
    \ejercicio La tercera parte de un número disminuida en 10 unidades equivale al
    triple del número.
    \begin{malla}[height=2cm]
    \end{malla}
    \ejercicio La suma de tres números pares consecutivos equivale a 42 unidades.
    \begin{malla}[height=2cm]
    \end{malla}
  \end{ejercicios}
  
  \parte Plantea una ecuación para cada problema y luego resuelve.
  \begin{ejercicios}(1)
    \ejercicio La producción de un evento tiene un costo de \$1.500.000. Si 
    cada entrada se vende a \$10.000, ¿cuántas entradas hay que vender 
    para obtener una ganancia de \$800.000?
    \begin{malla}[height=6cm]
    \end{malla}
    \ejercicio Sofía como 3/4 kg de pan, 5/8 kg de queso y
    gastó \$4.560. Si el precio de un kilogramo de pan es de \$1.800, ¿cuál 
    es el precio de 1 kg de queso?
    \begin{malla}[height=6cm]
    \end{malla}
    \ejercicio Tomás tiene tres cuartos de la edad de su hermana mayor. Si las
    edades de ambos suman 35 años, ¿qué edad tiene su hermana?
    \begin{malla}[height=6cm]
    \end{malla}
  \end{ejercicios}
\end{partes}
\end{document}