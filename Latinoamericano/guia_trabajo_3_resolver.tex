\def\colegio{Colegio Latinoamericano de Integración}
\def\titulo{Guía de trabajo}
\def\subtitulo{Plantear y resolver usando ecuaciones II}
\def\curso{Octavo Básico}

\documentclass[sin nombre]{plantilla-evaluacion-v1}

\begin{document}
\begin{partes}
\parte Plantea una ecuación para cada problema y luego resuelve.
\begin{ejercicios}(1)
  \ejercicio Si el perímetro de un rectángulo es 96,6 cm y la medida del largo
  es el doble que la medida del ancho, ¿cuáles son sus dimensiones?
  \begin{malla}[height=5cm]
  \end{malla}
  \ejercicio Una avenida está siendo asfaltada por etapas. En la primera etapa
  se asfaltó la mitad; en la segunda, la quinta parte, y en la tercera, la 
  cuarta parte del total. ¿Cuál es la longitud de la avenida si aún faltan 
  200 m por asfaltar?
  \begin{malla}[height=5cm]
  \end{malla}
  \ejercicio Un triángulo isósceles tiene un perímetro de $(3x+19)$ cm. Si la medida
  de su base es $(x+5)$ cm, ¿cuánto miden sus lados?
  \begin{malla}[height=5cm]
  \end{malla} 
  \ejercicio Las medidas de los lados de un rectángulo se diferencian en 12 cm. Si la 
  medida del lado de menor longitud es (2x+20), ¿cuál es el área y el perímetro del 
  rectángulo?
  \begin{malla}[height=5cm]
  \end{malla}
  \ejercicio Un joyero regalará su colección de relojes. La mitad se la dará a su hija,
  la tercera parte del resto se la regalará a su nieta, la mitad de los que quedan se los
  entregará a su sobrino y el resto, que son cinco relojes, se los dará a su hermano.
  \begin{preguntas}
    \pregunta ¿Cuántos relojes tiene su colección?
    \pregunta ¿Cuántos relejes recibirá cada uno?
  \end{preguntas}
  \begin{malla}[height=8cm]
  \end{malla}
\end{ejercicios}
\end{partes}
\end{document}