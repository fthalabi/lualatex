\def\colegio{Colegio Latinoamericano de Integración}
\def\titulo{Evaluación}
\def\subtitulo{Lenguaje Algebraico}
\def\puntaje{20}

\documentclass[sin curso]{plantilla-guia-v1}

\NewDocumentCommand{\comparar}{mmm}{
  \begin{tikzpicture}[line width=1pt,baseline=(X.base)]
    \matrix[column sep={6pt},ampersand replacement=\&]{
      \node (X) {#1}; \&[12pt]
      \node {#2}; \& 
      \node [draw,rounded corners,minimum height=30pt, minimum width=30pt] {}; \&
      \node {#3}; \\
    };
  \end{tikzpicture}
} 

\begin{document} 

\textbf{Objetivos}
\begin{lista}
  \item Transformar de lenguaje natural a algebraico.
  \item Evaluar expresiones algebraicas.
  \item Reducción de expresiones algebraicas.  
\end{lista}

% \begin{malla}[height=2cm,enlarge top by=10pt]
% \end{malla}
% \begin{respuesta}[height=1cm,enlarge top by=5pt]
% \end{respuesta}

\begin{enumerar}
  \item Expresar en lenguaje algebraico y señalé que letra corresponde a cada objeto [2 puntos cada una].
  \begin{ejercicios}(1)
    \item Un pantalón tiene el mismo valor que dos poleras.
    \begin{respuesta}[height=1.5cm]
    \end{respuesta}
    \item Una manzana vale 100 pesos más que una pera.
    \begin{respuesta}[height=1.5cm]
    \end{respuesta}
  \end{ejercicios}

  \item Analiza las situaciones y completa cada expresión con el signo \(>\), \(<\) o \(=\) según corresponda [1 punto cada una].
  \begin{ejercicios}(2)
    \item \comparar{Si $x=0.1$}{$\dfrac{2}{2x}$}{$\dfrac{1}{x}$}
    \item \comparar{Si $x=0.01$}{$\dfrac{2}{x}$}{$\dfrac{3}{x}$}
    \item \comparar{Si $x=0.1$}{$\dfrac{0.9+x}{0.1}$}{$\dfrac{0.1}{0.01}$}
    \item \comparar{Si $x=2$}{$\dfrac{1}{x}$}{$\dfrac{1}{x^2}$}
  \end{ejercicios}

  \item Calcula el valor de las expresiones [1 punto cada una].
  \begin{ejercicios}(2)
    \item Si \quad $a=2$ \quad y \quad $b=3$ \\ 
    $(a+b)-(a-b)$
    \begin{malla}[height=3cm]
    \end{malla}
    \item Si \quad $x=1$ \quad y \quad $y=-4$ \\ 
    $(-y-x)+(2x+y)$
    \begin{malla}[height=3cm]
    \end{malla}
    \item Si \quad $g=5$ \\ 
    $g^2-g+7$
    \begin{malla}[height=3cm]
    \end{malla}
    \item Si \quad $a=4$ \quad y \quad $b=-2$ \\ 
    $a\cdot{b} +a^2 -(-b)$
    \begin{malla}[height=3cm]
    \end{malla}
  \end{ejercicios}

  \item Expresa en forma reducida el \textbf{área} y \textbf{perimetro} 
  de la siguiente figura, considerando que la parte superior $A$ es un cuadrado y 
  la parte inferior $B$ es un rectángulo [4 puntos]. 

    \begin{tikzpicture}[line width=1pt]
      \def\x{1.5}
      \def\y{2}
      \def\z{5}
      \pgfmathsetmacro{\dif}{(\z-\x)/2}
      \draw[solid] (0,0) --++(90:\x cm) coordinate (A1)
        --++(0:\x cm) coordinate (A2) 
        --++(-90:\x cm) coordinate (A3)
        --++(0:\dif cm) coordinate (B1)
        --++(-90:\y cm) coordinate (B2)
        --++(-180:\z cm) coordinate (B3)
        --++(90:\y cm) --++(0:\dif cm);
      \draw[dashed] (0,0) --++(0:\x cm);
      
      \node at ($(A1)!0.5!(A3)$) {$A$};
      \node at ($(B1)!0.5!(B3)$) {$B$};
      \draw [decorate, decoration={calligraphic brace,raise=5pt,amplitude=5pt}] %% mirror para invertir
       (A2) -- (A3) node [midway,xshift=15pt] {$x$};
      \draw [decorate, decoration={calligraphic brace,raise=5pt,amplitude=5pt}] %% mirror para invertir
       (B1) -- (B2) node [midway,xshift=18pt] {$y$};
      \draw [decorate, decoration={calligraphic brace,raise=5pt,amplitude=5pt}] %% mirror para invertir
       (B2) -- (B3) node [midway,yshift=-18pt] {$z$};

      % \node [draw, rectangle, inner sep=5mm] (A) {$A$};
      % \node [draw, rectangle, inner xsep=20mm, inner ysep=7mm,anchor=north] (B) at (A.south) {$B$};
      % \draw [decorate, decoration={calligraphic brace,raise=5pt,amplitude=5pt}] %% mirror para invertir
      % (A.north east) -- (A.south east) node [midway,xshift=15pt] {$x$};
      % \draw [decorate, decoration={calligraphic brace,raise=5pt,amplitude=5pt}] %% mirror para invertir
      % (B.north east) -- (B.south east) node [midway,xshift=18pt] {$y$};
      % \draw [decorate, decoration={calligraphic brace,mirror,raise=5pt,amplitude=5pt}] %% mirror para invertir
      % (B.south west) -- (B.south east) node [midway,yshift=-18pt] {$z$};
    \end{tikzpicture}    

    \begin{respuesta}[height=3cm]
    \end{respuesta}

  \item Reduce las expresiones algebraicas [1 punto cada una].
  \begin{ejercicios}(1)
    \item $6x +4y -2z +5x -18y +9z$
    \begin{malla}[height=3cm]
    \end{malla}
    \item $15f -13g +5 -4f -8g +9$
    \begin{malla}[height=3cm]
    \end{malla}
    \item $7bc +15bd +4bc -12bd$
    \begin{malla}[height=3cm]
    \end{malla}
    \item $14uv -5 -7v +13 -12uv + 5v -2uv -1$
    \begin{malla}[height=3cm]
    \end{malla}
  \end{ejercicios}
  
\end{enumerar}



\end{document}
